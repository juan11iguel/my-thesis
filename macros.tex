% This allows to place QR codes in the caption of figures
\newsavebox{\captionqr}
\newsavebox{\captionqrleft}
\newsavebox{\captionqrright}

% Declare maths operators 
\DeclareMathOperator{\lew}{Le}
\DeclareMathOperator{\Me}{Me}
\DeclareMathOperator{\Rey}{Re}
\DeclareMathOperator{\Pra}{Pr}
\DeclareMathOperator{\Nus}{Nu}
\DeclareMathOperator{\LMTD}{LMTD}


% See glossary.tex for glossary entries and commands to add new entries
\newcommand{\fullgls}[1]{\glsentryfull{#1}} % Alias to print full form of a glossary entry

%----------------------------------------------------------------------------------------
%	BOXES
%----------------------------------------------------------------------------------------

\newcommand{\marginreminder}[3][0pt]{%
  \marginnote[#1]{%
    \begingroup
      \edef\tempTitle{Reminder: #2}%
      \begin{kaobox}[title=\tempTitle,
        colback=YellowGreen!25!white,
        colbacktitle=YellowGreen!25!white,
        colframe=black,
        fonttitle=\bfseries,
        titlerule=0.4pt
      ]
        #3
      \end{kaobox}
    \endgroup
  }%
}

\newcommand{\reminder}[2]{%
  \begingroup
    \edef\tempTitle{Reminder: #1}%
    \begin{kaobox}[title=\tempTitle,
      colback=YellowGreen!25!white,
      colbacktitle=YellowGreen!25!white,
      colframe=black,
      fonttitle=\bfseries,
      titlerule=0.4pt
    ]
      #2
    \end{kaobox}
  \endgroup
}

\newcommand{\annotation}[2]{%
  \begingroup
     \def\tempTitle{#1}%
      \begin{kaobox}[title=\tempTitle,
        colback=TealBlue!25!white,
        colbacktitle=TealBlue!25!white,
        colframe=black,
        fonttitle=\bfseries,
        titlerule=0.4pt
      ]
        #2
	\end{kaobox}
  \endgroup
}

\newcommand{\marginannotation}[3][0pt]{%
  \marginnote[#1]{%
    \begingroup
      \def\tempTitle{#2}%
      \begin{kaobox}[title=\tempTitle,
        colback=TealBlue!25!white,
        colbacktitle=TealBlue!25!white,
        colframe=black,
        fonttitle=\bfseries,
        titlerule=0.4pt
      ]
        #3
      \end{kaobox}
    \endgroup
  }%
}

\newcommand{\tldrbox}[1]{%
    \begin{kaobox}[title=TL;DR,
      colback=Orchid!25!white,
      colbacktitle=Orchid!25!white,
    %   colframe=black,
      fonttitle=\bfseries,
      titlerule=0.05pt
    ]
	#1
    \end{kaobox}
}

\newcommand{\wipbox}[1]{%
  \begingroup
      \def\tempTitle{\faIcon{wrench} \faIcon{snowplow} 
      \faIcon{user-clock} W O R K       I N       P R O G R E S S
      \faIcon{screwdriver} \faIcon[regular]{hourglass} 
      \faIcon{truck-loading} \faIcon{space-shuttle}}
      \begin{kaobox}[title=\tempTitle,colback=Dandelion!85!white,colbacktitle=Dandelion!85!white,]
        Esta sección no está terminada. Si quieres puedes echarle un ojo para ver
        la estructura y cómo encaja con el resto pero no merece la pena revisarla
        en detalle en el estado actual.

        #1
      \end{kaobox}
  \endgroup
}

% TODO: This should be a box with a counter so it can be referenced
\newcommand{\problemdefinitionbox}[2]{%
  \begingroup
    \edef\tempTitle{Problem: #1}%
    \begin{kaobox}[title=\tempTitle,
      colback=Snow3!25!white,
      colbacktitle=Snow3!25!white,
    ]
	#2
    \end{kaobox}
  \endgroup
}

\newcommand{\wideepigraph}[2]{%
  {%
    \setlength{\epigraphwidth}{0.95\linewidth}%
    \epigraph{#1}{#2}%
  }%
}

% Command: \trianglemarker[color][direction]
% color      -> HTML hex (default: D6B656) or any xcolor name
% direction  -> 'up' (default) or 'down'
\newcommand{\trianglemarker}[2][D6B656]{%
  \begingroup
    % define a temporary color from the provided hex (HTML model)
    \definecolor{tempcol}{HTML}{#1}%
    \tikz[baseline=-0.0ex,scale=1]{%
      \ifstrequal{#2}{down}{%
        % downward pointing triangle
        \fill[tempcol] (0,0.2)--(0.3,0.2)--(0.15,0)--cycle;
      }{%
        % upward pointing triangle (default)
        \fill[tempcol] (0,0)--(0.3,0)--(0.15,0.2)--cycle;
      }%
    }%
  \endgroup
}