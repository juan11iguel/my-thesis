\setchapterpreamble[u]{\margintoc}
\chapter{Hybrid modelling of a solar driven \gls{medLabel} system}
\labch{solarmed:modelling}

\glsresetall % Reset glossary entries

\tldrbox{ This chapter presents the discrete and complete dynamic modelling of
    the \gls{solarmedLabel} system.  First, dynamic physics-based models are
    developed for the solar field, heat exchanger, thermal storage, three-way
    valve, and a data-driven static model for the \gls{medLabel} plant.
    
    These models are then combined with the discrete behavior of the
    installation. This is represented by means of two supervisory
    \glspl{fsmLabel} that define the operation states of the \gls{sftsLabel} and
    the \gls{medfsmLabel}. Each \gls{fsmLabel} determines subsystem activation
    and transitions based on system inputs, internal rules, and configurable
    parameters such as cooldown or startup durations.
    
    The integrated hybrid model is evaluated under realistic operation
    conditions with different prediction horizons, showing good agreement with
    experimental data. Results demonstrate that the model accurately reproduces
    the coupled dynamics of the system, maintaining \gls{mapeLabel} below 15~\%
    for multi-hour predictions while preserving computational efficiency. }

%===================================
%===================================
\section{Introduction}

The behavior of the \gls{solarmedLabel} process can be abstracted into two
components, a continuous and a discrete one. Each component is described and
validated in the respective \nrefsec{solarmed:modelling:dynamic} and
\nrefsec{solarmed:modelling:discrete}. Then, they are combined to
create a complete model of the \gls{solarmedLabel} process in
\nrefsec{solarmed:modelling:complete}.

% The continuous-dynamic behavior of the process is described by a
% set of differential equations, while the discrete behavior is described by
% \glspl{fsmLabel}. The \glspl{fsmLabel} are used to model the
% operation state of the system while the process variables describe the
% continuous-dynamic behavior.

% \begin{itemize}
%     \item Operation state. 
%     \item Process variables
% \end{itemize}

%===================================
%===================================
\section{Dynamic modelling. Process variables}[Dynamic modelling]
\labsec{solarmed:modelling:dynamic}

The dynamic behavior of the \gls{solarmedLabel} governs the evolution of the
continuous process variables. It is represented by a set of models, one for each
system component. A discrete representation\sidenote{Not to be confused with the
discrete model, see \refsec{intro:modelling:discrete}} is used: process
variables are sampled at a fixed interval, $T_s$, and system dynamics are
expressed through difference equations.

In most cases, this representation captures the transient behavior of the
system. However, some models described in the following sections are
steady-state approximations. While this can introduce discrepancies during
transient events, the impact is minor. The model is intended primarily for
optimization, with sampling times on the order of minutes. Moreover, inputs to
slower components ---like the \gls{medLabel}--- are adjusted infrequently
(typically at intervals of 30 minutes or more) allowing sufficient time for the
system to reach steady state\sidenote[][*-5]{Further discussed in
\nrefsec{intro:modelling}}.


%===================================
%===================================
\subsection{Solar field}

\marginnote[*-1]{In all models, it has been assumed that model parameters
like heat transfer coefficient (U) or heat loss coefficients (H) are constant,
and not a function of temperature.}

% Describe input and outputs
The (flat-plate collector) solar field can be seen as a converter of electrical
to thermal energy, subject to irradiance availability. The main outputs, in
terms of operation of the solar field, are the thermal power obtained,
$\dot{Q}_{sf}$ (kW$_{th}$), at what temperature that heat is obtained,
$T_{sf,out}$ ($^\circ$C), and the electricity needed to do so, $C_{e,sf}$
(kW$_{\text{e}}$).


The diagram in \reffig{solarmed:diagrams:solar_field} illustrates the individual
loops that make up the field. In the model, it is assumed that all loops have
equal flow rates and temperatures\sidenote[][*-3]{\ie, a balanced flow
distribution with similar collectors, Which is the case in the experimental
facility for the considered loops 2--5}. As a result, the system can be
simplified to a single loop with a collector area equal to the sum of the
collector areas of the individual rows of collector loops.

% References
A first-principles model ---see \refmod{solarmed:sf}--- based on the one presented
in Ampuño et al.~\sidecite[*-3]{ampuno_modeling_2018} is used to model the solar
field. The model has two types of parameters: dynamic and constant\sidenote[][*2]{Also
called model and fixed model parameters, respectively}. The dynamic parameters
are the thermal loss coefficient, $H_{sf}\,\left( \frac{J}{s·^\circ C} \right)$,
which relates thermal losses to the environment and the gain coefficient,
$\beta$ (m), encompassing the collector transmisitivity and absorstance, and
determines the amount of irradiance that is transferred to the working fluid.
These two dynamic parameters are calibrated using experimental data, and their
values, together with the constant parameters, are presented in
\reftab{solarmed:modelling:models-parameters}.

The main difference of the model developed in this research work with respect to
the model presented in~\cite{ampuno_modeling_2018} is how the apparent transport
delay is
modelled~\sidecite[*-11]{ampuno_apparent_2019}\sidenote[][*-5]{Transport delays
are a common feature in dynamic systems, where the response of the system to an
input is not instantaneous, but rather delayed by a certain amount of time. This
delay can be caused by various factors. In this particular system, is due to the
time it takes for the water to flow through the solar field and reach the
temperature sensors. The apparent delay is the result of adding up the
individual ---different delays of each collector cell}. In this implementation,
the transport delay is simplified to a single steady state parameter based on
the work presented in Normey-Rico~\etal~\sidecite[*3]{normey-rico_robust_1998}
since delays vary less than 30~\% from the nominal value.

% Bloque de modelo con interfaz y ecuaciones
\begin{modelcounter}{Solar field}
    \begin{align*}
        T_{\text{out}}(k) &= \text{sf\:model}\left(
            T_{\text{out},k\!-\!1},\,
            \mathbf{T}_{\text{in},k\!-\!n:k},\,
            \boldsymbol{q}_{k\!-\!n:k},\,
            I_k,\,
            T_{\text{amb},k};\,
            \beta,\,
            H_{sf},\,
            \theta
        \right) \\
        & L_{pipe,eq} = \frac{T_s}{A_{pipe,eq}} \sum_{k=0}^{n} q_{sf}[k] \tag*{{\small\textit{Equivalent pipe length [m]}}} \\
        & L_{eq} = n_{c-s} \cdot L_{tb} \tag*{{\small\textit{Eq. collector tube length [m]}}} \\
        & c_f = n_{c-loop} \cdot n_{tb-c} \tag*{{\small\textit{Conversion factor [-]}}} \\
        & K_{1} = \beta / (\rho \cdot c_p \cdot A_{cs}) \tag*{{\small\textit{[K·m$^2$/J]}}} \\
        & K_{2} = H / (L_{pipe,eq} \cdot A_{cs} \cdot \rho \cdot c_p) \tag*{{\small\textit{[1/s]}}} \\
        & K_{3} = 1 / (L_{pipe,eq} \cdot A_{cs} \cdot c_f) \cdot (1/3600) \tag*{{\small\textit{[h/(3600·m$^3$·s)]}}} \\
        & T_{\text{out}}(k) = T_{\text{out}}(k-1) + \Big( \\
        & \qquad + K_1 \cdot I \tag*{{\small\textit{Solar contribution [K/s]}}} \\
        & \qquad - K_2\cdot (\overline{T}-T_{amb}) \tag*{{\small\textit{Environment losses [K/s]}}} \\ 
        & \qquad - K_3 \cdot \boldmath{q}_{k-n_d} \left( T_{\text{out},k-1} - T_{\text{in},k-n_{d}} \right) \tag*{{\small\textit{Heat absorbed [K/s]}}} \\
        & \Big) \cdot T_s 
    \end{align*}
    \labmod{solarmed:sf}
\end{modelcounter}

% Diagrama del componente con variables
\begin{marginfigure}[-5.5cm]
    \includegraphics[]{solarmed-diagrams-solar_field_compact.png}
    \caption{Solar field process diagram.}
    \labfig{solarmed:diagrams:solar_field}
\end{marginfigure}


The number of delay samples ($n_d$) depends on the model sample time and a system
parameter called the equivalent length. The following
procedure was followed to estimate it:

\begin{enumerate}
    \item Using a reference test with a fixed sample time, $T_s$, the
    number of delay samples ($n_{d}$) was manually fitted to the data, by
    visually inspecting the response of the system to a step change in the
    input flow.
    \item Estimate the equivalent length of the solar field by taking the
    average flow rate ($\overline{q}_{sf}$) across the delay samples
    span\sidenote{In reverse order, from newest to oldest}, and divide it
    by a fixed parameter- the solar field pipe equivalent cross-sectional
    area, $A_{pipe,eq}$.
    \[ 
        \overline{q}_{sf} = \sum_{k=-n_{d}}^{k=0}{q(k)/n_{d}} \\ 
    \]
    \[ 
        L_{pipe,eq} = \frac{\overline{q}_{sf} \times T_s \times n_{d}}{A_{pipe,eq}} 
    \]
    \item With this equivalent length ($L_{pipe,eq}$), the number of delay
    samples can be estimated for any sample time $T_s$ and flows vector
    $\pmb{q}_{sf}$ by iteratively adding the distance that flow travels at
    each sample time until the equivalent length is reached.
\end{enumerate}

%================================
\subsubsection{Electrical consumption}

\begin{definition}
    \textbf{Step train test}. Variations in the \gls{vfdLabel} pump speed
    from a minimum to a maximum value, with fixed increments.
\end{definition}

% Gráfica bonica para test 20240927 y 20240925 (unir ambos ensayos en el mismo
% dataframe y romper el eje x para evitar la discontinuidad grande entre días)
\begin{figure*}[]
    \includegraphics[]{solar_field_eletricity_caractherization_tests.png}
    \savebox\captionqr{\qrcode[hyperlink,height=0.5in]{\repositoryBaseUrl/assets/sf_tests_viz.zip}}
    \caption{Solar field and thermal storage electrical
    characterization tests.\hspace{1ex}\usebox\captionqr}
    \labfig{solarmed:modelling:sf:electrical_consumption_tests}
\end{figure*}

% Explicar configuración/situación hidráulica del campo
The \textit{AQUASOL-II} solar field consists of a set of pumps that recirculate
water through the system. The pumps are controlled by \glspl{vfdLabel}
that allow to vary the flow rate through the solar field. A main recirculation
pump ($P_{l0}$) is responsible for the primary flow, while additional pumps
($P_{l1}$, $P_{l2}$, etc.) are used in the individual loops to either increase
the total flow rate or to operate with the isolated loop. This redundancy means
that the same flow rate can be achieved with different pump configurations.

% Mostrar ensayo para caracterizar el consumo eléctrico del campo solar
The electrical consumption of the solar field is characterized by determining
the relationship between flow rate and power consumption for each configuration.
This allows for the identification of the configuration that minimizes
electrical consumption across the range of operating flow rates. Once this
characterization is established, the overall electrical consumption of the solar
field can then be modelled.

% Describir metodología del ensayo

% Results
A series of tests were performed as can be seen in
\reffig{solarmed:modelling:sf:electrical_consumption_tests}. The tests were
carried out in two different dates since they have to be performed early, before
the solar field is irradiated by the sun and the field heats up\sidenote{Observe
the trend in \reffig{solarmed:modelling:sf:electrical_consumption_tests} -
\textit{Temperatures}. The flow needs to be purposely regulated to maintain
safe solar field temperatures}. In the first day, step trains are applied to the
main loop and individual isolated loops\sidenote{20240925 07:15 -- 08:30}. On
the second day, different speeds levels were set for the main recirculation pump
(10--100~\%, 10~\% increments) while step trains were applied to the individual
loops (40--100~\%, 20~\% increments)\sidenote{In
\reffig{solarmed:modelling:sf:electrical_consumption_tests}, from 20240927 07:35
to 08:20}.

% Flow vs power consumption
\begin{figure}
    \includegraphics[width=\textwidth]{sf_flow_vs_power_consumption.png}
    \savebox\captionqr{\qrcode[hyperlink,height=0.5in]{\repositoryBaseUrl/figures/sf_flow_vs_power_consumption.html}}
    \caption[Solar field flow rate \vs consumption for different pump configurations]{Solar field flow rate for different pump configurations and their associated power consumption.\\[1ex] \usebox\captionqr}
    \labfig{solarmed:modelling:sf:flow_vs_power_consumption}
\end{figure}

% Describir selección de configuración para finalmente realizar calibración
\reffig{solarmed:modelling:sf:flow_vs_power_consumption} shows the relationship
between flow rate and power consumptions for different configurations and pump
speeds. Up to 91.9~l/min the best configuration is to just use the main
recirculation pump. Above this flow rate, the main pump is used in combination
with the individual loop pumps. First, a combination of main pump from 85 to
100~\% and individual loops fixed at their 40~\% minimum speed up until 105~l/min,
then the main pump fixed at 100~\% and individual loops at increasing values from
40 to 100~\% until a maximum flow rate of 148~l/min is achieved. 

\begin{equation}
    \text{Optimal configuration}(q_{sf}) =
    \begin{cases}
    l_{0,cv}\in [0,100]\,\% \land \: l_{i,cv}=0\,\%, & 6.2 < q_{sf} \leq 91.9 \:(\text{l/min})\\[6pt]
    l_{0,cv}\in [0,100] \land \: l_{i,cv}=40, & 91.9 < q_{sf} \leq 105 \\[6pt]
    l_{0,cv}=100 \land \: l_{i,cv}\in [40,100], & 105 < q_{sf} \leq 148
    \end{cases}
\end{equation}

With this optimal selection, a third-order polynomial regression is fitted to
the data, with a coefficient of determination of $R^2 = 0.99$.

% [-8.47935428e-02, 2.29091363e-02, -8.72483421e-04, 1.29582523e-05]
\begin{modelcounter}{Solar field electrical consumption}
    \begin{align*}
        C_{e,sf}\,[kW_e] &= \text{sf electrical consumption}(q_{sf}\,[m^3/h]) \\
        C_{e,sf} &= 1.3\cdot{10}^{-5} \cdot q_{sf}^3 + -8.72\cdot{10}^{-4} \cdot q_{sf}^2 + 2.29\cdot{10}^{-2} \cdot q_{sf} + -8.48\cdot{10}^{-2}
    \end{align*}
    \labmod{solarmed:sf:electrical_consumption}
\end{modelcounter}


Summarizing, the electrical consumption of the solar field is modelled as a
function of the flow rate through the solar field from a minimum value of
3.7~m$^3$/h (50~W) to a maximum value of 88.84 m$^3$/h (4.2~kW). This is
achieved as the result of different combinations of the main recirculation pump
and the individual loops depending on the working flow range.


%================================
\subsubsection{Validation}

\begin{figure*}[h!]
    \includegraphics[width=\linewidth]{solar_field_validation.png}
    \savebox\captionqrleft{\qrcode[hyperlink,height=0.5in]{\repositoryBaseUrl/figures/solar_field_validation_20231106.html}}
    \savebox\captionqrright{\qrcode[hyperlink,height=0.5in]{\repositoryBaseUrl/figures/solar_field_validation_20230511.html}}
    \caption[Solar field model calibration and validation tests]{Solar field model calibration and validation tests. The colors in
    the validation plot represent the different loops\hfill
    \usebox\captionqrleft\hspace{1ex}\usebox\captionqrright}
    \labfig{solarmed:modelling:validation:solar-field}
\end{figure*}

In order to calibrate the model parameters ($\beta$ and $H$), one
representative experimental test is used\sidenote{See
\reffig{solarmed:modelling:validation:solar-field}-- left} where the parameters
are fitted. %A collector conversion factor of $\beta_{sf}=4.36\cdot 10^{-2}\:
%(m)$ and a thermal loss coefficient $H_{sf}=13.57 \, (W/m^2)$.
The values can be found in \reftab{solarmed:modelling:models-parameters}.
Compared to the results presented in Ampuño~\etal~\cite{ampuno_modeling_2018},
double the gain coefficient is obtained while higher losses are found. This
could be explained by the fact that the calibration was performed including the
warm-up and cooldown periods of the solar field. 

As can be seen in the figure, most of the error is accumulated during the
cooldown of the field. In \reffig{solarmed:modelling:validation:solar-field}--
right the dynamic behavior of the model is validated with another test. The
dynamic response obtained is very similar to the one experimentally measured for
most of the test, similar to the calibration test despite the cloudy conditions.
A higher error is observed between calibration and the particular validation
test shown: R$^2$=0.97 compared to 0.87, but this can be explained because
during the latter part of the experiment the irradiance was very intermittent
while a high flow was kept which even manages to invert outlet and inlet
temperature. Nonetheless, in relative terms very similar errors are
obtained\sidenote{6.25 compared to 6.3 in terms of \gls{mapeLabel}}.

Several more tests (10) are evaluated, and the performance obtained is shown in
\reftab{solarmed:modelling:validation:solar-field}. Here it can be seen than
for many tests performance close to the calibration tests are obtained. There is
a notable difference in the performance of the model depending on the sample
rate used. R$^2$ goes from 0.96 to 0.93, and \gls{maeLabel} from 2.50 to
2.84~$^\circ$C, when moving from a fast sample rate ($T_s=5$~s) to a slow one
($T_s=400$~s). This is expected since the model loses performance in transient
periods which are a constant in cloudy days. However, the average error is still
accurate enough for the intended use of the model while having a significant
reduction in computational time, from 1.43~s to 0.02~s, 71 times faster. 

In general, good metrics are obtained for most tests, with maximum percentage
errors below 10~\% \gls{mapeLabel}, and in those who do not, the error is usually
accumulated while the solar field is heating up or cooling down, that is, when
no heat is being delivered to the load.


\begin{table*}[]
\caption{Summary table of the prediction results obtained with the solar field model for different test days and sample times.}
\labtab{solarmed:modelling:validation:solar-field}
\resizebox{\linewidth}{!}{%

\begin{tabular}{ccccccccccccccccccccc}
\hline
\multirow{3}{*}{\textbf{\begin{tabular}[c]{@{}c@{}}Predicted\\ variable\end{tabular}}} &  & \multirow{3}{*}{\textbf{\begin{tabular}[c]{@{}c@{}}Sample\\ time\\ (s)\end{tabular}}} &  & \multirow{3}{*}{\textbf{\begin{tabular}[c]{@{}c@{}}Test\\ date\end{tabular}}} &  & \multicolumn{15}{c}{\textbf{Performance metric}}
\\\cline{7-21}
 &  &  &  &  &  & \multicolumn{3}{c}{\textbf{\begin{tabular}[c]{@{}c@{}}R$^2$\\ (-)\end{tabular}}} &  & \multicolumn{3}{c}{\textbf{\begin{tabular}[c]{@{}c@{}}MAE\\ (s.u.)\end{tabular}}} &  & \multicolumn{3}{c}{\textbf{\begin{tabular}[c]{@{}c@{}}MAPE\\ (\%)\end{tabular}}} &  & \multicolumn{3}{c}{\textbf{\begin{tabular}[c]{@{}c@{}}Time\\ (s)\end{tabular}}}
 \\\cline{7-9}\cline{11-13}\cline{15-17}\cline{19-21}
 &  &  &  &  &  & Test &  & Avg. &  & Test &  & Avg. &  & Test &  & Avg. &  & Test &  & Avg.
 \\\cline{1-1}\cline{3-3}\cline{5-5}\cline{7-7}\cline{9-9}\cline{11-11}\cline{13-13}\cline{15-15}\cline{17-17}\cline{19-19}\cline{21-21}
\multirow{20}{*}{T$_{sf,out}$ ($^\circ$C)} &  & \multirow{10}{*}{5} &  & 20231030 &  & 0.97 &  & \multirow{10}{*}{0.96} &  & 4.41 &  & \multirow{10}{*}{2.50} &  & 9.17 &  & \multirow{10}{*}{9.37} &  & 1.48 &  & \multirow{10}{*}{1.43} \\
 &  &  &  & 20231106 &  & 0.97 &  &  &  & 2.97 &  &  &  & 6.25 &  &  &  & 1.40 &  &  \\
 &  &  &  & 20230630 &  & 0.81 &  &  &  & 9.77 &  &  &  & 17.95 &  &  &  & 1.70 &  &  \\
 &  &  &  & 20230703 &  & 0.92 &  &  &  & 5.07 &  &  &  & 6.05 &  &  &  & 1.60 &  &  \\
 &  &  &  & 20230508 &  & 0.89 &  &  &  & 5.59 &  &  &  & 7.86 &  &  &  & 1.42 &  &  \\
 &  &  &  & 20230628 &  & 0.90 &  &  &  & 4.93 &  &  &  & 6.30 &  &  &  & 1.61 &  &  \\
 &  &  &  & 20230511 &  & 0.87 &  &  &  & 5.02 &  &  &  & 6.30 &  &  &  & 1.43 &  &  \\
 &  &  &  & 20230629 &  & 0.85 &  &  &  & 7.22 &  &  &  & 11.61 &  &  &  & 1.58 &  &  \\
 &  &  &  & 20230505 &  & 0.76 &  &  &  & 10.13 &  &  &  & 13.74 &  &  &  & 1.52 &  &  \\
 &  &  &  & 20231031 &  & 0.96 &  &  &  & 2.50 &  &  &  & 9.37 &  &  &  & 1.43 &
 &  \\
 \cline{3-21}
 &  & \multirow{10}{*}{400} &  & 20231030 &  & 0.96 &  & \multirow{10}{*}{0.93} &  & 4.99 &  & \multirow{10}{*}{2.84} &  & 10.71 &  & \multirow{10}{*}{10.21} &  & 0.02 &  & \multirow{10}{*}{0.02} \\
 &  &  &  & 20231106 &  & 0.97 &  &  &  & 3.69 &  &  &  & 8.28 &  &  &  & 0.02 &  &  \\
 &  &  &  & 20230630 &  & 0.79 &  &  &  & 10.20 &  &  &  & 18.52 &  &  &  & 0.02 &  &  \\
 &  &  &  & 20230703 &  & 0.93 &  &  &  & 4.73 &  &  &  & 5.56 &  &  &  & 0.02 &  &  \\
 &  &  &  & 20230508 &  & 0.88 &  &  &  & 5.71 &  &  &  & 7.81 &  &  &  & 0.02 &  &  \\
 &  &  &  & 20230628 &  & 0.91 &  &  &  & 4.74 &  &  &  & 5.92 &  &  &  & 0.02 &  &  \\
 &  &  &  & 20230511 &  & 0.80 &  &  &  & 6.15 &  &  &  & 7.64 &  &  &  & 0.02 &  &  \\
 &  &  &  & 20230629 &  & 0.87 &  &  &  & 6.93 &  &  &  & 11.03 &  &  &  & 0.02 &  &  \\
 &  &  &  & 20230505 &  & 0.78 &  &  &  & 9.34 &  &  &  & 12.58 &  &  &  & 0.02 &  &  \\
 &  &  &  & 20231031 &  & 0.93 &  &  &  & 2.84 &  &  &  & 10.21 &  &  &  & 0.02 &  &  \\
\hline
\end{tabular}
}
\raggedright
\textcolor{darkgray}{\footnotesize\textit{s.u.} stands for \textit{same units} as the predicted variable}
\end{table*}

%===================================
%===================================
\subsection{Thermal storage}

A first-principles model of a two-tank thermal storage system is developed to
capture the key thermodynamic and fluid dynamic phenomena governing energy
transfer and stratification based on the methodology developed by
Duffie~\etal~\sidecite{duffie_energy_2013}.

The governing model equations and boundary conditions to simulate the transient
thermal behavior of the storage system, including mass and energy balances, heat
transfer mechanisms, and the stratification dynamics are shown in
\refmod{solarmed:modelling:thermal-storage}.

% Diagrama del componente con variables
\begin{marginfigure}[+3cm]
    \includegraphics[]{solarmed-diagrams-thermal_storage.png}
    \caption{Thermal storage process diagram.}
    \labfig{solarmed:diagrams:thermal_storage}
\end{marginfigure}

\begin{modelcounter}{Thermal storage}
    \begin{align*}
        \mathbf{T}_\text{h}(k),\; \mathbf{T}_\text{c}(k) &= \text{thermal storage model}\Big(
            \mathbf{T}_\text{h}(k{-}1),\;
            \mathbf{T}_\text{c}(k{-}1),\;
            T_\text{src}(k), \\
            &\quad T_\text{dis}(k),\;
            \dot{m}_\text{src}(k),\;
            \dot{m}_\text{dis}(k),\;
            T_\text{amb}(k);\;
            \boldsymbol{\theta}_\text{h},\;
            \boldsymbol{\theta}_\text{c}
        \Big)
    \end{align*}

    \vspace{0.5em}

    \textbf{if } $\dot{m}_\text{dis}(k) > \dot{m}_\text{src}(k)$: \hfill \textit{(cold to hot recirculation)}
    \begin{align*}
        \mathbf{T}_\text{c}(k) &= \text{single tank model}\Big(
            \mathbf{T}_\text{c}(k{-}1),\;
            T_\text{T}{=}0,\;
            T_\text{B}{=}T_\text{dis}(k),\;
            T_\text{amb}(k), \\
            &\quad \dot{m}_\text{in,T}{=}0,\;
            \dot{m}_\text{in,B}{=}\dot{m}_\text{dis}(k),\;
            \dot{m}_\text{out,T}{=}\dot{m}_\text{dis}(k) - \dot{m}_\text{src}(k),\;
            \dot{m}_\text{out,B}{=}\dot{m}_\text{src}(k);\;
            \boldsymbol{\theta}_\text{c}
        \Big) \\[0.5em]
        \mathbf{T}_\text{h}(k) &= \text{single tank model}\Big(
            \mathbf{T}_\text{h}(k{-}1),\;
            T_\text{T}{=}T_\text{src}(k),\;
            T_\text{B}{=}T_\text{c}^\text{out}(k),\;
            T_\text{amb}(k), \\
            &\quad \dot{m}_\text{in,T}{=}\dot{m}_\text{src}(k),\;
            \dot{m}_\text{in,B}{=}\dot{m}_\text{dis}(k) - \dot{m}_\text{src}(k),\;
            \dot{m}_\text{out,T}{=}\dot{m}_\text{dis}(k),\;
            \dot{m}_\text{out,B}{=}0;\;
            \boldsymbol{\theta}_\text{h}
        \Big)
    \end{align*}

    \vspace{0.5em}

    \textbf{else:} \hfill \textit{(hot to cold recirculation)}
    \begin{align*}
        \mathbf{T}_\text{h}(k) &= \text{single tank model}\Big(
            \mathbf{T}_\text{h}(k{-}1),\;
            T_\text{T}{=}T_\text{src}(k),\;
            T_\text{B}{=}0,\;
            T_\text{amb}(k), \\
            &\quad \dot{m}_\text{in,T}{=}\dot{m}_\text{src}(k),\;
            \dot{m}_\text{in,B}{=}0,\;
            \dot{m}_\text{out,T}{=}\dot{m}_\text{dis}(k),\;
            \dot{m}_\text{out,B}{=}\dot{m}_\text{src}(k) - \dot{m}_\text{dis}(k);\;
            \boldsymbol{\theta}_\text{h}
        \Big) \\[0.5em]
        \mathbf{T}_\text{c}(k) &= \text{single tank model}\Big(
            \mathbf{T}_\text{c}(k{-}1),\;
            T_\text{T}{=}T_\text{h}^\text{out}(k),\;
            T_\text{B}{=}T_\text{dis}(k),\;
            T_\text{amb}(k), \\
            &\quad \dot{m}_\text{in,T}{=}\dot{m}_\text{src}(k) - \dot{m}_\text{dis}(k),\;
            \dot{m}_\text{in,B}{=}\dot{m}_\text{dis}(k),\;
            \dot{m}_\text{out,T}{=}0,\;
            \dot{m}_\text{out,B}{=}\dot{m}_\text{src}(k);\;
            \boldsymbol{\theta}_\text{c}
        \Big)
    \end{align*}

    \textbf{where:} \\
        \begin{align*}
        \mathbf{T}(k) &= \text{single tank model}\Big(
            \mathbf{T}(k{-}1),\;
            T_\text{T,in}(k),\;
            T_\text{B,in}(k),\;
            \dot{m}_\text{in,T}(k),\;
            \dot{m}_\text{in,B}(k), \\
            &\quad \dot{m}_\text{out,T}(k),\;
            \dot{m}_\text{out,B}(k),\;
            T_\text{amb}(k);\;
            \boldsymbol{\theta};\;
        \Big)
    \end{align*}

    \vspace{0.5em}

    \begin{itemize}
    \item Top volume
    \[
    \begin{aligned}
    & -\rho \cdot V_T \cdot c_p \cdot \frac{T_{T,k} - T_{T,k-1}}{T_s}
    + \dot{m}_{src} \cdot T_{T,in} \cdot c_p
    - \dot{m}_{dis} \cdot T_{T,k} \cdot c_p \\
    & - \dot{m}_{src} \cdot T_{T,k} \cdot c_p
    + \dot{m}_{dis} \cdot T_{1,k} \cdot c_p
    - H_T \cdot (T_{T,k} - T_{amb})
    = 0
    \end{aligned}
    \]

    \item Bottom volume
    \[
    \begin{aligned}
    & -\rho \cdot V_B \cdot c_p \cdot \frac{T_{B,k} - T_{B,k-1}}{T_s}
    + \dot{m}_{src} \cdot T_{i-1,k} \cdot c_p
    + \dot{m}_{dis} \cdot T_{B,in} \cdot c_p \\
    & - \dot{m}_{src} \cdot T_{B,k} \cdot c_p
    - \dot{m}_{dis} \cdot T_{B,k} \cdot c_p
    - H_N \cdot (T_{B,k} - T_{amb})
    = 0
    \end{aligned}
    \]

    \item Inner volume
    \[
    \begin{aligned}
    & -\rho \cdot V_i \cdot c_p \cdot \frac{T_{i,k} - T_{i,k-1}}{T_s}
    + \dot{m}_{src} \cdot T_{i-1,k} \cdot c_p
    - \dot{m}_{dis} \cdot T_{i,k} \cdot c_p \\
    & - \dot{m}_{src} \cdot T_{i,k} \cdot c_p
    + \dot{m}_{dis} \cdot T_{i+1,k} \cdot c_p
    - H_i \cdot (T_{i,k} - T_{amb})
    = 0
    \end{aligned}
    \]
    \end{itemize}

    \labmod{solarmed:modelling:thermal-storage}
\end{modelcounter}

Three types of volumes are defined: the inner volume, the top volume and the
bottom volume:
\begin{itemize}
    \item Top volume ($V_{T}$): can receive external heat, and have heat
    extracted from it. It interacts with the inner volume that it interfaces
    with.
    \item Bottom volume ($V_{B}$): can also have external interactions, and
    exchanges with the inner volume above it.
    \item Inner volume ($V_{i}$): is any volume that is not the top or bottom,
    that is, is surrounded by other volumes with which it exchanges heat and
    mass by inner recirculation.
\end{itemize}

Similar to the solar field model, it has two parameters that need to be
calibrated using experimental data. These dynamic parameters are the
thermal loss coefficient ($H_{i}\,\left( \frac{J}{s·^\circ C} \right)$) which
relates heat losses to the environment and the volume of each of the considered
control volumes ($V_i$). Three temperature sensors are available in the
experimental facility, so three volume divisions are used to model the thermal
storage. With two tanks, this results in a total of 12 parameters to be
calibrated.

%================================
\subsubsection{Electrical consumption}

% Mostrar ensayo, si encaja incluirlo en la misma gráfica que el campo solar
% simplemente referenciar esa figura

The first step train given in
\reffig{solarmed:modelling:sf:electrical_consumption_tests} -- 20250925 from
06:50 to 07:15 is used to characterize the electrical consumption of
recirculating water ($q_{ts,src}$) in the thermal storage circuit. The
electrical consumption is modelled as a function of the flow rate through the
thermal storage from a minimum value of 1.4 m$^3$/h -- 0.05 kW$_e$ to a maximum
value of 8.4 m$^3$/h -- 0.75 kW$_e$ with a second-order polynomial regression,
with a coefficient of determination of $R^2 = 0.99$.

\begin{modelcounter}{Thermal storage electrical consumption}
    % 0.04883341, -0.00695794, 0.01049481
    \begin{align*}
        C_{e,ts}\,[kW_e] &= \text{ts electrical consumption}\left(q_{ts,src}\,[m^3/h]\right) \\
        & C_{e,ts} = 4.88\cdot{10}^{-1} \cdot q_{ts,src}^2 + -6.95\cdot{10}^{-3} \cdot q_{ts,src} + 0.01
    \end{align*}
    \labmod{solarmed:modelling:thermal-storage:electrical_consumption}
\end{modelcounter}

%================================
\subsubsection{Validation}

In order to calibrate the model parameters ($H_{i}$ and $V_{i}$), data from the
system was recorded during four consecutive days under different operating
conditions (see \reffig{solarmed:modelling:validation:thermal-storage} - left).
The first and last days included both charge and discharge cycles, while the
middle two days were dedicated to charging-only operations. In between these
days, the system was left idle to observe the natural thermal losses. The model
parameters were fitted\sidenote{See their values in
\reftab{solarmed:modelling:models-parameters}} to minimize a combined metric
averaging the three temperature measurements available per tank, obtaining a low
thermal loss coefficient in the order of 10$^{-2}$ to 10$^{-4}$~(W/K). On the
other hand, adding the volumes of the three control volumes totals around the
15~m$^3$ of the actual tank volume, which is a good indication that the model is
capturing the thermal behavior of the system well. However, they are not
distributed evenly; for both tanks the bottom volumes are significantly smaller
than the upper ones. This can be explained by the fact that the temperature
transmitters are not spaced evenly, and the bottom transmitter is located near
the tank's bottom.

\begin{figure*}[h!]
    \includegraphics[width=\linewidth]{thermal_storage_validation.png}
    \savebox\captionqrleft{\qrcode[hyperlink,height=0.5in]{\repositoryBaseUrl/figures/thermal_storage_validation_20230707.html}}
    \savebox\captionqrright{\qrcode[hyperlink,height=0.5in]{\repositoryBaseUrl/figures/thermal_storage_validation_20230505.html}}
    \caption[Thermal storage model calibration and validation tests]{Thermal storage model calibration and validation tests. \hfill \usebox\captionqrleft\hspace{1ex}\usebox\captionqrright}
    \labfig{solarmed:modelling:validation:thermal-storage}
\end{figure*}

In \reffig{solarmed:modelling:validation:thermal-storage} -- right, the model is
validated with a different test. It can be observed that the error between
calibration and validation is similar, 1.11~$^\circ$C (\gls{maeLabel}) compared
to 1.14~$^\circ$C\sidenote[][*-4]{See
\reftab{solarmed:modelling:validation:thermal-storage} -- $T_{ts,h}$ -- Sample
time 5 seconds. Similar value for the 400 seconds sample time. Higher
differences are observed for the cold tank, with even a better value obtained in
validation compared to calibration (3.15 \textit{v.s.} 1.78)}. The model seems
to have a slower dynamic response to changes in the load-discharge balance than
the actual system, which strangely stays impassive despite the changes in the
load until some point where it reacts more aggressively. This could be explained
by the interconnection between tanks. While the model assumes instantaneous and
continuos flow recirculation between tanks, in reality it seems that the flow is
discontinuous, only starting to flow when a certain pressure difference is
reached. Nonetheless, both model and experimental data converge to similar
values for all three measurements once the system stabilizes. Furthermore, the
two most important measurements, the top of the hot tank and the bottom of the
cold tank, which are the ones that interface with the rest of the system, have a
very low error throughout the test.

Finally, several more tests (7) are evaluated, and the performance obtained is
shown in \reftab{solarmed:modelling:validation:thermal-storage}. It should be
noted that this model only receives feedback from the process initially, and
then outputs are forecasted based on the inputs and the model own previously
forecasted states. This means that any error in the prediction will be
accumulated over time. This makes metrics like $R^2$ not representative of the
model performance, as a small offset in the prediction will make $R^2$ drop
significantly. For this reason, more emphasis is put on the \gls{maeLabel} and
\gls{mapeLabel} metrics. In general, almost identical performance is obtained
with the fast ($T_s=5$~s) and slow ($T_s=400$~s) sample rates, while the
computational time is significantly reduced, from 5.45~s to 0.07~s, almost 80
times faster. This is explained because the model on each iteration needs to
solve a system of equations, so it has associated a higher time per iteration,
making potential savings in the number of iterations more significant. On the
other hand, less accuracy is lost when reducing the sample rate compared to the
solar field since the dynamics of this system are naturally slower due to the
high thermal inertia of the tanks, thus making it more insensitive to the
sampling.

Analyzing the performance in terms of the point of measurement, the top of the
hot tank has the lowest error, with a \gls{maeLabel} of 1.14~$^\circ$C, while
the bottom of the cold tank has a slight higher error, with 1.78~$^\circ$C.
Overall good agreement between model and experimental data is observed with
maximum errors below 3~\% \gls{mapeLabel}. This means than the state of the
thermal storage can be predicted with a reasonable accuracy for hours ahead.

\begin{table*}[]
\caption{Summary table of the prediction results obtained with the thermal storage model for different test days and sample times.}
\labtab{solarmed:modelling:validation:thermal-storage}
\resizebox{\linewidth}{!}{%

\begin{tabular}{ccccccccccccccccccccc}
\hline
\multirow{3}{*}{\textbf{\begin{tabular}[c]{@{}c@{}}Predicted\\ variable\end{tabular}}} &  & \multirow{3}{*}{\textbf{\begin{tabular}[c]{@{}c@{}}Sample\\ time\\ (s)\end{tabular}}} &  & \multirow{3}{*}{\textbf{\begin{tabular}[c]{@{}c@{}}Test\\ date\end{tabular}}} &  & \multicolumn{15}{c}{\textbf{Performance metric}}
\\\cline{7-21}
 &  &  &  &  &  & \multicolumn{3}{c}{\textbf{\begin{tabular}[c]{@{}c@{}}R$^2$\\ (-)\end{tabular}}} &  & \multicolumn{3}{c}{\textbf{\begin{tabular}[c]{@{}c@{}}MAE\\ (s.u.)\end{tabular}}} &  & \multicolumn{3}{c}{\textbf{\begin{tabular}[c]{@{}c@{}}MAPE\\ (\%)\end{tabular}}} &  & \multicolumn{3}{c}{\textbf{\begin{tabular}[c]{@{}c@{}}Time\\ (s)\end{tabular}}}
 \\\cline{7-9}\cline{11-13}\cline{15-17}\cline{19-21}
 &  &  &  &  &  & Test &  & Avg. &  & Test &  & Avg. &  & Test &  & Avg. &  & Test &  & Avg.
 \\\cline{1-1}\cline{3-3}\cline{5-5}\cline{7-7}\cline{9-9}\cline{11-11}\cline{13-13}\cline{15-15}\cline{17-17}\cline{19-19}\cline{21-21}
\multirow{7}{*}{T$_{ts,h}$ ($^\circ$C)} &  & \multirow{14}{*}{5} &  & 20230630 &  & 0.13 &  & \multirow{7}{*}{0.51} &  & 0.99 &  & \multirow{7}{*}{1.14} &  & 1.03 &  & \multirow{7}{*}{1.15} &  & 5.91 &  & \multirow{14}{*}{5.35} \\
 &  &  &  & 20230508 &  & 0.79 &  &  &  & 1.47 &  &  &  & 1.58 &  &  &  & 5.36 &  &  \\
 &  &  &  & 20230707 &  & 0.88 &  &  &  & 1.11 &  &  &  & 1.27 &  &  &  & 55.49 &  &  \\
 &  &  &  & 20230628 &  & 0.76 &  &  &  & 1.02 &  &  &  & 1.16 &  &  &  & 5.89 &  &  \\
 &  &  &  & 20230511 &  & 0.22 &  &  &  & 2.26 &  &  &  & 2.52 &  &  &  & 5.28 &  &  \\
 &  &  &  & 20230629 &  & 0.98 &  &  &  & 0.34 &  &  &  & 0.36 &  &  &  & 5.84 &  &  \\
 &  &  &  & 20230505 &  & 0.51 &  &  &  & 1.14 &  &  &  & 1.15 &  &  &  & 5.35 &  &  \\
\multirow{7}{*}{T$_{ts,c}$ ($^\circ$C)} &  &  &  & 20230630 &  & 0.52 &  & \multirow{7}{*}{0.88} &  & 2.11 &  & \multirow{7}{*}{1.78} &  & 2.56 &  & \multirow{7}{*}{2.14} &  & 5.91 &  &  \\
 &  &  &  & 20230508 &  & 0.83 &  &  &  & 1.05 &  &  &  & 1.37 &  &  &  & 5.36 &  &  \\
 &  &  &  & 20230707 &  & 0.87 &  &  &  & 3.15 &  &  &  & 4.35 &  &  &  & 55.49 &  &  \\
 &  &  &  & 20230628 &  & 0.68 &  &  &  & 2.86 &  &  &  & 3.78 &  &  &  & 5.89 &  &  \\
 &  &  &  & 20230511 &  & 0.96 &  &  &  & 1.75 &  &  &  & 2.17 &  &  &  & 5.28 &  &  \\
 &  &  &  & 20230629 &  & 0.88 &  &  &  & 2.02 &  &  &  & 2.52 &  &  &  & 5.84 &  &  \\
 &  &  &  & 20230505 &  & 0.88 &  &  &  & 1.78 &  &  &  & 2.14 &  &  &  & 5.35 &
 &  \\ \hline
\multirow{7}{*}{T$_{ts,h}$ ($^\circ$C)} &  & \multirow{14}{*}{400} &  & 20230630 &  & 0.18 &  & \multirow{7}{*}{0.54} &  & 1.07 &  & \multirow{7}{*}{1.14} &  & 1.11 &  & \multirow{7}{*}{1.14} &  & 0.09 &  & \multirow{14}{*}{0.07} \\
 &  &  &  & 20230508 &  & 0.79 &  &  &  & 1.47 &  &  &  & 1.58 &  &  &  & 0.08 &  &  \\
 &  &  &  & 20230707 &  & 0.88 &  &  &  & 1.10 &  &  &  & 1.26 &  &  &  & 0.63 &  &  \\
 &  &  &  & 20230628 &  & 0.76 &  &  &  & 1.03 &  &  &  & 1.18 &  &  &  & 0.07 &  &  \\
 &  &  &  & 20230511 &  & 0.21 &  &  &  & 2.33 &  &  &  & 2.59 &  &  &  & 0.07 &  &  \\
 &  &  &  & 20230629 &  & 0.98 &  &  &  & 0.36 &  &  &  & 0.38 &  &  &  & 0.07 &  &  \\
 &  &  &  & 20230505 &  & 0.54 &  &  &  & 1.14 &  &  &  & 1.14 &  &  &  & 0.07 &  &  \\
\multirow{7}{*}{T$_{ts,c}$ ($^\circ$C)} &  &  &  & 20230630 &  & 0.41 &  & \multirow{7}{*}{0.84} &  & 2.22 &  & \multirow{7}{*}{2.05} &  & 2.69 &  & \multirow{7}{*}{2.44} &  & 0.09 &  &  \\
 &  &  &  & 20230508 &  & 0.74 &  &  &  & 1.25 &  &  &  & 1.60 &  &  &  & 0.08 &  &  \\
 &  &  &  & 20230707 &  & 0.87 &  &  &  & 3.09 &  &  &  & 4.26 &  &  &  & 0.63 &  &  \\
 &  &  &  & 20230628 &  & 0.68 &  &  &  & 2.81 &  &  &  & 3.73 &  &  &  & 0.07 &  &  \\
 &  &  &  & 20230511 &  & 0.94 &  &  &  & 1.89 &  &  &  & 2.40 &  &  &  & 0.07 &  &  \\
 &  &  &  & 20230629 &  & 0.88 &  &  &  & 1.97 &  &  &  & 2.47 &  &  &  & 0.07 &  &  \\
 &  &  &  & 20230505 &  & 0.84 &  &  &  & 2.05 &  &  &  & 2.44 &  &  &  & 0.07 &  &  \\
\hline
\end{tabular}
}
\raggedright
\textcolor{darkgray}{\footnotesize\textit{s.u.} stands for \textit{same units}
as the predicted variable. Alias: $T_{ts,h} == T_{ts,h,t}$ and $T_{ts,c} ==
T_{ts,c,b}$}
\end{table*}

%===================================
%===================================
\subsection{Heat exchanger}

The solar field and thermal storage are interfaced by a \gls{hexLabel} or
\texttt{hx}, particularly a water-to-water counter-flow heat exchanger. The
component is modelled using a first-principles steady state model based on the
effectiveness--NTU method~\sidecite{cengel_heat_2015,kays_compact_1958}. The
following assumptions are considered~\cite{cengel_heat_2015}: 

\begin{itemize}
    \item It has been assumed that the rate of change for the temperature of
    both fluids is proportional to the temperature difference; this assumption
    is valid for fluids with a constant specific heat, which is a good
    description of fluids changing temperature over a relatively small range.
    However, if the specific heat changes, the \gls{lmtdLabel} approach will no
    longer be accurate. 
    \item It has also been assumed that the heat transfer coefficient (U) is
    constant, and not a function of temperature.
    \item No phase change during heat transfer.
    \item Changes in kinetic energy and potential energy are neglected.
\end{itemize}

The model is described in \refmod{solarmed:hex}. It returns the outlet
temperatures from both primary circuit (solar field side), $p$, and  secondary
circuit $s$, the thermal storage side. As shown in \refmod{solarmed:hex}, first
the heat capacity $C$ is determined in order to calculate the effectiveness
($\epsilon$) of the heat exchanger. Finally, after determining the maximum heat
transfer rate ($\dot{Q}_{max}$), the outlet temperatures can be obtained.

%================================
\subsubsection{Validation}

In order to calibrate the only parameter of this model ($UA_{hx}$), one
representative experimental test is used where the parameters are fitted in
order to obtain the \textit{least-squares error} between the model and the
experimental data. A heat transfer conductance value of 13
547~W/K is obtained. In \reffig{solarmed:modelling:validation:heat-exchanger}
the dynamic behavior of the model is validated with another test. It can be seen
than the model performs fairly well even in transient conditions, with a
\gls{maeLabel} of $T_{hx,p,out}=1.38\:^\circ$C, $T_{hx,p,out}=1.39\:^\circ$C and
a coefficient of determination $R^2=99$~\% for both outputs\sidenote{With fast
sample rate --10 seconds}.

% Bloque de modelo con interfaz y ecuaciones
\begin{modelcounter}{Heat exchanger}
    \begin{align*}
        T_{hx,p,out},\,T_{hx,s,out}& = \text{hx\:model}(T_{hx,p,in},T_{hx,s,in},\dot{m}_p,\dot{m}_{s}, T_{amb}; (UA)_{hx}) \\
        & C_{hx,p} = \dot{m}_{hx,p} \cdot c_{p,Tp,in} \tag*{{\small\textit{Primary side heat cap. [J/K\cdot s]}}} \\
        & C_{hx,s} = \dot{m}_{hx,s} \cdot c_{p,Ts,in} \tag*{{\small\textit{Secondary side heat cap. [J/K\cdot s]}}} \\
        & C_{min} = \min(C_{hx,p}, C_{hx,s}) \\
        & C_{max} = \max(C_{hx,p}, C_{hx,s}) \\
        & C=\frac{C_{min}}{C_{max}} \\
        & \dot{Q}_{max} = C_{min} \cdot (T_{hx,p,in} - T_{hx,s,in}) \\
        & NTU = UA / C_{min} \\
        & \epsilon = \frac{1 - e^{ (-NTU \cdot (1 - C))}}{1 - C \cdot e^{-NTU \cdot (1 - C)}} \tag*{{\small\textit{Effectiveness [-]}}} \\
        & T_{hx,p,out} = T_{hx,p,in} - (\dot{Q}_{max} \cdot \epsilon) / (C_{hx,p}) \\
        & T_{hx,s,out} = T_{hx,s,in} + (\dot{Q}_{max} \cdot \epsilon) / (C_{hx,s}) \\
    \end{align*}

    \labmod{solarmed:hex}
\end{modelcounter}

% Diagrama del componente con variables
\begin{marginfigure}[-5.5cm]
    \includegraphics[]{solarmed-diagrams-heat_exchanger.png}
    \caption{Heat exchanger process diagram.}
    \labfig{solarmed:diagrams:heat_exchanger}
\end{marginfigure}



\begin{figure}
    \includegraphics[width=\textwidth]{heat_exchanger_validation_20231106.png}
    \savebox\captionqr{\qrcode[hyperlink,height=0.5in]{\repositoryBaseUrl/figures/heat_exchanger_validation_20231106.html}}
    \caption[Heat exchanger model validation for a particular test.]{Heat exchanger model validation for a particular test. The effectiveness--NTU is limited to the model because it is an estimated parameter.\\[1ex] \usebox\captionqr}
    \labfig{solarmed:modelling:validation:heat-exchanger}
\end{figure}

Several more tests (11) are evaluated, and the performance obtained is shown in
\reftab{solarmed:modelling:validation:heat-exchanger}. In general, almost
identical performance is obtained with the fast ($T_s=5$~s) and slow
($T_s=400$~s) sample rates, but as in the previous models, the computational
time is significantly reduced, from 0.45~s to 0.01~s, an order of magnitude
faster. In terms of accuracy, there seems to be a systematic higher error in the
outlet of the secondary circuit with respect to the primary side, with double
the error (\gls{maeLabel}: 1.08 compared to 2.16~$^\circ$C). In general, good
agreement between model and experimental data is observed with maximum errors
below 8~\% \gls{mapeLabel}.


\begin{table*}[]
\caption{Summary table of the prediction results obtained with the heat exchanger model for different test days and sample times.}
\labtab{solarmed:modelling:validation:heat-exchanger}
\resizebox{\linewidth}{!}{%

\begin{tabular}{ccccccccccccccccccccc}
\hline
\multirow{3}{*}{\textbf{\begin{tabular}[c]{@{}c@{}}Predicted\\ variable\end{tabular}}} &  & \multirow{3}{*}{\textbf{\begin{tabular}[c]{@{}c@{}}Sample\\ time\\ (s)\end{tabular}}} &  & \multirow{3}{*}{\textbf{\begin{tabular}[c]{@{}c@{}}Test\\ date\end{tabular}}} &  & \multicolumn{15}{c}{\textbf{Performance metric}}
\\\cline{7-21}
 &  &  &  &  &  & \multicolumn{3}{c}{\textbf{\begin{tabular}[c]{@{}c@{}}R$^2$\\ (-)\end{tabular}}} &  & \multicolumn{3}{c}{\textbf{\begin{tabular}[c]{@{}c@{}}MAE\\ (s.u.)\end{tabular}}} &  & \multicolumn{3}{c}{\textbf{\begin{tabular}[c]{@{}c@{}}MAPE\\ (\%)\end{tabular}}} &  & \multicolumn{3}{c}{\textbf{\begin{tabular}[c]{@{}c@{}}Time\\ (s)\end{tabular}}}
 \\\cline{7-9}\cline{11-13}\cline{15-17}\cline{19-21}
 &  &  &  &  &  & Test &  & Avg. &  & Test &  & Avg. &  & Test &  & Avg. &  & Test &  & Avg.
 \\\cline{1-1}\cline{3-3}\cline{5-5}\cline{7-7}\cline{9-9}\cline{11-11}\cline{13-13}\cline{15-15}\cline{17-17}\cline{19-19}\cline{21-21}
\multirow{11}{*}{T$_{hx,p,out}$ ($^\circ$C)} &  & \multirow{22}{*}{5} &  & 20231030 &  & 0.99 &  & \multirow{11}{*}{0.99} &  & 0.86 &  & \multirow{11}{*}{1.08} &  & 1.70 &  & \multirow{11}{*}{3.58} &  & 0.48 &  & \multirow{22}{*}{0.45} \\
 &  &  &  & 20231106 &  & 0.99 &  &  &  & 1.38 &  &  &  & 3.14 &  &  &  & 0.49 &  &  \\
 &  &  &  & 20230630 &  & 0.99 &  &  &  & 0.68 &  &  &  & 1.11 &  &  &  & 0.56 &  &  \\
 &  &  &  & 20230703 &  & 0.99 &  &  &  & 0.53 &  &  &  & 0.67 &  &  &  & 0.60 &  &  \\
 &  &  &  & 20230508 &  & 0.99 &  &  &  & 1.24 &  &  &  & 1.76 &  &  &  & 0.58 &  &  \\
 &  &  &  & 20230707 &  & 0.99 &  &  &  & 1.64 &  &  &  & 5.15 &  &  &  & 4.41 &  &  \\
 &  &  &  & 20230628 &  & 0.99 &  &  &  & 0.73 &  &  &  & 0.93 &  &  &  & 0.58 &  &  \\
 &  &  &  & 20230511 &  & 0.98 &  &  &  & 1.40 &  &  &  & 1.92 &  &  &  & 0.54 &  &  \\
 &  &  &  & 20230629 &  & 0.99 &  &  &  & 0.58 &  &  &  & 0.80 &  &  &  & 0.59 &  &  \\
 &  &  &  & 20230505 &  & 0.99 &  &  &  & 1.22 &  &  &  & 1.67 &  &  &  & 0.54 &  &  \\
 &  &  &  & 20231031 &  & 0.99 &  &  &  & 1.08 &  &  &  & 3.58 &  &  &  & 0.45 &  &  \\
\multirow{11}{*}{T$_{hx,s,out}$ ($^\circ$C)} &  &  &  & 20231030 &  & 0.98 &  & \multirow{11}{*}{0.96} &  & 2.58 &  & \multirow{11}{*}{2.16} &  & 5.54 &  & \multirow{11}{*}{7.55} &  & 0.48 &  &  \\
 &  &  &  & 20231106 &  & 0.97 &  &  &  & 3.19 &  &  &  & 6.88 &  &  &  & 0.49 &  &  \\
 &  &  &  & 20230630 &  & 0.98 &  &  &  & 2.71 &  &  &  & 4.63 &  &  &  & 0.56 &  &  \\
 &  &  &  & 20230703 &  & 0.98 &  &  &  & 1.72 &  &  &  & 2.47 &  &  &  & 0.60 &  &  \\
 &  &  &  & 20230508 &  & 0.96 &  &  &  & 2.57 &  &  &  & 3.70 &  &  &  & 0.58 &  &  \\
 &  &  &  & 20230707 &  & 0.98 &  &  &  & 2.90 &  &  &  & 7.24 &  &  &  & 4.41 &  &  \\
 &  &  &  & 20230628 &  & 0.95 &  &  &  & 2.78 &  &  &  & 3.96 &  &  &  & 0.58 &  &  \\
 &  &  &  & 20230511 &  & 0.95 &  &  &  & 2.88 &  &  &  & 3.95 &  &  &  & 0.54 &  &  \\
 &  &  &  & 20230629 &  & 0.98 &  &  &  & 2.46 &  &  &  & 4.05 &  &  &  & 0.59 &  &  \\
 &  &  &  & 20230505 &  & 0.97 &  &  &  & 2.93 &  &  &  & 4.31 &  &  &  & 0.54 &  &  \\
 &  &  &  & 20231031 &  & 0.96 &  &  &  & 2.16 &  &  &  & 7.55 &  &  &  & 0.45 &
 &  \\
 \hline
\multirow{11}{*}{T$_{hx,p,out}$ ($^\circ$C)} &  & \multirow{22}{*}{400} &  & 20231030 &  & 0.99 &  & \multirow{11}{*}{0.99} &  & 0.87 &  & \multirow{11}{*}{1.08} &  & 1.74 &  & \multirow{11}{*}{3.60} &  & 0.01 &  & \multirow{22}{*}{0.01} \\
 &  &  &  & 20231106 &  & 0.99 &  &  &  & 1.39 &  &  &  & 3.17 &  &  &  & 0.01 &  &  \\
 &  &  &  & 20230630 &  & 0.99 &  &  &  & 0.70 &  &  &  & 1.16 &  &  &  & 0.01 &  &  \\
 &  &  &  & 20230703 &  & 0.99 &  &  &  & 0.55 &  &  &  & 0.70 &  &  &  & 0.01 &  &  \\
 &  &  &  & 20230508 &  & 0.99 &  &  &  & 1.26 &  &  &  & 1.82 &  &  &  & 0.01 &  &  \\
 &  &  &  & 20230707 &  & 0.99 &  &  &  & 1.63 &  &  &  & 5.14 &  &  &  & 0.06 &  &  \\
 &  &  &  & 20230628 &  & 0.99 &  &  &  & 0.71 &  &  &  & 0.93 &  &  &  & 0.01 &  &  \\
 &  &  &  & 20230511 &  & 0.98 &  &  &  & 1.45 &  &  &  & 2.01 &  &  &  & 0.01 &  &  \\
 &  &  &  & 20230629 &  & 0.99 &  &  &  & 0.58 &  &  &  & 0.81 &  &  &  & 0.01 &  &  \\
 &  &  &  & 20230505 &  & 0.99 &  &  &  & 1.23 &  &  &  & 1.71 &  &  &  & 0.01 &  &  \\
 &  &  &  & 20231031 &  & 0.99 &  &  &  & 1.08 &  &  &  & 3.60 &  &  &  & 0.01 &  &  \\
\multirow{11}{*}{T$_{hx,s,out}$ ($^\circ$C)} &  &  &  & 20231030 &  & 0.98 &  & \multirow{11}{*}{0.96} &  & 2.60 &  & \multirow{11}{*}{2.13} &  & 5.64 &  & \multirow{11}{*}{7.46} &  & 0.01 &  &  \\
 &  &  &  & 20231106 &  & 0.98 &  &  &  & 3.19 &  &  &  & 6.93 &  &  &  & 0.01 &  &  \\
 &  &  &  & 20230630 &  & 0.98 &  &  &  & 2.69 &  &  &  & 4.65 &  &  &  & 0.01 &  &  \\
 &  &  &  & 20230703 &  & 0.98 &  &  &  & 1.77 &  &  &  & 2.60 &  &  &  & 0.01 &  &  \\
 &  &  &  & 20230508 &  & 0.96 &  &  &  & 2.61 &  &  &  & 3.79 &  &  &  & 0.01 &  &  \\
 &  &  &  & 20230707 &  & 0.98 &  &  &  & 2.92 &  &  &  & 7.26 &  &  &  & 0.06 &  &  \\
 &  &  &  & 20230628 &  & 0.95 &  &  &  & 2.77 &  &  &  & 3.98 &  &  &  & 0.01 &  &  \\
 &  &  &  & 20230511 &  & 0.95 &  &  &  & 2.84 &  &  &  & 3.90 &  &  &  & 0.01 &  &  \\
 &  &  &  & 20230629 &  & 0.98 &  &  &  & 2.52 &  &  &  & 4.19 &  &  &  & 0.01 &  &  \\
 &  &  &  & 20230505 &  & 0.97 &  &  &  & 2.92 &  &  &  & 4.34 &  &  &  & 0.01 &  &  \\
 &  &  &  & 20231031 &  & 0.96 &  &  &  & 2.13 &  &  &  & 7.46 &  &  &  & 0.01 &  &  \\
\hline
\end{tabular}
}
\raggedright
\textcolor{darkgray}{\footnotesize\textit{s.u.} stands for \textit{same units} as the predicted variable}
\end{table*}

%===================================
%===================================
\subsection{\gls{medLabel}}
\labsec{solarmed:modelling:med}

The \gls{medLabel} is modelled statically, considering changes in the system
operating conditions happen at a slow enough rate that the dynamic behavior
between stable states can be neglected, and thus, only those stable states are
considered. Two models are developed for the \gls{medLabel}: a data-driven model
based on experimental data from the pilot plant, and a first-principles model
based on thermodynamic equations. The data-driven model is the one integrated
in the overall plant model used in optimization applications (See
\refch{solarmed:optimization}), while the first-principles model is used for
comparison purposes and to gain insight into the operation of the \gls{medLabel}
(See \refch{solarmed:std}).

%================================
\subsubsection{Data-driven model}

A \gls{gprLabel} model is calibrated using data from the two experimental campaigns
described in \refsec{solarmed:facility:med}.

As observed in \refmod{solarmed:med}, the model has five inputs: the heat source
flow rate ($q_{s}$), the feedwater flow rate ($q_{f}$), the inlet
temperature of the heat source ($T_{s,in}$), the inlet temperature of the
cooling water ($T_{c,in}$) and the outlet temperature of the cooling water
($T_{c,out}$). The model returns three main outputs: the distillate flow
rate ($q_{d}$), the outlet temperature of the heat source ($T_{s,out}$)
and the cooling water flow rate ($q_{c}$). An additional output is
included with a validated value for the condenser outlet temperature
($T_{c,out}$), in cases where an unfeasible temperature is given as input.
In \refmod{solarmed:med}, the functions $f_{bb,q}$ and $f_{bb,T}$ represent the
black-box \gls{gprLabel} models for the model with the main outputs and the
auxiliary output, respectively. 

% Bloque de modelo con interfaz y ecuaciones
\begin{modelcounter}{\gls{medLabel} model}
    \begin{align*}
        q_{d}&,\,T_{s,out},\,q_{c},\,T_{c,out} = f\left( q_{s},q_{f}, T_{s,in}, T_{c,out}, T_{c,in} \right) \\
        & q_{d},\,T_{s,out},\,q_{c} = f_{bb,q}\left( q_{s},q_{f}, T_{s,in}, T_{c,out}, T_{c,in} \right) \\
        & \textbf{if } q_{c} > \overline{q_{c}} : q_{c} = \overline{q_{c}} \\
        & \textbf{or if } q_{c} < \underline{q_{c}} : q_{c} = \underline{q_{c}} \\
        & \textbf{then: } \\
        & \quad T_{c,out} = f_{bb,T}\!\left(q_{s},q_{f}, T_{s,in}, q_{c}, T_{c,in}\right) \\
        & \quad q_{d},\,T_{s,out} = f_{bb,q}\!\left(q_{s},q_{f}, T_{s,in}, T_{c,out}, T_{c,in}\right)
    \end{align*}
    \labmod{solarmed:med}
\end{modelcounter}

\begin{marginfigure}[-5.5cm]
    \includegraphics[]{solarmed-modelling-med-diagram.png}
    \caption{\gls{medLabel} process diagram.}
    \labfig{solarmed:diagrams:med}
\end{marginfigure}


%================================
\subsubsection{First-principles model}

The first-principles model is based on thermodynamic equations and mass and
energy balances. A detailed description of the model can be found in the
Appendix, \nrefch{appendix:med-model}.

%================================
\subsubsection{Electrical consumption}

A similar procedure to the one for the solar field and thermal storage was
followed, with some particular considerations:

The extraction pumps need to be evaluated under vacuum conditions, since this is
how the system operates under normal conditions. This has a direct influence on
the intake conditions (lower head pressure). The brine pump is evaluated with a
\textit{step train test} while the plant is inactive (no thermal input), in
vacuum conditions and with feedwater being pumped. The water will naturally fall
by gravity to the final effect\sidenote{Given the plant vertically-stacked
configuration}. The power consumption is measured using the \gls{vfdLabel}
integrated power meter. For the distillate pump, water does not reach the final
condenser unless vapor is generated, so the pump is evaluated with the plant
active in operating conditions that produce distillate at the midrange working
conditions of the \gls{vfdLabel}, aournd 35~Hz. The step train is then performed
by alternating high and low values\sidenote{For example, 35 $\rightarrow$ 40
$\rightarrow$ 30 $\rightarrow$ 45 $\rightarrow$ 25 $\rightarrow$ 50
$\rightarrow$ 20~Hz} and making sure the level stays within the operating range. 

For the rest of the system pumps: heat source, feedwater and cooling water, a
simple \textit{step train test} is performed. The vacuum system has tree levels:
high when generating vacuum, low when maintaining vacuum, and inactive. Each
with an associated constant consumption. Finally, the obtained electrical model
is shown in \refmod{solarmed:modelling:med:electrical_consumption}.

%================================
\subsubsection{Validation}

As explained, two \gls{gprLabel} models are used, so two models need to be
calibrated, one for the aforementioned desired system outputs ($q_{d}$,
$T_{s,out}$ and $q_{c}$) and an additional one for the condenser outlet
temperature. An 80/20 training/validation split is used. The training is
performed using an \gls{rbfLabel} kernel. The regression model is defined with
the \texttt{GPy} library, which includes a Gaussian likelihood with a noise term
by default. The kernel hyperparameters (variance, lengthscale, and noise
variance) are optimized by maximizing the log-marginal likelihood using
GPy's~\sidecite{gpy2014} built-in L-BFGS local optimizer. The regression results
for the main outputs are shown in \reffig{solarmed:modelling:med-regression}. As
can be seen, the model performs very well for all outputs, with $R^2$ values
above 0.90 in all cases. 

\begin{modelcounter}{\gls{medLabel} electrical consumption}
    \begin{align*}
        C_{e,med} &= \text{med electrical consumption}(q_{med,s}, q_{med,f}, q_{med,c}, q_{med,d}, q_{med,b}) \\
        &C_{e,med,s} = 0.0104 - 0.025 \, q_{med,s} + 0.0339 \, q_{med,s}^2 \quad \text{m}^3/\text{h} \rightarrow \text{kW}_e\\
        &C_{e,med,f} = 0.704 - 0.0947 \, q_{med,f} + 0.0191 \, q_{med,f}^2 \quad \text{m}^3/\text{h} \rightarrow \text{kW}_e\\
        &C_{e,med,c} = 5.218 - 0.924 \, q_{med,c} + 0.0567 \, q_{med,c}^2 \quad \text{m}^3/\text{h} \rightarrow \text{kW}_e\\
        &C_{e,med,d} = 4.150 - 3.657 \, q_{med,d} + 0.948 \, q_{med,d}^2 \quad \text{m}^3/\text{h} \rightarrow \text{kW}_e\\
        &C_{e,med,b} = 0.031 - 0.019 \, q_{med,b} + 1.33\times 10^{-3} \, q_{med,b}^2 \quad \text{m}^3/\text{h} \rightarrow \text{kW}_e\\
        &C_{e,med,vac} =
        \begin{cases}
            5, & \text{if } \text{med}_{\text{vac}} = 2 \\
            1, & \text{if } \text{med}_{\text{vac}} = 1 \\
            0, & \text{if } \text{med}_{\text{vac}} = 0
        \end{cases} \\
        &C_{e,med}  = C_{e,med,s} + C_{e,med,f} + C_{e,med,c} + C_{e,med,d} + C_{e,med,b} + C_{e,med,vac}
    \end{align*}
    \labmod{solarmed:modelling:med:electrical_consumption}
\end{modelcounter}



\begin{figure*}[h!]
    \includegraphics[]{med_model_regression_plot.png}
    \savebox\captionqr{\qrcode[hyperlink,height=0.5in]{\repositoryBaseUrl/figures/med_model_regression_plot.html}}
    \caption[\gls{medLabel} \gls{gprLabel} model regression for
    different outputs.]{\gls{medLabel} \gls{gprLabel} model regression for
    different outputs. Dataset includes data spanning 6 years of
    operation.\hfill\usebox\captionqr}
    \labfig{solarmed:modelling:med-regression}
\end{figure*}

\begin{figure*}[h!]
    \includegraphics[width=\linewidth]{med_validation.png}
    \savebox\captionqrleft{\qrcode[hyperlink,height=0.5in]{\repositoryBaseUrl/figures/med_validation_20230331.html}}
    \savebox\captionqrright{\qrcode[hyperlink,height=0.5in]{\repositoryBaseUrl/figures/med_validation_20230418.html}}
    \caption[\gls{medLabel} model validation tests]{\gls{medLabel} model validation tests. \hfill \usebox\captionqrleft\hspace{1ex}\usebox\captionqrright}
    \labfig{solarmed:modelling:validation:med}
\end{figure*}

\reffig{solarmed:modelling:validation:med} shows the model validation for two
particular tests. Several observations can be drawn. First, the model shows high
accuracy in predicting both distillate production and the heat source outlet
temperature, even during transient phases such as those observed in the early
part of the 31032023 test (\reffig{solarmed:modelling:validation:med}, left).

In contrast, the cooling water predictions are less reliable during these
transitions. As vapor gradually accumulates—visible as a pressure ramp in the
condenser between 10:30 and 12:30 in the 202303031 test ---it becomes difficult
to predict how much vapor ultimately reaches the final condenser. Once stable
conditions are established, however, the model can accurately estimate the
cooling water demand required to maintain equilibrium at a given operating
point, even under varying inlet water temperatures ($T_{c,in}$). This is evident
in the later phase of the 202303031 test and throughout most of the 20230418
test.

A further observation is the system's strong sensitivity to the cooling water
inlet temperature. In the pilot plant, which operates in a closed loop, these
inlet conditions are constantly shifting. Despite this, the static model is able
to capture the condenser's overall behavior once the system stabilizes and the
inlet temperature becomes the primary changing variable.

\begin{table*}[!b]
\centering
\caption{Models parameters}
\labtab{solarmed:modelling:models-parameters}
\resizebox{.8\linewidth}{!}{%
\begin{tabular}{rlrlllrlc}
\hline
\textbf{Subsystem}    &  & \textbf{Parameter}     &  & \textbf{Description}                            &  & \textbf{Value}                                &  & \textbf{Units}     \\ \hline \\ \hline
\multicolumn{9}{c}{Model parameters}                                                                                                                     \\ \hline
\multirow{2}{*}{Solar field}     &  & $\beta$      &  & Gain coefficient                       &  & 4.36$\times10^{-2}$                           &  & m         \\
                                 &  & H            &  & Heat loss coefficient                  &  & 13.67                                &  & W/m$^2$   \\
\cline{1-1}
\multirow{4}{*}{Thermal storage} &  & V$_h$        &  & Volume of control volume               &  & [5.94, 4.87, 2.19]                   &  & m$^3$     \\
                                 &  & H$_h$        &  & Heat loss coefficient                  &  & [6.98, 5.84, 30.41]$\times 10^-3$    &  & W/K       \\
                                 &  & V$_c$        &  & Volume of control volume               &  & [5.33, 7.56, 0.9]                    &  & m$^3$     \\
                                 &  & H$_c$        &  & Heat loss coefficient                  &  & [0.013.96, 0.1, 0.022]$\times 10^-3$ &  & W/K       \\
\cline{1-1}
Heat exchanger  &  & UA           &  & Heat transfer
conductance              &  & 1.35$\times 10^4$                    &  & W/K
\\ \hline

\multicolumn{9}{c}{Fixed model parameters}                                                                                                              \\ \hline
\multirow{6}{*}{Solar field}     &  & $A_{cs}$     &  & Collector tube cross-section area      &  & 7.85$\times 10^{-5}$                 &  & m$^2$     \\
                                 &  & $n_{tb-c}$  &  & Number of parallel tubes per collector &  & 1                                    &  & -         \\
                                 &  & $n_{c-loop}$ &  & Number of parallel collectors per loop &  & 7 $\times$ 5                         &  & -         \\
                                 &  & $L_{tb}$        &  & Individual collector tube length       &  & 23                                   &  & m         \\
                                 &  & $n_{c-s}$    &  & Number of collector row's in series    &  & 2                                    &  & -         \\
                                 &  & T$_{max}$    &  & Maximum working temperature            &  & 120                                  &  & $^\circ$C
     \\ \hline
\end{tabular}%
}
\end{table*}

\newpage

%===================================
%===================================
\section{Discrete modelling. Operation state}[Discrete modelling]
\labsec{solarmed:modelling:discrete}

In the previous section, the dynamic and static models of each component were
explained. This section focuses on the discrete behavior, its \textit{operation
state}. This component is modelled by means of \glspl{fsmLabel}. In order to
determine its state, the \fullgls{fsmLabel} uses information from its
inputs, internal state and the configured parameters.

\marginreminder[*-7]{\glspl[format=long]{fsmLabel}}{
    A finite state machine is a model of behavior composed of a finite number of
\textit{states} and \textit{transitions} between those states. Within each
state and transition some \textit{action} can be performed\footnote{See
    \nrefsec{intro:modelling:discrete} for a more detailed description.}.
}

The complete system is divided into two subsystems: the heat generation and
storage subsystem (\gls{sftsLabel}) and the separation subsystem
(\gls{medfsmLabel}).

%================================
\subsection{Heat generation and storage subsystem (\texttt{sfts})}
\labsec{solarmed:modelling:sfts_fsm}

This subsystem encompasses the \fullgls{sfLabel} and the \fullgls{tsLabel}. The
subsystem can be modelled with a simple \gls{fsmLabel}. The \gls{fsmLabel} has
two inputs, the recirculation flow rate on each circuit ($q_{sf}$ and
$q_{ts,src}$), and the \gls{fsmLabel} states are computed based on whether water
is being recirculated on each. Four states are defined as shown in
\reftab{solarmed:modelling:sfts_fsm_states} and the possible transitions are
visualized in \reffig{solarmed:modelling:fsm-diagrams} (a). The states involve:

\begin{itemize}
    \item \textbf{Off} (0). The system is off, no water is being recirculated in
    either circuit.
    \item \textbf{Warming up solar field} (1). Water is being recirculated in the
    solar field circuit but not in the thermal storage circuit. The solar field
    is being heated up.
    \item \textbf{Recirculating thermal storage} (2). Water is being recirculated
    in the thermal storage circuit but not in the solar field circuit. The
    thermal storage is being mixed.
    \item \textbf{Solar field heating up thermal storage} (3). Water is being
    recirculated in both circuits. The solar field is heating up the thermal
    storage.
\end{itemize}

% \begin{modelcounter}{\gls{sftsLabel} \gls{fsmLabel}}
%     \begin{align*}
%         \text{state}_{sfts} & = \text{sfts fsm model}(q_{sf}, q_{ts,src}; \theta) \\
%     \end{align*}
%     \labmod{solarmed:sfts_fsm}
% \end{modelcounter}

\begin{margintable}[*-9]
\caption[\gls{sftsLabel} \gls{fsmLabel} states definitions]{\gls{sftsLabel}
\gls{fsmLabel} states definitions. $\land$ represents the logical \texttt{AND}
operator.}
\labtab{solarmed:modelling:sfts_fsm_states}
\resizebox{\linewidth}{!}{%
\begin{tabular}{cll}
    \toprule
    \textbf{State} & \textbf{Name} & \textbf{Condition}  \\
    \midrule
    0 (00) & Off & $q_{sf}\land q_{ts,src}==0$ \\
1 (01) & \begin{tabular}[t]{@{}l@{}}Warming up\\\gls{sfLabel}\end{tabular} & $q_{sf} > 0 \land q_{ts,src}==0$ \\
2 (10) & \begin{tabular}[t]{@{}l@{}}Recirculating\\\gls{tsLabel}\end{tabular} & $q_{sf} ==0 \land q_{ts,src} > 0$ \\
    3 (11) & \gls{sfLabel} heating up \gls{tsLabel} & $q_{sf}\land q_{ts,src} > 0$
    \\
    \bottomrule
\end{tabular}
}
\end{margintable}

Additionally, some conditions are configured with the following
parameters\sidenote{In parentheses the value used}:

\begin{itemize}
    \item \textbf{Enable recirculating thermal storage} (false). Allow to
    recirculate water in the thermal storage circuit while no water is being
    heated in the solar field. This would be used to mix the hot and cold tanks.
    
    \item \textbf{Active cooldown time} (10 minutes). Time to wait before
    activating the system again after stopping it.
\end{itemize}

% Figura de la máquina de estado finito y gráfica con la evolución de estados
% durante un ensayo

\begin{figure*}[]
    \centering
    \subfloat[\centering \gls{sftsLabel}]{{\includegraphics[width=0.3\linewidth]{solarmed-sfts-fsm.png}}}%
    \hspace{0.1\linewidth}
    \subfloat[\centering \gls{medfsmLabel}]{{\includegraphics[width=0.58\linewidth]{solarmed-med-fsm.png}}}%
    \caption[Finite-state machines for the different subsystems]{Finite-state machines for the different subsystems}
    \labfig{solarmed:modelling:fsm-diagrams}
\end{figure*}


%================================
\subsection{Separation subsystem (\texttt{med})}
\labsec{solarmed:modelling:med_fsm}

% Figura de la máquina de estado finito y gráfica con la evolución de estados
% durante un ensayo

Two inputs are used in this machine, one logical which indicates the active
state the system when all of the required pumps for operation are enabled. The
other is an integer variable that regulates the vacuum state
($\text{med}_{\text{vac}}$). This latter
variable has three possible values: 0 when the vacuum pump is off, 1 when the
vacuum pump operates at low speed (maintaining vacuum) and 2 when the vacuum
pump operates at high speed (generating vacuum). The \gls{fsmLabel} states are
computed based on these two inputs and the conditions shown in
\reftab{solarmed:modelling:med_fsm_states}. The possible transitions are
visualized in \reffig{solarmed:modelling:fsm-diagrams}~(b). The states are
described in the following:

\begin{itemize}
    \item \textbf{Off} (0). The system is off, no pumps are operating and no vacuum is
    being generated.
    \item \textbf{Generating vacuum} (1). The system is off, no pumps are operating but
    vacuum is being generated.
    \item \textbf{Idle} (2). The system is off, no pumps are operating but vacuum is
    maintained.
    \item \textbf{Starting-up} (3). The system is starting up, all pumps start
    operation following the procedure outlined in
    \refsec{solarmed:std:monitoring-control}. No distillate is produced at this
    stage and the temperatures and pressures progressively ramp up until reaching
    equilibrium for the given inputs. 
    \item \textbf{Shutting down} (4). The system is being shut down, distillate
    production has stopped and the system is cooled-down progressively.
    Extraction pumps start cycles of operation to empty the system. Vacuum may
    be maintained at this stage.
    \item \textbf{Active} (5). The system is active, all pumps are operating and
    distillate is being produced. Vacuum is maintained.
\end{itemize}

As in the previous machine, the machine has additional \gls{fsmLabel} parameters
that regulate its behavior, specifically:

\begin{itemize}
    \item \textbf{Vacuum duration time} (30 minutes). Time to generate vacuum in the MED system.
    \item \textbf{Brine emptying time} (60 minutes). Time to extract brine from MED plant.
    \item \textbf{Startup duration time} (20 minutes). Time to start up the MED
    plant, once vacuum is generated.
    \item \textbf{Off cooldown time} (12 hours). Time to wait before activating the MED plant again after shutting it off.
    \item \textbf{Active cooldown time} (2 hours). Time to wait before activating the MED plant again after shutting it off or suspending it.
\end{itemize}

Finally, the machine has internal states that are updated during the machine
evaluation in order to keep track of the progress of the different timed
actions and whether they have been completed or not. These are for example the
vacuum elapsed samples, the startup elapsed samples or the brine emptying
elapsed samples. The associated logical states would be whether vacuum has been
generated, whether the startup procedure has been completed or whether the brine
emptying has been completed. 

\begin{table}
\caption[\gls{medfsmLabel} \gls{fsmLabel} states definitions]{\gls{medfsmLabel} \gls{fsmLabel} states definitions. $\land$ represents
the logical \texttt{AND} operator, $\exists$ represents that at least one meets
the condition, and $\forall$ represents that all meet the condition.}
\labtab{solarmed:modelling:med_fsm_states}
% \resizebox{\texwidth}{!}{%
\begin{tabular}{cll}
    \toprule
    \textbf{State} & \textbf{Name} & \textbf{Condition}  \\
    \midrule
    0 & Off & $\forall q == 0$ \\
    1 & Generating vacuum & $\text{med}_{vac} == 2$ \\
    2 & Idle & $\forall q == 0 \land \text{med}_{vac} == 1$ \\
    3 & Starting-up & $\forall q > \underline{q} \land \text{med}_{vac} \ge 1$ \\ % \land \forall T > \underline{T} $
    4 & Shutting down & $\exists \text{q} < \underline{\text{q}}$ \\
    5 & Active & $\forall \text{q} > \underline{\text{q}} \land \text{med}_{vac} \ge 1$ \\ % \land \forall T > \underline{T} $
    \bottomrule
\end{tabular}
% }
\end{table}

%================================
\subsection{Validation}

\reffig{solarmed:modelling:fsm} shows the evolution of the states of both
\glspl{fsmLabel} during a test. It can be seen how both machines evolve in
parallel in the bottom plot, while the upper one shows the cumulative input
values\sidenote{All variables are shown as integers and aliases are created to
group the logical active conditions \eg $\texttt{med\_active} = \forall q >
\underline{q}$}. The \gls{sftsLabel} first starts with the heating up of the
solar field state triggered by the activation of the solar field pump
(\texttt{sf\_active}), and once the temperature is high enough after few samples
the thermal storage starts recirculating (\texttt{ts\_active}). This continues
until the end of the tests when the temperature decreases so it gets deactivated
(at sample 80 in \reffig{solarmed:modelling:fsm}).

For the \gls{medfsmLabel}, there is no measurement for the vacuum state of the
system so it is assumed that it is set at a high-level from the beginning and
kept until the end of the test. The system starts in the off state, and once
vacuum is generated the start-up procedure gets triggered at sample 12-13. The
system then stays active producing separation until it is shut-off at sample 56
for two samples, equivalent to over 10 minutes.

\begin{figure*}[htbp]
    \includegraphics[width=.9\linewidth]{solarmed_fsm_evolution_20230703.png}
    \caption[SolarMED \gls{fsmLabel} states evolution during a test on 20230703]{\gls{solarmedLabel} \gls{fsmLabel} states evolution during a test on 20230703.
    Purple represents the \gls{medfsmLabel} state and orange the
    \gls{sftsLabel}.\\ \textbf{NOTE:} \textit{med active} is equivalent to the condition $\forall q_{med,i} > \underline{q_{med,i}}$} 
    \labfig{solarmed:modelling:fsm}
\end{figure*}


\section{Complete system model}
\labsec{solarmed:modelling:complete}

Finally, all the individual components described above (continuos and discrete)
are combined to form the complete system model\sidenote{To clearly distinguish
that a variable belongs to a subsystem, the subsystem acronym is added as a
subscript \eg~$q_s\,\rightarrow\,q_{med,s}$}. The complete model is a hybrid
model that orchestrates physics models for the solar field, heat exchanger,
thermal storage, three-way valve and data-driven for the \gls{medLabel}, plus
two supervisory finite state machines. The continuous states are the outputs of
the different subsystems, while the discrete states are the states of the two
\glspl{fsmLabel}.

The model ---defined in \refmod{solarmed:solarmed}--- is implemented following
an object-oriented approach\sidenote{Specifically as a Python class}. Once it is
initialized by being provided with the initial discrete states (\eg vacuum
generated, brine emptied from final effect, etc) and system parameters (\eg
thermal storage volumes, heat exchanger transfer conductance, \gls{medfsmLabel}
timings, etc). It also needs to be initialized with the initial solar field and
thermal storage temperatures.

Each step takes environment inputs (irradiance $I$, ambient temperature
$T_{amb}$, seawater temperature $T_{med,c,in}$) and operation decisions
(\eg, $q^{\ast}_{sf}$, $q^{\ast}_{ts,src}$, $q^{\ast}_{med,s}$,
$q^{\ast}_{med,f}$, $T^{\ast}_{med,s,in}$, $T^{\ast}_{med,c,out}$). Setpoints
reflect operator/optimizer intent, the model then validates and turns them into
realized outputs. Each step advances the plant one sampling interval.
Physically, the model treats the installation as three hydraulic/thermal loops that
exchange energy, all embedded in ambient conditions and subject to equipment
limits.

The supervisory logic (\ie, \glspl{fsmLabel}) first decides which subsystems are
allowed to act (solar field circulating, storage charging/discharging,
\gls{medLabel} producing). That logic turns setpoints into admissible operating
points: a requested flow or temperature may be permitted as is, clipped to a
limit, or forced to zero if a component is inactive, or set to a predefined
value\sidenote{For example, if the \gls{medLabel} subsystem is in the shutting
down process, no thermal load is extracted from the thermal storage, and only
the electrical consumption of the extraction pumps is considered (and of the
vacuum if kept active)}. Once the discrete state of the system is fixed for the
current step, the physics is evaluated in a sequence that mirrors causality and
heat flow. This set of dependencies between the different subsystems is
visualized in \reffig{solarmed:modelling:complete_model}.

The \gls{medLabel} is solved first because it defines the thermal demand that
the rest of the system must support. With the \gls{medLabel} demand known, the
three-way valve determines how much hot water must be extracted from storage and
how to mix it to meet the \gls{medLabel} hot-side inlet target. Physically, it
blends water drawn from the top of the hot tank with the \gls{medLabel} return,
choosing a mix ratio and discharge flow that close the \gls{medLabel}'s energy
balance. That fixes the storage discharge flow and the thermal duty the storage
must deliver in this step.

Next, the heat supply side is computed. When the solar field is circulating and
storage is being recharged, the solar field, heat exchanger, and tanks are
thermally coupled: the solar outlet temperature, the heat exchanger
primary/secondary outlet temperatures, and the evolving tank stratification all
depend on each other. In that coupled case, a small nonlinear subproblem is
solved so that energy balances are simultaneously satisfied across generation,
transfer, and load (storage), considering the previous tank states. When the
solar field and storage are not simultaneously active (for example, idle storage
or solar field, or storage discharging), the supply side decouples: the solar
field is first evaluated alone, producing a primary-side outlet temperature ;
then the tanks update their stratified temperatures by applying the computed
inlet/outlet temperatures and the discharge flows.


Once temperatures and flows are settled, the model computes electrical powers
using the actuators fitted curves. Finally, time-dependent states are rolled
forward: solar loop histories are shifted to carry the new inlet/outlet values
into the next step, and the tank stratification produced this step becomes the
initial condition for the next.


% Bloque de modelo con interfaz y ecuaciones
\begin{modelcounter}{SolarMED model}
    \begin{align*}
    q_{med,d},\,C_{e} &= f\bigl(
        q_{med,s}, q_{med,f}, T_{med,s,in}, T_{med,c,out}, \\
        &\qquad q_{ts,src}, q_{sf}, med_{vac,st}, T_{med,c,in}, T_{amb}, I; \\
        &\qquad \theta_{sf}, \theta_{hx}, \theta_{med}, \theta^{fsm}_{sfts}, \theta^{fsm}_{med}, \theta_{\infty}
    \bigr) \\ \\
    &\text{st}_{sfts}  = \text{sfts fsm model}(q_{sf}, q_{ts,src}; \theta^{fsm}_{sfts}) \\
    &\text{st}_{med}  = \text{med fsm model}(q_{med,s}, q_{med,f}, T_{med,c}; \theta^{fsm}_{med}) \\ \\
    &T_{\text{sf,out}} = \text{sf model}\!\left(
        T_{\text{sf,out},k-1},\,
        \mathbf{T}_{\text{sf,in},k-n:k},\,
        \boldsymbol{q}_{sf,k-n:k},\,
        I,\,
        T_{\text{amb}};\,
        \theta_{sf}
        \right) \\
        &\pmb{T}_\text{ts,h},\; \pmb{T}_\text{ts,c} = \text{ts model}\Bigl( \\
        & \qquad \qquad \pmb{T}_\text{ts,h}(k-1),\;
        \pmb{T}_\text{ts,c}(k-1),\;
        T_\text{hx,s,out}, \\
        &\qquad \qquad T_\text{med,s,out},\;
        q_\text{ts,src},\;
        q_\text{ts,dis},\;
        T_\text{amb};\;
        \boldsymbol{\theta}_\text{ts} \\
        & \Bigr) \\
        &T_{sf,in},\,T_{hx,s,out} = \text{hx model}(
            T_{sf,out}, T_{ts,c,b}, q_{sf}, q_{ts,src}, T_{amb}; \theta_{hx}
            ) \\
            &q_{ts,dis}=\text{3wv model}(q_{med,s}, T_{med,s,in}, T_{med,s,out}) \\
            &q_{med,d},\,T_{med,s,out},\,q_{med,c},\,T_{med,c,out} = \text{med model}\bigl(\\ 
            &\qquad \qquad q_{med,s},q_{med,f}, T_{med,s,in}, T_{med,c,out}, T_{med,c,in} \\ 
            & \bigr) \\ \\
            &C_{e,sf} = \text{sf electrical consumption}(q_{sf}) \\
            &C_{e,ts} = \text{ts electrical consumption}(q_{ts,src}) \\
            &C_{e,med} = \text{med electrical cons.}(q_{med,s}, q_{med,f}, q_{med,c}, q_{med,d}, q_{med,b}) \\
            &C_e=C_{e,sf}+C_{e,ts}+C_{e,med}
    \end{align*}
    \labmod{solarmed:solarmed}
\end{modelcounter}

\begin{marginfigure}[-13cm]
    \includegraphics[]{solarmed-modelling-complete_model.png}
    \caption{Complete \gls{solarmedLabel} model architecture.}
    \labfig{solarmed:modelling:complete_model}
\end{marginfigure}

\subsection{Validation}

% Gráfica tocha mostrado algún día y después una tabla para varios días
\begin{figure*}[h!]
    \includegraphics[width=\linewidth]{solar_med_validation_20230414.png}
    \savebox\captionqrleft{\qrcode[hyperlink,height=0.5in]{\repositoryBaseUrl/figures/solar_med_validation_20230414.html}}
    \caption[\gls{solarmedLabel} model validation]{\gls{solarmedLabel} model
    validation \hfill \usebox\captionqrleft\hspace{1ex}\usebox\captionqrright}
    \labfig{solarmed:modelling:validation:solarmed}
\end{figure*}


\reffig{solarmed:modelling:validation:solarmed} and
\reftab{solarmed:modelling:validation:solarmed} show the results obtained for
some days when evaluating the complete model. The model is evaluated with a
sample time of 400~s but four different prediction horizons. In
\reffig{solarmed:modelling:validation:solarmed} two of them are shown, 1 hour
and 8 hours. This means, \eg, that for a horizon of 1 hour, the model is
evaluated for 1 hour with no feedback from measured data. Then the model is
updated with the actual measured state at that time and the process is repeated
for the rest of the test. Finally, the error is computed between the model
output and the actual measurement. 

In \reffig{solarmed:modelling:validation:solarmed}, while all subsystems are
active, it can be seen a good agreement between the model prediction and the
actual measurements for both horizons. A higher error is observed during startup
and shutdown for the solar
field\sidenote[][*-3]{\reffig{solarmed:modelling:validation:solarmed} -- \textit{Solar
field} from 06:00 to 10:00 and from 15:00 to 17:45} (which propagates to the
heat exchanger).

The \gls{medLabel}, since it is a static model, is not affected by the horizon
time (as long as the hot tank top temperature is above the operating
temperature), while the solar field, and specially the thermal storage, show a
higher error that accumulates over time. As expected, the 8-hour horizon shows a
higher error than the 1-hour horizon, since the model is not updated with actual
measurements for the whole duration of the test. Nonetheless, the final state of
the thermal storage temperature profiles is not too far from the actual
measurements\sidenote[][*-3]{At the end of operation (15:00), for the 8-hour
horizon: $T_{ts,h,t}=$96.98~$^\circ$C \vs~95.69 and $T_{ts,c,b}=$80.2~$^\circ$C
\vs~81.7. In terms of energy stored: $Q_{ts,c}=$194.45~kWh$_{th}$ \vs~211.36 and
$Q_{ts,c}=$170.24~kWh$_{th}$ \vs~173.52}, showing that the model is able to
capture the overall behavior of the system. This is confirmed by the good
agreement observed in the energy stored throughout the
test\sidenote{\reffig{solarmed:modelling:validation:solarmed} -- \textit{Thermal
storage -- Energy}}.

For the \gls{medLabel}, as commented in the component section, the output with
the highest error is the cooling water flow rate, showing higher errors at the
beginning where there is a heat source temperature mismatch (the actual
temperature is higher that the upper limit of the data-driven model so is
clipped). In the second part of the test, the inlet cooling water temperature
increases significantly, and since the flow was kept constant in the test, the
condenser is not stable anymore\sidenote{Observe the trend in
\reffig{solarmed:modelling:validation:solarmed} -- \textit{MED --
Temperatures~(right) -- $P_{v,14}$}}.

\reftab{solarmed:modelling:validation:solarmed} shows the results obtained for
different metrics (\gls{maeLabel}, \gls{mapeLabel}) for two different test days.
This time two additional horizons are included, 30 minutes and 4 hours. The
table also shows the computation time required to evaluate the model for each
prediction horizon.

As mentioned, static variables such as the distillate production show no
variation with the prediction horizon (\eg \gls{maeLabel}:
$q_{med,d}=0.15$~m$^3$/h for all horizon times in test 20230414), while dynamic
variables such as the thermal storage hot tank top temperature show a clear
increase in error with longer horizons (\eg \gls{maeLabel}
$T_{ts,h}=2.36\,^\circ$C for a 30-minute horizon and $T_{ts,h}=3.59\,^\circ$C for a
4-hour horizon in test 20230414).

When judging the performance metrics, it is important to keep in mind that for
the complete model, feedback from the real system is only available at the
beginning of the day. From that point on, most subsystem inputs depend on states
from other subsystems. For example, the solar field inlet temperature depends on
the heat exchanger output, which in turn depends on the thermal storage state,
which itself depends on the MED plant operation and, ultimately, on the solar
field. Consequently, errors can propagate and accumulate over time.


Therefore, achieving consistent accuracy for multi-hour horizons in a fully
coupled system is challenging. For standalone component models,
\gls{mapeLabel} values above 5--10~\% would typically be considered poor;
however, for the complete integrated system, maintaining \gls{mapeLabel} values
below roughly 15~\% across several hours is a commendable result. This is
particularly the case for thermal state variables such as $T_{ts,h}$ and
$T_{ts,c}$, where the model captures the general dynamic behavior well even at
long horizons. The computation times shown in the last column remain reasonable,
with the 8-hours horizon evaluation taking around 5~s.

\marginnote[*-10]{
    \begin{kaobox}[title=Extended validation results]
    \begin{minipage}{0.6\linewidth}
     A visualization of all evaluated tests for every model validation are
    available in the thesis repository as a compressed folder
    \end{minipage}
    \hfill
    \begin{minipage}{0.3\linewidth}
    \centering
    \qrcode[hyperlink,height=0.5in]{\repositoryBaseUrl/assets/solarmed-models-validation-extented.7z}
    \end{minipage}
    \end{kaobox}
}

% \marginnote[*-10]{
%     \begin{kaobox}[title=Extended validation results]
%         \raggedright
%         % \begin{minipage}{\linewidth}
%         \begin{tabular}{m{0.3\linewidth} c}
%             A visualization of all evaluated tests for every model validation are
%             available in the thesis repository as a compressed folder  &
%             \qrcode[hyperlink,height=0.5in]{\repositoryBaseUrl/assets/solarmed-models-validation-extented.7z}
%         \end{tabular}
%         % \end{minipage}
%     \end{kaobox}
% }

\begin{table}[]
\caption[Summary table of the prediction results obtained with the
\gls{solarmedLabel} model]{Summary table of the prediction results obtained with the
\gls{solarmedLabel} model for different test days, sample time set to 400s and
different prediction horizons.}
\labtab{solarmed:modelling:validation:solarmed}
\resizebox{\textwidth}{!}{%

\begin{tabular}{ccccccccccc}
\hline
\multirow{2}{*}{\textbf{\begin{tabular}[c]{@{}c@{}}Predicted\\ variable\end{tabular}}} &  & \multirow{2}{*}{\textbf{\begin{tabular}[c]{@{}c@{}}Test\\ date\end{tabular}}} &  & \multirow{2}{*}{\textbf{\begin{tabular}[c]{@{}c@{}}Horizon\\ time\\ (s)\end{tabular}}} &  & \multicolumn{5}{c}{\textbf{Performance metric}}
\\\cline{7-11}
 &  &  &  &  &  & \textbf{\begin{tabular}[c]{@{}c@{}}MAE\\ (s.u.)\end{tabular}} &  & \textbf{\begin{tabular}[c]{@{}c@{}}MAPE\\ (\%)\end{tabular}} &  & \textbf{\begin{tabular}[c]{@{}c@{}}Time\\ (s)\end{tabular}}
 \\\cline{1-1}\cline{3-3}\cline{5-5}\cline{7-7}\cline{9-9}\cline{11-11}
\multirow{8}{*}{q$_{med,d}$ (m$^3$/h)} &  & \multirow{4}{*}{20230414} &  & 1800 &  & 0.15 &  & 16.80 &  & 13.42 \\
 &  &  &  & 3600 &  & 0.15 &  & 16.80 &  & 22.13 \\
 &  &  &  & 14400 &  & 0.15 &  & 16.80 &  & 8.15 \\
 &  &  &  & 28800 &  & 0.15 &  & 16.80 &  & 5.86 \\
\cline{3-11}
 &  & \multirow{4}{*}{20230418} &  & 1800 &  & 0.07 &  & 9.99 &  & 7.83 \\
 &  &  &  & 3600 &  & 0.07 &  & 9.99 &  & 4.52 \\
 &  &  &  & 14400 &  & 0.07 &  & 9.99 &  & 2.64 \\
 &  &  &  & 28800 &  & 0.07 &  & 9.99 &  & 2.52 \\
\cline{1-11}
\multirow{8}{*}{T$_{ts,h}$ ($^\circ$C)} &  & \multirow{4}{*}{20230414} &  & 1800 &  & 2.36 &  & 2.50 &  & 13.42 \\
 &  &  &  & 3600 &  & 3.15 &  & 3.32 &  & 22.13 \\
 &  &  &  & 14400 &  & 3.59 &  & 3.78 &  & 8.15 \\
 &  &  &  & 28800 &  & 3.18 &  & 3.37 &  & 5.86 \\
\cline{3-11}
 &  & \multirow{4}{*}{20230418} &  & 1800 &  & 0.48 &  & 0.52 &  & 7.83 \\
 &  &  &  & 3600 &  & 1.09 &  & 1.18 &  & 4.52 \\
 &  &  &  & 14400 &  & 3.95 &  & 4.29 &  & 2.64 \\
 &  &  &  & 28800 &  & 5.87 &  & 6.38 &  & 2.52 \\
\cline{1-11}
\multirow{8}{*}{T$_{ts,c}$ ($^\circ$C)} &  & \multirow{4}{*}{20230414} &  & 1800 &  & 2.66 &  & 3.76 &  & 13.42 \\
 &  &  &  & 3600 &  & 3.03 &  & 4.27 &  & 22.13 \\
 &  &  &  & 14400 &  & 4.20 &  & 5.93 &  & 8.15 \\
 &  &  &  & 28800 &  & 4.31 &  & 6.20 &  & 5.86 \\
\cline{3-11}
 &  & \multirow{4}{*}{20230418} &  & 1800 &  & 2.76 &  & 3.94 &  & 7.83 \\
 &  &  &  & 3600 &  & 3.19 &  & 4.52 &  & 4.52 \\
 &  &  &  & 14400 &  & 5.37 &  & 7.59 &  & 2.64 \\
 &  &  &  & 28800 &  & 6.85 &  & 9.74 &  & 2.52 \\
\cline{1-11}
\multirow{8}{*}{$\dot{Q}_{ts,src}$ (kW$_{th}$)} &  & \multirow{4}{*}{20230414} &  & 1800 &  & 14.12 &  & 16.85 &  & 13.42 \\
 &  &  &  & 3600 &  & 13.37 &  & 16.18 &  & 22.13 \\
 &  &  &  & 14400 &  & 13.71 &  & 15.79 &  & 8.15 \\
 &  &  &  & 28800 &  & 13.80 &  & 15.92 &  & 5.86 \\
\cline{3-11}
 &  & \multirow{4}{*}{20230418} &  & 1800 &  & 15.93 &  & 43.31 &  & 7.83 \\
 &  &  &  & 3600 &  & 18.35 &  & 47.19 &  & 4.52 \\
 &  &  &  & 14400 &  & 13.04 &  & 33.59 &  & 2.64 \\
 &  &  &  & 28800 &  & 12.85 &  & 30.99 &  & 2.52 \\
\cline{1-11}
\multirow{8}{*}{$\dot{Q}_{ts,dis}$ (kW$_{th}$)} &  & \multirow{4}{*}{20230414} &  & 1800 &  & 25.02 &  & 30.60 &  & 13.42 \\
 &  &  &  & 3600 &  & 23.59 &  & 30.07 &  & 22.13 \\
 &  &  &  & 14400 &  & 26.78 &  & 37.01 &  & 8.15 \\
 &  &  &  & 28800 &  & 25.81 &  & 36.43 &  & 5.86 \\
\cline{3-11}
 &  & \multirow{4}{*}{20230418} &  & 1800 &  & 13.02 &  & 45.83 &  & 7.83 \\
 &  &  &  & 3600 &  & 12.69 &  & 44.46 &  & 4.52 \\
 &  &  &  & 14400 &  & 19.82 &  & 78.38 &  & 2.64 \\
 &  &  &  & 28800 &  & 26.58 &  & 97.08 &  & 2.52 \\
\cline{1-11}
\multirow{8}{*}{Q$_{ts,h}$ (kWh$_{th}$)} &  & \multirow{4}{*}{20230414} &  & 1800 &  & 20.57 &  & 15.29 &  & 13.42 \\
 &  &  &  & 3600 &  & 17.02 &  & 12.65 &  & 22.13 \\
 &  &  &  & 14400 &  & 14.90 &  & 11.34 &  & 8.15 \\
 &  &  &  & 28800 &  & 19.23 &  & 15.01 &  & 5.86 \\
\cline{3-11}
 &  & \multirow{4}{*}{20230418} &  & 1800 &  & 8.73 &  & 6.23 &  & 7.83 \\
 &  &  &  & 3600 &  & 8.71 &  & 6.23 &  & 4.52 \\
 &  &  &  & 14400 &  & 16.92 &  & 11.64 &  & 2.64 \\
 &  &  &  & 28800 &  & 21.01 &  & 14.35 &  & 2.52 \\
\cline{1-11}
\multirow{8}{*}{Q$_{ts,c}$ (kWh$_{th}$)} &  & \multirow{4}{*}{20230414} &  & 1800 &  & 11.82 &  & 6.87 &  & 13.42 \\
 &  &  &  & 3600 &  & 14.58 &  & 8.17 &  & 22.13 \\
 &  &  &  & 14400 &  & 26.15 &  & 14.61 &  & 8.15 \\
 &  &  &  & 28800 &  & 22.15 &  & 12.37 &  & 5.86 \\
\cline{3-11}
 &  & \multirow{4}{*}{20230418} &  & 1800 &  & 9.30 &  & 4.64 &  & 7.83 \\
 &  &  &  & 3600 &  & 10.37 &  & 5.12 &  & 4.52 \\
 &  &  &  & 14400 &  & 21.75 &  & 10.76 &  & 2.64 \\
 &  &  &  & 28800 &  & 30.39 &  & 15.11 &  & 2.52 \\
\hline
\end{tabular}

}
\raggedright
\textcolor{darkgray}{\footnotesize\textit{s.u.} stands for \textit{same units} as the predicted variable}
\end{table}

% \begin{kaobox}[title=Extended validation results] 
%     \begin{tabular}{p{0.82\textwidth} c} 
%         A visualization of all evaluated tests for every model validation are available in the thesis
%     repository as a compressed folder  & \qrcode[hyperlink,height=0.5in]{\repositoryBaseUrl/figures/solarmed-models-validation-extented.7z}
%     \end{tabular}
% \end{kaobox}