\setchapterpreamble[u]{\margintoc}
\chapter{Hybrid modelling of a solar driven \gls{medLabel} system}
\labch{solarmed:modelling}


%===================================
%===================================
\section{Introduction}

The behavior of the \gls{solarmedLabel} process can be abstracted into two
components, a continuous and a discrete one. Each component is described and
validated in the respective \nrefsec{solarmed:modelling:dynamic} and
\nrefsec{solarmed:modelling:discrete} sections. Then, they are combined to
create a complete model of the \gls{solarmedLabel} process in
\nrefsec{solarmed:modelling:complete}.

% The continuous-dynamic behavior of the process is described by a
% set of differential equations, while the discrete behavior is described by
% \glspl{fsmLabel}. The \glspl{fsmLabel} are used to model the
% operation state of the system while the process variables describe the
% continuous-dynamic behavior.

% \begin{itemize}
%     \item Operation state. 
%     \item Process variables
% \end{itemize}

%===================================
%===================================
\section{Dynamic modelling. Process variables}[Dynamic modelling]
\labsec{solarmed:modelling:dynamic}

The dynamic behavior of the \gls{solarmedLabel} determines the values of the
process variables. This behavior is modelled by a set of models for each
component of the \gls{solarmedLabel} system. A discrete representation of the
system is used, where the process variables are sampled at a fixed sample time,
$T_s$, and systems are described by a set of difference equations. This is the
case for most cases, but for some, even though this modelling component
represents the dynamic behavior of the system, some models described in the
following sections are steady-state models. This can lead to discrepancies
between the model predictions and the actual system behavior, particularly
during transient events. However, this is not deemed a significant limitation
since the model is intended to be used for an optimization approach where the
model sample rate is in the order of minutes, and inputs for slower component
dynamics are changed sparingly, typically starting from 30 minutes and above,
more than enough time for the system to reach steady state.

%===================================
%===================================
\subsection{Solar field}

% Describe input and outputs
The solar field is basically a converter of electrical to thermal energy
dependent on the irradiance availability. The main outputs, in terms of
operation of the solar field, are the thermal power obtained
($Q_{sf}\,(kW_{th})$), at what temperature that heat is obtained ($T_{sf,out}$)
and the electricity needed to do so ($C_{e,sf}\,(kW_{e})$).

The diagram illustrates the individual loops that make up the field. In the
model, it is assumed that all loops have equal flow rates and temperatures
(\ie, a balanced flow distribution with similar collectors)\sidenote{Which is
the case in the experimental facility for loops 2 to 5, the ones used}. As a
result, the system can be simplified to a single loop with a collector area
equal to the sum of the collector areas of the individual rows of collector
loops.

% References
A first-principles based on the one presented in Ampuño et
al.~\sidecite{ampuno_modeling_2018} is used to model the solar field. The model
has two types of parameters: dynamic and constant. The dynamic parameters are
the thermal loss coefficient ($H\,\left( \frac{J}{s·^\circ C} \right)$) which
relates power losses to the environment and the collector conversion factor
($\beta\,(m)$) encompassing the collector transmisitivity and absorstance. It
determines the amount of irradiance that is transfered to the working fluid.
These two dynamic parameters are calibrated using experimental data. The
constant parameters are the ones defined in \reftab{tablaenapartadofacility}.

% Bloque de modelo con interfaz y ecuaciones
\begin{modelcounter}{Solar field}
    \begin{align*}
        T_{\text{out}}(k) &= \text{sf\:model}\left(
            T_{\text{out},k\!-\!1},\,
            \mathbf{T}_{\text{in},k\!-\!n:k},\,
            \boldsymbol{q}_{k\!-\!n:k},\,
            I_k,\,
            T_{\text{amb},k};\,
            \beta,\,
            H,\,
            \theta
        \right) \\
        & L_{pipe,eq} = \frac{T_s}{A_{pipe,eq}} \sum_{k=0}^{n} q_{sf}[k] \quad \text{Equivalent pipe length [m]} \\
        & L_{eq} = n_{c,s} \cdot L_t \quad \text{Eq. collector tube length [m]} \\
        & c_f = n_{c-loop} \cdot n_{tub-c} \quad \text{Conversion factor [-]}\\
        & K_{1} = \beta / (\rho \cdot c_p \cdot A_{cs}) \quad \text{Solar contribution [K·m$^2$/J]}\\
        & K_{2} = H / (L_{pipe,eq} \cdot A_{cs} \cdot \rho \cdot c_p) \quad \text{Environment losses [1/s]} \\
        & K_{3} = 1 / (L_{pipe,eq} \cdot A_{cs} \cdot c_f) \cdot (1/3600) \quad \text{Heat absorbed [h/(3600·m$^3$·s)]} \\
        & T_{\text{out}}(k) = T_{\text{out}}(k-1) + \Big( \\
        & \qquad + K_1 \cdot I \\
        & \qquad - K_2\cdot (\overline{T}-T_{amb}) \\ 
        & \qquad - K_3 \cdot \boldmath{q}_{k-n_d} \left( T_{\text{out},k-1} - T_{\text{in},k-n_{d}} \right) \\
        & \Big) \cdot T_s 
    \end{align*}

    \labmod{solarmed:sf}
\end{modelcounter}

% Diagrama del componente con variables
\begin{marginfigure}[-6.5cm]
    \includegraphics[]{figures/solarmed-diagrams-solar_field_compact.png}
    \caption{Solar field process diagram.}
    \labfig{solarmed:diagrams:solar_field}
\end{marginfigure}

The main difference with respect to the model presented
    in~\cite{ampuno_modeling_2018} is how the apparent transport delay is
    modelled~\sidecite{ampuno_apparent_2019}\sidenote{Transport delays are a
    common feature in dynamic systems, where the response of the system to an
    input is not instantaneous, but rather delayed by a certain amount of time.
    This delay can be caused by various factors, in this particular system, is
    due to the time it takes for the water to flow through the solar field and
    reach the temperature sensors. The apparent delay is the result of adding
    up the individual - different delays of each collector cell}. In this
    implementation, the transport delay is simplified to a single steady state parameter
    based on the work presented in Normey-Rico
    et.al~\sidecite{normey-rico_robust_1998} since delays vary less than 30\%
    from this nominal value.

    The number of delay samples depends on the model sample time, and a system
    parameter called the equivalent length. For this work, the following
    procedure was followed to estimate it:

    \begin{enumerate}
        \item Using a reference test with a fixed sample time, $T_s$, the
        number of delay samples ($n_{d}$) was manually fitted to the data, by
        visually inspecting the response of the system to a step change in the
        input flow.
        \item Estimate the equivalent length of the solar field by taking the
        average flow rate ($\overline{q}_{sf}$) across the delay samples
        span\sidenote{In reverse order, from newest to oldest}, and divide it
        by a fixed parameter: the solar field pipe equivalent cross-sectional
        area ($A_{pipe,eq}$). 
        \[ \overline{q}_{sf} = \sum_{k=-n_{d}}^{k=0}{q(k)/n_{d}} \\ \]
        \[ L_{pipe,eq} = \frac{\overline{q}_{sf} \times T_s \times n_{d}}{A_{pipe,eq}} \]
        \item With this equivalent length ($L_{pipe,eq}$), the number of delay samples can be
        estimated for any sample time $T_s$ and flows vector $\mathbf{q_{sf}}$ by
        iteratively adding the distance that flow travels at each sample time
        until the equivalent length is reached.
    \end{enumerate}

%================================
\subsubsection{Electrical consumption}

\begin{definition}
    \textbf{Step train test}. Variations in the \gls{vfdLabel} pump speed
    from a minimum to a maximum value, with fixed increments.
\end{definition}

% Explicar configuración/situación hidráulica del campo
The \textit{AQUASOL} solar field is composed by a set of pumps that recirculate
the water through the solar field. The pumps are controlled by \glspl{vfdLabel}
that allow to vary the flow rate through the solar field. A main recirculation
pump ($P_{l0}$) is responsible for the primary flow, while additional pumps
($P_{l1}$, $P_{l2}$, etc.) are used in the individual loops to either increase
the total flow rate or to operate with the isolated loop. This redundancy means
that the same flow rate can be achieved with different pump configurations.

% Mostrar ensayo para caracterizar el consumo eléctrico del campo solar
Then, prior to modelling the electrical consumption of the solar field, a prior
step is to characterize the electrical consumption of the system. This is done
by determining the relationship between flow rate and power consumption for
every configuration and to find the best configuration, that is, the one that
minimizes the electrical consumption across the range of flow rates that the
solar field operates at.

% Gráfica bonica para test 20240927 y 20240925 (unir ambos ensayos en el mismo
% dataframe y romper el eje x para evitar la discontinuidad grande entre días)
\begin{figure*}[h!]
    \includegraphics[]{figures/solar_field_eletricity_caractherization_tests.png}
    \savebox\captionqr{\qrcode[hyperlink,height=0.5in]{\repositoryBaseUrl/figures/sf_tests_viz.zip}}
    \caption{Solar field and thermal storage electrical
    characterization tests.\hspace{1ex}\usebox\captionqr}
    \labfig{solarmed:modelling:sf:electrical_consumption_tests}
\end{figure*}


% Describir metodología del ensayo

% Results
In order to characterize the electrical consumption of the solar field, a
series of tests were performed as can be seen in
\reffig{solarmed:modelling:sf:electrical_consumption_tests}. The tests were
carried out in two different dates since they have to be performed early,
before the solar field is irradiated by the sun and the field heats
up\sidenote{observe the trend in
\reffig{solarmed:modelling:sf:electrical_consumption_tests} -
\textit{Temperatures}}. In the first day, step trains are applied to the main
loop and individual isolated loops (20240925 07:15 - 08:30). On the second day,
different speeds levels were set for the main recirculation pump (10\% - 100\%,
10\% increments) while step trains were applied to the individual loops (40\% -
100\%, 20\% increments)\sidenote{In
\reffig{solarmed:modelling:sf:electrical_consumption_tests}, from 20240927
07:35 to 08:20}.

% Flow vs power consumption
\begin{figure}
    \includegraphics[width=\textwidth]{figures/sf_flow_vs_power_consumption.png}
    \savebox\captionqr{\qrcode[hyperlink,height=0.5in]{\repositoryBaseUrl/figures/sf_flow_vs_power_consumption.html}}
    \caption{Solar field flow for different pump configurations and their associated power consumption.\\[1ex] \usebox\captionqr}
    \labfig{solarmed:modelling:sf:flow_vs_power_consumption}
\end{figure}

% Describir selección de configuración para finalmente realizar calibración
\reffig{solarmed:modelling:sf:flow_vs_power_consumption} shows the relationship
between flow rate and power consumptions for different configuration and pump
speeds. Up to 90 l/min the best configuration is to just use the main
recirculation pump. Above this flow rate, the main pump is used in combination
with the individual loops. First a combination of main pump from 85 to 100\%
and individual loops at their 40\% minimum speed, then the main pump at 100\%
and individual loops at increasing values up to 100\%. With this selection, a
two degree order polynomial regression is fitted to the data, as shown in
\reffig{tobeadded3}.

Summarizing, the electrical consumption of the solar field is modelled as a
function of the flow rate through the solar field from a minimum value of XX
m$^3$/h to a maximum value of YY m$^3$/h. This is achieved as a result of the
combination of the main recirculation pump and the individual loops.

% [-8.47935428e-02, 2.29091363e-02, -8.72483421e-04, 1.29582523e-05]
\begin{modelcounter}{Solar field electrical consumption}
    % 0.04883341, -0.00695794, 0.01049481
    \begin{align*}
        C_{e,sf}\,[kW_e] &= \text{sf electrical consumption}(q_{sf}\,[m^3/h]) \\
        & C_{e,sf} = -8.48\cdot{10}^{-2} \cdot q_{sf}^3 + 2.29\cdot{10}^{-1} \cdot q_{sf}^2 + -8.72\cdot{10}^{-4} \cdot q_{sf} + 1.3\cdot{10}^{-5}
    \end{align*}
    \labmod{solarmed:sf:electrical_consumption}
\end{modelcounter}

% 

%================================
\subsubsection{Validation}

% Mostrar comparativa tanto de tiempos de muestreo, como el efecto de
% considerar el retraso por transporte
% Plot or table?

% Probably a table with the elapsed time, MAE and R2 for the validation days

% Plot with timeseries validation for one exemplary day
\begin{figure}
    \includegraphics[width=\textwidth]{figures/solarmed-modelling-solar_field_validation_20230630.png}
    \savebox\captionqr{\qrcode[hyperlink,height=0.5in]{\repositoryBaseUrl/figures/solarmed-modelling-solar_field_validation_20230630.html}}
    \caption[]{Solar field model validation for a particular test.\\[1ex] \usebox\captionqr}
    \labfig{::}
\end{figure}



%===================================
%===================================
\subsection{Thermal storage}

A first-principles model of a two-tank thermal storage system, developed to
capture the key thermodynamic and fluid dynamic phenomena governing energy
transfer and stratification. The system is based on the design principles
outlined by Duffie and Beckman\sidecite{duffie_energy_2013}, and consists of two
thermally insulated tanks: a hot tank and a cold tank, each serving distinct
roles in the thermal cycle. In normal operation, heat is extracted from the
bottom of the cold tank, and after being heated, it is injected into the top of
the hot tank. The load extracts heat from the top of the hot tank, and returns
it to the bottom of the cold tank, completing the cycle. The tanks are
connected from top of the cold tank to the bottom of the hot tank, allowing for
recirculation of the fluid between the two tanks.\todo{Esto debería
probablemente ir en la parte de la instalación}

The governing model equations and boundary conditions to simulate the transient
thermal behavior of the storage system, including mass and energy balances,
heat transfer mechanisms, and the stratification dynamics are shown in
\refmod{solarmed:ts}.

Similar to the solar field model, it has two parameters that need to be
calibrated using experimental data and are considered dynamic and constant
design parameters defined in \reftab{tablaenapartadofacility}. The dynamic
parameters are the thermal loss coefficient ($H_{i}\,\left(
\frac{J}{s·^\circ C} \right)$) which relates heat losses to the environment
and the volume of each of the considered control volumes ($V_i$). Three types of
volumes are defined: the inner volume, the top volume and the bottom volume:
\begin{itemize}
    \item Top volume ($V_{T}$): can receive external heat, and
    have heat extracted from it. It interacts with the
    inner volume that it interfaces with.
    \item Bottom volume ($V_{B}$): can also have external interactions, and
    exchanges with the inner volume above it.
    \item Inner volume ($V_{i}$): is any volume that is not the top or bottom,
    that is, is surrounded by other volumes with which it exchanges heat and
    mass by inner recirculation.
\end{itemize}

\begin{modelcounter}{Thermal storage}
    \begin{align*}
        \mathbf{T}_\text{h}(k),\; \mathbf{T}_\text{c}(k) &= \text{thermal storage model}\Big(
            \mathbf{T}_\text{h}(k{-}1),\;
            \mathbf{T}_\text{c}(k{-}1),\;
            T_\text{src}(k), \\
            &\quad T_\text{dis}(k),\;
            \dot{m}_\text{src}(k),\;
            \dot{m}_\text{dis}(k),\;
            T_\text{amb}(k);\;
            \boldsymbol{\theta}_\text{h};\;
            \boldsymbol{\theta}_\text{c}
        \Big)
    \end{align*}

    \vspace{0.5em}

    \textbf{if } $\dot{m}_\text{dis}(k) > \dot{m}_\text{src}(k)$: \hfill \textit{(cold to hot recirculation)}
    \begin{align*}
        \mathbf{T}_\text{c}(k) &= \text{single tank model}\Big(
            \mathbf{T}_\text{c}(k{-}1),\;
            T_\text{T}{=}0,\;
            T_\text{B}{=}T_\text{dis}(k),\;
            T_\text{amb}(k), \\
            &\quad \dot{m}_\text{in,T}{=}0,\;
            \dot{m}_\text{in,B}{=}\dot{m}_\text{dis}(k),\;
            \dot{m}_\text{out,T}{=}\dot{m}_\text{dis}(k) - \dot{m}_\text{src}(k),\;
            \dot{m}_\text{out,B}{=}\dot{m}_\text{src}(k);\;
            \boldsymbol{\theta}_\text{c}
        \Big) \\[0.5em]
        \mathbf{T}_\text{h}(k) &= \text{single tank model}\Big(
            \mathbf{T}_\text{h}(k{-}1),\;
            T_\text{T}{=}T_\text{src}(k),\;
            T_\text{B}{=}T_\text{c}^\text{out}(k),\;
            T_\text{amb}(k), \\
            &\quad \dot{m}_\text{in,T}{=}\dot{m}_\text{src}(k),\;
            \dot{m}_\text{in,B}{=}\dot{m}_\text{dis}(k) - \dot{m}_\text{src}(k),\;
            \dot{m}_\text{out,T}{=}\dot{m}_\text{dis}(k),\;
            \dot{m}_\text{out,B}{=}0;\;
            \boldsymbol{\theta}_\text{h}
        \Big)
    \end{align*}

    \vspace{0.5em}

    \textbf{else:} \hfill \textit{(hot to cold recirculation)}
    \begin{align*}
        \mathbf{T}_\text{h}(k) &= \text{single tank model}\Big(
            \mathbf{T}_\text{h}(k{-}1),\;
            T_\text{T}{=}T_\text{src}(k),\;
            T_\text{B}{=}0,\;
            T_\text{amb}(k), \\
            &\quad \dot{m}_\text{in,T}{=}\dot{m}_\text{src}(k),\;
            \dot{m}_\text{in,B}{=}0,\;
            \dot{m}_\text{out,T}{=}\dot{m}_\text{dis}(k),\;
            \dot{m}_\text{out,B}{=}\dot{m}_\text{src}(k) - \dot{m}_\text{dis}(k);\;
            \boldsymbol{\theta}_\text{h}
        \Big) \\[0.5em]
        \mathbf{T}_\text{c}(k) &= \text{single tank model}\Big(
            \mathbf{T}_\text{c}(k{-}1),\;
            T_\text{T}{=}T_\text{h}^\text{out}(k),\;
            T_\text{B}{=}T_\text{dis}(k),\;
            T_\text{amb}(k), \\
            &\quad \dot{m}_\text{in,T}{=}\dot{m}_\text{src}(k) - \dot{m}_\text{dis}(k),\;
            \dot{m}_\text{in,B}{=}\dot{m}_\text{dis}(k),\;
            \dot{m}_\text{out,T}{=}0,\;
            \dot{m}_\text{out,B}{=}\dot{m}_\text{src}(k);\;
            \boldsymbol{\theta}_\text{c}
        \Big)
    \end{align*}

    \textbf{where:} \\
        \begin{align*}
        \mathbf{T}(k) &= \text{single tank model}\Big(
            \mathbf{T}(k{-}1),\;
            T_\text{T,in}(k),\;
            T_\text{B,in}(k),\;
            \dot{m}_\text{in,T}(k),\;
            \dot{m}_\text{in,B}(k), \\
            &\quad \dot{m}_\text{out,T}(k),\;
            \dot{m}_\text{out,B}(k),\;
            T_\text{amb}(k);\;
            \boldsymbol{\theta};\;
        \Big)
    \end{align*}

    \vspace{0.5em}

    \begin{itemize}
    \item Top volume
    \[
    \begin{aligned}
    & -\rho \cdot V_T \cdot c_p \cdot \frac{T_{T,k} - T_{T,k-1}}{T_s}
    + \dot{m}_{src} \cdot T_{T,in} \cdot c_p
    - \dot{m}_{dis} \cdot T_{T,k} \cdot c_p \\
    & - \dot{m}_{src} \cdot T_{T,k} \cdot c_p
    + \dot{m}_{dis} \cdot T_{1,k} \cdot c_p
    - UA_T \cdot (T_{T,k} - T_{amb})
    = 0
    \end{aligned}
    \]

    \item Bottom volume
    \[
    \begin{aligned}
    & -\rho \cdot V_B \cdot c_p \cdot \frac{T_{B,k} - T_{B,k-1}}{T_s}
    + \dot{m}_{src} \cdot T_{i-1,k} \cdot c_p
    + \dot{m}_{dis} \cdot T_{B,in} \cdot c_p \\
    & - \dot{m}_{src} \cdot T_{B,k} \cdot c_p
    - \dot{m}_{dis} \cdot T_{B,k} \cdot c_p
    - UA_N \cdot (T_{B,k} - T_{amb})
    = 0
    \end{aligned}
    \]

    \item Inner volume
    \[
    \begin{aligned}
    & -\rho \cdot V_i \cdot c_p \cdot \frac{T_{i,k} - T_{i,k-1}}{T_s}
    + \dot{m}_{src} \cdot T_{i-1,k} \cdot c_p
    - \dot{m}_{dis} \cdot T_{i,k} \cdot c_p \\
    & - \dot{m}_{src} \cdot T_{i,k} \cdot c_p
    + \dot{m}_{dis} \cdot T_{i+1,k} \cdot c_p
    - UA_i \cdot (T_{i,k} - T_{amb})
    = 0
    \end{aligned}
    \]
    \end{itemize}

    \labmod{solarmed:ts}
\end{modelcounter}

% Diagrama del componente con variables
\begin{marginfigure}[-7.5cm]
    \includegraphics[]{figures/solarmed-diagrams-thermal_storage.png}
    \caption{Thermal storage process diagram.}
    \labfig{solarmed:diagrams:thermal_storage}
\end{marginfigure}



Three temperature sensors are available in the experimental facility, so three
volume divisions are used to model the thermal storage. The model is based on


%================================
\subsubsection{Electrical consumption}

% Mostrar ensayo, si encaja incluirlo en la misma gráfica que el campo solar
% simplemente referenciar esa figura

The first step train given in
\reffig{solarmed:modelling:sf:electrical_consumption_tests} - 20250925
at 06:50 - 07:15 is used to characterize the electrical consumption of
recirculating water ($q_{ts,src}$) in the thermal storage circuit. The
electrical consumption is modelled as a function of the flow rate through the
thermal storage from a minimum value of XX m$^3$/h - XX kW$_e$ to a maximum
value of YY m$^3$/h - YY kW$_e$.


\begin{modelcounter}{Thermal storage electrical consumption}
    % 0.04883341, -0.00695794, 0.01049481
    \begin{align*}
        C_{e,ts}\,[kW_e] &= \text{ts electrical consumption}(q_{ts,src}\,[m^3/h]) \\
        & C_{e,ts} = 4.88\cdot{10}^{-1} \cdot q_{ts,src}^2 + -6.95\cdot{10}^{-3} \cdot q_{ts,src} + 0.01
    \end{align*}
    \labmod{solarmed:ts:electrical_consumption}
\end{modelcounter}

%================================
\subsubsection{Validation}

\begin{figure}
    \includegraphics[width=\textwidth]{figures/solarmed-modelling-thermal_storage_validation_20230505.png}
    \savebox\captionqr{\qrcode[hyperlink,height=0.5in]{\repositoryBaseUrl/figures/solarmed-modelling-thermal_storage_validation_20230505.html}}
    \caption[]{Thermal storage model validation for a particular test.\\[1ex] \usebox\captionqr}
    \labfig{::}
\end{figure}

%===================================
%===================================
\subsection{Heat exchanger}

The solar field and thermal storage are interfaced by a \gls{hexLabel},
particularly a counter-flow heat exchanger. The component is modelled using a
first-principles steady state model based on the effectiveness-NTU
method\sidecite{cengel_heat_2015,kays_compact_1958}.

Modelling considerations \cite{cengel_heat_2015}: 
\begin{itemize}
    \item It has been assumed that the rate of change for the temperature of
    both fluids is proportional to the temperature difference; this assumption
    is valid for fluids with a constant specific heat, which is a good
    description of fluids changing temperature over a relatively small range.
    However, if the specific heat changes, the \gls{lmtdLabel} approach will no
    longer be accurate. 
    \item It has also been assumed that the heat transfer coefficient (U) is
    constant, and not a function of temperature.
    \item No phase change during heat transfer.
    \item Changes in kinetic energy and potential energy are neglected.
\end{itemize}


% Bloque de modelo con interfaz y ecuaciones
\begin{modelcounter}{Heat exchanger}
    \begin{align*}
        T_{hx,p,out},\,T_{hx,s,out}& = \text{hx\:model}(T_{hx,p,in},T_{hx,s,in},\dot{m}_p,\dot{m}_{s}, T_{amb}, (UA)_{hx}, H) \\
        & C_{hx,p} = \dot{m}_{hx,p} \cdot c_{p,Tp,in} \\
        & C_{hx,s} = \dot{m}_{hx,s} \cdot c_{p,Ts,in} \\
        & C_{min} = min(C_{hx,p}, C_{hx,s}) \\
        & C_{max} = max(C_{hx,p}, C_{hx,s}) \\
        & C=\frac{C_{min}}{C_{max}} \\
        & NTU = UA / C_{min} \\
        & \epsilon = \frac{1 - e^{ (-NTU \cdot (1 - C))}}{1 - C \cdot e^{-NTU \cdot (1 - C)}} \\
        & T_{hx,p,out} = T_{hx,p,in} - (\dot{Q}_{max} * \epsilon) / (C_{hx,p}) \\
        & T_{hx,s,out} = T_{hx,s,in} + (\dot{Q}_{max} * \epsilon) / (C_{hx,s}) \\
    \end{align*}

    \labmod{solarmed:hex}
\end{modelcounter}

% Diagrama del componente con variables
\begin{marginfigure}[-5.5cm]
    \includegraphics[]{figures/solarmed-diagrams-heat_exchanger.png}
    \caption{Heat exchanger process diagram.}
    \labfig{solarmed:diagrams:heat_exchanger}
\end{marginfigure}

Where $p$ references the primary circuit (solar field side) and $s$ the
secondary circuit (thermal storage side). As shown in the
\refmod{solarmed:hex}, first the heat capacity $C$ is determined in order to
calculate the effectiveness ($\epsilon$) of the heat exchanger. Finally, after
determining the maximum heat transfer rate ($\dot{Q}_{max}$), the outlet temperatures
can be obtained.



%================================
\subsubsection{Validation}

In order to calibrate the two parameters of this model ($UA$ and $H$), one
experimental test used where the parameters are varied in order to minimize the
error between the model and the experimental data. The obtained values are
shown in \reftab{} and the dynamic behavior of the model is shown in \reffig{}.
It can be seen than the model performs fairly well even in transient conditions,
with a mean absolute error of XX\% and a coefficient of determination $R^2$ of
YY\%. 

Several more tests are evaluated and the performance obtained is shown in
\reftab{}. On average, the model has a mean absolute error of XX\% and a coefficient
of determination $R^2$ of YY\%. The model is able to predict the outlet
temperatures of the heat exchanger with a good accuracy, even in transient
conditions, which is a good indication of the model's reliability.

\begin{figure}
    \includegraphics[width=\textwidth]{figures/solarmed-modelling-heat_exchanger_validation_20231106.png}
    \savebox\captionqr{\qrcode[hyperlink,height=0.5in]{\repositoryBaseUrl/figures/solarmed-modelling-heat_exchanger_validation_20231106.html}}
    \caption[]{Heat exchanger model validation for a particular test.\\[1ex] \usebox\captionqr}
    \labfig{::}
\end{figure}

%===================================
%===================================
\subsection{\gls{medLabel}}

The \gls{medLabel} is modelled statically, that is, considering that changes in
the system operating conditions happen at a slow enough rate that the dynamic
behavior between stable states can be neglected, and thus, only those stable
states are considered. The model is a data driven one, specifically a
\gls{gprLabel} model calibrated using data from an experimental campaign in the
pilot plant\sidenote{Referencia a donde se mencione o algún artículo de
Patricia}.



%================================
\subsubsection{Electrical consumption}


%================================
\subsubsection{Validation}


%===================================
%===================================
\section{Discrete modelling. Operation state}[Discrete modelling]
\labsec{solarmed:modelling:discrete}

The second modelling component defines the discrete state of the system, that
is, its \textit{operation state}. This component is modelled by means of
\glspl{fsmLabel}.

\reminder{\glspl[format=long]{fsmLabel}}{
    A finite state machine is a model of behavior composed of a finite number of
\textit{states} and \textit{transitions} between those states. Within each
state and transition some \textit{action} can be performed\footnote{See
    \nrefsec{intro:modelling:fsm} for a more detailed description.}.
}

The complete system is divided into two subsystems: the heat generation and
storage subsystem and the separation subsystem.

%================================
\subsection{Heat generation and storage subsystem (\texttt{sfts})}
\labsec{solarmed:modelling:sfts_fsm}

This subsystem encompasses the \fullgls{sfLabel} and the \fullgls{tsLabel}. The
subsystem can be modelled with a simple \gls{fsmLabel} as shown in \reffig{},
where the states are defined based on whether water is being recirculated in
each circuit. Four states are defined as shown in \reftab{solarmed:modelling:sfts_fsm_states}.

\begin{margintable}[*-3]
\caption{\gls{sftsLabel} \gls{fsmLabel} states definitions. $\land$ represents
the logical \texttt{AND} operator and $\forall$ represents that all meet the condition.}
\labtab{solarmed:modelling:sfts_fsm_states}
\resizebox{\linewidth}{!}{%
\begin{tabular}{cll}
    \toprule
    \textbf{State} & \textbf{Name} & \textbf{Condition}  \\
    \midrule
    0 (00) & Off & $q_{sf}\land q_{ts,src}==0$ \\
    1 (01) & Warming up \gls{sfLabel} & $q_{sf} > 0 \land q_{ts,src}==0$ \\
    2 (10) & Recirculating \gls{tsLabel} & $q_{sf} ==0 \land q_{ts,src} > 0$ \\
    3 (11) & \gls{sfLabel} heating up \gls{tsLabel} & $q_{sf}\land q_{ts,src} > 0$
    \\
    \bottomrule
\end{tabular}
}
\end{margintable}

% Figura de la máquina de estado finito y gráfica con la evolución de estados
% durante un ensayo

%================================
\subsection{Separation subsystem (\texttt{med})}
\labsec{solarmed:modelling:med_fsm}

% Figura de la máquina de estado finito y gráfica con la evolución de estados
% durante un ensayo

\begin{margintable}[*-3]
\caption{\gls{medfsmLabel} \gls{fsmLabel} states definitions. $\land$ represents
the logical \texttt{AND} operator, $\exists$ represents that at least one meets
the condition, and $\forall$ represents that all meet the condition.}
\labtab{solarmed:modelling:med_fsm_states}
\resizebox{\linewidth}{!}{%
\begin{tabular}{cll}
    \toprule
    \textbf{State} & \textbf{Name} & \textbf{Condition}  \\
    \midrule
    0 & Off & $\forall q == 0$ \\
    1 & Generating vacuum & $\text{med}_{vac} == 2$ \\
    2 & Idle & $\forall q == 0 \land \text{med}_{vac} == 1$ \\
    3 & Starting-up & $\forall q > \underline{q} \land \text{med}_{vac} \ge 1 \land \forall T >
    \underline{T} $ \\
    4 & Shutting down & $\exists$ q < \underline{q} \\
    5 & Active & $\forall q > \underline{q} \land \text{med}_{vac} \ge 1 \land \forall T >
    \underline{T} $\\
    \bottomrule
\end{tabular}
}
\end{margintable}

\section{Complete system model}
\labsec{solarmed:modelling:complete}

Aquí describir cómo se combinan los componentes en función del estado del
sistema y cómo ello depende de las máquinas de estado finito.

To refer to the operational state of the system, a three digit number is used,
where the first two digits represent the \gls{sftsLabel} state and the last one
the \gls{medfsmLabel} state. For example, the state \texttt{005} represents an
inactive \gls{sftsLabel} subsystem with an active \gls{medfsmLabel}.
\texttt{101} represents a warming-up solar field while vacuum is being
generated in the \gls{medLabel} system.

\blindtext\blindtext

\begin{marginfigure}[-3cm]
    \includegraphics[]{figures/solarmed-modelling-complete_model.png}
    \caption{Complete \gls{solarmedLabel} model architecture. TODO: Needs to be
    updated}
    \labfig{solarmed:modelling:complete_model}
\end{marginfigure}


\subsection{Validation}

% Gráfica tocha mostrado algún día y después una tabla para varios días