\setchapterpreamble[u]{\margintoc}
\chapter{Optimization of a combined cooling system}
\labch{cc:optimization}

\tldrbox{ This chapter describes optimization problems for a combined cooling
    system as well as different optimization strategies propositions to solve
    them. The objective is to minimize the daily cost of operation made up by
    the electricity and water costs, while ensuring the cooling demand is met.
    The key challenge is to manage the limited water resource. From the studied
    alternatives, this can only be effectively achieved by the proposed
    two-stage horizon optimization strategy. }

\section*{Introduction}
% Recodatorio: No hace falta introducir mucho el contexto porque eso es algo
% que ya se habrá hecho en la parte introductoria

% Estado del arte + Contribución del capítulo + Estructura del capítulo

% Estado del arte en optimización en sistemas de refrigeración
Over the years, various studies have compared wet and dry cooling systems for
\gls{cspLabel} plants. Most of these works are limited to studying the effect of
some operating parameters via a sensitivity analysis
\sidecite{asfand_thermodynamic_2020,mdallal_modelling_2024,hu_thermodynamic_2018,tang_study_2013,asvapoositkul_comparative_2014,barigozzi_performance_2014}.
Nonetheless, several have focused on improving cooling system performance
through optimization of the individual component operation. Among them, the
works from Martín~\etal stand out. In \sidecite{martin_optimal_2013} they were
the first to optimize the year-round operation of a \gls{cspLabel} system not
only considering the cooling side (a \gls{wctLabel}) but also integrating the
power block. The problem was formulated as a multiperiod \gls{nlpLabel} problem
with the cooling system air flow rate and outlet temperature as decision
variables. They showed that the obtained complex problem can feasibly be solved
and an average water consumption of 2.1 l/kWh was obtained with the least
efficient month amounting to 2.5 l/kWh. In \sidecite{martin_optimal_2015} the
same strategy was applied this time for a dry cooling alternative
(\gls{accLabel}) and formulating the optimization as a multiperiod
\gls{minlpLabel} problem. This integer extension to the problem was done to
account for the addition of a new decision variable: the discrete number of
units and fans that make up the \gls{accLabel} \ie their active state. The
problem was solved via relaxation of the integer variables and after evaluating
the annual operation they found that the optimized dry cooler consumed around
5~\% of the total generated power compared to 3.44~\% of the wet alternative,
and increasing a cent the \gls{lcoeLabel} (0.16 \textit{vs.} 0.15~\euro/kWh,
respectively). A limitation of both studies is the use of monthly average
values, which masks the significant daily temperature variations ---often
exceeding 10$^\circ$C--- that coincide with peak power production and can have a
substantial impact on cooling system performance.

Little discussion can be found in the literature regarding the operation
strategy of combined cooling systems. For water-enhanced dry cooling and
parallel configurations, the proposed operation
strategy~\sidecite{wiles_description_1978,zaloudek_study_1976,rohani_optimization_2021}
consists on always prioritizing the dry sections up until a set value in the
condenser pressure is reached, in which case the wet units are activated. This
strategy offers a simple and robust solution but leaves a lot of performance on
the table.

% Two distinct configurations can be found in the literature where a discussion is
% made about its operation strategy: water-enhanced dry cooling and parallel configuration.
% Rohani\todo{Realmente lo que proponen no es nada nuevo, exactamente esa
% operación ya se propuso por ejemplo en wiles\_description\_1978 y zaloudek\_study\_1976}
%~\etal \sidecite{rohani_optimization_2021} and Golkar et al
% \cite{golkar_determination_2019}. In the latter, Rohani~\etal implement a
% thorough model of water streams in a \gls{cspLabel} plant that was
% experimentally validated. Different scenarios and cooling alternatives were
% analyzed and each of them was simulated for a year of operation. In the hybrid
% configuration  Water will be left unused despite
% potentially being available to prioritize the more expensive dry cooler
% operation. While Golkar et al \cite{golkar_determination_2019} delved more in
% the design and sizing of the hybrid cooler by application of a genetic
% algorithm, it then applied a very similar operation strategy.

In Maulbetsch~\etal~\cite{maulbetsch_economic_2012}, a parallel combined system
is analyzed, where the operation strategy is set as follows: At some
temperature, the condensing pressure achieved will raise above a desired limit.
For ambient temperatures above that level, both systems are operated at full
design fan power. When the condensing pressure is below that limit, the capacity
of the wet section is reduced to maintain it while the dry section is operated
at full capacity. At lower temperature where the dry section can maintain the
condensing pressure by itself, the wet system is no longer operated. Finally, at
even lower temperatures, the fan power is gradually reduced on the dry section.

% Limitaciones en literatura
One inherent limitation that no optimization strategy can fully overcome is the
seasonal mismatch between ambient temperature and water availability. In many
locations, ambient temperatures are lowest ---favoring dry cooling--- during
timesq of the year when water is most abundant ---favoring wet cooling. The
opposite occurs during hot, dry summer periods, when cooling demand is highest
but water becomes a scarce resource. Many studies report annual water savings
figures, but this does not offer a complete picture and can be misleading, as it
may mask poor performance during critical periods. Reducing water use during
times of abundance, while failing to achieve significant savings during
water-scarce periods, does not represent an optimal solution ---even if total
annual water consumption appears lower.

\begin{kaobox}[title=The Increasing Variability of Rainfall] 
    Water scarcity, stress, and climate change are typically portrayed through a
    lens of averages and trends. But this is seldom an adequate representation
    of water availability throughout much of the world, where deviations from
    trends are widespread and are growing more frequent, as witnessed by the
    increased frequency of floods and droughts. Adapting to rainfall variability
    is often much more challenging than accommodating long-term trends because
    of the unpredictable duration of a deviation, its uncertain magnitude, and
    its unknown frequency. With climate change, deviations from trends are
    projected to become more pronounced and more frequent. Inter-annual
    variability in particular is expected to pose a large threat in some of the
    world's driest regions.\\

    \textcolor{darkgray}{\footnotesize\textbf{Source}: Uncharted Waters: The New
    Economics of Water Scarcity and Variability~\cite{damania_uncharted_2017}}
\end{kaobox}

Significant cost savings can be achieved with increasing water availability,
either from  the specification of a smaller condenser or by lowering operating
turbine exhaust pressures (increasing the wet ratio).
In conclusion, there remains significant potential for improved water management
through optimized system operation, particularly when resource availability is
explicitly considered in the decision-making process:

\begin{itemize}
    \item Humidity is higher at night where ambient temperatures are lower,
    partially alleviating the limitations of the dry system and making it less
    unfavorable. 
    \item Take full advantage of the cheaper and more efficient wet cooling when
    water is plentiful.
    \item Consider the availability of alternative water sources and their
    dynamic costs.
    \item When using a combined cooling system its operation is not trivial but
    inherently becomes more complex; thus requiring an operation strategy to, at
    a minimum, robustly satisfy the cooling demand, but preferably also minimize
    the cost of operation.
\end{itemize}

\begin{marginfigure}[*-10] % +5.5cm
    \includegraphics[]{cc-optimization-environment.png}
    \caption{Block diagram of the optimization scheme including environment components}
    \labfig{cc:optimization:environment}
\end{marginfigure}

% Contribución de este capítulo
This chapter analyzes the optimization of different cooling system
configurations, focusing on their two primary resource consumptions: electricity
and water. The optimization problems are formulated to minimize the total cost
of cooling a thermal load, with cost defined as the combined use of these two
resources. The thermal load is treated as an external requirement and is therefore
excluded from the decision space. This work addresses existing limitations in
the literature and presents, for the first time, an actual optimization of the
operation of a combined cooling system in the context of \gls{cspLabel}
applications.\sidenote{Although the proposed methodology is applicable to any
system requiring thermal load cooling, particular emphasis is placed on water
resource availability, given its critical importance in solar thermal
applications.\\See \nrefsec{intro:csp:cooling}}

% Estructura de este capítulo
This chapter is organized as follows: \nrefsec{cc:optimization:environment}
introduces the context by describing the key variables involved, including cost
factors, weather forecasts, thermal load, and water resource availability.
\nrefsec{cc:optimization:static} presents the first optimization approach,
focusing on static problems for the dry cooler, wet cooler, and their combined
configurations. \nrefsec{cc:optimization:horizon} introduces a shrinking
horizon optimization strategy applied to the combined cooler, optimizing its
performance over a prediction horizon. This final section also explores the
nature of the problem and outlines the proposed methodology for its resolution.


%===================================
%===================================
\section{Environment description}
\labsec{cc:optimization:environment}

The environment for the optimization problems includes the following components
(See \reffig{cc:optimization:environment}):

\begin{enumerate}
    \item \textbf{Costs context} There are mainly two operational costs
    associated with the cooling system: electricity ($J_e$) and water
    ($J_w$). In a system that generates electricity, such as a \gls{cspLabel}
    plant, this is electricity that could otherwise be sold to the market, so
    the associated price ($P_e$) is the pool market price. 
    
    As for the water, two sources are considered: the water price from source 1
    is denoted as $P_{w,s1}$, and from source 2 as $P_{w,s2}$, being
    $P_{w,s1}<P_{w,s2}$.

    \item \textbf{Weather conditions} The weather variables that have an impact
    on the cooling system are the ambient temperature ($T_{amb}$) and the
    relative humidity ($HR$) since they set the dry and wet bulb temperatures.

    \item \textbf{Thermal load} The thermal load is defined either by a vapor
    flow rate ($\dot{m}_v$) or a thermal power ($\dot{Q}$), which enters the
    condenser at a temperature $T_v$\sidenote{Vapor can also be referred as
    steam, usually steam is used when the vapor performs work, like in a
    turbine.}.
    
    \item \textbf{Water resource availability} Two sources of water are assumed
    to be available, one of them, the cheaper one coming from a dam/reservoir is
    limited in volume ($V_{avail}$). This cheaper source ($s_1$) is prioritized
    until it is depleted, when the more expensive alternative source ($s_2$) is
    used:

    \begin{align}
        & \quad C_{w,s1,i} = \frac{\min(V_{avail,i}, C_{w,i} \cdot T_s)}{T_s},
        \labeq{cc:optimization:wa1} \\
        & \quad C_{w,s2,i} = C_{w,i} - C_{w,s1,i},
        \labeq{cc:optimization:wa2} \\
        & \quad V_{avail,i} = V_{avail,i-1} - C_{w,s1,i} \cdot T_s,
        \labeq{cc:optimization:wa3}
    \end{align}

    where $i$ represents the step. At every step the amount used from each
    source is estimated and the source 1 availability is updated accordingly.
    $C_w$ represents the flow rate of water consumed and $T_s$ is the
    sample time at which steps are computed.
\end{enumerate}

% %================================
% \subsection{Water resource availability}
% \labsec{cc:optimization:environment-water}


\section{Static optimization}
\labsec{cc:optimization:static}

Static optimization problems are defined in a particular instant, given an
environment. Decisions are made without considering past states or actions, nor
their impact on future states.

From a process perspective this also characterizes the cooling process, except for
the water resource availability, being the only variable that depends on the
previous state, \ie is not static. Each time a static problem is evaluated, it
begins with a specific initial water volume ($V_{{avail},0}$) for that
step. After solving the problem, this volume must be updated before proceeding
to the next step. As a result, evaluating multiple consecutive steps requires a
sequential approach.

\reminder{Optimization problem definition}{
    The general optimization function is defined as:\footnote{See \nrefsec{intro:optimization}}
    \begin{equation*}
    \min_{\mathbf{x},\, \mathbf{e};\, \pmb{\theta}} \quad J = f(\mathbf{x}, \mathbf{e}; \pmb{\theta}) 
        \quad \text{s.t.} \quad g_i(\mathbf{x}) \leq 0, \quad i = 1, \ldots, m
    \end{equation*}

    where \(x\) is the decision vector, \(e\) represents the environment, and \(\theta\) contains the fixed parameters.
}

In order to streamline the problem formulation, a general combined cooling
system model is used for every scenario. This unified model incorporates both
the dry and wet coolers, as well as the shared surface condenser. For cases
where only one cooler is used, the other can be effectively disabled by setting
its associated variables to zero and configuring the hydraulic circuit to
prevent water circulation through it.

%================================
\subsection{Dry cooler}
\labsec{cc:optimization:static:dc}

In the first case study, the optimization focuses exclusively on the dry cooler.
Consequently, all variables and terms associated with the wet cooler, as well as
water resource management, are omitted from the formulation, making the problem
completely static\sidenote{Achieved by setting $R_p=0$ and $R_s$=0. See
\nrefsec{cc:modelling:complete-model} for reference}. This configuration is
illustrated in \reffig{cc:optimization:diagram-dc} and the problem is defined as
follows:


\marginnote[*5]{See \nrefsec{cc:modelling:dc} for a detailed description of the
dry cooler and \nrefsec{cc:modelling:condenser} for the condenser model.}

\begin{problemcounter}{\gls{dcLabel} - static}
    \begin{equation*}
        \min_{\mathbf{x},\, \mathbf{e};\, \pmb{\theta}} \quad J = f(\mathbf{x}, \mathbf{e}; \pmb{\theta}) = C_e \cdot P_e
    \end{equation*}

    \textbf{with}:
    \begin{align*}
        T_{dc,out},\,C_e,\,T_{c,in},\,T_{c,out} &= \text{ccs model}(q_c,\,\omega_{{dc}},\, T_{{amb}},\, T_v,\, \dot{m}_v)
    \end{align*}
    \begin{itemize}
        \item Decision variables
        \[
        x = [q_c,\, \omega_{{dc}}]
        \]
        \item Environment variables
        \[
        e = [T_{{amb}},\, P_e,\, T_v,\, \dot{m}_v]
        \]
        \item Fixed parameters
        \[
        \theta = [R_p = 0,\, R_s = 0,\, \omega_{{wct}} = 0]
        \]

    \end{itemize}

    \textbf{subject to}:
    \begin{itemize}
        
        \item Box-bounds
        \begin{itemize}
                \item $\omega_{dc} \in [\underline{\omega_{dc}}, \overline{\omega_{dc}}]$
                % \item $\omega_{wct} \in [\underline{\omega}_{wct}, \overline{w}_{wct}]$
                \item $q_{c} \in [\underline{q_{c}}, \overline{q_{c}}]$
                % \item $R_p \in [0,1]$
                % \item $R_s \in [0,1]$
        \end{itemize}

        \item Constraints
        \begin{itemize}
            \item $\left| T_{{dc,out}} - T_{{c,in}} \right| \leq \epsilon_1$
            \item $\left| Q_{{dc}}- Q_{{c,released}} \right| \leq \epsilon_2$
            \item $T_{{c,out}} \leq T_v - \Delta T_{{c-v,min}}$
        \end{itemize}

    \end{itemize}
    \labprob{cc:dc}
\end{problemcounter}

\begin{marginfigure}[*-20]
    \includegraphics[]{wascop-optimization-dc-diagram.png}
    \caption{Diagram of the dry cooler only cooling problem}
    \labfig{cc:optimization:diagram-dc}
\end{marginfigure}

The cost of cooling ($J$) is equivalent to the cost of electricity ($J_e$),
which in turn is the product of the electricity price ($P_e$) and the
electricity consumption ($C_e$). Only two decision variables are defined, the
cooling water recirculation flow rate ($q_c$) and the dry cooler fan speed
($\omega_{dc}$). Any two pair of values for these variables that satisfy the
bounds do not necessary yield a feasible solution. This is why three constrains
are introduced, the first one ensures that the outlet cooler temperature matches
the inlet condenser temperature (since they are directly connected), the second
one ensures that the cooling duty of the dry cooler matches the one of the
condenser while the last one ensures that the condenser outlet temperature
respects the minimum temperature difference with the vapor
temperature\sidenote[][*-4]{In order to better comprehend why mismatches
between cooler and condenser can exist, the reader is referred to
\nrefsec{cc:modelling:complete-model}}.


%================================
\subsection{Wet cooler}
\labsec{cc:optimization:static:wct}

Conversely to the dry cooler, the wet cooler optimization problem is configured
by setting $R_p=1$, effectively disabling the dry cooler. In this case, water
associated variables are included in the problem formulation\sidenote{See
\nrefsec{cc:modelling:wct} for a detailed description of the wet cooler and
condenser model.}:

\begin{problemcounter}{\gls{wctLabel} -- static}
    \begin{equation*}
        \min_{\mathbf{x},\, \mathbf{e};\, \pmb{\theta}} \quad J = f(\mathbf{x}, \mathbf{e}; \pmb{\theta}) = J_e + J_w
    \end{equation*}
    
    \textbf{with}:
    \begin{align*}
        J_e &= C_e \cdot P_e \\
        J_{w} &= C_{w,s1} \cdot P_{w,s1} + C_{w,s2} \cdot P_{w,s2} \\
        C_{w,s1} &= \min \left(V_{avail}/T_s, C_{w} \right) \\
        C_{w,s2} &= C_w-C_{w,s1} \\
        T_{wct,out},\,C_e,\,C_w,\,T_{c,in},\,T_{c,out}&=\text{ccs model}(q_c, \omega_{wct},T_{amb},HR,T_{v},\dot{m}_v)
    \end{align*}
    \begin{itemize}
        \item Decision variables
        \[
        x = [q_c,\, \omega_{{wct}}]
        \]
        \item Environment variables
        \[
        e = [T_{{amb}}, HR,\, P_e,\, P_{w,s1},\, P_{w,s2}, V_{avail},\, T_v,\, \dot{m}_v]
        \]
        \item Fixed parameters
        \[
        \theta = [R_p = 1,\, R_s = 0,\, \omega_{{dc}} = 0]
        \]
    
    \end{itemize}
    
    \textbf{subject to}:
    \begin{itemize}
        
        \item Box-bounds
        \begin{itemize}
                \item $\omega_{wct} \in [\underline{\omega_{wct}}, \overline{\omega_{wct}}]$
                % \item $w_{wct} \in [\underline{\omega}_{wct}, \overline{w}_{wct}]$
                \item $q_{c} \in [\underline{q_{c}}, \overline{q_{c}}]$
                % \item $R_p \in [0,1]$
                % \item $R_s \in [0,1]$
        \end{itemize}
    
        \item Constraints
        \begin{itemize}
            \item $\left| T_{{wct,out}} - T_{{c,in}} \right| \leq \epsilon_1$
            \item $\left| \dot{Q}_{{wct}} - \dot{Q}_{{c,released}} \right| \leq \epsilon_2$
            \item $T_{{c,out}} \leq T_v - \Delta T_{{c-v,min}}$
        \end{itemize}
    
    \end{itemize}
    \labprob{cc:wct}
\end{problemcounter}

\begin{marginfigure}[*-20]
    \includegraphics[]{wascop-optimization-wct-diagram.png}
    \caption{Diagram of the wet cooler only cooling problem}
    \labfig{cc:optimization:diagram-wct}
\end{marginfigure}

In this version of the problem, the decision vector is composed by the
recirculation flow rate, but now the fan speed of the wet cooler
($\omega_{wct}$) is included. The cost of cooling now includes the cost of
water ($J_w$) and its availability is updated using the water consumption
($C_{w}$) as described in
Equations~\ref{eq:cc:optimization:wa1}--\ref{eq:cc:optimization:wa3}. The
environment now includes the air relative humidity and water prices.


%================================
\subsection{Combined cooler}
\labsec{cc:optimization:static:cc}

The last static optimization problem is the combined cooler, which incorporates
both the dry and wet coolers, as well as the condenser. Here the hydraulic
distribution is not fixed but is part of the decision variables, allowing the
optimization to determine the optimal distribution between the two coolers. The
problem is defined as follows\sidenote{
    See~\nrefsec{cc:modelling:complete-model} for a detailed description of the
    combined cooler and condenser model.
}:

\begin{problemcounter}{\gls{ccLabel} - static}
    
    \begin{equation*}
        \min_{\mathbf{x},\, \mathbf{e};\, \pmb{\theta}} \quad J = f(\mathbf{x}, \mathbf{e}; \pmb{\theta}) = J_e + J_w
    \end{equation*}

    \textbf{with}:
    \begin{align*}
        J_e &= C_e \cdot P_e \\
        J_{w} &= C_{w,s1} \cdot P_{w,s1} + C_{w,s2} \cdot P_{w,s2} \\
        C_{w,s1} &= \min(V_{avail}/T_s, C_{w}) \\
        C_{w,s2} &= C_w-C_{w,s1} \\
        T_{cc,out},\,C_e,\,C_w,\,T_{c,in},\,T_{c,out}&=\text{ccs model}(q_c, R_p, R_s, \omega_{dc}, \omega_{wct},T_{amb},HR,T_{v},\dot{m}_v)
    \end{align*}
    \begin{itemize}
        \item Decision variables
        \[
        x = [q_c, R_p, R_s, \omega_{{dc}}, \omega_{{wct}}]
        \]
        \item Environment variables
        \[
        e = [T_{{amb}}, HR,\, P_e,\, P_{w,s1},\, P_{w,s2}, V_{avail},\, T_v,\, \dot{m}_v]
        \]

    \end{itemize}

    \textbf{subject to}:
    \begin{itemize}
        
        \item Box-bounds
        \begin{itemize}
                \item $\omega_{dc} \in [\underline{\omega_{dc}}, \overline{\omega_{dc}}]$
                \item $\omega_{wct} \in [\underline{\omega_{wct}}, \overline{\omega_{wct}}]$
                \item $q_{c} \in [\underline{q}_{c}, \overline{q}_{c}]$
                \item $R_p \in [0,1]$
                \item $R_s \in [0,1]$
        \end{itemize}

        \item Constraints
        \begin{itemize}
            \item $\left| T_{{cc,out}} - T_{{c,in}} \right| \leq \epsilon_1$
            \item $\left| \dot{Q}_{{cc}} - \dot{Q}_{{c,released}} \right| \leq \epsilon_2$
            \item $T_{{c,out}} \leq T_v - \Delta T_{{c-v,min}}$
        \end{itemize}

    \end{itemize}
    \labprob{cc:static}
\end{problemcounter}


\begin{marginfigure}[*-20]
    \includegraphics[]{wascop-optimization-cc-diagram.png}
    \caption{Diagram of the combined cooler and condenser problem}
    \labfig{cc:optimization:diagram-cc}
\end{marginfigure}

Since any given decision variables do not necessarily yield a feasible solution,
the same three constraints as in the previous problems are included to ensure
the proper operation of the system. The decision vector now includes the
recirculation flow rate, the fan speeds of both coolers, and the hydraulic
distribution variables ($R_p$ and $R_s$). The environment variables remain
unchanged with respect to the previously presented wet cooler case.

\begin{figure}
    \includegraphics[width=\textwidth]{cc-pareto_fronts_for_different_scenarios_and_power_and_hydraulic_dist_for_different_scenarios.png}
    \caption{Pareto fronts in different representative scenarios (top) and detailed power and hydraulic distribution for a specific scenario (bottom)}
    \labfig{cc:optimization:paretos}
\end{figure}

To better understand the combined cooler static optimization problem,
\reffig{cc:optimization:paretos} illustrates the various ways a combined
cooler\sidenote{Particularly for the pilot plant described in
\refch{cc:facility}} can meet a specific cooling load under four diverse
scenarios: two  different environment conditions and cooling loads. The optimal
operating points are evaluated in terms of the two consumptions: electricity
($C_e$) and water ($C_w$) and form the Pareto front. 

\marginreminder[*-38]{Pareto front}{When dealing with multiple objectives where
no single solution is optimal, but improvements in one objective lead to
trade-offs in others, a set of points is obtained that represents the best
trade-offs between the objectives ---known as a Pareto front\footnote{See
\nrefsec{intro:optimization:multi-objective}}.}

In \texttt{Case I}, which presents the highest water and electricity
consumptions, it can be observed that the water consumption is always above
zero. This indicates that the use of the \gls{wctLabel} is essential, as the DC
alone is not capable of cooling the nominal thermal load at 42~$^\circ$C, during
summer conditions. If the system operates only with the \gls{dcLabel}, the vapor
temperature (\ie turbine backpressure) would rise, negatively impacting the
power cycle performance of a \gls{cspLabel} plant. Maintaining the same ambient
conditions, when the thermal power is reduced (\texttt{Case II}), it becomes
feasible to operate using only the \gls{dcLabel}, although with a high parasitic
load (7.4~kW$_{\text{e}}$). By combining the \gls{dcLabel} with the \gls{wctLabel},
electricity consumption can be reduced by half, with water consumption remaining
below 50~l/h. Under more favorable winter conditions (\texttt{Cases III} and
\texttt{IV}), the \gls{dcLabel} alone becomes more efficient as demonstrated by
the significant reduction in electricity demand. Still, when coupled with the
\gls{wctLabel}, electricity consumption can decrease by about 35 \% at full
thermal load, with a limited water consumption of 50~l/h. At reduced thermal
load the additional benefit of the \gls{wctLabel} becomes negligible, making
this the only case where operating with \gls{dcLabel} alone is more favorable.

The figure also details the optimal cooling power and hydraulic distribution for
\texttt{Case III}. The background color represents the distribution of cooling
power: green indicates a greater contribution from the dry cooler, while purple
indicates a greater contribution from the wet cooler. To avoid water
consumption, steam must be cooled exclusively using the \gls{dcLabel}. At the
other extreme, to minimize electricity consumption, only the \gls{wctLabel}
should be used. At intermediate optimal points, both systems are combined. This
is achieved by series configurations at prioritized dry cooling and
progressively increasing parallel configurations for predominantly
\gls{wctLabel} use. As expected, the cooling water flow rate
(\reffig{cc:optimization:paretos} - \textit{Hydraulic distribution}) is higher
(17~m$^3$/h) for a drier operation, since the reduced temperature component of
the cooling driving force is limited and therefore needs to be compensated with
a higher flow. Higher compared to the wet operation and its inherently higher
temperature difference available, allowing for a better (lower) flow of
10~m$^3$/h. In \reffig{cc:optimization:paretos} - \textit{Cooling power
distribution} it is also interesting to highlight how the series-parallel
flexibility of the system enables it to achieve almost a continuous division of
the cooling power from full dry to full wet, at least for this particular case.  


\section{Horizon optimization}
\labsec{cc:optimization:horizon}

The problem structure is very similar to the static alternative, the main
difference is that now the decision and environment vectors are composed not
from the expected value for the optimization step, but an array of values from
the current optimization step ($i$) until the end of the prediction horizon
($n_{steps}$)\sidenote[][*-3]{
    Bold notation is used to indicate that the variable is an array and not a
    single value, \eg~$\mathbf{x}$
}:

\marginnote[*10]{$\forall i = 1 \ldots n_{steps}$ is a notation to indicate that
    a condition must be held at every step $i$ in the optimization horizon
    ($n_{steps}$)}


\begin{problemcounter}{\gls{ccLabel} - horizon}
    \begin{equation*}
        \min_{\mathbf{x},\, \mathbf{e};\, \pmb{\theta}} \quad J = f(\mathbf{x}, \mathbf{e}; \pmb{\theta}) = \sum_{i=1}^{n_{steps}} \left( J_{e,i} + J_{w,i} \right) \cdot T_s
    \end{equation*}

    \textbf{with}:
    \begin{align*}
        \quad for&\: i = 1 \ldots n_{steps}: \\
        & J_{e,i} = C_{e,i} \cdot P_{e,i} \\
        & J_{w,i} = C_{w,s1,i} \cdot P_{w,s1,i} + C_{w,s2,i} \cdot P_{w,s2,i} \\
        & C_{w,s1,i} = \min(V_{avail,i}/T_s, C_{w,i}) \\
        & C_{w,s2,i} = C_{w,i}-C_{w,s1,i} \\
        & V_{avail,i} = V_{avail,i-1}-C_{w,s1,i}\cdot T_s \\
        & T_{cc,out,i},\,C_{e,i},\,C_{w,i},\,T_{c,out,i}=f(q_{c,i}, R_{p,i}, R_{s,i}, \omega_{dc,i}, \omega_{wct,i},T_{amb,i},HR_i,T_{v,i},\dot{m}_{v,i})
    \end{align*}
    \begin{itemize}
        \item Decision variables
        \[
        \mathbf{x} = [\mathbf{q_c}, \mathbf{R_p}, \mathbf{R_s}, \pmb{\omega}_{{dc}}, \pmb{\omega}_{{wct}}]
        \]
        where $\mathbf{x}=[x_{1,1},\,\ldots,\, x_{1,n_{steps}},\, \ldots,\, x_{n_{x},n_{steps}}]$
        \item Environment variables
        \[
        \mathbf{e} = [\mathbf{T_{amb}}, \mathbf{HR},\, \mathbf{P_e},\, \mathbf{P_{w,s1}},\, \mathbf{P_{w,s2}}, \mathbf{V_{avail,0}},\, \mathbf{T_v},\, \mathbf{\dot{m}_v}]
        \]
        where $\mathbf{e}=[e_{1,1},\,\ldots,\, e_{1,n_{steps}},\, \ldots,\, e_{n_{e},n_{steps}}]$

    \end{itemize}

    \textbf{subject to}:
    \begin{itemize}
        
        \item Box-bounds
        \begin{itemize}
                \item $\pmb{\omega_{dc}} \in [\underline{\omega_{dc}}, \overline{\omega_{dc}}]$
                \item $\pmb{\omega_{wct}} \in [\underline{\omega_{wct}}, \overline{\omega_{wct}}]$
                \item $\mathbf{q_{c}} \in [\underline{q_{c}}, \overline{q_{c}}]$
                \item $\mathbf{R_p} \in [0,1]$
                \item $\mathbf{R_s} \in [0,1]$
        \end{itemize}

        \item Constraints, $\forall i = 1 \ldots n_{steps}$:
        \begin{itemize}
            \item $\left| T_{{cc,out},i} - T_{{c,in},i} \right| \leq \epsilon_1$
            \item $\left| \dot{Q}_{{cc},i} - \dot{Q}_{{c,released},i} \right| \leq \epsilon_2$
            \item $T_{{c,out},i} \leq T_{v,i} - \Delta T_{{c-v,min}}$
        \end{itemize}

    \end{itemize}
    \labprob{cc:horizon}
\end{problemcounter}



Although this formulation allows for an arbitrarily long prediction horizon, in
practice it is limited to the number of steps for which reliable forecasts of
the environmental variables are available. In this study, water availability is
allocated on a daily basis; therefore, the prediction horizon begins at the
current time when the optimization is launched and extends it until the end of
the operation day. 

\begin{remark}
    A fixed sample time (\(T_s\)) is assumed, although the formulation can be
    adapted to handle variable sample times if the problem data are provided as
    time series with non-uniform time steps.
\end{remark}

% \begin{enumerate} 
%     \item An operation plan of the thermal load (power block or MED operating
%     conditions) needs to be defined.
%     \item An estimation of the costs context evolution. Water price is unlikely
%     to change often, but electricity can be more dynamic, specially market prices.
%     \item 
% \end{enumerate}

%================================
\subsection{A discussion on solving the optimization problem}[Problem discussion]


% Aquí comentar cómo no es factible resolver el problema directamente porque es
% muy difícil encontrar soluciones factibles debido a la estructura del problema.

% Comentar número de elementos en el vector de decisión, crecimiento exponencial
% de la complejidad del problema con el número de pasos en el horizonte, etc..

As defined, the \gls{ccsLabel} problem decision vector is composed by five
variables that are direct inputs on the process, \ie decision variables
represent actuators in the system. As explained in
\refsec{cc:modelling:complete-model}, not every combination of these variables
yields a feasible solution. In the real system, this implies that stable
operation would not be achieved for such a set of inputs\sidenote[][*10]{Either the
system would be overcooled, causing the vapor pressure and temperature to
decrease until a new equilibrium is reached, or undercooled, leading them to
increase instead}. To check for feasible operation the three mentioned
constraints are introduced. However, this increases the complexity of the
solution space significantly, since the solution space will not be continuous,
but as seen in \reffig{cc:optimization:search-space}, it will be formed by
islands of feasible solution space regions separated by infeasible regions. This
means that finding a feasible solution is not trivial, and the optimization
algorithm will need to explore the solution space-a global search algorithm-in
an attempt to find the global minimum.

\begin{marginfigure}[-6.5cm]
    \includegraphics[]{optimization-diagrams-search-space.png}
    \caption{Visualization of a constrained search space for two decision variables}
    \labfig{cc:optimization:search-space}
\end{marginfigure}

For one single step (equivalent to the static \refprob{cc:static}), most global
search algorithms with multiple runs are able to consistently find the global
optima. Local gradient-based algorithms are not suitable in this case because of
its sensitivity to the initial solution and often converge to local minima,
even when coupled with other techniques such as Generalized Monotonic Basin
Hopping~\sidecite{wales_global_1997}.

The problem becomes significantly more complex when the prediction horizon is
extended, the decision vector grows five-fold for each additional step in the
prediction horizon, and the optimization algorithm is tasked with finding a
feasible solution for this much larger decision vector, in a very complex
solution space, at once for all steps. The chances of finding a feasible
solution decrease significantly, and this was reflected during implementation in
the failure to find a single feasible solution. Even when providing an initial
guess composed by the static problem solutions for each step in a 24 steps
horizon, the returned solution was that same initial guess.


%================================
\subsection{Proposed solution: Decomposition-based multi-objective optimization with trajectory planning}[Proposed solution]

A two-stage optimization strategy is proposed to solve a multi-step decision
problem\sidenote[][*-3]{Alternative wording: Pareto front chaining, multi-step
Pareto optimization, path planning on Pareto surfaces.}. At each step of the
prediction horizon, a multi-objective optimization problem is independently
solved, yielding a Pareto front. A global optimization problem is then
formulated to select a path through the sequence of Pareto fronts, minimizing a
cumulative objective \ie the cost of cooling\sidenote{This is akin to a
pathfinding or the traveling salesman problem over the Paretos set}.

The methodology is illustrated in \reffig{cc:optimization:horizon-methodology}
and its components are described in the following sections.

\begin{figure*}[h!]
    % \begin{flushleft}
    %     {\textbf{0.} Decompose the problem horizon into its $n$ steps elements.}
    % \end{flushleft}
    % \vspace{1ex}

    % \begin{flushleft}
    % {\textbf{1.} Solve a multi-objective optimization problem,
    % independently for each step, to obtain a set of pareto fronts:}
    % \end{flushleft}
    % \vspace{0.5ex}
    % \includegraphics[]{optimization-diagrams-paretos.png}
    % \vspace{1ex}

    % \textdownarrow

    % \begin{flushleft}
    % {\textbf{2.} Select a path through these
    % pareto fronts that minimizes a global objective: the cumulative operation
    % cost of operation}
    % \end{flushleft}
    % \vspace{0.5ex}
    % \includegraphics[]{optimization-diagrams-path-selection.png}
    % % \vspace{1ex}
    \includegraphics[]{cc-horizon-optimization-diagram.png}

    \caption[Proposed methodology. Decomposition-based multi-objective
    optimization with trajectory planning]{Proposed methodology.
    Decomposition-based multi-objective optimization with trajectory planning.
    \textcolor[HTML]{6C8EBF}{\textit{Blue-dots} ($\bullet$)} represent
    points on the Pareto front. Three paths are illustrated: a water-greedy
    \textcolor[HTML]{9673A6}{dash-purple (- -) path}, a water-conservative
    \textcolor[HTML]{82B366}{green-dotted path (\texttt{..})} and an
    optimized-approach \textcolor[HTML]{FFCE9F}{solid-\textit{orangey} path
    (\texttt{--})}}
    \labfig{cc:optimization:horizon-methodology}
\end{figure*}

\subsubsection{Solving the multi-objective optimization problems}

To limit the complexity of the problem, the decision space can be reduced by one
variable by analyzing how the complete model is solved and described in
\nrefsec{cc:modelling:complete-model}; firstly, the condenser can be solved just
by using the recirculation flow rate ($q_c$), it follows the dry cooler by
adding the first valve ratio ($R_p$) and dry cooler fan speed ($\omega_{dc}$).
The only remaining component to solve is the wet cooler. The wet cooler inlet
conditions ($q_{wct}$, $T_{wct,in}$) can be determined by using the second valve
ratio ($R_s$). As for the outlet conditions, from the condenser evaluation, its
inlet temperature is known and it sets the value of the combined cooler outlet
temperature ($T_{cc,out}$), which in turn is the result of the mixing from the
\gls{dcLabel} and \gls{wctLabel} outlet temperatures ($T_{dc,out}$ and
$T_{wct,out}$, respectively). 

The result of this analysis is that the wet cooler fan speed is not a decision
variable anymore, but an output of the model, which can be computed by inverting the
wet cooler model, where an outlet temperature is provided as input, and the fan
speed is computed as an output. Summarizing, the decision vector can be reduced
from five to four variables:\sidenote[][*-5]{This reasoning works only for a system
with this particular configuration. A different combined cooler layout would
require a different analysis.}

\begin{equation*}
    x = [q_c, R_p, R_s, \omega_{dc}]
\end{equation*}

More importantly, now the optimization algorithm does not need to find a set of
five inputs that produce a feasible solution in a complex solution space, but
only four values from which a feasible wet cooling tower fan speed
exists\sidenote{\ie within its bounds $\omega_{wct}\,\in
[\underline{\omega_{wct}}, \overline{\omega_{wct}}]$}, thus greatly simplifying
the solution-space complexity.

A straightforward approach to solve the multi-objective optimization is to do a
grid-search over the decision space, evaluating the model for every combination
of decision variables, and then storing only the points for which a feasible
$\omega_{wct}$ exists. This approach is not recommended for large decision
spaces, but for the four-dimensional decision space and with a model that can
be evaluated in fractions of a second, it is feasible.

Next, the Pareto front is computed from the feasible points, which are
evaluated in terms of the two consumptions: electricity ($C_e$) and water
($C_w$). By definition, the Pareto front is the set of points that cannot be
improved in one objective without worsening the other, and it is computed by
checking for each point if there is another point that is better in both
objectives, and if so, it is removed from the set of feasible points. The
remaining points form the Pareto front. This process is repeated for
each step in the horizon, resulting in a set of Pareto fronts as visualized in
\reffig{cc:optimization:horizon-methodology}--\texttt{1}.


\subsubsection{Path selection subproblem}


The path selection subproblem is a combinatorial optimization problem over a
layered, dynamic-weighted directed graph. Each layer corresponds to a time step
in the prediction horizon, and each node within a layer represents a point on
the corresponding Pareto front. The objective is to find a path \(\mathbf{p} =
(p_1, p_2, \ldots, p_{n_{\text{steps}}})\), where \(p_i\) is the selected node
at time step \(i\), that minimizes the total cumulative cost \(J\) along the
path. Each layer weight (\ie,~the water cost component) is dynamic ---that is,
it depends on the trajectory followed to reach that layer. The problem can be
formulated as:

\begin{equation}
    \min_{\mathbf{p}, \mathbf{e}} \quad J = \sum_{i=1}^{n_{{steps}}} 
    J_i \left(p_i, V_{avail,i-1} \right).
\end{equation}

The step cost contribution \(J_{i}\) depends on both the consumptions (\ie
electricity and water) of the node \(p_i\), as well as on a dynamic cost
function determined by the traversed path. In particular, the water cost
component is correlated with the water resource availability
(\(J_{w,i}=f(V_{avail,i-1})\)\sidenote{See
Equations~\ref{eq:cc:optimization:wa1}--\ref{eq:cc:optimization:wa3}}). The step
associated cooling cost is:

\begin{equation}
    J_{{i}} = P_{e,i} \cdot C_e(p_{i}) + P_{w,s1,i} \cdot C_{w,s1}(p_{i}, V_{avail,i-1}) + P_{w,s2,i} \cdot C_{w,s2}(p_{i}, V_{avail,i-1}),
\end{equation}

where:
\begin{itemize}
    \item \(C_{e}(p_{i})\), \(C_{w,sx}(p_{i},V_{avail,i-1})\): electricity and water consumption
    (from $s_1$ and $s_2$) at node \(p_{i}\).
    \item \(P_{e,i}\), \(P_{w,sx,i}\): prices for electricity and water (from
    $s_1$ and $s_2$) at step \(i\).
\end{itemize}

This dynamic cost function introduces path-dependency into the cost function,
and makes the problem non-trivial to solve via simple shortest path algorithms.
Nonetheless, its calculation is straightforward and can be performed almost
instantly, thus making it suitable for dynamic programming, graph search (like
Dijkstra or A*), or metaheuristics such as genetic algorithms.

The subproblem is illustrated in
\reffig{cc:optimization:horizon-methodology}--2. Each layer corresponds with the
Pareto front of a step in the horizon and each node,
\textcolor[HTML]{6C8EBF}{blue-dot ($\bullet$)}, represents a point in the
particular step Pareto and is associated with a dynamic cost ($J_i$).
Sequentially traversing the layers selecting one point per layer up to a certain
layer, yields the cumulative cost up to that step ($J^{(i)}$). Three paths are
illustrated in \reffig{cc:optimization:horizon-methodology}--2. The
\textcolor[HTML]{9673A6}{dash-purple (- -) path} 
is a path that chooses nodes with a high water use\sidenote{In
\reffig{cc:optimization:horizon-methodology}, nodes are ordered with increasing
values of \(C_w\) from bottom to top.}, so in the first split it can be seen it
achieves the lowest cost of cooling ($J^{(2)}$), but also leaves the least
water available for the next steps ($V_{avail}^{(2)}$), resulting in a higher
total cost of cooling at the end of the horizon ($J==J^{(n)}$). On the other
hand, the \textcolor[HTML]{82B366}{green-dotted path (\texttt{..})} selects the
nodes with the lowest water use, this translates in a consistently higher cost
of operation and leaving some water available at the end of the horizon. Because
of the formulation of the problem, this is sub-optimal since this unused water
is considered lost. Finally, the \textcolor[HTML]{FFCE9F}{solid-\textit{orangey}
path (\texttt{--})} is a compromise between the two, it uses water
efficiently, leaving no water available at the end of the horizon and minimizing
the overall cost of cooling.
