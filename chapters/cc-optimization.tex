\setchapterpreamble[u]{\margintoc}
\chapter{Optimization of a combined cooling system}
\labch{cc:optimization}

\tldrbox{
    This chapter describes optimization problems for a combined cooling system,
    a \gls{dcLabel} and a \gls{wctLabel} as well as different optimization
    strategies propositions to solve them. The objective is to minimize the
    daily cost of operation made up by the electricity and water costs, while
    ensuring the cooling demand is met. They key challenge is to manage the
    available water resource, since there is a limited amount of cheap
    rainwater available and any excess water required must be purchased at a
    significantly higher cost. From the alternatives, this can only be
    effectively achieved by the shrinking horizon optimization strategy applied
    to the combined cooler for which an implementation methodology is proposed.
}

\section{Environment definition}

The environment for the optimization problems described in this section
includes the following components and is visualized in \reffig{cc:optimization:environment}:

\begin{marginfigure}[+5.5cm]
    \includegraphics[]{figures/cc-optimization-environment.png}
    \caption{Block diagram of the environment components}
    \labfig{cc:optimization:environment}
\end{marginfigure}


\begin{description}
    \item[Costs context] The cooling system has mainly two associated
    operational costs: electricity and water use. For the electricity the sale
    price of electricity is used since whatever is consumed by the cooling
    system, it's electricity that cannot be sold to the market in the case of a
    CSP plant, and it is a electricity that needs to be purchased in the case
    of an MED plant. As for the water, ... This module provides values for the price of
    electricity ($P_e$) and the prices of water from source 1 ($P_{w,s1}$) and
    source 2 ($P_{w,s2}$).

    \item[Weather forecast] The only two weather variables that have an impact
    on the cooling system are the ambient temperature ($T_{amb}$) and the
    relative humidity ($HR$) since they set the dry and wet bulb temperatures,
    and the psycrometric properties ... 

    \item[Thermal load] 
    
    \item[Water resource availability] Two sources of water are available, one
    of them, the cheaper one coming from a dam is limited in volume. The cheaper
    source ($s_1$) is prioritized until it is depleted, then the alternative source
    ($s_2$) is used:

    \begin{align}
        & \quad C_{w,s1,i} = \frac{\min(V_{avail,i}, C_{w,i} \cdot T_s)}{T_s} \\
        & \quad C_{w,s2,i} = C_{w,i}-C_{w,s1,i} \\
        & \quad V_{avail,i} = V_{avail,i-1}-C_{w,s1,i}\cdot T_s
    \end{align}

    where $i$ represents the step, at every step the amount used from each source
    is estimated and the dam-water left is updated accordingly.
\end{description}


% %================================
% \subsection{Water resource availability}
% \labsec{cc:optimization:environment-water}


\section{Static optimization}

As a first approach, the optimization problems evaluated are static. They are
defined in a particular time given an environment, and decisions do not take
into account prior decisions, neither consider the effect on future state.

\reminder{Optimization problem definition}{
    The general optimization function is defined as:\footnote{See \nrefsec{intro:optimization}}
    \begin{equation*}
    \min_{\mathbf{x},\, \mathbf{e};\, \boldsymbol{\theta}} \quad J = f(\mathbf{x}, \mathbf{e}; \boldsymbol{\theta}) 
        \quad \text{s.t.} \quad g_i(\mathbf{x}) \leq 0, \quad i = 1, \ldots, m
    \end{equation*}

    where \(x\) is the decision vector, \(e\) represents the environment, and \(\theta\) contains the fixed parameters.
}

Every time a problem is evaluated, it will start with some initial volume
($V_{avail,0}$) for the particular step, and this volume needs to be updated
before evaluating the next step. This yields that in order to evaluate several
consecutive steps, they must do so sequentially.

%================================
\subsection{Dry cooler}
\labsec{cc:optimization:static:dc}

In the first case study, only the dry cooler is involved, and so all terms
related to the wet cooler are set to zero and every water resource related term
can be ignored as can be seen in \reffig{cc:optimization:diagram-dc}. The components are defined as follows:

\marginnote[*5]{See \nrefsec{cc:modelling:dc} for a detailed description of the
dry cooler and condenser model.}

\problemdefinitionbox{\gls{dcLabel} - static}{
    \begin{equation*}
        \min_{\mathbf{x},\, \mathbf{e};\, \boldsymbol{\theta}} \quad J = f(\mathbf{x}, \mathbf{e}; \boldsymbol{\theta}) = C_e \cdot P_e
    \end{equation*}

    \textbf{with}:
    \begin{align*}
        T_{dc,out},\,C_e,\,T_{c,out} &= f(q_c,\, \omega_{\text{dc}},\, T_{\text{amb}},\, T_v,\, \dot{m}_v)
    \end{align*}
    \begin{itemize}
        \item Decision variables
        \[
        x = [q_c,\, \omega_{\text{dc}}]
        \]
        \item Environment variables
        \[
        e = [T_{\text{amb}},\, P_e,\, T_v,\, \dot{m}_v]
        \]
        \item Fixed parameters
        \[
        \theta = [R_p = 0,\, R_s = 0,\, \omega_{\text{wct}} = 0]
        \]

    \end{itemize}

    \textbf{subject to}:
    \begin{itemize}
        
        \item Box-bounds
        \begin{itemize}
                \item $w_{dc} \in [\underline{w}_{dc}, \overline{w}_{dc}]$
                % \item $w_{wct} \in [\underline{w}_{wct}, \overline{w}_{wct}]$
                \item $q_{c} \in [\underline{q}_{c}, \overline{q}_{c}]$
                % \item $R_p \in [0,1]$
                % \item $R_s \in [0,1]$
        \end{itemize}

        \item Constraints
        \begin{itemize}
            \item $\left| T_{\text{dc,out}} - T_{\text{c,in}} \right| \leq \epsilon_1$
            \item $T_{\text{c,out}} \leq T_v - \Delta T_{\text{c-v,min}}$
            \item $\left| Q_{\text{dc}}- Q_{\text{c,released}} \right| \leq \epsilon_2$
        \end{itemize}

    \end{itemize}
    }

\begin{marginfigure}[*-20]
    \includegraphics[]{figures/wascop-optimization-dc-diagram.png}
    \caption{Diagram of the dry cooler only cooling problem}
    \labfig{cc:optimization:diagram-dc}
\end{marginfigure}



%================================
\subsection{Wet cooler}
\labsec{cc:optimization:static:wct}

\marginnote[*5]{See \nrefsec{cc:modelling:wct} for a detailed description of the
wet cooler and condenser model.}

\problemdefinitionbox{\gls{wctLabel} - static}{
    \begin{equation*}
        \min_{\mathbf{x},\, \mathbf{e};\, \boldsymbol{\theta}} \quad J = f(\mathbf{x}, \mathbf{e}; \boldsymbol{\theta}) = J_e + J_w
    \end{equation*}

    \textbf{with}:
    \begin{align*}
        J_e &= C_e \cdot P_e \\
        J_{w} &= C_{w,s1} \cdot P_{w,s1} + C_{w,s2} \cdot P_{w,s2} \\
        C_{w,s1} &= \min \left((V_{avail}, C_{w} \cdot T_s)/T_s \right) \\
        C_{w,s2} &= C_w-C_{w,s1} \\
        T_{wct,out},\,C_e,\,C_w,\,T_{c,out}&=f(q_c, \omega_{wct},T_{amb},HR,T_{v},\dot{m}_v)
    \end{align*}
    \begin{itemize}
        \item Decision variables
        \[
        x = [q_c,\, \omega_{\text{wct}}]
        \]
        \item Environment variables
        \[
        e = [T_{\text{amb}}, HR,\, P_e,\, P_{w,s1},\, P_{w,s2}, V_{avail},\, T_v,\, \dot{m}_v]
        \]
        \item Fixed parameters
        \[
        \theta = [R_p = 1,\, R_s = 0,\, \omega_{\text{dc}} = 0]
        \]

    \end{itemize}

    \textbf{subject to}:
    \begin{itemize}
        
        \item Box-bounds
        \begin{itemize}
                \item $w_{wct} \in [\underline{w}_{wct}, \overline{w}_{wct}]$
                % \item $w_{wct} \in [\underline{w}_{wct}, \overline{w}_{wct}]$
                \item $q_{c} \in [\underline{q}_{c}, \overline{q}_{c}]$
                % \item $R_p \in [0,1]$
                % \item $R_s \in [0,1]$
        \end{itemize}

        \item Constraints
        \begin{itemize}
            \item $\left| T_{\text{wct,out}} - T_{\text{c,in}} \right| \leq \epsilon_1$
            \item $T_{\text{c,out}} \leq T_v - \Delta T_{\text{c-v,min}}$
            \item $\left| Q_{\text{wct}} - Q_{\text{c,released}} \right| \leq \epsilon_2$
        \end{itemize}

    \end{itemize}
}

\begin{marginfigure}[*-20]
    \includegraphics[]{figures/wascop-optimization-wct-diagram.png}
    \caption{Diagram of the wet cooler only cooling problem}
    \labfig{cc:optimization:diagram-wct}
\end{marginfigure}


%================================
\subsection{Combined cooler}
\labsec{cc:optimization:static:cc}


\marginnote[*5]{See \nrefsec{cc:modelling:complete-model} for a detailed description of the
combined cooler and condenser model.}

\problemdefinitionbox{\gls{ccLabel} - static}{
    \begin{equation*}
        \min_{\mathbf{x},\, \mathbf{e};\, \boldsymbol{\theta}} \quad J = f(\mathbf{x}, \mathbf{e}; \boldsymbol{\theta}) = J_e + J_w
    \end{equation*}

    \textbf{with}:
    \begin{align*}
        J_e &= C_e \cdot P_e \\
        J_{w} &= C_{w,s1} \cdot P_{w,s1} + C_{w,s2} \cdot P_{w,s2} \\
        C_{w,s1} &= \frac{\min(V_{avail}, C_{w} \cdot T_s)}{T_s} \\
        C_{w,s2} &= C_w-C_{w,s1} \\
        T_{cc,out},\,C_e,\,C_w,\,T_{c,out}&=f(q_c, R_p, R_s, \omega_{dc}, \omega_{wct},T_{amb},HR,T_{v},\dot{m}_v)
    \end{align*}
    \begin{itemize}
        \item Decision variables
        \[
        x = [q_c, R_p, R_s, \omega_{\text{dc}}, \omega_{\text{wct}}]
        \]
        \item Environment variables
        \[
        e = [T_{\text{amb}}, HR,\, P_e,\, P_{w,s1},\, P_{w,s2}, V_{avail},\, T_v,\, \dot{m}_v]
        \]
        \item Fixed parameters
        \[
        \theta = [R_p = 1,\, R_s = 0,\, \omega_{\text{dc}} = 0]
        \]

    \end{itemize}

    \textbf{subject to}:
    \begin{itemize}
        
        \item Box-bounds
        \begin{itemize}
                \item $w_{dc} \in [\underline{w}_{dc}, \overline{w}_{dc}]$
                \item $w_{wct} \in [\underline{w}_{wct}, \overline{w}_{wct}]$
                \item $q_{c} \in [\underline{q}_{c}, \overline{q}_{c}]$
                \item $R_p \in [0,1]$
                \item $R_s \in [0,1]$
        \end{itemize}

        \item Constraints
        \begin{itemize}
            \item $\left| T_{\text{cc,out}} - T_{\text{c,in}} \right| \leq \epsilon_1$
            \item $T_{\text{c,out}} \leq T_v - \Delta T_{\text{c-v,min}}$
            \item $\left| Q_{\text{cc}} - Q_{\text{c,released}} \right| \leq \epsilon_2$
        \end{itemize}

    \end{itemize}
}

\begin{marginfigure}[*-20]
    \includegraphics[]{figures/wascop-optimization-cc-diagram.png}
    \caption{Diagram of the combined cooler and condenser problem}
    \labfig{cc:optimization:diagram-cc}
\end{marginfigure}


\section[]{Shrinking horizon optimization}[Horizon optimization]

The problem structure is very similar to the static alternative, the main
difference is that now the decision and environment vectors are composed not
from the expected value for the optimization step, but an array of values from
the current optimization step until the end of the prediction horizon
($n_{steps}$), this means that forecasts for each variable in the environment
are needed, that is:
\begin{enumerate} 
    \item An operation plan of the thermal load (power block or MED operating
    conditions) needs to be defined.
    \item An estimation of the costs context evolution. Water price is unlikely
    to change often, 
\end{enumerate}

\marginnote[*5]{$\forall i = 1 \ldots n_{steps}$ is a notation to indicate that
    a condition must be held at every step $i$ in the optimization horizon
    ($n_{steps}$)}

\problemdefinitionbox{\gls{ccLabel} - horizon}{
    \begin{equation*}
        \min_{\mathbf{x},\, \mathbf{e};\, \boldsymbol{\theta}} \quad J = f(\mathbf{x}, \mathbf{e}; \boldsymbol{\theta}) = \sum_{i=1}^{n_{steps}} \left( J_{e,i} + J_{w,i} \right) \cdot T_s
    \end{equation*}

    \textbf{with}:
    \begin{align*}
        \quad for\: i &= 1 \ldots n_{steps}: \\
        & \quad J_{e,i} = C_{e,i} \cdot P_{e,i} \\
        & \quad J_{w,i} = C_{w,s1,i} \cdot P_{w,s1,i} + C_{w,s2,i} \cdot P_{w,s2,i} \\
        & \quad C_{w,s1,i} = \frac{\min(V_{avail,i}, C_{w,i} \cdot T_s)}{T_s} \\
        & \quad C_{w,s2,i} = C_{w,i}-C_{w,s1,i} \\
        & \quad V_{avail,i} = V_{avail,i-1}-C_{w,s1,i}\cdot T_s \\
        & \quad T_{cc,out,i},\,C_{e,i},\,C_{w,i},\,T_{c,out,i}=f(q_{c,i}, R_{p,i}, R_{s,i}, \omega_{dc,i}, \omega_{wct,i},T_{amb,i},HR_i,T_{v,i},\dot{m}_{v,i})
    \end{align*}
    \begin{itemize}
        \item Decision variables
        \[
        \mathbf{x} = [\mathbf{q_c}, \mathbf{R_p}, \mathbf{R_s}, \mathbf{\omega_{\text{dc}}}, \mathbf{\omega_{\text{wct}}}]
        \]
        where $x=[x_{0,0},\,\ldots\, x_{0,n_{steps}},\, \ldots,\, x_{n_{x},n_{steps}}]$
        \item Environment variables
        \[
        \mathbf{e} = [\mathbf{T_{\text{amb}}}, \mathbf{HR},\, \mathbf{P_e},\, \mathbf{P_{w,s1}},\, \mathbf{P_{w,s2}}, \mathbf{V_{avail,0}},\, \mathbf{T_v},\, \mathbf{\dot{m}_v}]
        \]
        where $e=[e_{0,0},\,\ldots\, e_{0,n_{steps}},\, \ldots,\, e_{n_{e},n_{steps}}]$

    \end{itemize}

    \textbf{subject to}:
    \begin{itemize}
        
        \item Box-bounds
        \begin{itemize}
                \item $\mathbf{w_{dc}} \in [\underline{w}_{dc}, \overline{w}_{dc}]$
                \item $\mathbf{w_{wct}} \in [\underline{w}_{wct}, \overline{w}_{wct}]$
                \item $\mathbf{q_{c}} \in [\underline{q}_{c}, \overline{q}_{c}]$
                \item $\mathbf{R_p} \in [0,1]$
                \item $\mathbf{R_s} \in [0,1]$
        \end{itemize}

        \item Constraints, $\forall i = 1 \ldots n_{steps}$:
        \begin{itemize}
            \item $\left| T_{\text{cc,out},i} - T_{\text{c,in},i} \right| \leq \epsilon_1$
            \item $T_{\text{c,out},i} \leq T_{v,i} - \Delta T_{\text{c-v,min}}$
            \item $\left| Q_{\text{cc},i} - Q_{\text{c,released},i} \right| \leq \epsilon_2$
        \end{itemize}

    \end{itemize}
}

%================================
\subsection[]{A discussion on solving the optimization problem}[Problem discussion]
Aquí comentar cómo no es factible resolver el problema directamente porque es
muy difícil encontrar soluciones factibles debido a la estructura del problema.

Comentar número de elementos en el vector de decisión, crecimiento exponencial
de la complejidad del problema con el número de pasos en el horizonte, etc..

%================================
\subsection[]{Proposed solution: Decomposition-based multi-objective optimization with trajectory planning}[Proposed solution]

We propose a two-level optimization strategy for a multi-stage decision
problem\sidenote[][*-5]{Alternative wording: Pareto front chaining, multi-stage
Pareto optimization, path planning on Pareto surfaces. }. At each stage of a
prediction horizon, we independently solve a multi-objective optimization
problem, yielding a Pareto front\marginreminder[*-3]{Pareto front}{
When dealing with multiple objectives where no single solution is optimal, but
improvements in one objective lead to trade-offs in others, we obtain a set of
points that represent the best trade-offs between the objectives, known as a
Pareto front\footnote{See \nrefsec{intro:optimization:multi-objective}}
}. We then formulate a global optimization problem to select a consistent path
through the sequence of Pareto fronts, minimizing a cumulative objective (e.g.,
cost or distance), similar to a pathfinding or TSP-like problem over
Pareto-optimal points.

\begin{enumerate}
    \item Decompose a multi-stage problem into $N$ stages.
    \item Solve a multi-objective optimization problem at each stage
    independently to obtain a Pareto front.
    \item Formulate a second-level problem to select a path through these
    Pareto fronts that minimizes a global objective, the cumulative operation
    cost\sidenote{analogously to a Traveling Salesman Problem (TSP) on the
    Pareto surfaces}.
\end{enumerate}

\subsubsection{Solving the multi-objective optimization problems}

\subsubsection{Path selection subproblem}

Problem nature description.

The path selection subproblem can be formulated as a graph traversal problem,
where each node represents a point in the Pareto front of a stage, and edges
represent the transition costs between these points. The goal is to find a path
through the graph that minimizes the cumulative cost. This subproblem is a combinatorial
optimization problem, in particular, a layered weighted directed graph.

Definición formal del problema

The transition cost is correlated to the current resource availability and will
depend on the current state of the system, which is a function of the
previous decisions.


Mencionar algoritmo seleccionado

The path optimization could be handled via dynamic programming, graph search
(like Dijkstra or A*), or metaheuristics depending on the problem size.
Metaheuristics:
- Genetic Algorithms
- Simulated Annealing
- Ant Colony Optimization
- Tabu Search
- Particle Swarm Optimization

%===================================
%===================================
\section{Limitations}
\labsec{cc:optimization:limitations}

Modelado de disponibilidad de agua en función de las precipitaciones. Esto es
lo más complejo. Aunque haya mucha disponibilidad, puede ser que haya mucha
demanda de agua (agricultura, etc), si hay poca disponibilidad puede que ni
siquiera se permitiese su implementación. Si finalmente se ajusta el volumen
máximo para que coincida con lo que consumiría el sistema húmedo
exclusivamente, esto va a hacer que el húmedo no sea factible parte del año.
Discutir.
