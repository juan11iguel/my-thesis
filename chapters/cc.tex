\pagelayout{wide} % No margins
\addpart{Optimal water and electricity management in a combined cooling system}
\pagelayout{margin} % Restore margins

% \section*{TL;DR}
% \addcontentsline{toc}{section}{TL;DR}

\tldrbox{
    In the pursuit of extending the use and feasibility of solar thermal
    applications, a case study consisting of a commercial 50~MW-8 hours of
    storage \gls{cspLabel} plant, Andasol-II, is analyzed in annual simulations
    where the cooling solution makes use of the novel proposed \gls{ccsLabel}.
    To obtain these results, a model of the \gls{ccsLabel} has been developed
    based on the same configuration as the \gls{psaLabel} pilot plant.
    Different optimization strategies based on evolutionary algorithms have
    been implemented to adapt the system operation to the changing conditions.
    The strategy has been experimentally validated in the pilot plant and the
    simulated results show than the proposed scheme can yield ... compared to
    the \gls{dcLabel} only and ... with the \gls{wctLabel} only
    alternatives.
}
    
\begin{figure}[htbp]
    \includegraphics[width=\textwidth]{figures/cc-visual-abstract.png}
    % {\footnotesize \textbf{(a)} Visual abstract of the part}
    \vspace{2ex}
    \includegraphics[]{figures/cc-visual-abstract-results.png}
    {\footnotesize \textbf{(a)} Andasol-II \gls{cspLabel} plant case study results}
    \vspace{2ex}

    % \includegraphics[]{figures/cc-visual-abstract-results.png}
    % {\footnotesize \textbf{(b)} MED case study results}

    \caption{Results for the analyzed case study in this part. The first bar
    plot compares the specific cost of the different cooling alternatives
    (lower is better). The second pie chart shows the breakdown of the costs,
    and the third and fourth pie charts show the cooling power distribution
    between \gls{dcLabel} and \gls{wctLabel} and the hydraulic distribution,
    respectively.}
    \labfig{cc:visual-abstract}
\end{figure}

\todo{In pie chart, rename pumping to recirculation. Q to $\dot{Q}$}    

\section*{Derived scientific contributions}
\addcontentsline{toc}{section}{Derived scientific contributions}


\section*{Structure}

This part is structured as follows: in the first two chapters the methodology is
described, specifically the modelling in (\nrefch{cc:modelling}) is presented
for the \gls{ccsLabel}, and the optimization framework in
\nrefch{cc:optimization}. The third chapter presents the experimental facility
(\nrefch{cc:facility}) that is used to experimentally validate the model and
the optimization strategy integrated in a hierarchical control scheme in
\nrefch{cc:validation}. The final chapter, \nrefch{cc:simulation}, describes
and analyzes the results of the annual simulations performed for the Andasol-II
CSP plant.

%===================================
%===================================
%===================================
\setchapterpreamble[u]{\margintoc}
\chapter{CSP overview}
\labch{intro:csp}

In the pursuit of eliminating reliance on fossil fuels sources for energy
generation and replacing them by renewable sources, \gls[format=long]{cspLabel}
has proven to be a reliable contributor. In particular, in providing much
needed energy storage, dispatchability and ensuring grid stability.

% Introducción de artículo: Wet cooling tower performance prediction in
% \gls{cspLabel} plants: A comparison between artificial neural networks and
% Poppe’s model
\gls[format=long]{cspLabel} plants use mirrors to concentrate the sun's energy
to finally drive a turbine that generates electricity. This technology
currently represents a minor part of renewable energy generation in Europe.
Only approximately 5~GW are installed globally (of which 2.3~GW in Europe are
concentrated in Spain). However, the potential for growth is significant given
the capability of \gls{cspLabel} to provide renewable electricity when needed
thanks to in-built energy storage continuing the production even in the absence
of sunlight, unlike other renewable technologies that are dependent on the
availability of the energy source. Of increasing importance is also their
potential application in improving the manageability of the grid, replacing
fossil fuel alternatives. Their dispatchability enables plants to respond to
peaks in demand, and provide ancillary services to the grid. According to the
International Energy Agency forecasts, \gls{cspLabel} has a huge potential in
the long term, ranging from the 986~TWh by 2030 up to 4186~TWh by 2050
\sidecite{iea_energy_2014}, which means that \gls{cspLabel} will account for
11\% of the electricity generated worldwide and for 4\% in the case of Europe. 

%================================
%================================
\section{Cooling and water use}
\labsec{intro:csp:cooling}
% \labch{intro:cooling}

% Introducción de artículo: Wet cooling tower performance prediction in
% \gls{cspLabel} plants: A comparison between artificial neural networks and
% Poppe’s model


The cooling of the power block in this technology plays a crucial role in its
feasibility. The cheapest and most efficiency cooling technology is evaporative
cooling, and that is why most plants, specially in Spain where built using this
alternative (XX \% \sidecite{thonig_cspguru_2023}), however, the high-radiation
areas in which they are located are usually regions with rapidly-degrading
water availability due to climate change, so water has become a scarce
resource. Nowadays most likely those plants would have been built with dry
cooling technologies, significantly increasing the cost (up to 8\% during
periods of high ambient temperatures when energy demand and prices peak
\sidecite{}).


\gls{cspLabel} plants are, in general, located in arid areas, where sun
irradiance is high but water is scarce. The efficiency of these plants is
highly dependent on the temperature at which the steam is condensed. To date,
the conventional systems used to remove excess heat from \gls{cspLabel} plants
are either wet (water-cooled) or dry (air-cooled). The lowest attainable
condensing temperature is achieved in wet cooling systems that depend on the
wet-bulb temperature, allowing \gls{cspLabel} plants to achieve higher
efficiencies. However, this efficiency increase is at the expense of a high
cost: excessive water use. Dry cooling systems eliminate the water use but they
lead to lower plant efficiencies when the ambient air temperature is high.
Those hot periods are often the periods of peak system demand and higher
electricity sale price. The combination of the advantages of each of them into
an innovative cooling system is thus of great interest. There are different
types of innovative cooling systems: those that integrate the dry and wet
cooling systems into the same cooling device, which are called hybrid cooling
systems
\sidecite{rezaei_reducing_2010,asvapoositkul_comparative_2014,hu_thermodynamic_2018} 
and those that combine separate dry and wet cooling systems, which are called
combined cooling systems. In the case of hybrid cooling systems, the dry
section are composed of compact heat exchangers included in a wet cooling tower
\cite{rezaei_reducing_2010}. This kind of cooling systems can be considered as
an efficient cooling solution for \gls{cspLabel} plants
\sidecite{elmarazgioui_impact_2022} due to the energy conservation and water
and greenhouse gas emissions savings. In the case of combined cooling systems,
different configurations can be found. The most commonly proposed in the
literature is the one that considers an \gls{accLabel} in parallel with a
\gls{wctLabel}, as can be seen in
\sidecite{barigozzi_wet_2011,barigozzi_performance_2014}. In this kind of
configuration, the exhaust steam from the turbine is condensed either through
the \gls{accLabel} or through a \gls{scLabel} coupled with the \gls{wctLabel}.
Another configuration, recently proposed in
\sidecite{palenzuela_experimental_2022} is a wet cooling tower and a
\gls{dcLabel} (type \gls{acheLabel}) sharing a surface condenser. In this case,
the exhaust steam from the turbine is condensed through the surface condenser
and the heated cooling water is cooled either through the \gls{wctLabel} or
through the dry cooler. This kind of combined cooling systems are proposed as
the most suitable option for a flexible operation as a function of the ambient
conditions, since they allow to select the best operation strategies to achieve
an optimum water and electricity consumption compromise
\sidecite{asfand_thermodynamic_2020}. In addition, if the optimization is
combined with energy demand forecasting as described in
\sidecite{wazirali_stateoftheart_2023}, the expected results can be even
better.

