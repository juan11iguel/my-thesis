\pagelayout{wide} % No margins
\addpart{Energy management in MED processes driven by variable energy sources}
\pagelayout{margin} % Restore margins

\wipbox{}

\tldrbox{
    ...
}

% Visual abstract
a\todo{Add visual abstract}

\section*{Derived scientific contributions}
\addcontentsline{toc}{section}{Derived scientific contributions}

\section*{Structure}

%===================================
%===================================
%===================================
\setchapterpreamble[u]{\margintoc}
\chapter{Thermal desalination}
\labch{intro:desalination}

\wipbox{}

Desalination is increasingly recognized as a key strategy to address global
freshwater scarcity, driven by the combined pressures of climate change and
population growth. Regions already facing drought and water stress, such as
parts of Spain, are expected to see growing dependence on desalinated water to
meet rising demand. While desalination technologies—particularly membrane-based
systems like \gls[format=long]{roLabel}—have seen rapid expansion, the energy
intensity of the process remains a major challenge. To mitigate this, efforts
have focused on improving energy efficiency and integrating renewable energy
sources such as solar or geothermal heat. In particular, thermal desalination
technologies like \gls{medLabel} are gaining renewed interest due to their
compatibility with low-exergy heat sources (\eg waste heat) and the ability to
treat high-salinity brines. These thermal processes also align better with
circular economy approaches, allowing the concentration of brine and the
recovery of valuable minerals such as lithium or magnesium, an emerging field
known as brine mining.