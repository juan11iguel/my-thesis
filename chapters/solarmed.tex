\pagelayout{wide} % No margins
\myepigraphhead[500]{Such arbitrariness had its origins in the example of the
Catalan sage, for whom wisdom was not worthwhile if it could not be used to
invent a new way of preparing chickpeas\\//\\Tanta arbitrariedad tenía origen en el ejemplo del sabio catalán, para quien la sabiduría no valía la pena si no era posible servirse de ella para inventar una manera nueva de preparar los garbanzos}{Gabriel García Márquez, \textit{One
Hundred Years of Solitude // Cien años de soledad}}
\part{Energy management in \gls{medLabel} processes driven by variable energy
sources}

\glsresetall % Reset glossary entries

\tldrbox{ 

This work advances the integration, evaluation, and optimal operation of
low-temperature \gls{medLabel} systems powered by variable energy sources such
as thermal energy. As desalination becomes increasingly important to address
global freshwater scarcity, modern thermal systems ---particularly low-temperature
\gls{medLabel}--- offer a robust and complementary pathway for brine
concentration and resource recovery, especially when coupled with waste-heat or
renewable sources.\\

A standardized methodology is developed to evaluate \gls{medLabel} performance under
realistic and highly variable operating conditions. The approach defines
instrumentation requirements, key performance indicators, and uncertainty
quantification procedures, complemented by an automated steady-state detection
algorithm that enhances the robustness of experimental assessments. Tests at
elevated top-brine temperatures confirm the feasibility of high-temperature
operation without significant scale formation, although design changes are
required to unlock further gains in concentration and thermal efficiency.\\

To reproduce system behavior, a complete hybrid model of the solar-driven
\gls{medLabel} installation at \gls{psaLabel} is formulated. This includes
physics-based and data-driven dynamic models for heat generation, storage, and
desalination subsystems, combined with discrete supervisory finite-state
machines that encode operational logic. The resulting model captures multi-hour
system dynamics with mean absolute percentage errors below 15~\%, ensuring both
fidelity and computational tractability for optimization and control tasks.\\

The core contribution of the work is a novel hierarchical optimization strategy
that governs the autonomous operation of the coupled solar-\gls{medLabel}
system. The upper-level controller solves a mixed-integer nonlinear economic
problem that determines subsystem activation, startup, shutdown, and regulation
while exploiting solar availability and storage flexibility. When compared over
a full week of operation, the strategy outperforms both a heuristic rule-based
baseline and a continuous-only optimization approach by 32~\% and 21~\%,
respectively. These gains stem from the ability to maximize useful temperature
differences and align \gls{medLabel} operation with the temporal value of heat.
This performance levels are achieved with a similar operation strategy of a
waste-heat driven system.\\

\begin{minipage}{\linewidth}
    \centering
    \includegraphics[width=\linewidth]{solarmed-visual-abstract.png}\\
    \small \textit{Visual abstract}
\end{minipage}
}

\section*{Part structure}

This part is structured as follows: first in \nrefch{intro:desalination} a
context of thermal desalination technologies is provided and their applicability
to new applications such as brine-mining, specifically for the case of
\gls[format=long]{medLabel}. Then, the experimental solar-driven \gls{medLabel}
pilot at \gls{psaLabel} is presented in \refch{solarmed:facility}. A standard
methodology for the experimental assessment of \gls{medLabel} systems under
variable operating conditions is detailed in \refch{solarmed:std}, and its
application to the experimental plant is shown in \refch{solarmed:std} in a
high-TBT experimental campaign. The methodology for modelling and optimizing the
operation of the system are described in \refch{solarmed:modelling} and
\refch{solarmed:optimization}, respectively.

\pagelayout{margin} % Restore margins
%===================================
%===================================
%===================================
\setchapterpreamble[u]{\margintoc}
\chapter{Thermal desalination}
\labch{intro:desalination}

\glsresetall

\tldrbox{ Desalination is often seen as a solution to mitigate freshwater
    scarcity in the face of climate change and population
    growth~\cite{pltonykova_united_2020,jones_state_2019}. It already plays a
    fundamental role in many regions~\cite{eke_global_2020}, but, as more
    frequent droughts and water shortages are expected, the demand for
    desalinated water is likely to increase.\smallskip

    However, desalination also faces several challenges, the most important
    being the intense energy required to separate salts from seawater, which
    makes it an expensive process compared to other methods to obtain fresh
    water.\smallskip

    Low-temperature \gls{medLabel} systems constitute a mature, robust, and
    technically sound solution for brine concentration and resource recovery,
    especially when driven by waste or renewable heat sources. Rather than being
    seen as outdated compared to mechanical desalination, \gls{medLabel}
    technology can play a complementary role in integrated water-energy systems
    aimed at sustainability and minimum-liquid discharge. }

%===================================
%===================================
\section{Water crisis}
\labsec{solarmed:intro:water-crisis}

One of the many bad consequences of climate change is the rapid desertification
of the planet. According to World Resources Institute
projections~\sidecite{kuzma_aqueduct_2023} (see
\reffig{solarmed:water-stress-map}), 51 countries will suffer from high water
stress by 2050. Many regions, including the Arabian Peninsula, Iran, India, and
North Africa, are expected to consume at least 80~\% of their water supply. The
issue is not confined to emerging economies, as Southern European countries like
Spain, Italy, and Portugal are also significantly affected with projections of
extremely high water scarcity.

Desalination is often seen as a solution to mitigate freshwater scarcity in the
face of climate change and population
growth~\sidecite[*-4]{pltonykova_united_2020,jones_state_2019}. It already plays a
fundamental role in many regions~\sidecite{eke_global_2020}, but, as more
frequent droughts and water shortages are expected, the demand for desalinated
water is likely to increase~\sidecite{mekonnen_four_2016}. However, desalination
also faces several challenges, the most important being the intense energy
required to separate salts from seawater, which makes it an expensive process
compared to other methods to obtain fresh water.

In a water scarcity scenario, the priority should always be to reduce the water
demand in the first place~\sidecite[*-2]{semiat_energy_2008}. Secondly, to
recycle as much water as possible so that it can be
reused~\sidecite[*-1]{howe_principles_2012}. However, for many parts of the world,
these are only palliative measures. Simply put, there will not be enough water
available to satisfy its needs, and thus the energy intense process of
desalination is the only viable alternative.

Great efforts are being made to reduce the energy consumption of desalination
and use renewable energies to increase its
sustainability~\sidecite[*-4]{shekarchi_comprehensive_2019,allouhi_green_2024,schar_optimization_2023}.
Another path to follow is to generate added value from the separation process. 

%===================================
%===================================
\section{Brine concentration and mining}
\labsec{solarmed:intro:brine}

\begin{figure}
    \includegraphics[width=.7\textwidth]{water-stress-map.jpeg}
    \savebox\captionqr{\qrcode[hyperlink,height=0.5in]{https://www.statista.com/chart/26140/water-stress-projections-global/}}
    \caption{Global water stress map. \\[1ex]Source:
    \href{https://www.statista.com/chart/26140/water-stress-projections-global/}{Statista}.\\[1ex]
    \usebox\captionqr}
    \labfig{solarmed:water-stress-map}
\end{figure}

Brine, a byproduct of the desalination process, typically exhibits salinities
1.6--2.1 times higher than seawater, along with residual process chemicals such
as coagulants, antiscalants, and desinfectants~\sidecite{panagopoulos_study_2022}. Its conventional disposal
methods ---such as marine discharge, deep-well injection, evaporation ponds, and
land application--- are often unsustainable, leading to marine ecosystem stress,
soil salinization, and groundwater
contamination~\sidecite{panagopoulos_environmental_2020}. High-salinity plumes
can cause osmotic shock in marine organisms, disrupt seagrass and coral
communities, and alter local biogeochemical conditions. These impacts are
particularly pronounced in semi-enclosed basins such as the Mediterranean or Red
Sea, where dilution capacity is limited.

At the same time, desalination brine represents a largely untapped resource.
Brines are rich in sodium, chloride, magnesium, calcium, and potassium, as well
as trace valuable metals like lithium, rubidium, and celsium, which have high
commercial value~\cite{panagopoulos_environmental_2020}. This recognition has
led to growing interest in brine management and valorization, an approach that
aligns with the principles of a circular water economy. Through \gls{mldLabel}
and \gls{mldLabel} systems, it is possible to recover up to 95--100~\% of
freshwater and extract valuable salts and minerals, turning waste into a
secondary source of raw materials~\cite{panagopoulos_environmental_2020}. This
strategy not only reduces environmental impacts but also offers potential
economic benefits, offsetting part of the desalination cost.

However, technical and economic barriers still limit large-scale implementation
of brine mining. Challenges include high energy demand, low extraction
efficiencies for trace elements, and the immaturity of integrated hybrid systems
combining membrane, thermal, and chemical processes. Moreover, the dominance of
sodium chloride ---by far the most abundant constituent--- means that saturation
processes generate vast quantities of common salt, creating logistical and
market challenges for its reuse. Recent studies are exploring new applications
for desalination brines, such as using them as sources of chloride and nitrate
ions in hydrometallurgical leaching or as inputs for industrial chemical
production~\sidecite[*-4]{hernandez_use_2020}.

%===================================
%===================================
\section{Overview of Desalination Technologies}
\labsec{solarmed:intro:technologies}


Desalination refers to the set of processes that remove dissolved salts and
impurities from saline water to produce freshwater suitable for drinking,
irrigation, or industrial use. These technologies can be broadly divided into
thermal and mechanical-based processes, depending on the dominant physical
mechanism of salt separation~\sidecite{el-dessouky_fundamentals_2002}.


%================================
\subsection{Mechanical Technologies}
\labsec{solarmed:intro:technologies:mechanical}

Membrane processes rely on selective transport through semipermeable membranes,
driven by pressure, concentration, or electrical potential differences, without
phase change. They have become dominant in global desalination capacity because
of lower energy requirements and modular scalability. The main categories are:

\begin{itemize}
    \item \gls{roLabel}. The most widely adopted method, where high-pressure
    pumps (50-80 bar for seawater) force water through semipermeable membranes,
    rejecting dissolved salts. \gls{roLabel} systems achieve high recovery and
    energy efficiency, particularly when coupled with modern energy recovery
    devices, but require careful pretreatment to prevent fouling and scaling. 
    \item \gls{nfLabel} and \gls{foLabel}. Emerging variants designed for
    partial desalination, pretreatment, or hybrid systems that improve overall
    process efficiency.
\end{itemize}

%================================
\subsection{Thermal Technologies}
\labsec{solarmed:intro:technologies:thermal}

Thermal desalination processes are based on phase change, involving the
evaporation of saline water and condensation of vapor as pure distillate. They
were the first large-scale desalination methods to be commercialized and remain
widely used, particularly in areas with access to low-cost fuel or waste heat.
The main thermal processes are:

\begin{itemize}
    \item \gls{msfLabel}. In \gls{msfLabel}, seawater is heated and then flashed
    into vapor in a series of chambers operating at successively lower
    pressures. The vapor is condensed to produce distilled water, and the
    released heat is recovered to preheat the feed. \gls{msfLabel} systems are
    robust and well-proven for large capacities but have higher thermal and
    electrical energy requirements.
    
    \item \fullgls{medLabel}.~\gls{medLabel} involves a sequence of
    evaporation-condensation stages (or ``effects'') at decreasing pressures.
    Vapor produced in one effect serves as the heating medium for the next,
    significantly improving thermal efficiency. \gls{medLabel} systems typically
    operate at 60--70~$^\circ$C to minimize scaling but can reach up to
    120~$^\circ$C when properly pretreated. They are highly reliable and
    well-suited to integration with waste heat or solar thermal sources.
    
    \item \gls{vcLabel}. In Vapor Compression, the vapor generated from the feed
    is compressed either mechanically (\glsentryshort{mvcLabel}) or thermally
    (\glsentryshort{tvcLabel}) to raise its temperature and pressure so it can
    serve as the heat source for further evaporation. \glsentryshort{mvcLabel}
    systems are compact and efficient for small to medium capacities, while
    \glsentryshort{tvcLabel} is often combined with \gls{medLabel} to improve
    energy recovery.

    \item \gls{mdLabel}. In Membrane Distillation a hydrophobic, microporous
    membrane acts as a physical barrier that prevents liquid intrusion
    while allowing water vapor to pass. The liquid and vapor phases at the
    membrane interface remain in thermodynamic equilibrium, and the driving
    force is a vapor pressure difference across the membrane, normally
    established by a temperature gradient. Under this gradient, vapor formed at
    the hot feed-membrane interface flows through the membrane pores toward the
    side with lower vapor pressure, where it condenses. \gls{mdLabel} can
    utilize low-grade or waste heat (<~80~$^\circ$C) and can operate with very
    high salinity feeds (even near saturation). Salt rejection is typically
    high, although the exact salt rejection factor depends strongly on membrane
    properties, on operating conditions and the membrane configuration.

    Common configurations include Direct Contact \gls{mdLabel}
    (DC\gls{mdLabel}), Air-Gap \gls{mdLabel} (AG\gls{mdLabel}), Vacuum \gls{mdLabel}
    (V\gls{mdLabel}), Sweeping-Gas \gls{mdLabel} (SG\gls{mdLabel}), and
    Vacuum-Air-Gap \gls{mdLabel} (V-AG\gls{mdLabel}). Although still emerging at
    industrial scale, \gls{mdLabel} is particularly promising for brine
    concentration, \gls{mldLabel} systems, and solar-driven desalination.
\end{itemize}


\subsection{Thermal desalination timeline and comparison with \gls{roLabel}}

Thermal desalination technologies, such as \gls{medLabel}, \gls{msfLabel}, and
newer hybrid configurations, have evolved through several decades of incremental
improvements focused primarily on maximizing thermal efficiency and reducing
specific energy consumption. Early developments emphasized heat recovery,
exemplified by the integration of
\gls{tvcLabel}~\sidecite[*-10]{milow_advanced_1997} and absorption heat pumps
(DEAHP)~\sidecite[*-7]{alarcon-padilla_application_2007}, both designed to
increase the \gls{gorLabel}\sidenote[][*-3]{A variety of metrics can be used to
measure the energy efficiency of a desalination plant, each with different
purposes and conveying different information. \gls{gorLabel} is a metric that
measures the heat provided to the system per unit of distillate produced. Also
relates to the number of times latent heat is reused in the system.\\[2ex]A
detailed description of all metrics can be found in \nrefch{solarmed:std}} by
reusing latent heat more effectively. Research throughout the 1980s and 1990s
concentrated on optimizing heat exchanger design, corrosion resistance, and
system modularity, leading to steady improvements in performance and
reliability~\sidecite[*3]{khawaji_advances_2008}. However, despite these advances,
the fundamental thermodynamic limits of phase-change separation and the high
capital cost of metallic heat exchangers have constrained further cost
reductions and scalability.

By the late 1990s and early 2000s, it became increasingly clear that
mechanically driven separation ---particularly reverse osmosis
(\gls{roLabel})--- offered superior performance from both energetic and economic
perspectives. \gls{roLabel}'s rise was enabled by major advances in polymer
science, leading to thin-film composite membranes with high salt rejection and
flux, as well as by the widespread deployment of energy recovery devices that
dramatically lowered specific energy consumption.

Under any accurately defined metric in use today, \gls{roLabel} outperforms
\gls{medLabel} on the basis of primary energy consumption\sidenote{Primary
energy consumption is the metric most closely tied to fuel consumption and,
ultimately, to operating costs. This metric provides the fairest comparison
between desalination technologies~\cite{bouma_metrics_2020}}. \gls{roLabel}'s
efficiency and economic advantage results from the ``conductivity'' and cost
advantages of membranes over heat exchangers. Significant improvements in heat
exchanger costs or heat transfer coefficients would be needed to make thermal
desalination technologies such as \gls{medLabel} competitive in this
respect~\sidecite[]{bouma_metrics_2020}. 

If we extend the comparison to primary energy consumption ---which correlates
directly with fuel usage--- while \gls{medLabel} uses approximately three times
the exergy of an \gls{roLabel} system at the desalination system inlet, it
requires less than twice the primary energy (11.3~\% compared to 20.6~\% in
terms of second law efficiency)~\cite{bouma_metrics_2020}. The smaller gap
arises because of the thermodynamic penalty associated with converting primary
energy into electricity rather than into steam. Thus, while \gls{roLabel}
remains more efficient overall, the difference narrows when considering primary
energy or co-generation scenarios, where electricity and water are produced
together. In such cases, overall second-law efficiencies can approach 70~\%,
with only about a 1~\% difference between \gls{roLabel} and thermal
technologies~\cite{bouma_metrics_2020}. 

In terms of costs, desalination plants are designed to minimize the
\gls{lcowLabel}, which depends not only on energy consumption but also on
capital costs, material properties, and maintenance requirements. For example,
while the cost of energy is comparable in co-production systems, \gls{roLabel}
benefits from the lower material and fabrication costs of polymer membranes
compared to the metallic heat exchangers used in thermal systems. Furthermore,
membranes exhibit much higher effective conductance than heat exchangers,
contributing to their overall economic and energetic advantage.

Finally, in terms of environmental impact, \gls{roLabel} membranes are only
partially recyclable. There are established reuse/recycling routes (direct
reuse, conversion to other applications (\eg~\gls{nfLabel}), mechanical
valorisation, thermal/chemical recovery). Recycling is technically feasible and
can bring large environmental benefits versus landfilling or incineration, but
practical recycling rates are limited today by mixed-material construction,
fouling/contamination, cost of separation/transport/pretreatment, and economics
of scaling recycling streams~\sidecite[*-5]{lejarazu-larranaga_thin_2022}. Thermal
desalination technologies, on the other hand, primarily use metals and alloys
that are widely recyclable.

\begin{marginfigure}[-1.5cm]
    \includegraphics[]{desalination-plants-worldwide.png}
    \caption{Desalination technologies used at plants worldwide in 2019\\[1ex]Source:
    Panagopoulos \etal~\cite{panagopoulos_environmental_2020}}
    \labfig{solarmed:intro:plants-worldwide}
\end{marginfigure}

In summary, \gls{roLabel} technology is the most prevalent, with 74~\% ---See
\reffig{solarmed:intro:plants-worldwide}--- of the world's installed capacity
using this technology in 2019, while another 21~\% and 3~\% remained in the use
of thermal technologies (namely, \gls{medLabel} and
\gls{msfLabel})~\cite{panagopoulos_environmental_2020}. However, there has been
a renewed research focus on low-temperature, small-scale thermal desalination.
This resurgence is driven by the increasing availability of low-grade or waste
heat, the desire for robust, low-maintenance systems in off-grid or remote
regions, and the need for brine concentration and minimum-liquid discharge
applications where \gls{roLabel} performance declines sharply. The development
of membrane distillation (\gls{mdLabel}) ---a low temperature thermal separation
process--- is an example of this shift.



%===================================
%===================================
\subsubsection{The Case for \glsentrylong{medLabel}}
\labsec{solarmed:conclusions}

In light of the comparative analysis between thermal and membrane desalination
technologies, \glsentrylong{medLabel} stands out as a resilient and adaptable
option ---particularly for brine concentration and \gls{mldLabel}
applications~\sidecite{zaragoza_coupling_2022}. While \gls{roLabel} dominates
conventional seawater desalination due to its lower specific energy consumption,
thermal systems offer unique advantages in scenarios where high salinity, waste
heat availability, or stringent water quality requirements become critical
factors.

Recent studies reinforce the viability of modern \gls{medLabel} configurations.
Panagopoulos~\sidecite{panagopoulos_process_2020} developed a comprehensive
techno-economic model for a \gls{medLabel}-thermal vapor compression
(\gls{medLabel}-\gls{tvcLabel}) system designed to treat high-salinity brines.
The analysis demonstrated that a four-effect \gls{medLabel}-\gls{tvcLabel} unit
operating with steam at 120~$^\circ$C achieved a freshwater production
cost ($\approx$3.0 USD$_{2020}\cdot$m$^{-3}$) under conventional heat supply.
When integrated with industrial waste heat, the cost decreased substantially to
$\approx$1.7 USD$_{2020}\cdot$m$^{-3}$, with a payback period below two years.
These results confirm that waste-heat-driven \gls{medLabel} represents an
economically viable pathway for sustainable brine management and resource recovery.

From a thermodynamic perspective, \gls{medLabel} processes are capable of
concentrating brines up to \gls{mldLabel} conditions, particularly when
constructed with advanced corrosion-resistant alloys (\eg, super-duplex or
hyper-duplex stainless steels) that withstand chloride concentrations above
18,000~mg/L. Exergy analyses from Panagopoulos~\cite{panagopoulos_process_2020}
show that the largest irreversibilities occur in the thermal vapor compressor
(\gls{tvcLabel}) and evaporation stages (effects); nonetheless, the overall
exergy efficiency remains competitive for low-temperature thermal systems
coupled with a better compatibility with low-grade heat sources.
%  Consequently, \gls{ltLabel}
% \gls{medLabel} systems can be regarded as a mature, robust, and technically
% sound solution for brine concentration and resource recovery.
% , especially when
% coupled with renewable or waste heat sources. 
% Far from being obsolete, modern thermal
% configurations complement mechanical desalination technologies in integrated
% water-energy systems aimed at achieving sustainability, resource circularity,
% and near-zero discharge.

% Overall, thermal desalination processes are characterized by high reliability,
% excellent product water quality, and strong compatibility with co-generation and
% heat-integration strategies. Although their specific energy consumption remains
% higher than that of \gls{roLabel}, this gap diminishes in high-salinity and
% brine concentration applications, where mechanically driven processes suffer
% sharp efficiency losses due to osmotic
% limitations~\sidecite{lienhard_thermodynamics_2017}.

Also, hybrid desalination systems that combine both membrane (\gls{roLabel}) and
thermal (\gls{medLabel}) processes are also seen as a promising path for
combined desalination and brine concentration
applications~\sidecite{nassrullah_energy_2020,feria-diaz_commercial_2021}.

In conclusion, while \gls{roLabel} remains the benchmark technology for
large-scale seawater desalination, thermal and hybrid systems ---particularly
\gls{ltLabel}-\gls{medLabel} and \gls{medLabel}-\gls{tvcLabel}--- play an
indispensable complementary role. They enable the efficient use of low-grade
heat, ensure superior product water quality, and provide a sustainable route for
brine minimization and resource recovery.

%===================================
%===================================
\section{(Variable) Energy sources for thermal separation processes}
\labsec{solarmed:intro:energy-sources}

\begin{marginfigure}[]
    \includegraphics[]{GHG-emissions-per-m3-freshwater.png}
    \caption{\gls{ghgLabel}s emissions per cubic meter of freshwater
    produced\\Source: Panagopoulos
    \etal~\cite{panagopoulos_environmental_2020}.}
    \labfig{solarmed:intro:emissions}
\end{marginfigure}

Coupling desalination plants with renewable energy sources ---such as solar,
geothermal, wind, tidal energy, or alternative sources like industrial waste
heat--- has become an increasingly attractive strategy. Unlike fossil fuels,
renewables are abundant and more sustainable. This advantage is reflected in
\reffig{solarmed:intro:emissions} (green bars), which presents the
\gls{ghgLabel} emissions per~m$^3$ of freshwater produced by major desalination
technologies when powered by renewable energy or waste heat. As shown, the
associated \gls{ghgLabel} emissions are significantly lower compared to
conventional fossil-fuel-based operation.

Furthermore, \reffig{solarmed:intro:plants-renewable-worldwide} illustrates the
global distribution of renewable-driven desalination plants. In contrast to
\reffig{solarmed:intro:plants-worldwide}, this distribution highlights a
larger share of thermal desalination technologies when supplied by
renewable energy sources or waste heat.


% Furthermore, some
% desalination technologies are self-sufficient and use excess energy from one
% stage of the cycle to lower pressure or boost temperature at another stage, such
% as in the thermal-based technologies (\gls{msfLabel} and \gls{medLabel}).

\begin{marginfigure}[]
    \includegraphics{desalination-technologies-with-renewable-worldwide.png}
    \caption{Desalination technologies coupled with renewable energy sources at
    plants worldwide. \\Source: Panagopoulos
    \etal~\cite{panagopoulos_environmental_2020}}
    \labfig{solarmed:intro:plants-renewable-worldwide}
\end{marginfigure}


%================================
\subsection{Solar thermal}
\labsec{solarmed:intro:energy-sources:solar-thermal}

There are two ways in which solar thermal energy can be coupled with thermal
desalination processes. Either a solar field can be purposely built to drive the
desalination process (standalone or partially with other heat source), or using
a co-generation scheme of energy and water in a \gls{cspdLabel} configuration.
This integration offers significant advantages, including reduced costs for
co-producing electricity and freshwater, improved cost-effectiveness through
shared infrastructure and economies of scale, and additional savings in
greenhouse gas emissions. One critical synergy arises from the fact that during
high solar irradiance periods, when solar plants generate maximum (even excess)
power coincides with water scarcity periods, making solar-driven desalination
particularly effective. The choice between \gls{cspLabel}+\gls{medLabel} and
\gls{cspLabel}+\gls{roLabel} is highly dependent on regional
conditions~\sidecite{palenzuela_concentrating_2015}:

\begin{itemize}
    \item In regions with low seawater salinity and lower ambient temperatures
    (like the Mediterranean), \gls{cspLabel}+\gls{roLabel} is generally more
    favorable. The penalty on power production from integrating \gls{medLabel}
    is often higher than the electricity consumption of \gls{roLabel} in these
    conditions.
        
    \item In regions with high seawater salinity and temperature (like the
    Arabian Gulf), \gls{cspLabel}+\gls{medLabel} becomes more attractive. The
    high salinity increases \gls{roLabel}'s electricity consumption, making the
    thermal route more efficient and cost-effective, especially when dry cooling
    is used for the power cycle.
\end{itemize}
    
To mitigate the risks of fully replacing the power plant's condenser with a
desalination unit (which makes power production dependent on the desalination
plant), hybrid configurations like LT-\gls{medLabel}-TVC have been developed.
This concept uses a combination of exhaust steam and extracted steam to drive
the desalination, offering a good balance of efficiency and operational
flexibility. In some cases, especially with dry cooling, it can outperform
\gls{cspLabel}+\gls{roLabel}.


One example of a \gls{cspdLabel} system is Sundrop Farms in Port Augusta, South
Australia, representing the world's first commercial application using
concentrated solar thermal power to co-generate electricity, freshwater, and
heating for horticulture~\sidecite{palenzuela_concentrating_2019}. Its central
solar tower, targeted by 23,000 mirrors, produces steam to generate electricity,
heat and cool greenhouses, and power a desalination plant. The system annually
yields 1,700~MWh of electricity, 250,000 m$^3$ of desalinated water from the
saline Spencer Gulf, and 20,000~MWh of thermal energy. The freshwater is used in
the greenhouses and then recycled, while the resulting brine is managed by
sending it to existing power station outflows, with ongoing research into
mineral recovery.

%================================
\subsection{Waste heat}
\labsec{solarmed:intro:energy-sources:waste-heat}

A significant portion of the world's primary energy
consumption\sidenote{Estimated between 20 and 50~\%} is ultimately released as
waste heat from industrial processes and power generation
facilities~\sidecite{elsaid_recent_2020a, bruckner_industrial_2015}. This
thermal energy, often regarded as a byproduct or liability, represents an
immense and largely untapped resource. Instead of being vented to the
environment, it can be harnessed either directly or through conversion systems
to supply energy for various desalination technologies, thereby lowering
operational costs and reducing the environmental footprint of freshwater
production.

Broadly speaking, two main recovery pathways exist. The first involves direct
heat-to-heat recovery, typically implemented through heat exchangers or heat
pumps. This route is highly efficient and is particularly well suited to
thermally driven desalination processes, such \gls{medLabel}, \gls{msfLabel} and
\gls{mdLabel}. The second approach converts waste heat into mechanical work or
electricity, most commonly using an Organic Rankine cycle systems or other
thermodynamic engines. The generated power can then drive electricity-based
desalination units.

Waste heat has therefore emerged as one of the most promising sustainable heat
sources for low-temperature thermal desalination systems. Its integration into
desalination processes can take several forms. In hybrid configurations, waste
heat can be combined with other renewable sources, such as solar thermal energy,
to increase the overall temperature level or availability of heat supplied to
the thermal separator~\sidecite{christ_boosted_2015}. Alternatively, depending
on the quantity and temperature of the available waste stream, it can operate
standalone, driving thermally based desalination units without external fuel
input.

The temperature grade of waste heat\sidenote{Generally categorized into low
(<100~$^\circ$C), medium (100--400~$^\circ$C), and high (>400~$^\circ$C)} plays a
decisive role in determining its recovery potential and the appropriate
desalination technology~\cite{bruckner_industrial_2015}. This classification is
crucial because it dictates not only the technical feasibility but also the
economic viability of energy recovery.

In any case, effective utilization of waste heat not only enhances overall
energy efficiency but also contributes to decarbonization efforts in the water
and energy sectors. However, challenges remain in terms of temporal
availability, temperature matching, and economic competitiveness.

\annotation{Solar energy and waste heat are free --- so why care about efficiency?}{
{ Is it not solar energy free? Yes and no. This has been a recurring topic of debate
in the literature. Although the fuel, the Sun, is indeed free and
practically inexhaustible, converting that energy into a useful form (whether
electrical, thermal, or otherwise) requires a transformation process that
entails costs.\\

In the case of solar fields, whether photovoltaic or solar-thermal, the more
energy you need, the larger the field area required, and consequently, the higher
the investment cost. Therefore, renewable sources do not provide free energy. A
similar argument applies to waste heat: to obtain usable heat for a thermal
separator, a transformation process is often required, typically involving
costly heat exchangers.\\

In summary, generating heat solely to power a thermal separation
process (renewable or not) is generally inefficient and capital-intensive.\\

A separate question ---explored in \nrefch{solarmed:std}--- is what we actually
mean by efficiency, and how it should be defined according to the energy source.}}