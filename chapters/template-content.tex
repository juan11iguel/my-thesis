%=====================================
%=====================================
%=====================================
%=====================================
\pagelayout{wide} % No margins
\addpart{Template content}
\pagelayout{margin} % Restore margins
%=====================================
%=====================================
%=====================================
%=====================================

\setchapterpreamble[u]{\margintoc}
\chapter{Introduction}
\labch{intro}

\section{The Main Ideas}

Many modern printed textbooks have adopted a layout with prominent 
margins where small figures, tables, remarks and just about everything 
else can be displayed. Arguably, this layout helps to organise the 
	discussion by separating the main text from the ancillary material, 
	which at the same time is very close to the point in the text where 
	it is referenced.

This document does not aim to be an apology of wide margins, for there 
are many better suited authors for this task; the purpose of all these 
words is just to fill the space so that the reader can see how a book 
written with the kaobook class looks like. Meanwhile, I shall also try 
to illustrate the features of the class.

The main ideas behind kaobook come from this 
\href{https://3d.bk.tudelft.nl/ken/en/2016/04/17/a-1.5-column-layout-in-latex.html}{blog 
	post}, and actually the name of the class is dedicated to the author 
of the post, Ken Arroyo Ohori, which has kindly allowed me to create a 
class based on his thesis. Therefore, if you want to know more reasons 
to prefer a 1.5-column layout for your books, be sure to read his blog 
post.

Another source of inspiration, as you may have noticed, is the 
\href{https://github.com/Tufte-LaTeX/tufte-latex}{Tufte-Latex Class}. 
The fact that the design is similar is due to the fact that it is very 
difficult to improve something which is already so good. However, I like 
to think that this class is more flexible than Tufte-Latex. For 
instance, I have tried to use only standard packages and to implement as 
little as possible from scratch;\sidenote{This also means that 
understanding and contributing to the class development is made easier. 
Indeed, many things still need to be improved, so if you are interested, 
check out the repository on github!} therefore, it should be pretty easy 
to customise anything, provided that you read the documentation of the 
package that provides that feature.

In this book I shall illustrate the main features of the class and 
provide information about how to use and change things. Let us get 
started.

\section{What This Class Does}
\labsec{does}

The \Class{kaobook} class focuses more about the document structure than 
about the style. Indeed, it is a well-known \LaTeX\xspace principle that 
structure and style should be separated as much as possible (see also 
\vrefsec{doesnot}). This means that this class will only provide 
commands, environments and in general, the opportunity to do things, 
which the user may or may not use. Actually, some stylistic matters are 
embedded in the class, but the user is able to customise them with ease.

The main features are the following:

\begin{description}
	\item[Page Layout] The text width is reduced to improve readability 
	and make space for the margins, where any sort of elements can be 
	displayed.
	\item[Chapter Headings] As opposed to Tufte-Latex, we provide a 
	variety of chapter headings among which to choose; examples will be 
	seen in later chapters.
	\item[Page Headers] They span the whole page, margins included, and, 
	in twoside mode, display alternatively the chapter and the section 
	name.\sidenote[][-2mm]{This is another departure from Tufte's 
	design.}
	\item[Matters] The commands \Command{frontmatter}, 
	\Command{mainmatter} and \Command{backmatter} have been redefined in 
	order to have automatically wide margins in the main matter, and 
	narrow margins in the front and back matters. However, the page 
	style can be changed at any moment, even in the middle of the 
	document.
	\item[Margin text] We provide commands \Command{sidenote} and 
	\Command{marginnote} to put text in the 
	margins.\sidenote[][-2mm]{Sidenotes (like this!) are numbered while 
	marginnotes are not}
	\item[Margin figs/tabs] A couple of useful environments is 
	\Environment{marginfigure} and \Environment{margintable}, which, not 
	surprisingly, allow you to put figures and tables in the margins 
	(\cfr \reffig{marginmonalisa}).
	\item[Margin toc] Finally, since we have wide margins, why don't add 
	a little table of contents in them? See \Command{margintoc} for 
	that.
	\item[Hyperref] \Package{hyperref} is loaded and by default we try 
	to add bookmarks in a sensible way; in particular, the bookmarks 
	levels are automatically reset at \Command{appendix} and 
	\Command{backmatter}. Moreover, we also provide a small package to 
	ease the hyperreferencing of other parts of the text.
	\item[Bibliography] We want the reader to be able to know what has 
	been cited without having to go to the end of the document every 
	time, so citations go in the margins as well as at the end, as in 
	Tufte-Latex. Unlike that class, however, you are free to customise 
	the citations as you wish.
\end{description}

\begin{marginfigure}[-5.5cm]
	\includegraphics{monalisa}
	\caption[The Mona Lisa]{The Mona Lisa.\\ 
	\url{https://commons.wikimedia.org/wiki/File:Mona_Lisa,_by_Leonardo_da_Vinci,_from_C2RMF_retouched.jpg}}
	\labfig{marginmonalisa}
\end{marginfigure}

The order of the title pages, table of contents and preface can be 
easily changed, as in any \LaTeX\ document. In addition, the class is 
based on \KOMAScript's \Class{scrbook}, therefore it inherits all the 
goodies of that.

\section{What This Class Does Not Do}
\labsec{doesnot}

As anticipated, further customisation of the book is left to the user. 
Indeed, every book may have sidenotes, margin figures and so on, but 
each book will have its own fonts, toc style, special environments and 
so on. For this reason, in addition to the class, we provide only 
sensible defaults, but if these features are not needed, they can be 
left out. These special packages are located in the \Path{style} 
directory, which is organised as follows:

\begin{description}
	\item[kao.sty] This package contains the most important definitions 
    of macros and specifications of page layout. It is the heart of the 
    \Class{kaobook}. Special features include: most commonly used 
    \LaTeX\ packages are already loaded; there is some flexibility to 
    change the default layout; some fancy environments (with coloured 
    boxes around them, floating, and/or with a counter) are predefined.
    \item[kaobiblio.sty] Where commands to print citations in the 
    margins are defined.\sidenote{See \refch{references}.} It is the 
    \Class{kao-} analog of \Package{biblatex}.
    \item[kaorefs.sty] It contains some useful commands to manage 
    labeling and referencing, again to ensure that the same elements are 
    referenced always in a consistent way.
    \item[kaotheorems.sty] For the style of mathematical environments, 
    which can be optionally wrapped in a colourful \Package{mdframed} 
    environment, like in this document, or not.
    \end{description}

\marginnote[2mm]{The audacious users might feel tempted to edit some of 
these packages. I'd be immensely happy if they sent me examples of what 
they have been able to do!}

In the rest of the book, I shall assume that the reader is not a novice 
in the use of \LaTeX, and refer to the documentation of the packages 
used in this class for things that are already explained there. 
Moreover, I assume that the reader is willing to make minor edits to the 
provided packages for styles, environments and commands, if he or she 
does not like the default settings.

\section{How to Use This Class}

Either if you are using the template from 
\href{http://latextemplates.org/template/kaobook}{latextemplates}, or if 
you cloned the GitHub 
\href{https://www.github.com/fmarotta/kaobook}{repository}, there are 
infinite ways to use the \Class{kaobook} class in practice, but we will 
discuss only two of them. The first is to find the \Path{main.tex} file 
which I used to write this book, and edit it; this will probably involve 
a lot of text-deleting, copying-and-pasting, and rewriting. The second 
way is to start almost from scratch and use the 
\Path{./examples/minimal\_book/main.tex} file, which is a cleaned-up 
version of the \Path{./examples/documentation/main.tex}; even if you 
choose the second way, you may find it useful to draw inspiration from 
the \Path{./examples/documentation/main.tex} file.

To compile the document, assuming that its name is \Path{main.tex}, you 
will have to run the following sequence of commands:

\begin{lstlisting}[style=kaolstplain,linewidth=1.5\textwidth]
pdflatex main # Compile template
makeindex main.nlo -s nomencl.ist -o main.nls # Compile nomenclature
makeindex main # Compile index
biber main # Compile bibliography
makeglossaries main # Compile glossary
pdflatex main # Compile template again
pdflatex main # Compile template again
\end{lstlisting}

You may need to compile the template some more times in order for some 
errors to disappear. For any support requests, please ask a question on 
\url{tex.stackexchange.org} with the tag \enquote{kaobook}, open an 
issue on GitHub, or contact the author via e-mail.


\setchapterpreamble[u]{\margintoc}

%=====================================
%=====================================
%=====================================
%=====================================
\pagelayout{wide} % No margins
\addpart{Class Options, Commands and Environments}
\pagelayout{margin} % Restore margins
%=====================================
%=====================================
%=====================================
%=====================================

\chapter{Class Options}
\labch{options}

In this chapter I will describe the most common options used, both the 
ones inherited from \Class{scrbook} and the \Class{kao}-specific ones. 
Options passed to the class modifies its default behaviour; beware 
though that some options may lead to unexpected results\ldots

\section{\Class{KOMA} Options}

The \Class{kaobook} class is based on \Class{scrbook}, therefore it 
understands all of the options you would normally pass to that class. If 
you have a lot of patience, you can read the \KOMAScript\xspace 
guide.\sidenote{The guide can be downloaded from 
\url{https://ctan.org/pkg/koma-script?lang=en}.} Actually, the reading 
of such guide is suggested as it is very instructive.

Every \KOMAScript\xspace option you pass to the class when you load it 
is automatically activated. In addition, in \Class{kaobook} some options 
have modified default values. For instance, the font size is 9.5pt and 
the paragraphs are separated by space,\sidenote[][-7mm]{To be precise, 
they are separated by half a line worth of space: the \Option{parskip} 
value is \enquote{half}.} not marked by indentation.

\section{\Class{kao} Options}

In the future I plan to add more options to set the paragraph formatting 
(justified or ragged) and the position of the margins (inner or outer in 
twoside mode, left or right in oneside mode).\sidenote{As of now, 
paragraphs are justified, formatted with \Command{singlespacing} (from 
the \Package{setspace} package) and \Command{frenchspacing}.}

I take this opportunity to renew the call for help: everyone is 
encouraged to add features or reimplement existing ones, and to send me 
the results. You can find the GitHub repository at 
\url{https://github.com/fmarotta/kaobook}.

\begin{kaobox}[title=To Do]
Implement the \Option{justified} and \Option{margin} options. To be 
consistent with the \KOMAScript\xspace style, they should accept a 
simple switch as a parameter, where the simple switch should be 
\Option{true} or \Option{false}, or one of the other standard values for 
simple switches supported by \KOMAScript. See the \KOMAScript\xspace 
documentation for further information.
\end{kaobox}

The above box is an example of a \Environment{kaobox}, which will be 
discussed more thoroughly in \frefch{mathematics}. Throughout the book I 
shall use these boxes to remarks what still needs to be done.

\section{Other Things Worth Knowing}

A bunch of packages are already loaded in the class because they are 
needed for the implementation. These include:

\begin{itemize}
	\item etoolbox
	\item calc
	\item xifthen
	\item xkeyval
	\item xparse
	\item xstring
\end{itemize}

Many more packages are loaded, but they will be discussed in due time. 
Here, we will mention only one more set of packages, needed to change 
the paragraph formatting (recall that in the future there will be 
options to change this). In particular, the packages we load are:

\begin{itemize}
	\item ragged2e
	\item setspace
	\item hyphenat
	\item microtype
	\item needspace
	\item xspace
	\item xcolor (with options \Option{usenames,dvipsnames})
\end{itemize}

Some of the above packages do not concern paragraph formatting, but we 
nevertheless grouped them with the others. By default, the main text is 
justified and formatted with singlespacing and frenchspacing; the margin 
text is the same, except that the font is a bit smaller.

As a last warning, please be aware that the \Package{cleveref} package 
is not compatible with \Class{kaobook}. You should use the commands 
discussed in \refsec{hyprefs} instead.

\section{Document Structure}

We provide optional arguments to the \Command{title} and 
\Command{author} commands so that you can insert short, plain text 
versions of this fields, which can be used, typically in the half-title 
or somewhere else in the front matter, through the commands 
\Command{@plaintitle} and \Command{@plainauthor}, respectively. The PDF 
properties \Option{pdftitle} and \Option{pdfauthor} are automatically 
set by hyperref to the plain values if present, otherwise to the normal 
values.\sidenote[][*-1]{We think that this is an important point so 
we remark it here. If you compile the document with pdflatex, the PDF 
metadata will be altered so that they match the plain title and author 
you have specified; if you did not specify them, the metadata will be 
set to the normal title and author.}

There are defined two page layouts, \Option{margin} and \Option{wide}, 
and two page styles, \Option{plain} and \Option{fancy}. The layout 
basically concern the width of the margins, while the style refers to 
headers and footer; these issues will be 
discussed in \frefch{layout}.\sidenote[][6mm]{For now, suffice it to say that pages with 
the \Option{margin} layout have wide margins, while with the 
\Option{wide} layout the margins are absent. In \Option{plain} pages the 
headers and footer are suppressed, while in \Option{fancy} pages there 
is a header.} 

The commands \Command{frontmatter}, \Command{mainmatter}, and 
\Command{backmatter} have been redefined in order to automatically 
change page layout and style for these sections of the book. The front 
matter uses the \Option{margin} layout and the \Option{plain} page 
style. In the mainmatter the margins are wide and the headings are 
fancy. In the appendix the style and the layout do not change; however 
we use \Command{bookmarksetup\{startatroot\}} so that the bookmarks of 
the chapters are on the root level (without this, they would be under 
the preceding part). In the backmatter the margins shrink again and we 
also reset the bookmarks root.

\setchapterpreamble[u]{\margintoc}
\chapter{Margin Stuff}

Sidenotes are a distinctive feature of all 1.5-column-layout books. 
Indeed, having wide margins means that some material can be displayed 
there. We use margins for all kind of stuff: sidenotes, marginnotes, 
small tables of contents, citations, and, why not?, special boxes and 
environments.

\section{Sidenotes}

Sidenotes are like footnotes, except that they go in the margin, where 
they are more readable. To insert a sidenote, just use the command 
\Command{sidenote\{Text of the note\}}. You can specify a 
mark\sidenote[O]{This sidenote has a special mark, a big O!} with \\ 
\Command{sidenote[mark]\{Text\}}, but you can also specify an offset, 
which moves the sidenote upwards or downwards, so that the full syntax is:

\begin{lstlisting}[style=kaolstplain]
\sidenote[mark][offset]{Text}
\end{lstlisting}

If you use an offset, you always have to add the brackets for the mark, 
but they can be empty.\sidenote{If you want to know more about the usage 
of the \Command{sidenote} command, read the documentation of the 
\Package{sidenotes} package.}

In \Class{kaobook} we copied a feature from the \Package{snotez} 
package: the possibility to specify a multiple of \Command{baselineskip} 
as an offset. For example, if you want to enter a sidenote with the 
normal mark and move it upwards one line, type:

\begin{lstlisting}[style=kaolstplain]
\sidenote[][*-1]{Text of the sidenote.}
\end{lstlisting}

As we said, sidenotes are handled through the \Package{sidenotes} 
package, which in turn relies on the \Package{marginnote} package.

\section{Marginnotes}

This command is very similar to the previous one. You can create a 
marginnote with \Command{marginnote[offset]\{Text\}}, where the offset 
argument can be left out, or it can be a multiple of 
\Command{baselineskip},\marginnote[-1cm]{While the command for margin 
notes comes from the \Package{marginnote} package, it has been redefined 
in order to change the position of the optional offset argument, which 
now precedes the text of the note, whereas in the original version it 
was at the end. We have also added the possibility to use a multiple of 
\Command{baselineskip} as offset. These things were made only to make 
everything more consistent, so that you have to remember less things!} 
\eg

\begin{lstlisting}[style=kaolstplain]
\marginnote[-12pt]{Text} or \marginnote[*-3]{Text}
\end{lstlisting}

\begin{kaobox}[title=To Do]
A small thing that needs to be done is to renew the \Command{sidenote} 
command so that it takes only one optional argument, the offset. The 
special mark argument can go somewhere else. In other words, we want the 
syntax of \Command{sidenote} to resemble that of \Command{marginnote}.
\end{kaobox}

We load the packages \Package{marginnote}, \Package{marginfix} and 
\Package{placeins}. Since \Package{sidenotes} uses \Package{marginnote}, 
what we said for marginnotes is also valid for sidenotes. Side- and 
margin- notes are shifted slightly upwards 
(\Command{renewcommand\{\textbackslash marginnotevadjust\}\{3pt\}}) in 
order to align them to the bottom of the line of text where the note is 
issued. Importantly, both sidenotes and marginnotes are defined as 
floating if the optional argument (\ie the vertical offset) is left 
blank, but if the offset is specified they are not floating. Recall that 
floats cannot be nested, so in some rare cases you may encounter errors 
about lost floats; in those cases, remember that sidenotes and 
marginnotes are floats. To solve the problem, it may be possible to 
transform them into non-floating elements by specifying an offset of 
0pt.

\section{Footnotes}

Even though they are not displayed in the margin, we will discuss about 
footnotes here, since sidenotes are mainly intended to be a replacement 
of them. Footnotes force the reader to constantly move from one area of 
the page to the other. Arguably, marginnotes solve this issue, so you 
should not use footnotes. Nevertheless, for completeness, we have left 
the standard command \Command{footnote}, just in case you want to put a 
footnote once in a while.\footnote{And this is how they look like. 
Notice that in the PDF file there is a back reference to the text; 
pretty cool, uh?}

\section{Margintoc}

Since we are talking about margins, we introduce here the 
\Command{margintoc} command, which allows one to put small table of 
contents in the margin. Like other commands we have discussed, 
\Command{margintoc} accepts a parameter for the vertical offset, like 
so: \Command{margintoc[offset]}.

The command can be used in any point of the document, but we think it 
makes sense to use it just at the beginning of chapters or parts. In 
this document I make use of a \KOMAScript\xspace feature and put it in 
the chapter preamble, with the following code:

\marginnote{The font used in the margintoc is the same as the one for 
	the chapter entries in the main table of contents at the beginning 
	of the document.}

\begin{lstlisting}[style=kaolstplain]
\setchapterpreamble[u]{\margintoc}
\chapter{Chapter title}
\end{lstlisting}

As the space in the margin is a valuable resource, there is the 
possibility to print a shorter version of the title in the margin toc. 
Thus, there are in total three possible versions for the title of a 
section (or subsection): the one for the main text, the one for the main 
table of contents, and the one for the margintoc. These versions can be 
specified at the same time when the section is created in the source 
\TeX file:
\begin{lstlisting}[style=kaolstplain]
\section[alternative-title-for-toc]{title-as-written-in-text}[alternative-title-for-margintoc]
\end{lstlisting}

By default, the margintoc includes sections and subsections.
If you only want to show sections, add
\begin{lstlisting}[style=kaolstplain]
\setcounter{margintocdepth}{\sectiontocdepth}
\end{lstlisting}
somewhere in your preamble.

\section{Marginlisting}

On some occasions it may happen that you have a very short piece of code 
that doesn't look good in the body of the text because it breaks the 
flow of narration: for that occasions, you can use a 
\Environment{marginlisting}. The support for this feature is still 
limited, especially for the captions, but you can try the following 
code:

\begin{marginlisting}
	\caption{An example of a margin listing.}
	\begin{lstlisting}[language=Python,style=kaolstplain]
print("Hello World!")
	\end{lstlisting}
\end{marginlisting}

\begin{verbatim}
\begin{marginlisting}
	\caption{My caption}
	\begin{lstlisting}[language=Python,style=kaolstplain]
	... code ...
	\end{lstlisting}
\end{marginlisting}
\end{verbatim}

Since we are here, let me spend a few words about listsings. Thanks to contributions from \href{https://github.com/fmarotta/kaobook/pull/225}{Kazuhiko Sakaguchi}, kaobook now supports both \Package{listings} and \Package{minted}. Just pass the option \Option{listing=listings} or \Option{listing=minted} when you load the \Class{kaobook} class, and the appropriate pacakge will be loaded.

Not only textual stuff can be displayed in the margin, but also figures. 
Those will be the focus of the next chapter.

\setchapterimage[7.5cm]{seaside}
\setchapterpreamble[u]{\margintoc}
%\chapter[Figures and Tables]{Figures and Tables\footnotemark[0]}
\chapter{Figures and Tables}

\footnotetext{The credits for the image above the chapter title go to:
	Bushra Feroz, CC~BY-SA~4.0, \url{https://commons.wikimedia.org/w/index.php?curid=68724647}}

\section{Normal Figures and Tables}

Figures and tables can be inserted just like in any standard 
\LaTeX\xspace document. The \Package{graphicx} package is already loaded 
and configured in such a way that the figure width is equal to the 
textwidth and the height is adjusted in order to maintain the original 
aspect ratio. As you may have imagined, the captions will be 
positioned\ldots well, in the margins. This is achieved with the help of 
the \Package{floatrow} package.

Here is a picture of Mona Lisa (\reffig{normalmonalisa}), as an example. 
The captions are formatted as the margin- and the side-notes; If you 
want to change something about captions you can use the command 
\Command{captsetup} from the \Package{caption} package. Remember that if 
you want to reference a figure, the label must come \emph{after} the 
caption!

\begin{figure}[hb]
	\includegraphics[width=0.45\textwidth]{monalisa}
	\caption[Mona Lisa, again]{It's Mona Lisa again. \blindtext}
	\labfig{normalmonalisa}
\end{figure}

While the format of the caption is managed by \Package{caption}, its 
position is handled by the \Package{floatrow} package. Achieving this 
result has been quite hard, but now I am pretty satisfied. In two-side 
mode, the captions are printed in the correct margin.

Tables can be inserted just as easily as figures, as exemplified by the 
following code:

\begin{lstlisting}[caption={Caption of a listing.}]
\begin{table}
\begin{tabular}{ c c c c }
	\toprule
	col1 & col2 & col3 & col 4 \\
	\midrule
	\multirow{3}{4em}{Multiple row} & cell2 & cell3 & cell4\\ &
	cell5 & cell6 & cell7 \\ &
	cell8 & cell9 & cell10 \\
	\multirow{3}{4em}{Multiple row} & cell2 & cell3 & cell4 \\ &
	cell5 & cell6 & cell7 \\ &
	cell8 & cell9 & cell10 \\
	\bottomrule
\end{tabular}
\end{table}
\end{lstlisting}

which results in the useless \vreftab{useless}.

\begin{table}[ht]
\caption[A useless table]{A useless table.}
\labtab{useless}
\begin{tabular}{ c c c c }
	\toprule
	col1 & col2 & col3 & col 4 \\
	\midrule
	\multirow{3}{4em}{Multiple row} & cell2 & cell3 & cell4\\ &
	cell5 & cell6 & cell7 \\ &
	cell8 & cell9 & cell10 \\
	\multirow{3}{4em}{Multiple row} & cell2 & cell3 & cell4 \\ &
	cell5 & cell6 & cell7 \\ &
	cell8 & cell9 & cell10 \\
	\bottomrule
\end{tabular}
\end{table}

I don't have much else to say, so I will just insert some blind text. 
\blindtext

\section{Margin Figures and Tables}

Marginfigures can be inserted with the environment 
\Environment{marginfigure}. In this case, the whole picture is confined 
to the margin and the caption is below it. \reffig{marginmonalisa} is 
obtained with something like this:

\begin{lstlisting}[caption={Another caption.}]
\begin{marginfigure}
	\includegraphics{monalisa}
	\caption[The Mona Lisa]{The Mona Lisa.}
	\labfig{marginmonalisa}
\end{marginfigure}
\end{lstlisting}

There is also the \Environment{margintable} environment, of which 
\reftab{anotheruseless} is an example. Notice how you can place the 
caption above the table by just placing the \Command{caption} command 
before beginning the \Environment{tabular} environment. Usually, figure 
captions are below, while table captions are above. This rule is also 
respected for normal figures and tables: the captions are always on the 
side, but for figure they are aligned to the bottom, while for tables to 
the top.

\begin{margintable}
\caption[Another useless table]{Another useless table.}
\labtab{anotheruseless}
\raggedright
\begin{tabular}{ c c c c }
	\hline
	col1 & col2 & col3 \\
	\hline
	\multirow{3}{4em}{Multiple row} & cell2 & cell3 \\ & cell5 & cell6 
	\\ & cell8 & cell9 \\ \hline
\end{tabular}
\end{margintable}

Marginfigures and tables can be positioned with an optional offset 
command, like so:

\begin{lstlisting}
\begin{marginfigure}[offset]
	\includegraphics{seaside}
\end{marginfigure}
\end{lstlisting}

Offset ca be either a measure or a multiple of \Command{baselineskip}, 
much like with \Command{sidenote}, \Command{marginnote} and 
\Command{margintoc}.\todo{Improve this part.} If you are wondering how I 
inserted this orange bubble, have a look at the \Package{todo} package.

\section{Wide Figures and Tables}

With the environments \Environment{figure*} and \Environment{table*} you 
can insert figures which span the whole page width. For example, here 
are a wide figure and a wide table.

\begin{figure*}[h!]
	\includegraphics{seaside}
	\caption[A wide seaside]{A wide seaside, and a wide caption.
		Credits: By Bushra Feroz, CC BY-SA 4.0, \url{https://commons.wikimedia.org/w/index.php?curid=68724647}}
\end{figure*}

\begin{table*}[h!]
    \caption{A wide table with invented data about three people living in the UK. Note that wide figures and tables are centered and their caption also extends into the margin.}
    \begin{tabular}{p{2.0cm} p{2.0cm} p{2.0cm} p{2.0cm} p{2.0cm} p{2.0cm} p{1.5cm}}
        \toprule
        Name    & Surname   & Job       & Salary           & Age   & Height    & Country \\
        \midrule
        Alice   & Red       & Writer    & 4.000 \pounds    & 34    & 167 cm     & England \\
        Bob     & White     & Bartender & 2.000 \pounds    & 24    & 180 cm     & Scotland \\
        Drake   & Green     & Scientist & 4.000 \pounds    & 26    & 175 cm     & Wales \\
        \bottomrule
    \end{tabular}
\end{table*}

It is the user's responsibility to adjust the width of the table, if 
necessary, until it is aesthetically pleasing. The previous table was 
obtained with the following code:

\begin{lstlisting}[caption=How to typeset a wide table]
\begin{table*}[h!]
    \caption{A wide table with invented data about three people living in the UK. Note that wide figures and tables are centered and their caption also extends into the margin.}
    \begin{tabular}{p{2.0cm} p{2.0cm} p{2.0cm} p{2.0cm} p{2.0cm} p{2.0cm} p{1.5cm}}
        \toprule
        Name    & Surname   & Job       & Salary           & Age   & Height    & Country \\
        \midrule
        Alice   & Red       & Writer    & 4.000 \pounds    & 34    & 167 cm     & England \\
        Bob     & White     & Bartender & 2.000 \pounds    & 24    & 180 cm     & Scotland \\
        Drake   & Green     & Scientist & 4.000 \pounds    & 26    & 175 cm     & Wales \\
        \bottomrule
    \end{tabular}
\end{table*}
\end{lstlisting}

The \Package{floatrow} package provides the \enquote{H} specifier to 
instruct \LaTeX to position the figure (or table) in precisely the same 
position it occupies in the source code. However, this specifier does 
not work with wide figures or tables: you should use \enquote{h!} 
instead, like so: \lstinline|\begin{figure*}[h!]|.

You may have noticed the full width image at the very beginning of this
chapter: that, however, is set up in an entirely different way, which
you'll read about in \vrefch{layout}.

\Class{kaobook} also supports paginated tables (have a look at the 
\Package{longtable} package). The 
\Environment{longtable}\sidenote{Interestingly, \Environment{longtable}s 
may require up to four rounds of compilation before they are typeset 
correctly.} environment behaves a bit differently from 
\Environment{table}, in that \Environment{longtable} encompasses both 
\Environment{table} and \Environment{tabular}, so that you can write, 
\eg,

\begin{lstlisting}[caption=Example of a longtable]
\begin{longtable}{|l c c|}
    \hline
    One & Two & Three \\
    Left & Center & Center \\
    \hline
    \caption{Caption of the longtable.}
\end{longtable}
\end{lstlisting}

to obtain the following table:
\begin{longtable}{|l c c|}
    \hline
    One & Two & Three \\
    Left & Center & Center \\
    \hline
    \caption{Caption of the longtable.}
\end{longtable}

The caption of a \Environment{longtable} is always positioned below the 
table, and it has the same width as the text (it doesn't extend into the 
margin). However, sometimes you may need a \Environment{longtable} that 
is so wide that it trespass into the margins; in those cases, you may 
want to also increase the width of the caption. To do so, you'll have to 
write two additional commands, one before and one after the 
\Environment{longtable}:

\begin{lstlisting}[caption=Increasing the width of the caption of a \Environment{longtable}.]
\floatsetup[longtable]{margins=centering,LTcapwidth=table} % Add this line before the longtable to increase the caption width
\begin{longtable}{lp{8cm}p{5cm}p{2cm}}
...
\end{longtable}
\floatsetup[longtable]{margins=raggedright,LTcapwidth=\textwidth} % Add this line after the longtable to revert the previous change
\end{lstlisting}

Having seen figures and tables, it is now time to tackle 
hyperreferences.

\setchapterstyle{kao}
%\setchapterpreamble[u]{\margintoc}
\chapter{References}
\labch{references}

\section{Citations}

\index{citations}
To cite someone \sidecite{Visscher2008,James2013} is very simple: just 
use the \Command{sidecite}\index{\Command{sidecite}} command. It does 
not have an offset argument yet, but it probably will in the future. 
This command supports multiple entries, as you can see, and by default 
it prints the reference on the margin as well as adding it to the 
bibliography at the end of the document. Note that the citations have 
nothing to do with the text,\sidecite{James2013} but they are completely 
random as they only serve the purpose to illustrate the feature.

For this setup I wrote a separate package, \Package{kaobiblio}, which 
you can find in the \Package{styles} directory and include in your main 
tex file. This package accepts all the options that you can pass to 
\Package{biblatex}, and actually it passes them to \Package{biblatex} 
under the hood. Moreover, it also defines some commands, like 
\Command{sidecite}, and environments that can be used within a 
\Class{kao} book.\sidenote[][-.9cm]{For this reason you should always 
use \Package{kaobiblio} instead of \Package{biblatex}, but the syntax 
and the options are exactly the same.}

If you want to use \Package{bibtex} instead of \Package{biblatex},
pass the option \Option{backend=bibtex} to \Package{kaobiblio}.
\Package{kaobiblio} also supports two options that are not shared with
\Package{biblatex}: \Option{addspace} and \Option{linkeverything},
both of which are boolean options, meaning that they can take
either \enquote{true} or \enquote{false} as a value. If you
pass \Option{addspace=true} when loading \Package{kaobiblio},
a space will be automatically added before the citation marks.
If you pass \Option{linkeverything=true}, the author's name in
the authoryear-* and authortitle-* styles will be a hyperlink
like the year.\sidenote{The fact that the author name is not
a hyperlink bothers more than one biblatex user. There are
\href{https://github.com/plk/biblatex/issues/428}{strong arguments}
\emph{against} hyperlinking the author name, but in my personal opinion, 
linking the author's name does not result in any problems in most 
practical cases.}

As you have seen, the \Command{sidecite} command will print a citation 
in the margin. However, this command would be useless without a way to 
customise the format of the citation, so the \Class{kaobook} provides 
also the \Command{formatmargincitation} command. By \enquote{renewing} 
that command, you can choose which items will be printed in the margins. 
The best way to understand how it works is to see the actual definition 
of this command.

\begin{lstlisting}[style=kaolstplain,linewidth=1.5\textwidth]
\newcommand{\formatmargincitation}[1]{%
	\parencite{#1}: \citeauthor*{#1} (\citeyear{#1}), \citetitle{#1}%
}
\end{lstlisting}

Thus, the \Command{formatmargincitation} accepts one parameter, which is 
the citation key, and prints the parencite followed by a colon, then the 
author, then the year (in brackets), and finally the 
title.\sidecite{Battle2014} Now, suppose that you wish the margin 
citation to display the year and the author, followed by the title, and 
finally a fixed arbitrary string; you would add to your document:

\begin{lstlisting}[style=kaolstplain,linewidth=1.5\textwidth]
\renewcommand{\formatmargincitation}[1]{%
	\citeyear{#1}, \citeauthor*{#1}: \citetitle{#1}; very interesting!%
}
\end{lstlisting}

\renewcommand{\formatmargincitation}[1]{%
	\citeyear{#1}, \citeauthor*{#1}: \citetitle{#1}; very interesting!%
}

The above code results in citations that look like the 
following.\sidecite{Zou2005} Of course, changing the format is most 
useful when you also change the default bibliography style. For 
instance, if you want to use the \enquote{philosophy-modern} style for 
your bibliography, you might have something like this in the preamble:

\begin{lstlisting}[style=kaolstplain,linewidth=1.5\textwidth]
\usepackage[style=philosophy-modern]{styles/kaobiblio}
\renewcommand{\formatmargincitation}[1]{%
	\sdcite{#1}%
}
\addbibresource{main.bib}
\end{lstlisting}

\renewcommand{\formatmargincitation}[1]{%
	\parencite{#1}: \citeauthor*{#1} (\citeyear{#1}), \citetitle{#1}%
}

The commands like \Command{citeyear}, \Command{parencite}
and \Command{sdcite} are just examples. A full
reference of the available commands can be found in this
\href{http://tug.ctan.org/info/biblatex-cheatsheet/biblatex-cheatsheet.pdf}{cheatsheet},
under the \enquote{Citations} section.

Finally, to compile a document containing citations, you need to use an 
external tool, which for this class is biber. You need to run the 
following (assuming that your tex file is called main.tex):

\begin{lstlisting}[style=kaolstplain]
$ pdflatex main
$ biber main
$ pdflatex main
\end{lstlisting}

\section{Glossaries and Indices}

\index{glossary}
The \Class{kaobook} class loads the packages \Package{glossaries} and 
\Package{imakeidx}, with which you can add glossaries and indices to 
your book. For instance, I previously defined some glossary entries and 
now I am going to use them, like this: \gls{computer}. 
\Package{glossaries} also allows you to use acronyms, like the 
following: this is the full version, \acrfull{fpsLabel}, and this is the 
short one \acrshort{fpsLabel}. These entries will appear in the glossary 
in the backmatter.

Unless you use \href{https://www.overleaf.com}{Overleaf} or some other 
fancy IDE for \LaTeX, you need to run an external command from your 
terminal in order to compile a document with a glossary. In particular, 
the commands required are:\sidenote{These are the commands you would run 
in a UNIX system, but see also \nrefsec{compiling}; I have no idea about 
how it works in Windows.}

\begin{lstlisting}[style=kaolstplain]
$ pdflatex main
$ makeglossaries main
$ pdflatex main
\end{lstlisting}

Note that you need not run \texttt{makeglossaries} every time you 
compile your document, but only when you change the glossary entries.

\index{index}
To create an index, you need to insert the command 
\lstinline|\index{subject}| whenever you are talking about 
\enquote{subject} in the text. For instance, at the start of this 
paragraph I would write \lstinline|index{index}|, and an entry would be 
added to the Index in the backmatter. Check it out!

\marginnote[2mm]{In theory, you would need to run an external command 
for the index as well, but luckily the package we suggested, 
	\Package{imakeidx}, can compile the index automatically.}

\index{nomenclature}
A nomenclature is just a special kind of index; you can find one at the end of
this book. To insert a nomenclature, we use the package \Package{nomencl} and
add the terms with the command \Command{nomenclature}. We put then a
\Command{printnomenclature} where we want it to appear.

Also with this package we need to run an external command to compile the 
document, otherwise the nomenclature will not appear:

\begin{lstlisting}[style=kaolstplain]
$ pdflatex main
$ makeindex main.nlo -s nomencl.ist -o main.nls
$ pdflatex main
\end{lstlisting}

These packages are all loaded in 
\href{style/packages.sty}{packages.sty}, one of the files that come with 
this class. However, the configuration of the elements is best done in 
the main.tex file, since each book will have different entries and 
styles.

Note that the \Package{nomencl} package caused problems when the 
document was compiled, so, to make a long story short, I had to prevent 
\Package{scrhack} to load the hack-file for \Package{nomencl}. When 
compiling the document on Overleaf, however, this problem seem to 
vanish.

\marginnote[-19mm]{This brief section was by no means a complete 
reference on the subject, therefore you should consult the documentation 
of the above package to gain a full understanding of how they work.}

\section{Hyperreferences}
\labsec{hyprefs}

\index{hyperreferences}
Together with this class we provide a handy package to help you 
referencing the same elements always in the same way, for consistency 
across the book. First, you can label each element with a specific 
command. For instance, should you want to label a chapter, you would put 
\lstinline|\labch{chapter-title}| right after the \Command{chapter} 
directive. This is just a convenience, because \Command{labch} is
actually just an alias to \lstinline|\label{ch:chapter-title}|, so it 
spares you the writing of \enquote{ch:}. We defined similar commands for 
many typically labeled elements, including:

\begin{multicols}{2}
\setlength{\columnseprule}{0pt}
\begin{itemize}
	\item Page: \Command{labpage}
	\item Part: \Command{labpart}
	\item Chapter: \Command{labch}
	\item Section: \Command{labsec}
	\item Figure: \Command{labfig}
	\item Table: \Command{labtab}
	\item Definition: \Command{labdef}
	\item Assumption: \Command{labassum}
	\item Theorem: \Command{labthm}
	\item Proposition: \Command{labprop}
	\item Lemma: \Command{lablemma}
	\item Remark: \Command{labremark}
	\item Example: \Command{labexample}
	\item Exercise: \Command{labexercise}
\end{itemize}
\end{multicols}

Of course, we have similar commands for referencing those elements. 
However, since the style of the reference should depend on the context, 
we provide different commands to reference the same thing. For instance, 
in some occasions you may want to reference the chapter by name, but 
other times you want to reference it only by number. In general, there 
are four reference style, which we call plain, vario, name, and full.

The plain style references only by number. It is accessed, for chapters, 
with \lstinline|\refch{chapter-title}| (for other elements, the syntax 
is analogous). Such a reference results in: \refch{references}.

The vario and name styles rest upon the \Package{varioref} package. 
Their syntax is \lstinline|\vrefch{chapter-title}| and 
\lstinline|\nrefch{chapter-title}|, and they result in: 
\vrefch{references}, for the vario style, and: \nrefch{references}, for 
the name style. As you can see, the page is referenced in 
\Package{varioref} style.

The full style references everything. You can use it with 
\lstinline|\frefch{chapter-title}| and it looks like this: 
\frefch{references}.

Of course, all the other elements have similar commands (\eg for parts 
you would use \lstinline|\vrefpart{part-title}| or something like that). 
However, not all elements implement all the four styles. The commands 
provided should be enough, but if you want to see what is available or 
to add the missing ones, have a look at the 
\href{styles/kaorefs.sty}{attached package}.

In order to have access to all these features, the \Package{kaorefs} 
should be loaded in the preamble of your document. It should be loaded 
last, or at least after \Package{babel} (or \Package{polyglossia}) and 
\Package{plaintheorems} (or \Package{mdftheorems}). Options can be 
passed to it like to any other package; in particular, it is possible to 
specify the language of the captions. For instance, if you specify 
\enquote{italian} as an option, instead of \enquote{Chapter} it will be 
printed \enquote{Capitolo}, the Italian analog. If you know other 
languages, you are welcome to contribute the translations of these 
captions! Feel free to contact the author of the class for further 
details. 

The \Package{kaorefs} package also include \Package{cleveref}, so it is 
possible to use \Command{cref} in addition to all the previously 
described referencing commands.

\section{A Final Note on Compilation}
\labsec{compiling}

Probably the easiest way to compile a latex document is with the 
\Package{latexmk} script, as it can take care of everything, if properly 
configured, from the bibliography to the glossary. The command to issue, 
in general, is:

\begin{lstlisting}
latexmk [latexmk_options] [filename ...]
\end{lstlisting}

\Package{latexmk} can be extensively configured (see
\url{https://mg.readthedocs.io/latexmk.html}). For convenience, I print 
here an example configuration that would cover all the steps described 
above.

\begin{lstlisting}
# By default compile only the file called 'main.tex'
@default_files = ('main.tex');

# Compile the glossary and acronyms list (package 'glossaries')
add_cus_dep( 'acn', 'acr', 0, 'makeglossaries' );
add_cus_dep( 'glo', 'gls', 0, 'makeglossaries' );
$clean_ext .= " acr acn alg glo gls glg";
sub makeglossaries {
   my ($base_name, $path) = fileparse( $_[0] );
   pushd $path;
   my $return = system "makeglossaries", $base_name;
   popd;
   return $return;
}

# Compile the nomenclature (package 'nomencl')
add_cus_dep( 'nlo', 'nls', 0, 'makenlo2nls' );
sub makenlo2nls {
    system( "makeindex -s nomencl.ist -o \"$_[0].nls\" \"$_[0].nlo\"" );
}
\end{lstlisting}

However, if you'd rather not use an external package and want to do 
everything manually, here are some tips.\sidenote{As the author only 
uses Linux and compiles everything from the command line, he doesn't 
know how the compilation works in Windows or Mac. The tips, therefore, 
refer to the usage with Linux from the command line.}

\minisec{Compiling the examples in the kaobook repository}
To compile the examples, and in particular the documentation, that are 
in the \Path{examples} directory of the 
\href{https://github.com/fmarotta/kaobook}{kaobook repository} on 
GitHub, do as follows. \lstinline[language=bash]|cd| into the root 
directory of the repository, and run
\lstinline|pdflatex -output-directory examples/documentation main.tex|. 
With this trick, you can compile the documentation using the class files 
pertaining to the repository (and not, say, those in your texmf tree). 
The \enquote{-output-directory} option works with the other 
\LaTeX-related commands such as biber and makeglossaries.

A note of warning: sometimes \LaTeX\ needs more than one run to get the
correct position of each element; this is true in particular for the
positioning of floating elements like figures, tables, and margin notes.
Occasionally, \LaTeX\ can need up to four re-runs, so If the alignment
of margin elements looks odd, or if they bleed into the main text, try
running pdflatex one more time.

%=====================================
%=====================================
%=====================================
%=====================================
\pagelayout{wide} % No margins
\addpart{Design and Additional Features}
\pagelayout{margin} % Restore margins
%=====================================
%=====================================
%=====================================
%=====================================

\setchapterimage[6cm]{seaside}
\setchapterpreamble[u]{\margintoc}
\chapter{Page Design}
\labch{layout}

\section{Headings}

So far, in this document I used two different styles for the chapter
headings: one has the chapter name, a rule and, in the margin, the
chapter number; the other has an image at the top of the page, and
the chapter title is printed in a box (like for this chapter). There
is one additional style, which I used only in the \nrefch{appendix};
there, the chapter title is enclosed in two horizontal rules, and
the chapter number (or letter, in the case of the appendix) is above
it.\sidenote[][.7cm]{To be honest, I do not think that mixing heading
styles like this is a wise choice, but in this document I did it only to
show you how they look.}

Every book is unique, so it makes sense to have different styles from 
which to choose. Actually, it would be awesome if whenever a 
\Class{kao}-user designs a new heading style, he or she added it to the 
three styles already present, so that it will be available for new users 
and new books.

The choice of the style is made simple by the \Command{setchapterstyle} 
command. It accepts one option, the name of the style, which can be: 
\enquote{plain}, \enquote{kao}, \enquote{bar}, or 
\enquote{lines}.\sidenote{Plain is the default \LaTeX\xspace title 
style; the other ones are self explanatory.} If instead you want the 
image style, you have to use the command \Command{setchapterimage}, 
which accepts the path to the image as argument; you can also provide an 
optional parameter in square brackets to specify the height of the 
image. \Command{setchapterimage} automatically sets the chapter style to 
\enquote{bar} for that chapter (and also for subsequent chapters).

Let us make some examples. In this book, I begin a normal chapter with 
the lines:
\begin{lstlisting}
\setchapterstyle{kao}
\setchapterpreamble[u]{\margintoc}
\chapter{Title of the Chapter}
\labch{title}
\end{lstlisting}

In Line 1 I choose the style for the title to be \enquote{kao}. Then, I 
specify that I want the margin toc. The rest is ordinary administration 
in \LaTeX, except that I use my own \Command{labch} to label the 
chapter. Actually, the \Command{setchapterpreamble} is a standard 
\KOMAScript\xspace one, so I invite you to read about it in the KOMA
documentation. Once the chapter style is set, it holds until you change 
it.\sidenote{The \Command{margintoc} has to be specified at every 
chapter. Perhaps in the future this may change; it all depends on how 
this feature will be welcomed by the users, so keep in touch with me if 
you have preferences!} Whenever I want to start a chapter with an image, 
I simply write:

\begin{lstlisting}
\setchapterimage[7cm]{path/to/image.png} % Optionally specify the height
\setchapterpreamble[u]{\margintoc}
\chapter{Catchy Title} % No need to set a chapter style
\labch{catchy}
\end{lstlisting}

If you prefer, you can also specify the style at the beginning of the 
main document, and that style will hold until you change it again.

\section{Headers \& Footers}

Headers and footers in \KOMAScript\xspace are handled by the 
\Package{scrlayer-scrpage} package. There are two basic style: 
\enquote{scrheadings} and \enquote{plain.scrheadings}. The former is 
used for normal pages, whereas the latter is used in title pages (those 
where a new chapter starts, for instance) and, at least in this book, in 
the front matter. At any rate, the style can be changed with the 
\Command{pagestyle} command, \eg 
\lstinline|\pagestyle{plain.scrheadings}|.

In both styles, the footer is completely empty. In plain.scrheadings,
also the header is absent (otherwise it wouldn't be so plain\ldots), but 
in the normal style the design is reminiscent of the \enquote{kao} style
for chapter titles.

\section{\Option{twoside} mode}

\begin{kaobox}[title=To Do]
The \Option{twoside} class option is still unstable and may lead to 
unexpected behaviours. Great strides have been done since the first 
version of \Class{kaobook}, but some work still needs to be done. As 
always, any help will be greatly appreciated.
\end{kaobox}

By passing the \Option{twoside} option to the \Class{kaobook}, the
style of left and right pages will be different, similarly to a
printed book. In digital books, having a symmetrical layout for
left and right pages is less important, and you may be tempted
to use the \Option{twoside=false} option. However, keep in mind
that in \enquote{oneside} mode the \Command{uppertitleback} and
\Command{lowertitleback} commands are not available.\sidenote{Another
useful thing to keep in mind is that, when \Option{twoside=true}, an
extra white page will be added to the frontmatter.} If you want to have 
the upper/lower titleback in a one-side document, just add manually the 
contents that you'd put using the upper/lower titleback commands.

\section{Table of Contents}

Another important part of a book is the table of contents. By default, 
in \Class{kaobook} there is an entry for everything: list of figures, 
list of tables, bibliographies, and even the table of contents itself. 
Not everybody might like this, so we will provide a description of the 
changes you need to do in order to enable or disable each of these 
entries. In the following \reftab{tocentries}, each item corresponds to 
a possible entry in the \acrshort{tocLabel}, and its description is the 
command you need to provide to have such entry. These commands are 
specified in the attached \href{style/style.sty}{style 
package},\sidenote{In the same file, you can also choose the titles of 
these entries.} so if you don't want the entries, just comment the 
corresponding lines.

Of course, some packages, like those for glossaries and indices, will 
try to add their own entries.\marginnote{In a later section, we will see 
how you can define your own floating environment, and endow it with an 
entry in the \acrshort{tocLabel}.} In such cases, you have to follow the 
instructions specific to that package. Here, since we have talked about 
glossaries and notations in \refch{references}, we will briefly see how
to configure them.

\begin{table}
\footnotesize
\caption{Commands to add a particular entry to the table of contents.}
\labtab{tocentries}
\begin{tabular}{ l@{\hspace{1mm}}l }
	\toprule
	Entry & Command to Activate \\
	\midrule
	Table of Contents & \lstinline|\setuptoc{toc}{totoc}| \\
    List of Figs/Tabs & \lstinline|\PassOptionsToClass{toc=listof}{\@baseclass}| \\
	Bibliography & \lstinline|\PassOptionsToClass{toc=bibliography}{\@baseclass}| \\
	\bottomrule
\end{tabular}
\end{table}

For the \Package{glossaries} package, use the \enquote{toc} option when 
you load it: \lstinline|\usepackage[toc]{glossaries}|. For 
\Package{nomencl}, pass the \enquote{intoc} option at the moment of 
loading the package. Both \Package{glossaries} and \Package{nomencl} are 
loaded in the attached \href{style/packages.sty}{\enquote{packages} 
package}.

Additional configuration of the table of contents can be performed 
through the packages \Package{etoc}, which is loaded because it is 
needed for the margintocs, or the more traditional \Package{tocbase}. 
Read the respective documentations if you want to be able to change the 
default \acrshort{tocLabel} style.\sidenote[][*-1]{(And please, send me 
a copy of what you have done, I'm so curious!)}

\section{Paper Size}

Recent versions of Kaobook support paper sizes different from the
default A4. It is possible to pass the name of the paper as an option
to the class, as we are accustomed for any other \LaTeX\ class. For
example, the class option \Option{b5paper} would set the paper size
to the B5 format.

We also support the paper sizes specified in
\href{https://www.bod.de/hilfe/hilfe-und-service.html?cmd=SINGLE\&entryID=2494\_GER\_WSS\&eo=2\&title=welche-buchformate-gibt-es}{this
web page} and some additional sizes requested by the users, with the 
option names specified in \reftab{papersizes}.

\begin{margintable}[*-6]
	\caption{Some non-standard paper sizes supported by kaobook.}
	\labtab{papersizes}
	\begin{tabular}{ll}
		\toprule
		Dimension & Option name \\
		\midrule
		12.0cm x 19.0cm & smallpocketpaper \\
		13.5cm x 21.5cm & pocketpaper \\
		14.8cm x 21.0cm & a5paper \\
		15.5cm x 22.0cm & juvenilepaper \\
		17.0cm x 17.0cm & smallphotopaper \\
		21.0cm x 15.0cm & appendixpaper \\
		17.0cm x 22.0cm & cookpaper \\
		19.0cm x 27.0cm & illustratedpaper \\
		17.0cm x 17.0cm & photopaper \\
		16.0cm x 24.0cm & f24paper \\
		%21.0cm x 29.7cm & a4paper \\
		\bottomrule
	\end{tabular}
\end{margintable}

For instance, to use the \enquote{smallpocketpaper} add the correct 
description at the beginning of the documentclass instruction:
\begin{lstlisting}
\documentclass[
		smallpocketpaper,
		fontsize=10pt,
		twoside=false,
		%open=any,
		secnumdepth=1,
]{kaobook}
\end{lstlisting}

Sometimes it is convenient to adopt a landscape view; \Class{kaobook} 
provides two additional options, \Option{a4paperlandscape} and 
\Option{169paperlandscape}, which set the page in landscape mode with 
width-to-height ratios of, respectively, 1.414 and 16:9.

\section{Page Layout}

Besides the page style, you can also change the width of the content of 
a page. This is particularly useful for pages dedicated to part titles, 
where having the 1.5-column layout might be a little awkward, or for 
pages where you only put figures, where it is important to exploit all 
the available space.

In practice, there are two layouts: \enquote{wide} and \enquote{margin}. 
The former suppresses the margins and allocates the full page for 
contents, while the latter is the layout used in most of the pages of 
this book, including this one. The wide layout is also used 
automatically in the front and back matters.

\marginnote{Sometimes it is desirable to increase the width for just one 
or a few paragraphs; the \Environment{widepar} environment does that: 
wrap your paragraphs in this environment, and they will occupy the full 
width of the page.}

To change page layout, use the \Command{pagelayout} command. For 
example, when I start a new part, I write:

\begin{lstlisting}
\pagelayout{wide}
\addpart{Title of the New Part}
\pagelayout{margin}
\end{lstlisting}

Beyond these two basic layouts, it is also possible to finely tune the 
page layout by redefining the \Command{marginlayout} command. This 
command is called internally by the higher-level \Command{pagelayout}, 
and it is responsible for setting the width of the margins and of the 
text. The default definition is:

\begin{lstlisting}
\newcommand{\marginlayout}{%
	\newgeometry{
		top=27.4mm,				% height of the top margin
		bottom=27.4mm,			% height of the bottom margin
		inner=24.8mm,			% width of the inner margin
		textwidth=107mm,		% width of the text
		marginparsep=8.2mm,		% width between text and margin
		marginparwidth=49.4mm,	% width of the margin
	}%
}
\end{lstlisting}

so if you want to, say, decrease the width of the margin while 
increasing the width of the text, you could write in the preamble of 
your document something like:

\begin{lstlisting}
\renewcommand{\marginlayout}{%
	\newgeometry{
		top=27.4mm,				% height of the top margin
		bottom=27.4mm,			% height of the bottom margin
		inner=24.8mm,			% width of the inner margin
		textwidth=117mm,		% width of the text
		marginparsep=8.2mm,		% width between text and margin
		marginparwidth=39.4mm,	% width of the margin
	}%
}
\end{lstlisting}

where the text width has been increased by 10mm and the margin width has 
been decreased by 10mm.

\section{Numbers \& Counters}

In this short section we shall see how dispositions, sidenotes and 
figures are numbered in the \Class{kaobook} class.

By default, dispositions are numbered up to the section in \Class{kaobook}
and up to the subsection in \Class{kaohandt}. This can be changed by
passing the option \Option{secnumdepth} to\Class{kaobook} or
\Class{kaohandt} (e.g. 1 corresponds to section and 2 corresponds to
subsections).

The sidenotes counter is the same across all the document, but if you 
want it to reset at each chapter, just uncomment the line

\begin{lstlisting}[style=kaolstplain]
\counterwithin*{sidenote}{chapter}
\end{lstlisting}

in the \Package{styles/style.sty} package provided by this class.

Figure and Table numbering is also per-chapter; to change that, use 
something like:

\begin{lstlisting}[style=kaolstplain]
\renewcommand{\thefigure}{\arabic{section}.\arabic{figure}}
\end{lstlisting}

\section{White Space}

One of the things that I find most hard in \LaTeX\xspace is to finely 
tune the white space around objects. There are not fixed rules, each 
object needs its own adjustment. Here we shall see how some spaces are 
defined at the moment in this class.\marginnote{Attention! This section 
may be incomplete.}

\textbf{Space around sidenotes and citations marks}

There should be no space before or after sidenotes and citation marks, 
like so:

sidenote\sidenote{This paragraph can be used to diagnose any problems:
if you see whitespace around sidenotes or citation marks, probably
a \% sign is missing somewhere in the definitions of the class
macros.}sidenote\newline
citation\cite{James2013}citation

\textbf{Space around figures and tables}

\begin{lstlisting}[style=kaolstplain]
\renewcommand\FBaskip{.4\topskip}
\renewcommand\FBbskip{\FBaskip}
\end{lstlisting}

\textbf{Space around captions}

\begin{lstlisting}[style=kaolstplain]
\captionsetup{
	aboveskip=6pt,
	belowskip=6pt
}
\end{lstlisting}

\textbf{Space around displays (\eg equations)}

\begin{lstlisting}[style=kaolstplain]
\setlength\abovedisplayskip{6pt plus 2pt minus 4pt}
\setlength\belowdisplayskip{6pt plus 2pt minus 4pt}
\abovedisplayskip 10\p@ \@plus2\p@ \@minus5\p@
\abovedisplayshortskip \z@ \@plus3\p@
\belowdisplayskip \abovedisplayskip
\belowdisplayshortskip 6\p@ \@plus3\p@ \@minus3\p@
\end{lstlisting}

\setchapterstyle{kao}
\setchapterpreamble[u]{\margintoc}
\chapter{Mathematics and Boxes}
\labch{mathematics}

\section{Theorems}

Despite most people complain at the sight of a book full of equations, 
mathematics is an important part of many books. Here, we shall 
illustrate some of the possibilities. We believe that theorems, 
definitions, remarks and examples should be emphasised with a shaded 
background; however, the colour should not be to heavy on the eyes, so 
we have chosen a sort of light yellow.\sidenote{The boxes are all of the 
same colour here, because we did not want our document to look like 
\href{https://en.wikipedia.org/wiki/Harlequin}{Harlequin}.}

\begin{definition}
\labdef{openset}
Let $(X, d)$ be a metric space. A subset $U \subset X$ is an open set 
if, for any $x \in U$ there exists $r > 0$ such that $B(x, r) \subset 
U$. We call the topology associated to d the set $\tau\textsubscript{d}$ 
of all the open subsets of $(X, d).$
\end{definition}

\refdef{openset} is very important. I am not joking, but I have inserted 
this phrase only to show how to reference definitions. The following 
statement is repeated over and over in different environments.

\begin{theorem}
A finite intersection of open sets of (X, d) is an open set of (X, d), 
i.e $\tau\textsubscript{d}$ is closed under finite intersections. Any 
union of open sets of (X, d) is an open set of (X, d).
\end{theorem}

\begin{proposition}
A finite intersection of open sets of (X, d) is an open set of (X, d), 
i.e $\tau\textsubscript{d}$ is closed under finite intersections. Any 
union of open sets of (X, d) is an open set of (X, d).
\end{proposition}

\marginnote{You can even insert footnotes inside the theorem 
environments; they will be displayed at the bottom of the box.}

\begin{lemma}
A finite intersection\footnote{I'm a footnote} of open sets of (X, d) is 
an open set of (X, d), i.e $\tau\textsubscript{d}$ is closed under 
finite intersections. Any union of open sets of (X, d) is an open set of 
(X, d).
\end{lemma}

You can safely ignore the content of the theorems\ldots I assume that if 
you are interested in having theorems in your book, you already know 
something about the classical way to add them. These example should just 
showcase all the things you can do within this class.

\begin{corollary}[Finite Intersection, Countable Union]
A finite intersection of open sets of (X, d) is an open set of (X, d), 
i.e $\tau\textsubscript{d}$ is closed under finite intersections. Any 
union of open sets of (X, d) is an open set of (X, d).
\end{corollary}

\begin{proof}
The proof is left to the reader as a trivial exercise. Hint: \blindtext
\end{proof}

\begin{definition}
Let $(X, d)$ be a metric space. A subset $U \subset X$ is an open set 
if, for any $x \in U$ there exists $r > 0$ such that $B(x, r) \subset 
U$. We call the topology associated to d the set $\tau\textsubscript{d}$ 
of all the open subsets of $(X, d).$
\end{definition}

\marginnote{
	Here is a random equation, just because we can:
	\begin{equation*}
  x = a_0 + \cfrac{1}{a_1
          + \cfrac{1}{a_2
          + \cfrac{1}{a_3 + \cfrac{1}{a_4} } } }
	\end{equation*}
}

\begin{example}
Let $(X, d)$ be a metric space. A subset $U \subset X$ is an open set 
if, for any $x \in U$ there exists $r > 0$ such that $B(x, r) \subset 
U$. We call the topology associated to d the set $\tau\textsubscript{d}$ 
of all the open subsets of $(X, d).$
\end{example}

\begin{remark}
Let $(X, d)$ be a metric space. A subset $U \subset X$ is an open set 
if, for any $x \in U$ there exists $r > 0$ such that $B(x, r) \subset 
U$. We call the topology associated to d the set $\tau\textsubscript{d}$ 
of all the open subsets of $(X, d).$
\end{remark}

As you may have noticed, definitions, example and remarks have 
independent counters; theorems, propositions, lemmas and corollaries 
share the same counter.

\begin{remark}
Here is how an integral looks like inline: $\int_{a}^{b} x^2 dx$, and 
here is the same integral displayed in its own paragraph:
\[\int_{a}^{b} x^2 dx\]
\end{remark}

There is also an environment for exercises.

\begin{exercise}
Prove (or disprove) the Riemann hypothesis.
\end{exercise}

We provide one package for the theorem styles: 
\href{kaotheorems.sty}{kaotheorems.sty}, to which you can pass the 
\Option{framed} option you do want coloured boxes around theorems, like 
in this document.\sidenote{The styles without \Option{framed} are not 
showed, but actually the only difference is that they don't have the 
yellow boxes.} You may want to edit this files according to your taste 
and the general style of the book. However, there is an option to 
customise the background colour of the boxes if you use the 
\Option{framed} option: when you load this package, you can pass it the 
\Option{background=mycolour} option (replace \enquote{mycolour} with the 
actual colour, for instance, \enquote{red!35!white}). This will change 
the colour of all the boxes, but it is also possible to override the 
default colour only for some elements. For instance, the 
\Option{propositionbackground=mycolour} option will change the colour 
for propositions only. There are similar options for theorem, 
definition, lemma, corollary, remark, and example.

\section[Boxes \& Environments]{Boxes \& Custom Environments
\sidenote[][*1.8]{Notice that in the table of contents and in the 
	header, the name of this section is \enquote{Boxes \& Environments}; 
	we achieved this with the optional argument of the \texttt{section} 
	command.}}

Say you want to insert a special section, an optional content or just 
something you want to emphasise. We think that nothing works better than 
a box in these cases. We used \Package{mdframed} to construct the ones 
shown below. You can create and modify such environments by editing the 
provided file \href{kao.sty}{kao.sty}.

\begin{kaobox}[title=Title of the box]
\blindtext
\end{kaobox}

If you set up a counter, you can even create your own numbered 
environment.

\begin{kaocounter}
	\blindtext
\end{kaocounter}

\section{Experiments}

It is possible to wrap marginnotes inside boxes, too. Audacious readers 
are encouraged to try their own experiments and let me know the 
outcomes.

\marginnote[-2.2cm]{
	\begin{kaobox}[title=title of margin note]
		Margin note inside a kaobox.\\
		(Actually, kaobox inside a marginnote!)
	\end{kaobox}
}

I believe that many other special things are possible with the 
\Class{kaobook} class. During its development, I struggled to keep it as 
flexible as possible, so that new features could be added without too 
great an effort. Therefore, I hope that you can find the optimal way to 
express yourselves in writing a book, report or thesis with this class, 
and I am eager to see the outcomes of any experiment that you may try.

%\begin{margintable}
	%\captionsetup{type=table,position=above}
	%\begin{kaobox}
		%\caption{caption}
		%\begin{tabular}{ |c|c|c|c| }
			%\hline
			%col1 & col2 & col3 \\
			%\hline
			%\multirow{3}{4em}{Multiple row} & cell2 & cell3 \\ & cell5 
			%%& cell6 \\ 
			%& cell8 & cell9 \\
			%\hline
		%\end{tabular}
	%\end{kaobox}
%\end{margintable}

%=====================================
%=====================================
%=====================================
%=====================================
\pagelayout{wide} % No margins
\addpart{Appendix}
\pagelayout{margin} % Restore margins
%=====================================
%=====================================
%=====================================
%=====================================

\setchapterstyle{lines}
\labch{appendix}
\blinddocument

\setchapterpreamble[u]{\margintoc}
\chapter{Fonts Testing}

\section{Font Sizes}

{\tiny The quick brown fox jumps over the lazy dog.}

{\scriptsize The quick brown fox jumps over the lazy dog.}

{\footnotesize The quick brown fox jumps over the lazy dog.}

{\small The quick brown fox jumps over the lazy dog.}

{\normalsize The quick brown fox jumps over the lazy dog.}

{\large The quick brown fox jumps over the lazy dog.}

{\Large The quick brown fox jumps over the lazy dog.}

{\LARGE The quick brown fox jumps over the lazy dog.}

{\huge The quick brown fox jumps over the lazy dog.}

{\Huge The quick brown fox jumps over the lazy dog.}


\section{Font Families}

\sffamily\blindtext

\textmd{The quick brown fox jumps over the lazy dog. Medium.}

\textbf{The quick brown fox jumps over the lazy dog. Bold.}

\textup{The quick brown fox jumps over the lazy dog. Upright.}

\textit{The quick brown fox jumps over the lazy dog. Italics.}

\textsl{The quick brown fox jumps over the lazy dog. Slanted.}

\textsc{The quick brown fox jumps over the lazy dog. Small Caps.}

\ttfamily\blindtext

\textmd{The quick brown fox jumps over the lazy dog. Medium.}

\textbf{The quick brown fox jumps over the lazy dog. Bold.}

\textup{The quick brown fox jumps over the lazy dog. Upright.}

\textit{The quick brown fox jumps over the lazy dog. Italics.}

\textsl{The quick brown fox jumps over the lazy dog. Slanted.}

\textsc{The quick brown fox jumps over the lazy dog. Small Caps.}

\rmfamily\blindtext

\textmd{The quick brown fox jumps over the lazy dog. Medium.}

\textbf{The quick brown fox jumps over the lazy dog. Bold.}

\textup{The quick brown fox jumps over the lazy dog. Upright.}

\textit{The quick brown fox jumps over the lazy dog. Italics.}

\textsl{The quick brown fox jumps over the lazy dog. Slanted.}

\textsc{The quick brown fox jumps over the lazy dog. Small Caps.}

\pagelayout{wide}
\setchapterstyle{kao}
\chapter{Test}
\blindtext

