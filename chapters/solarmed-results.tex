% \setchapterpreamble[u]{\margintoc}
% \chapter{Simulation results of the optimal coupling and operation of a solar driven \gls{medLabel} system}
% \labch{solarmed:results}


% %===================================
% %===================================
% \section{Introduction}

% %===================================
% %===================================
% \section{Models validation}
% \labsec{solarmed:results:modelling}


%===================================
%===================================
% \newpage
\section{Optimization results}
\labsec{solarmed:optimization:results}

\subsection{Candidate problems generation}

\begin{figure*}[!b]
    \centering
    \subfloat[\centering Potential operation mode changes for Days 23 (clear) and 26 (cloudy)]{{\includegraphics[width=0.52\linewidth]{figures/solarmed_optim_results_op_modes_candidates.png}}}%
    \hspace{0.01\linewidth}
    \subfloat[\centering Candidate operation plans generated. Only best 10 are shown (best candidate from bottom to top)]{{\includegraphics[width=0.44\linewidth]{figures/solarmed_optim_results_op_plan_candidates.png}}}%
    \caption{Operation plan layer (start-up) candidate problem generation results for specific dates}
    \labfig{solarmed:optimization:results:op-plan-candidates}
\end{figure*}

\reffig{solarmed:optimization:results:op-plan-candidates} shows the initial
evaluation of the operation plan layer -- startup. In
\reffig{solarmed:optimization:results:op-plan-candidates}~(a) it is shown how
the available degrees of freedom are distributed for Days~23 and 26. Three
\gls{dofLabel} are available for the initial startup, two for the shutdown, and
none ---\ie fixed schedule--- for the second day in the horizon. From this
selection the operation plans are generated as shown in
\reffig{solarmed:optimization:results:op-plan-candidates}~(b) -- Day 23, where
the \textcolor[HTML]{666666}{dashed-gray (- -)} vertical lines are equivalent to
the arrows in (a) and the horizontal lines represent the active operation span.
For this particular day the optimization favors the shortest operation time and
most delayed start (Problem 29 in
\reffig{solarmed:optimization:results:op-plan-candidates}~(b)).

Another thing to notice, is how despite the unstable irradiance in Day 26, the
smoothed prediction (1 hour samples) results in overall similar update times---
only being slightly delayed.

\subsection{Choosing an algorithm}

Once the optimization problem(s) are defined, an algorithm must be selected to
explore the solution space and identify a decision vector that minimizes the
objective function.

The solution space has proven to be non-convex, exhibiting numerous local
minima ---poor results were obtained when using gradient-based local search
algorithms. The size of the decision vector depends on the duration of the
active periods, typically around 120 elements. Moreover, simulating two days of
operation (even when inactive periods are skipped) requires approximately
5--10~seconds of computation time. Algorithm-level parallelization is not
advantageous in this context, since many candidate problems are already
evaluated concurrently. The objective, therefore, is to identify a
global large-scale optimization algorithm capable of finding near-optimal
solutions within 200 to 300 objective function evaluations, corresponding to a
total computation time of roughly 2--4~hours.

To select the most suitable algorithm, one candidate problem was arbitrarily
chosen, and a library of global evolutionary algorithms from the open-source
\textit{PyGMO} Python package was tested\sidenote{See \nrefsec{intro:tools}}.
The algorithms considered include: \gls{deLabel}, Self-adaptive Differential
Evolution (SADE), \gls{seaLabel}, Covariance Matrix Adaptation Evolution
Strategy (CMA-ES), and \gls{psoLabel}\sidenote{All described in
\nrefsec{intro:optimization:algorithms}}. The evolution results are shown in
\reffig{solarmed:optimization:results:fitness_comparison}~(a), indicating that,
for this particular problem, the best-performing algorithm is the (N+1)-ES
Simple Evolutionary Algorithm\sidenote{\texttt{sea\_pop\_1\_gen\_1000} in
\reffig{solarmed:optimization:results:fitness_comparison}~(a)}.

\subsection{Choosing a candidate problem}

Once the algorithm is selected, all $n_{\text{problems}}$ can be evaluated,
where the algorithm is required only to determine values for the continuous
process variables. The results of this evaluation are presented in
\reffig{solarmed:optimization:results:fitness_comparison}~(b), showing the
fitness evolution as a function of the number of objective function evaluations
for all 101 problems. Problems~8, 18, and~48 yielded the best fitness values
after the evolution process. Notably, Problem~18 was not among the top
candidates during the initial half of the evaluation, but it eventually
surpassed the others to achieve the second-best fitness. This highlights the
importance of setting correct parameters for the candidate problems ranked
evaluation in order not to early drop potential best-performing candidates.



%================================
\subsection{Simulation results}
\labsec{solarmed:optimization:results:nnlp}

\marginreminder[*-15]{Optimization objective and alternatives}{The aim is to
define an operational strategy that manages both subsystems autonomously and
efficiently and seeks to maximize water production while minimizing
electrical consumption.\\
This is formulated with an objective that minimizes the total operating cost
($J$). Revenue from fresh water production ($q_{med,d}$) sold at price $P_w$
contributes a negative cost term ($J_w$), while electrical consumption
($C_e$) at price $P_e$ contributes a positive cost term ($J_e$). The net
benefit ($B$) is the opposite of the operating cost.\\[1ex]
The operation strategies evaluated are:\\[1ex]
\textbf{Heuristic:} Rule-based approach.\\ 
\textbf{NLP:} Optimized alternative with fixed subsystem start-up and
shutdown schedule based on rules.\\ 
\textbf{nNLP:} Optimized approach where the operation schedule is included in
the decision space.}

\reffig{solarmed:optimization:results:timeseries} shows the time-series results
for one episode composed of seven consecutive days of operation applying the
proposed n\gls{nlpLabel} methodology. As seen in
\reffig{solarmed:optimization:results:timeseries} -- \textit{Environment}, the
solar irradiance remains stable under clear-sky conditions during the first five
days, while the last two are cloudy. The weather corresponds to typical
end-of-summer conditions for Almería (southeast Spain). The seawater inlet
temperature ($T_{med,c,in}$) remains nearly constant at the characteristic
temperature of the Mediterranean Sea (22.1~$^\circ$C on average for the
episode), obtained from the \textit{Copernicus Marine
Programme}~\sidecite[*15]{cnr_mediterranean_2024}. The figure also displays the
results for the alternative \gls{nlpLabel} operation strategies across selected
variables and the fitness evolution comparison for all studied strategies in
\reffig{solarmed:optimization:results:timeseries}~--~\textit{Cumulative
benefit}\sidenote[][*-2]{Constant prices for water and electricity are used, where
$P_w=3\,[u.m./m^3]$ and $P_e=0.05\,[u.m./kWh_e]$, respectively.}. A detailed comparison of these strategies is
presented in the next section.

\begin{figure*}[!htpb]
    \centering
    \savebox\captionqrleft{\qrcode[hyperlink,height=0.5in]{\repositoryBaseUrl/figures/.html}}
    \savebox\captionqrright{\qrcode[hyperlink,height=0.5in]{\repositoryBaseUrl/figures/.html}}
    
    \subfloat[Algorithms fitness evolution \hspace{1ex} \usebox\captionqrleft]{%
        \includegraphics[width=0.50\linewidth]{solarmed-operation_plan-algo_comparison.png}
    }
    \hspace{0.03\linewidth}
    \subfloat[Start-up problems fitness evolution \hspace{1ex} \usebox\captionqrright]{%
    \includegraphics[width=0.42\linewidth]{solarmed-operation_plan-startup-problems_fitness_comparison.png}
    }
    \caption{Fitness evolution comparison for a representative start-up problem.}
    \labfig{solarmed:optimization:results:fitness_comparison}
\end{figure*}

In \reffig{solarmed:optimization:results:timeseries} -- \textit{Solar Field}, it
can be observed that both (\gls{nlpLabel} and n\gls{nlpLabel}) strategies
optimize the solar field operation almost identically, maximizing the outlet
temperature (around 120~$^\circ$C) and achieving large temperature differences
(averaging more than 33~$^\circ$C over the episode). This operation pattern is
consistent with the objective of maximizing energy transfer, which can be
achieved either by increasing the temperature gain in the solar field and by
raising the mass flow rate ---the latter incurring additional pumping costs.

Regarding energy management, \reffig{solarmed:optimization:results:timeseries}
-- \textit{Thermal Storage} shows that during the first day the system starts
with highly charged tanks (significant stored energy) and gradually depletes
them. From the second day onwards, most days follow a similar charge-discharge
cycle: initial charging during the first part of the active period, followed by
discharge toward the end, with varying final storage levels and temperatures. A
clear increasing trend is seen from Days~22 to~24, culminating in a marked rise
on Day~25\sidenote{Particularly in temperature, see
\reffig{solarmed:optimization:results:timeseries}~--~\textit{Hot tank temp.}}.
This behavior is likely a preparation for the drop in irradiance observed at the
beginning of Day~25 (see \reffig{solarmed:optimization:results:timeseries} --
\textit{Environment}). For the remaining two days, the algorithm prioritizes
production on Day~26 while reducing operating time and heat-source
temperatures ---especially on Days~26 and~27. The dotted lines represent the
alternative \gls{nlpLabel} strategy, which maintains a higher amount of stored
energy at the end of each day, resulting in a more balanced storage level
throughout the episode.

\begin{figure*}[!htpb]
    \centering
    % This centers the oversized figure properly
    \makebox[\linewidth][c]{%
        \includegraphics[width=1.2\linewidth]{figures/solarmed_optim_nNLP_timeseries.png}%
    }

    \savebox\captionqr{\qrcode[hyperlink,height=0.5in]{\repositoryBaseUrl/figures/solarmed_optim_nNLP_timeseries.html}}

    \caption[\gls{solarmedLabel} n\gls{nlpLabel} optimization strategy results]{%
    \gls{solarmedLabel} n\gls{nlpLabel} optimization
    strategy results\hspace{.5\linewidth}\usebox\captionqr
    }
    \labfig{solarmed:optimization:results:timeseries}
\end{figure*}


% \FloatBarrier % Prevents figures from jumping to next section

% Operation plan candidates The link should take to a zip with the html files
% for each operation plan evaluation {\newgeometry{margin=0.5cm}  % Remove
% margins completely \begin{sidewaysfigure*} \centering
% \savebox\captionqr{\qrcode[hyperlink,height=0.5in]{\repositoryBaseUrl/figures/solarmed-result.html}}
% \includegraphics[width=.9\paperheight, keepaspectratio]{solarmed-result.png}
% \caption{A rotated figure \hspace{1ex}\usebox\captionqr} \end{sidewaysfigure*}
% \restoregeometry
% }

% {%
% \newgeometry{margin=0.5cm} \begin{figure*} % \centerfloat
% \savebox\captionqr{\qrcode[hyperlink,height=1cm]{\repositoryBaseUrl/figures/solarmed-result.html}}
% \makebox[\textwidth][c]{\includegraphics[angle=90,width=0.7\paperwidth]{solarmed-result.png}}%
% \caption{A rotated figure \hspace{1ex}\usebox\captionqr} \end{figure*}%
% \restoregeometry
% }

% ================================
\subsection{Operation and performance comparison between strategies}
\labsec{solarmed:optimization:results:comparison}

% \marginreminder[*-5]{Strategies nomenclature}{}

To better understand how each alternative operates the system and how this
affects overall performance, different visualizations are presented in
Figures~\ref{fig:solarmed:optimization:results:key_vars_comp},
\ref{fig:solarmed:optimization:results:schedule_comparison}, and
\ref{fig:solarmed:optimization:results:final_comparison}. Each operation
strategy is compared on a day-by-day basis\sidenote[][]{In
Figures~\ref{fig:solarmed:optimization:results:key_vars_comp} and
\ref{fig:solarmed:optimization:results:final_comparison}, the cumulative daily
irradiance is shown in the background as a reference of the available solar
energy, which is the primary determinant of system operation.}.

% Figures description

\reffig{solarmed:optimization:results:key_vars_comp}~(a) displays the initial
and final temperatures of the upper (hot) tank, while
\reffig{solarmed:optimization:results:key_vars_comp}~(b) presents the average
\gls{medLabel} hot-side inlet and outlet temperatures.

\reffig{solarmed:optimization:results:schedule_comparison} shows the daily
operation schedule of each subsystem. Yellow bars correspond to the heat
generation and storage subsystem (\gls{sftsLabel}), while purple bars represent
the \gls{medLabel} desalination subsystem. The start and end of each bar
indicate subsystem start-up and shutdown times, respectively, with the bar
length representing the active period duration\sidenote[][*-3]{Also displayed
numerically next to each bar, and the mean value summarized in the legend}.

Finally, \reffig{solarmed:optimization:results:final_comparison} compares
the cumulative daily benefit (left bars) and the total benefit obtained at the
end of the episode (rightmost yellow bars).

\paragraph{Results discussion.}

As expected, the Heuristic alternative fully depletes thermal storage at the end
of each operation day (see
\reffig{solarmed:optimization:results:key_vars_comp}~(a), Heuristic). This means
that each cycle starts and ends at the same storage level. While this strategy
delivers regular daily benefits under clear conditions (see
\reffig{solarmed:optimization:results:final_comparison}, Heuristic), it is
strongly penalized during unstable conditions (\eg, Day~27), where the benefit
drops to nearly half of the other alternatives. This occurs because the
Heuristic \gls{medLabel} operation (see
\reffig{solarmed:optimization:results:key_vars_comp}~(b)) consistently targets
higher hot-side inlet temperatures than the optimized approaches. As a result,
the \gls{medLabel} operates for significantly shorter periods (\eg, 6.9~hours on
Day~27, compared to over 9~hours on average for other days; see
\reffig{solarmed:optimization:results:schedule_comparison} -- \gls{medLabel} --
Heuristic).

The comparison also highlights the differences between the \gls{nlpLabel} and
n\gls{nlpLabel} alternatives. The n\gls{nlpLabel} strategy consistently operates
the \gls{medLabel} at lower temperatures and for longer durations (9.8~h vs.
8.1~h, see \reffig{solarmed:optimization:results:schedule_comparison}). This is
particularly evident on the cloudy Day~27, where the nNLP maintains operation at
the lowest temperature ($T_{med,s,in} = 66.4~^{\circ}\mathrm{C}$).


Overall, the Heuristic approach tends to delay \gls{medLabel} start-up relative
to the other strategies, while keeping it active later into the evening.
Although it operates the \gls{medLabel} longer than the \gls{nlpLabel}, this
does not lead to higher benefits (see
\reffig{solarmed:optimization:results:final_comparison}, Heuristic
vs.~\gls{nlpLabel}).


Another interesting observation is the difference in \gls{medLabel} heat source
temperature span between the Heuristic and optimized strategies: 2.34~$^\circ$C
for the Heuristic versus 3.51~$^\circ$C (\gls{nlpLabel}) and 3.32~$^\circ$C
(n\gls{nlpLabel}) ---see
\reffig{solarmed:optimization:results:schedule_comparison}. This indicates that
maximizing \gls{solarmedLabel} system performance is achieved by operating for
longer periods at lower temperatures and with high temperature differences--- in
both the solar field and \gls{medLabel} sides--- thereby maximizing the
utilization of sensible heat.

\bigskip
\begin{figure*}[htpb]
    \centering
    
    \savebox\captionqrleft{\qrcode[hyperlink,height=0.5in]{\repositoryBaseUrl/figures/solarmed_optim_thermal_storage_temp_comparison.html}}
    \savebox\captionqrright{\qrcode[hyperlink,height=0.5in]{\repositoryBaseUrl/figures/thermal_storage_validation_20230505.html}}

    \subfloat[\centering Energy
    management\hspace{1ex} \usebox\captionqrleft]{{\includegraphics[width=0.48\linewidth]{figures/solarmed_optim_thermal_storage_temp_comparison.png}}}%
    \hspace{0.01\linewidth}
    \subfloat[\centering \gls{medLabel}
    operation\hspace{1ex} \usebox\captionqrright]{{\includegraphics[width=0.48\linewidth]{figures/solarmed_optim_heat_source_in_out_temp_comparison.png}}}%
    \caption[Daily key variables differences comparison]{Daily key variables
    differences comparison}

    % \hfill\usebox\captionqrleft\hspace{1ex}\usebox\captionqrright}
    \labfig{solarmed:optimization:results:key_vars_comp}
\end{figure*}


A key distinction between the \gls{nlpLabel} and n\gls{nlpLabel} strategies
appears during cloudy conditions (Day~27). The n\gls{nlpLabel} strategy depletes
storage on the previous day ---Day~26, see
\reffig{solarmed:optimization:results:key_vars_comp}~(a)--- and adapts the
following day by delaying start-up more significantly than on other days ---Day
27 in \reffig{solarmed:optimization:results:schedule_comparison}--- while
operating at a lower temperature ---Day~27 in
\reffig{solarmed:optimization:results:key_vars_comp}~(b). In contrast, the
\gls{nlpLabel} strategy reserves energy, allowing it to operate in a more
consistent manner across the episode.

\begin{marginfigure}[*-7]
    \includegraphics[]{figures/solarmed_optim_startup_shutdown_comparison.png}
    \caption[Operation schedule comparison]{Operation schedule comparison between optimization strategies}
    \labfig{solarmed:optimization:results:schedule_comparison}
\end{marginfigure}

As shown in Figures~\ref{fig:solarmed:optimization:results:schedule_comparison}
and \ref{fig:solarmed:optimization:results:final_comparison}, the \gls{nlpLabel}
strategy generally stops subsystem operation earlier than the n\gls{nlpLabel}
alternative. Adjusting the shutdown threshold to match the average stop time of
the n\gls{nlpLabel} strategy could potentially improve \gls{nlpLabel}
performance. Still, as observed in
\reffig{solarmed:optimization:results:schedule_comparison}, there is notable
variability in start and shutdown times, particularly on cloudy days (Days~27
and~28). This adaptability to changing conditions is something the fixed
\gls{nlpLabel} alternative can only approximate through heuristic rules (\eg,
advancing or delaying operation based on predicted irradiance availability).
Overall, the \gls{nlpLabel} strategy manages stored energy and \gls{medLabel}
operation in a consistent but moderately adaptive manner.

By far, the best-performing strategy (see
\reffig{solarmed:optimization:results:final_comparison}) is the n\gls{nlpLabel}
alternative, achieving a total benefit of 508.99~u.m., followed by the
\gls{nlpLabel} (403.97~u.m.) and Heuristic (346.86~u.m.) alternatives. As
mentioned earlier, some performance gap could be reduced by incorporating
smarter rule-based logic into the \gls{nlpLabel} and Heuristic
strategies\sidenote[][*8]{The results presented here can serve as useful
guidance for defining such rules.}, but they would still remain suboptimal
compared to the n\gls{nlpLabel} approach. Since the \gls{nlpLabel} is
effectively a subset of the n\gls{nlpLabel}, it cannot consistently outperform
it.


\paragraph{Conclusions.}

The n\gls{nlpLabel} alternative outperforms the \gls{nlpLabel} and Heuristic
strategies by 21~\% and 32~\%, respectively. These results correspond to a
seven-day episode characterized mainly by clear-sky conditions, with cloudy days
occurring only toward the end of the period. It is expected that evaluating
performance over an entire year ---with its substantially higher environmental
variability--- would further accentuate the advantages of the n\gls{nlpLabel}
strategy proposed in this chapter.

This study demonstrated that sensible-driven thermal separation processes are
optimally operated when temperature differences are maximized. This finding
contrasts with conventional operation strategies for thermal (latent-driven)
desalination systems, which typically consider the process in isolation and aim
to minimize energy consumption. When the entire system is evaluated from the
perspective of primary energy utilization, the optimal operation of a
solar-driven \gls{medLabel} unit more closely resembles that of a
waste-heat-driven process.

\begin{figure}[!htpb]
    \includegraphics[width=\textwidth]{figures/solarmed_optim_final_comparison.png}
    \caption[Daily and total cumulative benefits comparison]{Daily and total cumulative benefits comparison. \textit{u.m.} represents arbitrary monetary units.}
    \labfig{solarmed:optimization:results:final_comparison}
\end{figure}

Furthermore, it was shown that achieving optimal performance in a thermally
driven separation process powered by variable energy sources requires a
combination of factors: a sufficiently long optimization horizon, careful
selection of decision variables, flexible scheduling of subsystem operation, and
a fitness function that accurately reflects performance relative to primary
energy input. Together, these elements enable a truly optimized and resilient
operation of solar-thermal separation systems.

