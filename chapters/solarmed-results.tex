% \setchapterpreamble[u]{\margintoc}
% \chapter{Simulation results of the optimal coupling and operation of a solar driven \gls{medLabel} system}
% \labch{solarmed:results}


% %===================================
% %===================================
% \section{Introduction}

% %===================================
% %===================================
% \section{Models validation}
% \labsec{solarmed:results:modelling}


%===================================
%===================================
\section{Optimization results}
\labsec{solarmed:optimization:results}

\subsection{Choosing an algorithm}

Once the optimization problem(s) is defined, an algorithm must be chosen that
explores the solution space and finds a decision vector that minimizes the
objective function.


\begin{figure*}
    \centering
    \savebox\captionqrleft{\qrcode[hyperlink,height=0.3in]{\repositoryBaseUrl/figures/.html}}
    \savebox\captionqrright{\qrcode[hyperlink,height=0.3in]{\repositoryBaseUrl/figures/.html}}

    \subfloat[Algorithms fitness evolution \hspace{1ex} \usebox\captionqrleft]{%
        \includegraphics[width=0.50\linewidth]{solarmed-operation_plan-algo_comparison.png}
    }
    \hspace{0.03\linewidth}
    \subfloat[Startup-problems fitness evolution \hspace{1ex} \usebox\captionqrright]{%
        \includegraphics[width=0.42\linewidth]{solarmed-operation_plan-startup-problems_fitness_comparison.png}
    }

    \caption{Fitness evolution for a particular startup-problem}
\end{figure*}

The solution space has proven to be non-convex, with many local minimums (poor
results were obtained when using local-gradient-based algorithms). The size of
the decision vector depends on the active periods duration, around 120
elements. In addition, simulation of two days of operation (even when inactive
periods are skipped) requires 5-10 seconds of computation time. Algorithm
parallelization capabilities are of no use in this case, since many candidate
problems will already be evaluated in parallel. The objective is then to find a
global large-scale optimization algorithm that can find near-optimal solutions
with 200 to 300 objective function evaluations (totaling 2-4 hours of
computation time). In order to find the best algorithm, one of the candidate
problems is arbitrary chosen and a library of global-evolutionary optimization
algorithms is used from the PyGMO open-source Python library, specifically:
Differential Evolution (DE), Self-adaptive DE (SADE), (N+1)-ES Simple
Evolutionary Algorithm (SAE), Covariance Matrix Adaptation Evolution Strategy
(CMA-ES) and Particle Swarm Optimization (PSO). Evolution results are shown in
Fig.\ref{fig:algo_comparison}, showcasing that for this particular problem the
best alternative is the (N+1)-ES Simple Evolutionary Algorithm.

\subsection{Choosing a candidate problem}

Once an algorithm was chosen, all $n_{problems}$ were evaluated where the
algorithm is only required to choose values for the process variables
(continuous). The results of this evaluation are shown in
Fig.\ref{fig:problems_comparison}, 101 problems were evaluated and visualized
is their fitness evolution as a function of objective function evaluations.
Problems 8, 18 and 48 resulted in the best fitness after the evolution process
and their operation plan can be visualized in Fig.
\ref{fig:best_candidates_op_plan}.


%================================
\subsection{Simulation results}
\labsec{solarmed:optimization:results}

\reffig{} shows results for the simulated system in a total of X days. Where
the first two days present favorable - sunny -
conditions, followed by a cloudy day, and finishing with a sunny day (\reffig{}
- \textit{Environment}).


% Operation plan candidates
% The link should take to a zip with the html files for each operation plan evaluation
% {\newgeometry{margin=0.5cm}  % Remove margins completely
% \begin{sidewaysfigure*}
%     \centering
%     \savebox\captionqr{\qrcode[hyperlink,height=0.5in]{\repositoryBaseUrl/figures/solarmed-result.html}}
%     \includegraphics[width=.9\paperheight, keepaspectratio]{solarmed-result.png}
%     \caption{A rotated figure \hspace{1ex}\usebox\captionqr}
% \end{sidewaysfigure*}
% \restoregeometry
% }

{%
\newgeometry{margin=0.5cm}
\begin{figure*}
    % \centerfloat
    \savebox\captionqr{\qrcode[hyperlink,height=1cm]{\repositoryBaseUrl/figures/solarmed-result.html}}
    \makebox[\textwidth][c]{\includegraphics[angle=90,width=0.7\paperwidth]{solarmed-result.png}}%
    \caption{A rotated figure \hspace{1ex}\usebox\captionqr}
\end{figure*}%
\restoregeometry
}

%================================
\subsection{Performance comparison with alternative strategies}
\labsec{solarmed:optimization:results:comparison}

baseline operation and just operation optimization