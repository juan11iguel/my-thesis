\chapter*{How to read this thesis}
\addcontentsline{toc}{chapter}{How to read this thesis}


% I believe that a document of this size will be more often viewed on computer
% than on paper. This how I personally view most papers I read anyway.
% As such, I believe that the document should be adapted for such a medium, unlike
% most \LaTeX{} documents.
% While trying to make my own kind of document, I stumbled upon the great
% \href{https://github.com/fmarotta/kaobook/}{kaobook} class which corresponds
% almost exactly to what I was looking for: citations, notes, reminders can be put
% in a large margin.

% Citations like~\sidecite{KOMAScriptDoc} will go in the
% margin, as well as in the \refbib.
% The margin also contain side notes\sidenote{
%   No more going back and forth between the end and top of the page.
% }, replacing the usual footnotes.

% I will also use margin notes that are not anchored in the main document but are
% usually relevant to the paragraph or figure on their left.
% \marginnote[-0.4cm]{
%   Margin notes are not placed entirely automatically. I tend to adjust the
%   height so it is at the right place.
% }

% \begin{theorem}
%   Theorems and definitions will be in these yellow boxes.
% \end{theorem}

% On the other hand, I will sometimes refer to concepts for the first time without
% taking the time to introduce them because they are secondary to the main point.
% In some cases those will be accompanied by a short definition on the side, in
% a yellowish box.