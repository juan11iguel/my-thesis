\chapter*{How to read this thesis}
\addcontentsline{toc}{chapter}{How to read this thesis}

\tldrbox{
    This preliminary chapter explains how to read this thesis, mainly the
    different environment boxes used throughout the manuscript, why the large
    margins, what is placed in them, and how to use the interactive features
    of the manuscript. This is an example of a \gls{tldrLabel} box. It contains
    an Abstract/Summary of the main point of the chapter and are placed at the
    beginning of every chapter.
}

This \LaTeX{} template is designed with large margins, on the one hand this
allows to have shorter lines, which makes for an easier reading experience but
most interestingly, it also allows to place additional information in the
margins, such as side notes, side citations, figures, tables… your
imagination is the limit! Or rather \LaTeX{} compilation errors and your
patience are. Throughout this manuscript I will add side notes\sidenote{
  Like this one! They are like footnotes, but placed in the margin of the page
} to provide additional information and comments that would otherwise be too
distracting and verbose to include in the main text, constantly interrupting
the flow of the reading. The side notes are not essential to understand the
content of the document, but mostly complementary.

\section*{Boxed environments}

Both problem definition boxes (\eg ref) and model definition boxes (\eg
\refmod{example}) are countered environments and can (and will) be referenced
in the text.

\begin{marginfigure}[]
    \includegraphics[]{figures/WASCOP-facility-WCT.png}
    % \caption{Back view of the WCT}
    \savebox\captionqr{\qrcode[hyperlink,height=0.4in]{\repositoryBaseUrl/figures/day_viz_20220614_eval_at_20250414T1247_test_water_price.html}}
    % \caption{\parbox{0.8\textwidth}{Line 1 of caption\\Line 2 of caption}}
	\caption{Example figure. Try clicking or scanning the QR code to access the
	interactive version.\\[1ex]\usebox\captionqr}
    \labfig{how-to-read:example-figure}
\end{marginfigure}


\problemdefinitionbox{Problem definition box example}{
    This is an example of a problem definition box. It is used to formally and
    concisely define an optimization problem.
  \begin{equation*}
    \min_{\mathbf{x},\, \mathbf{e};\, \boldsymbol{\theta}} \quad J = f(\mathbf{x}, \mathbf{e}; \boldsymbol{\theta}) = XXXX
  \end{equation*}

  \textbf{with}:
  \begin{align*}
      out_1,\, out_2 &= f(in_1,\, in_2,\, \ldots,\, in_N)\\
      out_1,\, out_2 &= f(in_1,\, in_2,\, \ldots,\, in_N)
  \end{align*}
  \begin{itemize}
    \item Decision variables
    \[
      \mathbf{x} = [x_1,\, x_2]
    \]
    \item Environment variables
    \[
      \mathbf{e} = [e_1,\, e_2,\, \ldots,\, e_3]
    \]
    \item Fixed parameters
    \[
      \boldsymbol{\theta} = [\theta_1 = X,\, \theta_2 = Y]
    \]
  \end{itemize}

  \textbf{subject to}:
  \begin{itemize}
    \item Box-bounds
    \begin{itemize}
      \item $x_{1} \in [\underline{x}_{1}, \overline{x}_{1}]$
      \item $x_{2} \in [\underline{x}_{2}, \overline{x}_{2}]$
    \end{itemize}
    \item Constraints
    \begin{itemize}
      \item $\left| out_X - out_Y \right| \leq \epsilon_1$
      \item $out_X \leq out_Z - \Delta Z$
    \end{itemize}
  \end{itemize}
}

\begin{modelcounter}{Model definition box example}
    \begin{equation*}
        out_{1},\,out_{2} = \text{some\:cool\:model}(in_{1},\, in_{2},\, in_{3})
    \end{equation*}
    \labmod{example}
\end{modelcounter}

\begin{kaobox}[title=Other boxes]
    Other boxes are used to highlight important points, or to provide
    additional information that is not essential to the main text.
\end{kaobox}

% On the other hand, I will sometimes refer to concepts for the first time without
% taking the time to introduce them because they are secondary to the main point.
% In some cases those will be accompanied by a short definition on the side, in
% a yellowish box.

In order to make the book more interactive and link-friendly, I have enabled
hyperlinks in the PDF. This means that you can click on the references,
citations, and links to external resources, and they will take you to the
corresponding location. This is standard latex, however to maintain a
consistent experience in the physical version, QR codes are inserted in the
margin next to the links. The reader is invited to scan them with a QR code
reader to access the corresponding online resource\sidenote[][*-8]{
    I believe that this is a good way to make the document more
accessible and to encourage readers to explore the content in more depth.
However, the interactive features are \textbf{optional} and not necessary to
understand the content of the document.
}. Some figures also include QR codes that link to an interactive (HTML)
version of the figure, see \reffig{how-to-read:example-figure} as an example.

The additional material as
well as the source code of this document are hosted in a \href{\repositoryBaseUrl}{Zenodo repository}\sidenote[][*-2]{
    \qrcode[hyperlink,height=0.5in]{\repositoryBaseUrl}
}. Alternatively, a mirror repository is also available at: \\
\begin{center}
    \href{https://github.com/juan11iguel/my-thesis}{https://github.com/juan11iguel/my-thesis} \\
\end{center}
It seems unlikely that both Zenodo and GitHub will go down at a time where this
document is still relevant, and if they do, I think there will be more
important things to worry about than losing access to the interactive content
of this thesis.\sidenote[{\color{gray} \faToiletPaper}]{Like hoarding toilet paper}
