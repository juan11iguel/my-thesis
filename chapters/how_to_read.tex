\chapter{How to read this thesis}
\labch{how_to_read}

% Structure

% Description of the 1.5 layout with references
% Justification of why it's a better way to consume content
% Guided tour of the layout and elements used throught the thesis

This document makes use of a so-called \textit{1.5 column
layout}~\sidecite{kenarroyoohori_column_2016}. This layout helps to organise the
discussion by separating the main text from ancillary material, which at the
same time is very close to the point in text where it is referenced: citations,
notes, reminders, figures and tables can be put in a large margin.
 

Citations like~\sidecite{kenarroyoohori_column_2016} will go in the
margin, as well as in the \refbib.
The margin also contain side notes\sidenote{
    The idea is that the reader is able to know what has 
	been cited without having to go to the end of the document every 
	time, so citations go in the margins as well as at the end.
}, replacing the usual footnotes.


\reminder[-0.7cm]{of something}{
  Body of the reminder.
}
I will also use the margin to set up reminders in green boxes. Their purpose
is to avoid the reader having to go back to a previous chapter they might not
have even read and still grasp the meaning of the statements at hand.

\sidedef[-0.7cm]{of something else}{
  This is supposedly a definition
}
On the other hand, I will sometimes refer to concepts for the first time without
taking the time to introduce them because they are secondary to the main point.
In some cases those will be accompanied by a short definition on the side, in
a yellowish box.