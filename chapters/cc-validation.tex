\setchapterpreamble[u]{\margintoc}
\chapter{Validation in the combined cooling pilot plant} % Optimization strategy experimental validation
\labch{cc:validation}

\begin{kaobox}[title=To Do]
After the chapter is complete, find and replace all mentions to RBF, ANN, RMSE
and all other acronyms with the acronym with \Command{gls}.
\end{kaobox}

% ====================================
% ====================================
\section{Modelling}

The two main components of the system (\gls{wctLabel} and \gls{dcLabel}) are
modelled with different approaches and compared in detail. Afterward, the
integration of the selected modelling approach with the rest of the system
components (\refsec{cc:modelling:other-components}) is validated in
\nrefsec{cc:validation:complete-system}.

% \begin{enumerate}
% 	\item FP model
% 	\item DB from experimental campaign
% 	\item DB from FP
% 	\item Comparison
% \end{enumerate}


% ================================
\subsection{Wet cooler model alternatives comparison and validation}
\labsec{cc:validation:wct}

\subsubsection{Physical model}

As previously mentioned\sidenote{See \nrefsec{cc:facility:exp-wct}}, three
experimental campaigns have been performed, shown in
\reffig{cc:validation:modelling:method} as \hyperref[sec:cc:facility:exp:1]{Exp
1}, \hyperref[sec:cc:facility:exp:2]{\hyperref[sec:cc:facility:exp:2]{Exp 2}},
and \hyperref[sec:cc:facility:exp:3]{\hyperref[sec:cc:facility:exp:3]{Exp 3}}.
\hyperref[sec:cc:facility:exp:1]{Exp 1} corresponds to the Poppe model
calibration campaign and it was designed for the calibration of the first
principles model. The aims of such campaign was to fit a function (mapping)
that relates the air mass flow rate at the outlet of the tower, $\dot{m}_a$,
with the frequency of the fan, $f_{fan}$:\todo{Lidia, aquí la correlación no
usa la temperatura ambiente}

\begin{figure}
  \includegraphics[]{figures/wascop-wct-exp-campaigns-diagram.png}
  \caption{Calibration, tuning, validation and comparison procedure}
  \labfig{cc:validation:modelling:method} 
\end{figure}

\begin{equation} 
    \dot{m}_a=-0.0014 f_{fan}^2+0.1743f_{fan}-0.7251.
    \labeq{cc:maffan} 
\end{equation}

and to calibrate a WCT performance coefficient: the Merkel number, $\Me$.
\reffig{cc:validation:Me_LG}\todo{Make text in figure larger} shows the
variation of the Merkel number as a function of the water-to-air mass flow
ratio ($\dot{m}_w/\dot{m}_a$) using data from
\hyperref[sec:cc:facility:exp:1]{Exp1}. As can be seen, the $\Me$ decreases
with $\dot{m}_w/\dot{m}_a$ values following a linear trend on log-log scale.   

% TODO: Make text in this figure larger
\begin{marginfigure}
    \includegraphics[width=1.1\linewidth]{figures/cc-poppe-Me_vs_mwma.png}
    \caption{Experimental results for the $\Me$ number as a function of $\dot{m}_w/\dot{m}_a$.}
    \labfig{cc:validation:Me_LG} 
\end{marginfigure}

Following the correlation for the Merkel number of a wet cooling tower
described in \refsec{cc:modelling:wct:poppe}, the parameters $c$ and $n$ obtained
from the data fitting are 1.516 and 0.693, respectively.

\subsubsection{Data-driven}

In order to generate the data-driven from first-principles alternative, the
most relevant input variables identified in \refsec{cc:modelling:wct:samples}
are discretized using a fixed number of resolution steps for each variable,
within ranges based on expected operating conditions, as
defined in \reftab{cc:validation:wct:samples-params}.

\begin{margintable}[]
    \caption{Bounds and discretization of the model input variables.}
    \labtab{cc:validation:wct:samples-params}
    \resizebox{\linewidth}{!}{%
    \begin{tabular}{lcccc}
        \toprule
        $\mathbf{x}$ & \textbf{Units} & \textbf{lb} & \textbf{ub} & \textbf{n} \\
        \midrule
        $T_{amb}$       & $^\circ$C            & 3   & 50   & 7 \\
        $\Delta T_{amb-in}$       & $^\circ$C            & 3   & 30    & 7  \\
        $q_{dc}$       & m$^3$/h            & 0.00   & 1.00    & 5  \\
        $\omega_{\text{dc}}$   & \%         & 11.00  & 99.18   & 10 \\
        $\omega_{\text{wct}}$  & \%         & 21.00  & 93.42   & 10 \\
        \bottomrule
    \end{tabular}
    }
\end{margintable}

\begin{figure*}
    \includegraphics[width=\linewidth]{figures/cc-validation-samples_wct.png}
    \savebox\captionqr{\qrcode[hyperlink,height=0.3in]{\repositoryBaseUrl/figures/cc-validation-samples_wct.html}}
    \caption[\gls{wctLabel} samples distribution visualization]{Data-driven
    from first-principles. Samples distribution visualization.\hspace{1ex}
    \usebox\captionqr}
    \labfig{cc:validation:wct:samples-distribution}
\end{figure*}

\reffig{cc:validation:wct:samples-distribution} shows the generated input space
distribution. The upper plot shows the frequency distribution of the samples
while the lower one the actual values per input, where the x-axis represents
the samples and the y-axis the values for each of the input variables.

\subsubsection{Prediction capabilities}
Tabla tocha añadiendo casos (GPR, DD from FP, RF, GB)

The results of each modelling alternative and its comparison can be visualized
in \reffig{cc:validation:wct:prediction-comparison} and
\reftab{cc:validation:wct:results}. The results of each modelling alternative
and its comparison can be visualized
\reffig{cc:validation:wct:prediction-comparison} shows the results obtained
with the models using \hyperref[sec:cc:facility:exp:3]{Exp 3}. It shows the
perfect fit together with the results obtained with Poppe's model, MIMO FF,
cascade CF, and MIMO RBF. In \reftab{cc:validation:wct:results}, the
performance of the studied modelling approaches are included for the different
performance metrics\sidenote{Described in \refsec{intro:modelling:metrics}}.
\texttt{T} represents the performance metric value  for the training /
calibration dataset (\hyperref[sec:cc:facility:exp:1]{Exp 1} or
\hyperref[sec:cc:facility:exp:2]{Exp 2} depending on the case), and \texttt{V}
for the validation and comparison one (\hyperref[sec:cc:facility:exp:3]{Exp
3}). In all cases the model representing each alternative is in the best case
scenario, \ie maximum number of points available. On the other hand,
\texttt{s.u.} indicates that the units of the column are the same as from the
source variable.

\begin{figure}
    \includegraphics[width=.9\textwidth]{figures/cc-validation-wct-regression.png}
    \savebox\captionqr{\qrcode[hyperlink,height=0.5in]{\repositoryBaseUrl/figures/cc-validation-wct-regression.html}}
    \caption[\gls{wctLabel} Models performance regression]{\gls{wctLabel} models performance comparison between the different modelling approaches.\\[1ex] \usebox\captionqr}
    \labfig{cc:validation:wct:regression}
\end{figure}


Comparing both modelling approaches (see
\reffig{cc:validation:wct:prediction-comparison}), it can be outlined that both
models provide a good prediction of the output variables, falling most of the
discrepancies (errors) within the uncertainty range. Poppe's model provides a
better prediction of the outlet temperature, obtaining an RMSE of 0.33
$^\circ$C and an R$^2$ of 0.98. In comparison, the best ANN alternative (RBF
MIMO) has a slight worse performance with an RMSE of 0.51 $^\circ$C and
R$^2=0.95$. In terms of water consumption, the physical model has a better
prediction accuracy in terms of RMSE and  R$^2$ (8.5 l/h and 0.97) compared to
11.24 l/h and 0.95 for the best ANN model (cascade CF). It can be stated that,
although the results are better for the physical model (specially in the case
of the outlet temperature prediction), both approaches produce valid results
with high accuracy levels.\todo{Incluir gráfica comparativa de evolución de
RMSE (incluyendo GPR) para argumentar por qué GPR es mejor no solo en términos
de error, si no también en requerimiento de datos.}

\begin{table*}[]
\caption{Summary table of the prediction results obtained with the different modelling approaches studied.}
\labtab{cc:validation:wct:results}
\resizebox{\linewidth}{!}{%
    
\begin{tabular}{cclccccccccccccccccccccccc}
\hline
\multicolumn{1}{c}{\multirow{3}{*}{\textbf{\begin{tabular}[c]{@{}c@{}}Predicted\\ variable\end{tabular}}}} & &
\multicolumn{1}{c}{\multirow{3}{*}{\textbf{\begin{tabular}[c]{@{}c@{}}Modelling\\ alternative\end{tabular}}}} & &
\multicolumn{1}{c}{\multirow{3}{*}{\textbf{\begin{tabular}[c]{@{}c@{}}Model\\ config\end{tabular}}}} & &
\multicolumn{1}{c}{\multirow{3}{*}{\textbf{\begin{tabular}[c]{@{}c@{}}Topology\end{tabular}}}} & &
\multicolumn{15}{c}{\textbf{Performance metric}} & &
\multicolumn{1}{c}{\multirow{3}{*}{\textbf{\begin{tabular}[c]{@{}c@{}}Evaluation\\ time (s)\end{tabular}}}} \\ 
\cline{9-23} \multicolumn{1}{c}{} & & \multicolumn{1}{c}{} & & & & & &
\multicolumn{3}{c}{\textbf{\begin{tabular}[c]{@{}c@{}}R$^2$\\ (-)\end{tabular}}} & &
\multicolumn{3}{c}{\textbf{\begin{tabular}[c]{@{}c@{}}RMSE\\ (s.u.)\end{tabular}}} & &
\multicolumn{3}{c}{\textbf{\begin{tabular}[c]{@{}c@{}}MAE\\ (s.u.)\end{tabular}}} & &
\multicolumn{3}{c}{\textbf{\begin{tabular}[c]{@{}c@{}}MAPE \\ (\%)\end{tabular}}} & &
\multicolumn{1}{c}{} \\
\cline{9-11} \cline{13-15} \cline{17-19} \cline{21-23}
\multicolumn{1}{c}{}  & & \multicolumn{1}{c}{}  & & \multicolumn{1}{c}{}  & & \multicolumn{1}{c}{}  & &
\multicolumn{1}{c}{T} & & \multicolumn{1}{c}{V} & &
\multicolumn{1}{c}{T} & & \multicolumn{1}{c}{V} & &
\multicolumn{1}{c}{T} & & \multicolumn{1}{c}{V} & &
\multicolumn{1}{c}{T} & & \multicolumn{1}{c}{V} & & \multicolumn{1}{c}{} \\
\cline{1-1} \cline{3-3} \cline{5-5} \cline{7-7} \cline{9-9} \cline{11-11} \cline{13-13} \cline{15-15} \cline{17-17} \cline{19-19} \cline{21-21} \cline{23-23} \cline{25-25}
\multirow{11}{*}{T$_{wct,out}$ ($^\circ$C)}
    % Include Poppe manually
    & & \textbf{Physical model} & & - & & - & & - & & \textbf{0.98} & & - & & \textbf{0.33} & & - & & \textbf{0.27} & & - & & \textbf{0.87} & & 6.288 & \\
     & & Feedforward ANN & & MIMO & & 20-2 & & 0.90 & & 0.81 & & 0.60 & & 0.97 & & 0.42 & & 0.67 & & 1.36 & & 2.36 & & 0.004 & \\ & & Cascade-forward ANN & & MIMO & & 10-10-2 & & 0.90 & & 0.82 & & 0.60 & & 0.93 & & 0.44 & & 0.65 & & 1.42 & & 2.27 & & 0.005 & \\ & & Radial basis ANN & & MIMO & & 34-2 & & 0.97 & & 0.97 & & 0.34 & & 0.41 & & 0.21 & & 0.28 & & 0.66 & & 0.94 & & 0.007 & \\ & & Feedforward ANN & & Cascade & & 20-1 & & 0.90 & & 0.82 & & 0.60 & & 0.93 & & 0.43 & & 0.65 & & 1.41 & & 2.26 & & 0.011 & \\ & & Cascade-forward ANN & & Cascade & & 10-10-1 & & 0.90 & & 0.83 & & 0.60 & & 0.92 & & 0.43 & & 0.64 & & 1.40 & & 2.24 & & 0.010 & \\ & & Radial basis ANN & & Cascade & & 92-1 & & 0.97 & & -1.44 & & 0.33 & & 3.45 & & 0.10 & & 2.12 & & 0.32 & & 7.43 & & 0.009 & \\ & & \textbf{Gaussian PR} & & Cascade & & N/A & & 0.99 & & \textbf{0.97} & & 0.20 & & \textbf{0.37} & & 0.15 & & \textbf{0.26} & & 0.47 & & \textbf{0.89} & & 0.001 & \\ & & Random forest & & Cascade & & N/A & & 0.75 & & 0.30 & & 0.96 & & 1.85 & & 0.60 & & 1.46 & & 2.03 & & 5.05 & & 0.078 & \\ & & Gradient boosting & & Cascade & & N/A & & 1.00 & & 0.68 & & 0.00 & & 1.24 & & 0.00 & & 0.95 & & 0.01 & & 3.29 & & 0.015 & \\ & & \textbf{Gaussian PR (FP)} & & Cascade & & N/A & & 1.00 & & \textbf{0.94} & & 0.32 & & \textbf{0.54} & & 0.15 & & \textbf{0.41} & & 0.52 & & \textbf{1.32} & & 0.105 & \\ \hline
\multirow{11}{*}{C$_{w}$ (l/h)}
    % Include Poppe manually
    & & \textbf{Physical model} & & - & & - & & - & &  \textbf{0.97} & & - & & \textbf{8.47} & & - & & \textbf{6.74} & & - & & \textbf{3.74} & & 6.288 & \\
     & & Feedforward ANN & & MIMO & & 20-2 & & 0.92 & & 0.83 & & 14.77 & & 21.58 & & 11.98 & & 18.64 & & 9.91 & & 10.75 & & 0.004 & \\ & & Cascade-forward ANN & & MIMO & & 10-10-2 & & 0.92 & & 0.84 & & 15.47 & & 20.90 & & 12.51 & & 17.84 & & 10.48 & & 10.22 & & 0.005 & \\ & & Radial basis ANN & & MIMO & & 34-2 & & 0.99 & & 0.97 & & 5.58 & & 9.34 & & 3.81 & & 7.47 & & 3.23 & & 4.68 & & 0.007 & \\ & & Feedforward ANN & & Cascade & &  20-1 & & 0.92 & & 0.88 & & 15.00 & & 18.45 & & 11.97 & & 15.77 & & 10.20 & & 8.92 & & 0.011 & \\ & & Cascade-forward ANN & & Cascade & &  10-10-1 & & 0.92 & & 0.85 & & 15.01 & & 20.34 & & 12.11 & & 17.66 & & 10.00 & & 10.18 & & 0.010 & \\ & & Radial basis ANN & & Cascade & &  33-1 & & 0.99 & & 0.93 & & 4.99 & & 14.28 & & 3.45 & & 10.14 & & 2.68 & & 6.22 & & 0.009 & \\ & & \textbf{Gaussian PR} & & Cascade & & N/A & & 0.99 & & \textbf{0.95} & & 4.74 & & \textbf{12.00} & & 3.61 & & \textbf{9.96} & & 3.09 & & \textbf{6.32} & & 0.001 & \\ & & Random forest & & Cascade & & N/A & & 0.89 & & 0.80 & & 17.35 & & 23.23 & & 10.51 & & 18.51 & & 7.58 & & 9.73 & & 0.078 & \\ & & Gradient boosting & & Cascade & & N/A & & 1.00 & & 0.77 & & 0.24 & & 25.07 & & 0.07 & & 17.21 & & 0.05 & & 9.55 & & 0.015 & \\ & & \textbf{Gaussian PR (FP)} & & Cascade & & N/A & & 0.98 & & \textbf{0.95} & & 10.85 & & \textbf{11.63} & & 4.81 & & \textbf{8.14} & & 3.74 & & \textbf{4.52} & & 0.105 & \\ \hline
\end{tabular}%
}
\end{table*}


% \subsubsection{Experimental data requirements}

% In order to estimate the minimum number of tests required to obtain
% satisfactory results with both modelling strategies, an analysis was performed
% in which each modelling alternative was calibrated/tuned for different case
% studies with different amounts of available data, and then the performance
% metrics were  evaluated. In this way, trends in the predictive accuracy of the
% models as a function of the available data can be identified. When the
% variation becomes small, it can be stated that the model has converged and
% adding more information provides diminishing returns. 

% For the physical model, the number of tests from Exp 1, used to calibrate $\Me$
% correlation, was varied from 2 up to 16 data points added sequentially. In the
% case of the ANN models, the available tuning data (Exp 2) was increased in
% steps of 10 \%, starting from the availability of 10 \% up to the entire data
% set (\mbox{100 \%}).

% In both cases, the criteria for selecting the data was not random, but it was
% done by applying physical knowledge. The water-to-air mass flow ratio,
% $\dot{m}_w/\dot{m}_a $, is a good indicator for selecting the operation points
% to be fed to the model. The trend observed in \reffig{cc:validation:Me_LG} (decreasing
% Me for increasing $\dot{m}_w/\dot{m}_a $) has been extensively reported in the
% literature. This behavior is explained by the increase in the amount of water
% per unit of air that lead to a less effective cooling \sidecite{ruiz_thermal_2022}.
% The situation corresponding to the minimum $\dot{m}_w/\dot{m}_a $ can be
% interpreted as the maximum air flow rate for a given water flow rate to be
% cooled. This results in the maximum driving force and, therefore, maximum
% Merkel number. As $\dot{m}_a $ decreases progressively, the driving force
% decreases for a given $\dot{m}_w $, and Me decreases accordingly. Based on this
% knowledge, the selection starts by choosing extreme points for the water-to-air
% mass flow ratio in the Me-$\dot{m}_a/\dot{m}_w$ relationship from the available
% data, which gives information of the system operating in its limits.
% Subsequently intermediate points are added, covering this way the whole
% operating range of the cooling system.

% \begin{figure}
%     \includegraphics[width=\textwidth]{figures/wct-rmse-evolution.png}
%     \caption{\gls{rmseLabel} evolution as a function of the number of points used for calibration/training of the Poppe's and data-driven approaches}
%     \labfig{cc:wct:rmse-evolution}
% \end{figure}

% The results of this study are presented in \reffig{cc:wct:rmse-evolution}, where
% the x-axis represents the number of available data points and the y-axis a
% model performance metric (RMSE) obtained when the model outputs are compared to
% data from Exp 3. From the results obtained, it can be clearly seen the
% advantage of the physical model in terms of data requirements, since with the
% minimal amount of points, good results are obtained, and by enlarging the
% available data points to 8-10, low variation in the RMSE evolution can be
% observed for both predicted variables. In the case of the ANN-based approaches,
% the results differ depending on the ANN alternative. 

% In terms of the outlet temperature, very good results (low error and variation)
% are obtained with the minimal dataset (10 \% of available data, 12 data points)
% for feedforward and cascade-forward in any configuration (MIMO and cascade). If
% more data is added, RMSE is reduced from 1.1 up to 0.7 $^\circ$C. Although the
% MIMO RBF outperforms the results of the other ANN alternatives, it does so only
% from 90 points onwards. For this case, the downward trend is much more
% noticeable but constant, which can not be stated for the cascade RBF,
% displaying an erratic evolution up to 70 points. 

% Similar conclusions can be drawn for the water consumption, except that in this
% case the two RBF configurations achieve satisfactory results much earlier,
% starting from 23 points.

% Summarizing, both modelling approaches, Poppe's model and ANNs, produce
% satisfactory results since their predictions fall well within the range of
% uncertainty for all the case studies, although the obtained results, in terms
% of RMSE, favor the physical model. Therefore, while the ANN model benefits from
% as much data as possible, the Poppe model is already able to produce
% satisfactory results with just two properly selected points. These two points
% are easy to identify in advance because they are related to the maximum and
% minimum $\dot{m}_w/\dot{m}_a$ ratio of the wet cooling tower. In practice, to
% minimize the error prediction, around 5 points are often used. Out of the ANN
% alternatives, considering both output variables, if less than 70 data points
% are available, cascade-forward and feedforward alternatives with any
% configuration are the best option, producing satisfactory results with as low
% as 10 points. On the other hand, if enough data is available, MIMO RBF should
% be considered as a strong candidate, but not in the cascade configuration
% alternative.


% \subsubsection{Sensitivity analysis}


% In \reffig{cc:validation:wct:sa}, only total-order sensitivity
% indices\marginreminder{
%     How to interpret \gls{saLabel} results}{ 
%     The results are different sensitivity indices such as total sensitivity
%     indices (total-order), first-order sensitivity indices (first-order), and
%     interaction sensitivity indices (second-order). First-order measures the
%     direct effect of an input variable on the output, excluding interaction
%     effects with other variables, while the second-order measures specifically
%     these interaction effects. Finally, total-order indices account for the
%     total effect of an input variable, including both direct and interaction
%     effects.\footnote{More in \nrefch{intro:sa}} 
% } are represented in the y-axis for the two output variables (outlet
% temperature on the top and water consumption on the bottom). Its value ranges
% from 0 to 1, where 0 means the variable has no effect, and 1 means it has a
% significant effect on the output\sidenote[][*10]{values can go slightly above 1 due to
% computing errors. This is due to the Sobol' sequence sample generator producing
% some unfeasible test samples that need to be discarded}. The x-axis represents
% the system's inputs and includes a bar for some of the obtained models with
% different calibration or training data points. 

% Comparing the results obtained for the different modelling approaches in
% \reffig{cc:validation:wct:sa}, it can be seen that very homogeneous results are
% obtained in all cases, except for the Cascade RBF case, which was the worst
% performing of all the alternatives. These results serve to confirm that, at
% least from a sensitivity analysis point of view, all valid approaches are
% similarly sensitive to variations in the same inputs, which is desirable since
% they are trying to predict the same physical system. In the case of Cascade
% RBF,  a discrepancy can be observed; less relevant input variables ($T_{amb}$
% and $\phi_{\infty}$) are overestimated and overall higher uncertainties in
% sensitivity are observed.

% It is also important to highlight that the observed results are in agreement
% with the underlying physics of the heat and mass transfer processes occurring
% in the exchange area of the tower. The frequency of the fan and the volumetric
% flow rate are directly related to $\dot{m}_a$ and $\dot{m}_w$, respectively,
% and they have a high influence on the heat transfer coefficients. These
% coefficients govern the evaporation processes, which impact the evaporation
% rate (water lost due evaporation) and the outlet water temperature. On the
% other hand, the ambient conditions and the inlet water temperature also affect
% the outputs, but less significantly, since the driving force for the
% evaporation is the difference between the inlet air enthalpy and the enthalpy
% of saturated air evaluated at water temperature.

% \begin{figure}
%     \includegraphics[width=\textwidth]{figures/wct-sensitivity_analysis.png}
%     \caption{Sobol's sensitivity analysis result for different case studies}
%     \labfig{cc:validation:wct:sa}
% \end{figure}


% =================================
\subsection{Dry cooler model alternatives comparison and validation}

\subsubsection{Physical model}

\subsubsection{Data-driven}

In order to generate the data-driven from first-principles alternative, the
most relevant input variables identified in \refsec{cc:modelling:dc:samples}
are discretized using a fixed number of resolution steps for each variable,
within ranges based on expected operating conditions, as defined in
\reftab{cc:validation:dc:samples-params}. and
\nrefsec{cc:validation:dc:samples-distribution} visualizes the generated input
space distribution where it can be appreciated that the samples are well
distributed across the entire input space.

\begin{margintable}[]
    \caption{Bounds and discretization of the model input variables.}
    \labtab{cc:validation:dc:samples-params}
    \resizebox{\linewidth}{!}{%
    \begin{tabular}{lcccc}
        \toprule
        $\mathbf{x}$ & \textbf{Units} & \textbf{lb} & \textbf{ub} & \textbf{n} \\
        \midrule
        $T_{amb}$       & $^\circ$C            & 3   & 50   & 7 \\
        $\Delta T_{amb-dc,in}$       & $^\circ$C            & 3   & 30    & 7  \\
        $q_{dc}$       & m$^3$/h            & 6   & 24    & 7  \\
        $T_{dc,in}$   & $^\circ$C         & 25  & 45   & - \\
        $\omega_{\text{dc}}$  & \%         & 11  & 99.18   & 6 \\
        \bottomrule
    \end{tabular}
    }
\end{margintable}

\begin{figure}
    \includegraphics[width=\textwidth]{figures/cc-validation-samples_dc.png}
    \savebox\captionqr{\qrcode[hyperlink,height=0.5in]{\repositoryBaseUrl/figures/cc-validation-samples_dc.html}}
    \caption[\gls{dcLabel} samples distribution visualization]{Data-driven from first-principles. Samples distribution visualization.\\[1ex] \usebox\captionqr}
    \labfig{cc:validation:dc:samples-distribution}
\end{figure}


\subsubsection{Prediction capabilities}
Tabla tocha añadiendo casos (GPR, DD from FP, RF, GB)\todo{completar esta
sección antes del domingo}

\begin{figure}
    \includegraphics[width=.9\textwidth]{figures/cc-validation-dc-regression.png}
    \savebox\captionqr{\qrcode[hyperlink,height=0.5in]{\repositoryBaseUrl/figures/cc-validation-dc-regression.html}}
    \caption[\gls{dcLabel} Models performance regression]{\gls{dcLabel} models performance comparison between the different modelling approaches.\\[1ex] \usebox\captionqr}
    \labfig{cc:validation:dc:regression}
\end{figure}

\begin{table*}[]
\caption{Summary table of the prediction results obtained with the different modelling approaches studied.}
\labtab{cc:validation:wct:results}
\resizebox{\linewidth}{!}{%
    
\begin{tabular}{cclccccccccccccccccccccccc}
\hline
\multicolumn{1}{c}{\multirow{3}{*}{\textbf{\begin{tabular}[c]{@{}c@{}}Predicted\\ variable\end{tabular}}}} & &
\multicolumn{1}{c}{\multirow{3}{*}{\textbf{\begin{tabular}[c]{@{}c@{}}Modelling\\ alternative\end{tabular}}}} & &
\multicolumn{1}{c}{\multirow{3}{*}{\textbf{\begin{tabular}[c]{@{}c@{}}Model\\ config\end{tabular}}}} & &
\multicolumn{1}{c}{\multirow{3}{*}{\textbf{\begin{tabular}[c]{@{}c@{}}Topology\end{tabular}}}} & &
\multicolumn{15}{c}{\textbf{Performance metric}} & &
\multicolumn{1}{c}{\multirow{3}{*}{\textbf{\begin{tabular}[c]{@{}c@{}}Evaluation\\ time (s)\end{tabular}}}} \\ 
\cline{9-23} \multicolumn{1}{c}{} & & \multicolumn{1}{c}{} & & & & & &
\multicolumn{3}{c}{\textbf{\begin{tabular}[c]{@{}c@{}}R$^2$\\ (-)\end{tabular}}} & &
\multicolumn{3}{c}{\textbf{\begin{tabular}[c]{@{}c@{}}RMSE\\ (s.u.)\end{tabular}}} & &
\multicolumn{3}{c}{\textbf{\begin{tabular}[c]{@{}c@{}}MAE\\ (s.u.)\end{tabular}}} & &
\multicolumn{3}{c}{\textbf{\begin{tabular}[c]{@{}c@{}}MAPE \\ (\%)\end{tabular}}} & &
\multicolumn{1}{c}{} \\
\cline{9-11} \cline{13-15} \cline{17-19} \cline{21-23}
\multicolumn{1}{c}{}  & & \multicolumn{1}{c}{}  & & \multicolumn{1}{c}{}  & & \multicolumn{1}{c}{}  & &
\multicolumn{1}{c}{T} & & \multicolumn{1}{c}{V} & &
\multicolumn{1}{c}{T} & & \multicolumn{1}{c}{V} & &
\multicolumn{1}{c}{T} & & \multicolumn{1}{c}{V} & &
\multicolumn{1}{c}{T} & & \multicolumn{1}{c}{V} & & \multicolumn{1}{c}{} \\
\cline{1-1} \cline{3-3} \cline{5-5} \cline{7-7} \cline{9-9} \cline{11-11} \cline{13-13} \cline{15-15} \cline{17-17} \cline{19-19} \cline{21-21} \cline{23-23} \cline{25-25}
\multirow{7}{*}{T$_{dc,out}$ ($^\circ$C)}
    % Include physical model results manually
     & & \textbf{Physical model} & & - & & - & & - & & \textbf{0.98} & & - & & \textbf{0.50} & & - & & \textbf{0.42} & & - & & \textbf{1.28} & & 0.035 & \\
     & & Feedforward ANN & & - & & 20-1 & & 0.77 & & 0.78 & & 1.42 & & 1.62 & & 1.13 & & 1.18 & & 3.29 & & 3.85 & & 0.005 & \\ & & Cascade-forward ANN & & - & & 10-10-1 & & 0.78 & & 0.85 & & 1.39 & & 1.37 & & 1.12 & & 1.02 & & 3.23 & & 3.24 & & 0.007 & \\ & & \textbf{Gaussian PR} & & - & & N/A & & 0.99 & & \textbf{0.99} & & 0.24 & & \textbf{0.32} & & 0.19 & & \textbf{0.25} & & 0.56 & & \textbf{0.77} & & 0.005 & \\ & & Random forest & & - & & N/A & & 0.84 & & 0.61 & & 1.19 & & 2.17 & & 0.72 & & 1.36 & & 2.05 & & 4.69 & & 0.022 & \\ & & Gradient boosting & & - & & N/A & & 1.00 & & 0.86 & & 0.00 & & 1.31 & & 0.00 & & 0.86 & & 0.00 & & 2.92 & & 0.035 & \\ & & \textbf{Gaussian PR (FP)} & & - & & N/A & & 1.00 & & \textbf{0.98} & & 0.03 & & \textbf{0.53} & & 0.02 & & \textbf{0.44} & & 0.07 & & \textbf{1.35} & & 0.002 & \\ \hline
\end{tabular}%
}
\end{table*}

\subsubsection{Experimental data requirements}
% \subsubsection{Sensitivity analysis}



%================================
\subsection{Main components modelling conclusions}

This section presents a comparison between two modelling alternatives:
data-driven and first-principles. It applies to wet cooling towers and dry
coolers, specifically to \fullgls{acheLabel}. The main conclusions obtained
during the investigation and final recommendations can be summarized as
follows:

\subsubsection{Wet cooling tower}

Regarding\todo{Pendiente de actualizar estas conclusiones con resultados
actualizados} the prediction of the output variables, in the case of the outlet
water temperature, both models reported good results, with low errors falling
within the uncertainty range of the experimental equipment. Nonetheless, the
physical model performs better than the best data-driven alternative (MIMO
RBF): $R^2=0.98$ and RMSE$=0.33 \: ^\circ$C compared to $R^2=0.95$ and
RMSE$=0.51 \: ^\circ$C, respectively.

For the predictions of water consumption, it was shown that the Poppe model
accurately predicts this variable, with results of $R^2=0.97$ and RMSE$=8.47$
l/h. The best ANN alternative (cascade CF) achieves close results with an
$R^2=0.95$ and RMSE$=11.24$ l/h.

However, the Poppe model reached such reliable prediction levels with a much
lower number of tests, needing only 2. In comparison, the ANN alternatives need
more data, at least 10 (with a good distribution over the operating range) for
the FF and CF ANN models.


\subsubsection{Air-cooled heat exchanger}



\subsubsection{Conclusions and recommendations}

For the proposed optimization strategy in \nrefsec{cc:optimization}, a fast,
reliable model that can be scaled to different system sizes is required.

On the one hand, the first-principle models execution time is much higher than
the data-driven alternatives, which is a significant drawback when it comes to
the optimization strategy, where the model is evaluated many times in a short
period of time. On the other hand, the data-driven counterparts are only
applicable to the conditions and the particular system with which they are
developed.

Conversely, one of the main strengths of both physical models presented in this
chapter, is their ability to predict the operation of the coolers regardless of
the conditions tested; while the data-driven execution time is faster by orders
of magnitude, it can be vectorized and its execution time is more constant
regardless of the input conditions.

Therefore, as combining a wet cooler and a dry cooler into a combined cooler
offers potential advantages compared to the individual systems, combining both
modelling approaches is the chosen solution to model the system. The best
performing data-driven model, the \gls{gprLabel} is calibrated using data from
the first-principle models, where physical models are adapted dynamically to
the required scale and finally the data-driven model can be generated. This
approach provides a way of having on-demand models that can be adapted to the
particular case study, while still being fast and efficient in terms of
computational resources.


%================================
\subsection{Condenser model validation}
\labsec{cc:validation:condenser}

For the surface condenser\sidenote{See
\nrefsec{cc:modelling:surface-condenser}} a physical model is used, with
the heat transfer coefficient as the only parameter to calibrate. Seven
different alternative estimations of the heat transfer coefficient were
calculated, using the data from the experimental campaign described in
\nrefsec{cc:facility:exp-condenser}. They are as follows:

\begin{enumerate}
    \item Empirical correlation using the condenser flow rate ($q_c$) and the
    vapor temperature (\(T_{v}\)) as inputs.
    \item Empirical correlation using the cooling water inlet temperature
    (\(T_{c,in}\)) and \(T_{v}\) as inputs.
    \item Empirical correlation using the flow rate per condenser tube
    (\(q_{c,tube}=q_c/n_{tubes}=q_c/24\)) and the cooling water inlet temperature.
    \item Nominal value from the manufacturer, which equals 1.838
    W/m$^2\,^\circ$C
    \item Calibra\_Uexp\_original\todo{Estos qué son? Generar una nueva versión
    de la figura una vez se seleccionen los métodos finales}
    \item Calibra\_Uexp\_recortado
\end{enumerate}

\begin{figure}
    \includegraphics[width=\textwidth]{figures/cc-validation-condenser-U.png}
    \caption{Heat transfer coefficient calibration results}
    \labfig{cc:validation:condenser-U}
\end{figure}

The results of the calibration are shown in \reffig{cc:validation:condenser-U},
where the y-axis shows the thermal
power obtained and the x-axis holds different bars for the different heat transfer
coefficient estimation methods, with bars also for the experimental heat
released by the vapor and absorbed by the coolant. As can be seen in the
figure. The shown results are for steady-state conditions with the condenser in
an equilibrium state (\(Q_{\text{released}} \approx Q_{\text{absorbed}}\)), and with a
large variation in the condenser conditions (120 to 200~kW, the whole operating
range of the condenser). The results show that the heat transfer coefficient
obtained with the method 3 is the one that best fits the experimental data,
with a \gls{maeLabel} of 17.6~kW and a maximum error of 33.41~kW (15\%).


% =================================
\subsection{Complete system model validation}
\labsec{cc:validation:complete-system}

Esto\todo{a completar una vez se tengan resultados experimentales, hay que
implementar la función para generar la visualización}, o bien se hace
comparando puntos en estático cuando todo el sistema está en estacionario, o en
la gráfica de validación de la estrategia de optimización se muestra también
una línea con las predicciones del modelo. Y después una tabla con cada una de
las salidas del sistema, mostrando el error entre cada una de las predicciones
del modelo (cada vez que se evalúa la optimización), en comparación al valor
real obtenido en la planta. 

Cuando haya un cambio en la planificación, las predicciones de predicciones
antes a conocerse el cambio, cambiar su color a un gris para mostrar que esas
predicciones ya no son válidas pues han cambiado las condiciones.

\begin{figure*}[h!]
    \includegraphics[]{figures/cc-validation-complete-model.png}
    \savebox\captionqr{\qrcode[hyperlink,height=0.5in]{\repositoryBaseUrl/figures/cc-validation-complete-model.html}}
    \caption[Figure caption]{Figure caption.\hspace{1ex}\usebox\captionqr}
    \labfig{cc:validation:complete-model}
\end{figure*}

% Figure description
\reffig{cc:validation:complete-model} shows the model validation results for
the complete system model. Different plots are shown for the main output
variables\sidenote{$C_w$, $T_{dc,out}$, etc}. Solid lines represent the
measured variables in the real facility, while the different markers represent
the predictions generated at different times. The plots also include a
right-axis to display a metric error, specifically the \gls{mapeLabel} of the
predictions with respect to the measured values. The error is shown with a bar
for each prediction.

% Results assessment
The results show that the model is able to predict the main output variables
of the system with a good accuracy. As expected the errors compound over time,
specially after the change in the operation schedule at XX:XX. Nonetheless, the
model is able to adapt after a new evaluation including the change and overall
predictions below XX\% of error are achieved.

% A mi me gustaría que esta sección sea su propio capítulo
% Optimization validation
%================================
%================================
\section{Control and optimization results}
\labsec{cc:validation:optimization}

Once the models of the main components of the system have been validated,
the next step is to validate the optimization strategy proposed in
\nrefsec{cc:modelling:optimization}.
First, an optimization algorithm is chosen by comparing different alternatives
in \nrefsec{cc:validation:algorithm}. Then, the two proposed variants for the
combined cooler are compared in simulation for one operation day in the
simulated pilot plant in order to see which one performs better in
\nrefsec{cc:validation:optimization:static-vs-horizon}. Finally, two validation
scenarios are tested in the real facility, one where a regular operation
schedule is followed throughout the operation, and a second one where planned
changes are introduced in the operation schedule, in order to validate how the
optimization strategy adapts to changing conditions.

%================================
\subsection{Choosing an optimization algorithm}
\labsec{cc:validation:algorithm}

\subsubsection{Static problems}

For every static optimization problem (referencias a problemas) three different
algorithms are tested: \gls{seacstrLabel}, \gls{ihsLabel} and \gls{decstrLabel}.
For each alternative the same number of objective function evaluations are
given (800) but they are distributed differently depending on the algorithm:
\begin{itemize}
    \item \gls{seacstrLabel} and \gls{decstrLabel} make use of the
    \gls{cstrLabel} wrapper algorithm, which allows them to the constrained
    problems. 10 iterations are performed for this wrapper algorithm, leaving
    80 iterations to spare for the inner algorithm.
    
    \item For all alternatives, three values are tested for the initial
    population size: 50, 100 and 400 individuals\sidenote{The initial
    population fitness evaluation is not counted for the budget of
    objective function evaluations}. 
    
    \item Depending on the algorithm only one individual is evolved (
    \gls{ihsLabel} and \gls{seaLabel}) or the whole population (
    \gls{deLabel}). This means that 800 generations are available for
    \gls{ihsLabel}, 80 generations for \gls{seacstrLabel} and for
    \gls{decstrLabel}, 1 generation is available for the population of 50
    individuals, while only the initial generation is for the population of 100
    and 400 individuals.
\end{itemize}

\begin{table}[]
\caption{Static optimization algorithm comparison results}
\labtab{cc:validation:static-algo-comparison}
\resizebox{\linewidth}{!}{%
\begin{tabular}{llllclclccccccccc}
\hline
\multirow{2}{*}{\textbf{System}} &  & \multicolumn{1}{c}{\multirow{2}{*}{\textbf{Algorithm}}} &  & \multicolumn{5}{c}{\textbf{Parameters}}                                                                                             &  & \multicolumn{7}{c}{\textbf{Average fitness per obj. fun. evaluations}}                                                                 \\ \cline{5-9} \cline{11-17} 
                                 &  & \multicolumn{1}{c}{}                                    &  & \textbf{\begin{tabular}[c]{@{}c@{}}pop\\ size\end{tabular}} &  & \textbf{gen}     &  & \textbf{\begin{tabular}[c]{@{}c@{}}wrapper\\ algo\\ iters\end{tabular}} &  & \multicolumn{1}{c}{\textbf{0}} & \textbf{} & \multicolumn{1}{c}{\textbf{50}} & \textbf{} & \multicolumn{1}{c}{\textbf{150}} & \textbf{} & \multicolumn{1}{c}{\textbf{800}} \\ \cline{1-1} \cline{3-3} \cline{5-5} \cline{7-7} \cline{9-9} \cline{11-11} \cline{13-13} \cline{15-15} \cline{17-17} 


\multirow{9}{*}{\textbf{\protect\gls{dcLabel}}}       &  & \multirow{3}{*}{\protect\gls{ihsLabel}}                              &  & 50                                       &  & 800 &  & N/A                                                    &  & $1.28\pm0.82$             &           & $1.05\pm0.29$              &           & $0.80\pm0.10$               &           & $0.77\pm0.09$               \\
                                 &  &                                                         &  & 100                                       &  & 800 &  & N/A                                                    &  & $0.92\pm0.18$             &           & $0.87\pm0.14$              &           & $0.81\pm0.11$               &           & $0.77\pm0.10$               \\
                                 &  &                                                         &  & 400                                       &  & 800 &  & N/A                                                    &  & $0.81\pm0.11$             &           & $0.80\pm0.11$              &           & $0.79\pm0.10$               &           & $0.77\pm0.10$               \\ \cline{3-3} \cline{5-9} \cline{11-17} 
                                 &  & \multirow{3}{*}{\protect\gls{seaLabel}}                              &  & 50                                       &  & 80 &  & 10                                                    &  & $1.19\pm0.28$             &           & $0.95\pm0.11$              &           & $0.79\pm0.10$               &           & $0.77\pm0.09$               \\ 
                                 &  &                                                         &  & 100                                       &  & 80 &  & 10                                                    &  & $0.92\pm0.13$             &           & $0.86\pm0.10$              &           & $0.80\pm0.10$               &           & $0.77\pm0.09$               \\
                                 &  &                                                         &  & 400                                       &  & 80 &  & 10                                                    &  & $0.82\pm0.10$             &           & $0.80\pm0.10$              &           & $0.78\pm0.10$               &           & $0.77\pm0.09$               \\ \cline{3-3} \cline{5-9} \cline{11-17} 
                                 &  & \multirow{3}{*}{\protect\gls{seaLabel}}                              &  & 50                                       &  & 1 &  & 10                                                    &  & $1.06\pm0.40$             &           & $0.97\pm0.18$              &           & $0.83\pm0.10$               &           & $1.04\pm1.04$               \\ 
                                 &  &                                                         &  & 100                                       &  & 0 &  & 10                                                    &  & $0.95\pm0.16$             &           & $0.95\pm0.16$              &           & $0.95\pm0.95$               &           & $0.95\pm0.95$               \\
                                 &  &                                                         &  & 400                                       &  & 0 &  & 10                                                    &  & $0.83\pm0.10$             &           & $0.83\pm0.10$              &           & $0.83\pm0.10$               &           & $0.83\pm0.83$               \\ \hline


\multirow{9}{*}{\textbf{\protect\gls{wctLabel}}}       &  & \multirow{3}{*}{\protect\gls{ihsLabel}}                              &  & 50                                       &  & 800 &  & N/A                                                    &  & $0.24\pm0.08$             &           & $0.18\pm0.04$              &           & $0.10\pm0.00$               &           & $0.07\pm0.00$               \\
                                 &  &                                                         &  & 100                                       &  & 800 &  & N/A                                                    &  & $0.12\pm0.02$             &           & $0.11\pm0.01$              &           & $0.08\pm0.00$               &           & $0.07\pm0.00$               \\
                                 &  &                                                         &  & 400                                       &  & 800 &  & N/A                                                    &  & $0.07\pm0.00$             &           & $0.07\pm0.00$              &           & $0.07\pm0.00$               &           & $0.07\pm0.00$               \\ \cline{3-3} \cline{5-9} \cline{11-17} 
                                 &  & \multirow{3}{*}{\protect\gls{seaLabel}}                              &  & 50                                       &  & 80 &  & 10                                                    &  & $0.25\pm0.04$             &           & $0.16\pm0.01$              &           & $0.07\pm0.00$               &           & $0.06\pm0.00$               \\ 
                                 &  &                                                         &  & 100                                       &  & 80 &  & 10                                                    &  & $0.17\pm0.03$             &           & $0.11\pm0.00$              &           & $0.07\pm0.00$               &           & $0.06\pm0.00$               \\
                                 &  &                                                         &  & 400                                       &  & 80 &  & 10                                                    &  & $0.07\pm0.00$             &           & $0.07\pm0.00$              &           & $0.07\pm0.00$               &           & $0.06\pm0.00$               \\ \cline{3-3} \cline{5-9} \cline{11-17} 
                                 &  & \multirow{3}{*}{\protect\gls{seaLabel}}                              &  & 50                                       &  & 1 &  & 10                                                    &  & $0.29\pm0.07$             &           & $0.17\pm0.02$              &           & $0.09\pm0.00$               &           & $0.07\pm0.07$               \\ 
                                 &  &                                                         &  & 100                                       &  & 0 &  & 10                                                    &  & $0.11\pm0.00$             &           & $0.11\pm0.00$              &           & $0.11\pm0.11$               &           & $0.11\pm0.11$               \\
                                 &  &                                                         &  & 400                                       &  & 0 &  & 10                                                    &  & $0.07\pm0.00$             &           & $0.07\pm0.00$              &           & $0.07\pm0.00$               &           & $0.07\pm0.07$               \\ \hline


\multirow{9}{*}{\textbf{\protect\gls{ccLabel}}}       &  & \multirow{3}{*}{\protect\gls{ihsLabel}}                              &  & 50                                       &  & 1000 &  & N/A                                                    &  & $0.77\pm0.12$             &           & $0.80\pm0.11$              &           & $0.77\pm0.11$               &           & $0.59\pm0.11$               \\
                                 &  &                                                         &  & 100                                       &  & 1000 &  & N/A                                                    &  & $0.70\pm0.12$             &           & $0.78\pm0.10$              &           & $0.82\pm0.15$               &           & $0.61\pm0.13$               \\
                                 &  &                                                         &  & 400                                       &  & 1000 &  & N/A                                                    &  & $0.79\pm0.19$             &           & $0.82\pm0.21$              &           & $0.80\pm0.22$               &           & $0.65\pm0.16$               \\ \cline{3-3} \cline{5-9} \cline{11-17} 
                                 &  & \multirow{3}{*}{\protect\gls{seaLabel}}                              &  & 50                                       &  & 100 &  & 10                                                    &  & $0.92\pm0.13$             &           & $0.86\pm0.14$              &           & $0.74\pm0.16$               &           & $0.51\pm0.10$               \\ 
                                 &  &                                                         &  & 100                                       &  & 100 &  & 10                                                    &  & $0.88\pm0.16$             &           & $0.82\pm0.16$              &           & $0.75\pm0.21$               &           & $0.62\pm0.16$               \\
                                 &  &                                                         &  & 400                                       &  & 100 &  & 10                                                    &  & $0.84\pm0.21$             &           & $0.80\pm0.18$              &           & $0.74\pm0.21$               &           & $0.69\pm0.19$               \\ \cline{3-3} \cline{5-9} \cline{11-17} 
                                 &  & \multirow{3}{*}{\protect\gls{seaLabel}}                              &  & 50                                       &  & 2 &  & 10                                                    &  & $0.83\pm0.16$             &           & $0.79\pm0.13$              &           & $0.73\pm0.14$               &           & $0.56\pm0.13$               \\ 
                                 &  &                                                         &  & 100                                       &  & 1 &  & 10                                                    &  & $0.82\pm0.17$             &           & $0.80\pm0.13$              &           & $0.77\pm0.10$               &           & $0.64\pm0.13$               \\
                                 &  &                                                         &  & 400                                       &  & 0 &  & 10                                                    &  & $0.73\pm0.16$             &           & $0.73\pm0.16$              &           & $0.73\pm0.16$               &           & $0.73\pm0.73$               \\ \hline


\end{tabular}
}
\end{table}

Table X shows the results obtained, in terms of fitness at
different stages in the evolution. From the results it can be seen that for all
alternatives the best performing and most consistent algorithm is ...

\subsubsection{Horizon optimization. Path selection}

A methodology similar to the static comparison is used. This time the
algorithms evaluated are:  \gls{gacoLabel}, \gls{ihsLabel}, \gls{sgaLabel} and
\gls{psoLabel}. Three different population sizes are tested (80, 150 and 1000)
if the particular algorithm evolves more than one individual; the number of
generations is calculated accordingly so that all alternatives have the same
budget of objective function evaluations, equal to 200k
evaluations\sidenote{Only up to 50k evaluations is shown in the figure for
clarity}. The results are visualized in
\reffig{cc:validation:path_explorer_algo_comp}, where there are different plots
for different dates, the y-axis represents the fitness and the x-axis shows the
number of objective function evaluations. The results show that consistently
the \gls{sgaLabel} outperforms the alternatives, and particularly, the smaller
population size (80) configuration followed very closely by the 150 population
size configuration.

\begin{figure*}[h!]
    \includegraphics[]{figures/cc-validation-path_explorer_algo_comp.png}
    \savebox\captionqr{\qrcode[hyperlink,height=0.5in]{\repositoryBaseUrl/figures/cc-validation-path_explorer_algo_comp.html}}
    \caption[Horizon optimization -- path selection subproblem. Fitness
    evolution comparison for four different dates.]{Horizon optimization --
    path selection subproblem. Fitness evolution comparison for different
    algorithms in four different
    dates.\hspace{1ex}\usebox\captionqr}
    \labfig{cc:validation:path_explorer_algo_comp}
\end{figure*}


\subsection{Comparing the static and horizon optimization strategies}
\labsec{cc:validation:optimization:static-vs-horizon}

\begin{kaobox}[title=TODO]
    Poner la figura de resultados del horizonte para SOLO un día detallado aquí
    (más días hace que no se distingan bien las barras, tampoco se puede poner el
    pareto). Debe incluir la distribución hidráulica en barras comparando estático
    con horizonte, el frente de pareto del horizonte, y la comparativa de coste
    acumulado.
\end{kaobox}


\begin{figure*}[h!]
    \includegraphics[]{figures/cc-validation-comp_static_horizon.png}
    \savebox\captionqr{\qrcode[hyperlink,height=0.5in]{\repositoryBaseUrl/figures/cc-validation-comp_static_horizon.html}}
    \caption[Detailed timeseries simulation results. \gls{ccLabel}--horizon]{Detailed simulation results for the horizon
    optimization compared to the static alternative.\hspace{1ex}\usebox\captionqr}
    \labfig{cc:validation:static-vs-horizon}
\end{figure*}

Comentar la figura\todo{La figura es provisional. Actualizar la figura con cambios mencionados}, sobre el frente de pareto que se muestra, cómo la estática
al principio abusa del agua y para el final del día aumenta muchos sus costes, etc.

%================================
\subsection{Validation at pilot plant}

A hierarchical control strategy has been implemented in order to validate the
optimization strategy in the real facility.
\reffig{cc:validation:optimization:diagram} shows a diagram of the methodology,
where the left side represents the upper layer with the proposed
shrinking horizon optimization\sidenote{See \nrefsec{optimization:horizon}} and
the right side shows the low-level regulatory control layer, which directly
interfaces with the actuators and sensors of the facility. 

\begin{figure*}
    \includegraphics[]{figures/cc-validation-diagram.png}
    \caption{Implementation of the optimization strategy in the real facility.
    Hierarchical control}
    \labfig{cc:validation:optimization:diagram}
\end{figure*}

\begin{margintable}[*-6]
    \caption{Box-bounds for the decision variables.}
    \labtab{cc:validation:optimization-bounds}
    \begin{tabular}{lccc}
        \toprule
        $\mathbf{x}$ & \textbf{Units} & \textbf{lb} & \textbf{ub} \\
        \midrule
        $q_c$       & m$^3$/h            & 5.22   & 24.15   \\
        $R_p$       & --            & 0.00   & 1.00    \\
        $R_s$       & --            & 0.00   & 1.00    \\
        $\omega_{\text{dc}}$   & \%         & 11.00  & 99.18   \\
        $\omega_{\text{wct}}$  & \%         & 21.00  & 93.42   \\
        \bottomrule
    \end{tabular}
\end{margintable}

\textbf{Environment}. To generate the environment for the optimization, weather
forecasts using the OpenWeather API were used\sidecite{openweater_api}, for the
electricity costs data from the 2022 spanish grid was used, updating the year
to the one in which the experiment was performed, and the water cost was set to
$C_{w,s1}=X$ and $C_{w,s2}=Y$. For the thermal load a profile was generated by
setting a constant vapor temperature of $T_v=45\,^\circ$C while an arbitrary
cooling power was generated considering the heat availability from the
flat-plate collector field, which is the heat source of the system, for the
particular day. Finally, an initial value for the water availability was set to
$V_{avail,0}=0.5\,m^3$, and from there it is updated by reading the actual
system consumption online.

\textbf{Optimization layer}. The optimization
algorithm is run every 30 minutes, and generates a new set of results for the
remaining operation time. The results of the optimization are then passed to
the regulatory control layer by setting them as setpoints for the low-level
control. The box-bounds for the decision variables are shown in
\reftab{cc:validation:optimization-bounds}.

\textbf{Control layer}. Four controllers are implemented in this
layer...\todo{aiuda Lidia!}


% \begin{kaobox}[title=Consulta]
%     Y si en lugar de mostrar varios escenarios mostramos un único día donde
%     haya una primera parte donde se hacen varias evaluaciones siguiendo un
%     plan, y a mitad de día se introduce un cambio en el plan de operación. Así
%     nos ahorramos describir varios ensayos, y tener que estar mostrando varias
%     gráficas. Mejor mostrar solo una en grande y dedicarse a comentar lo que
%     suceda en ella.
% \end{kaobox}

% \subsubsection{Regular operation validation}
% \labsec{cc:validation:regular-operation}

% As mentioned, the first validation scenario consists of a regular operation
% schedule, where the optimization strategy is applied to the system without
% introducing any changes in the scheduled operation. 

% %================================
% \subsection{Planned changes in operation validation}[Planned changes]
% \labsec{cc:validation:planned-operation}
% a\todo{al menos redactar lo que se pretende sin resultados antes del domingo}

% %================================
% \subsection{Unanticipated operational changes validation}[Unanticipated changes]
% \labsec{cc:validation:unanticipated-operation}

% A\todo{Quitar esta sección? Esto es más de control que de validar la
% estrategia de optimización y no creo que sea necesario incluirlo}