\setchapterpreamble[u]{\margintoc}
\chapter{Validation in the combined cooling pilot plant} % Optimization strategy experimental validation
\labch{cc:validation}

\tldrbox{ This chapter presents the modelling and optimization results for the
combined cooling pilot plant. First calibration and training results of various
modelling approaches, followed by validation of the complete integrated model.
The best performing data-driven model, the \gls{gprLabel} is calibrated using
synthetic data from the first-principle models. This surrogate and on-demand
model that can be adapted to the particular case study, while still being fast
and efficient in terms of computational resources achieves a \gls{maeLabel}
below 0.97~$^\circ$C for temperatures and 19.4~l/h for water consumption. In the
second part of the chapter, the proposed optimization strategies are compared.
The horizon optimization alternative, which is also validated experimentally,
shows superior performance.}

% ====================================
% ====================================
\section{Modelling}

The system consists of two main components, the \gls{wctLabel} and 
the \gls{dcLabel}, which are modelled using different approaches and compared. 
The alternatives considered are:

\begin{enumerate}
    \item A first-principles approach, described in 
    \refsec{cc:modelling:wct:physical} and \refsec{cc:modelling:dc:physical} 
    for the \gls{wctLabel} and \gls{dcLabel}, respectively.
    \item \fullgls{annLabel}, as introduced in \refsec{intro:modelling:ann}, 
    with different network architectures: Feedforward, Cascade-forward, 
    and Radial Basis.
    \item Random Forest, described in \refsec{intro:modelling:other-ml}.
    \item Gradient Boosting, described in \refsec{intro:modelling:other-ml}.
    \item Gaussian Process Regression, described in 
    \refsec{intro:modelling:gpr}.
\end{enumerate}

In addition, the physical model of the surface condenser is validated. 
Finally, the selected modelling alternative is integrated with the 
remaining system components (\refsec{cc:modelling:other-components}) 
and validated at the system level in \nrefsec{cc:validation:complete-system}.


% ================================
\subsection{Wet cooler model alternatives comparison and validation}
\labsec{cc:validation:wct}

\subsubsection{Physical model}

\begin{marginfigure}[*-5]
    \includegraphics[width=1.1\linewidth]{figures/cc-poppe-Me_vs_mwma.png}
    \caption{Experimental results for the $\Me$ number as a function of $\dot{m}_{wct}/\dot{m}_{wct,air}$.}
    \labfig{cc:validation:Me_LG} 
\end{marginfigure}

As explained in \refsec{cc:facility:exp}, three experimental campaigns have been
performed. Using data from the fan calibration tests
(\reftab{cc:exp-campaigns} -- \textit{\gls{wctLabel}-fan}),
a function (mapping) is fitted that relates the air mass flow rate at the outlet
of the tower, $\dot{m}_{wct,air}$, with the fan speed, $\omega_{wct}$:

\begin{equation} 
    \dot{m}_{wct,air}=-0.0014 \omega_{wct}^2+0.1743\omega_{wct}-0.7251.
    \labeq{cc:maffan} 
\end{equation}

And with data from the calibration campaign (\reftab{cc:exp-campaigns} --
\textit{\gls{wctLabel}-cal}), the Merkel performance number, $\Me$.
\reffig{cc:validation:Me_LG} shows the variation of the Merkel number as a
function of the water-to-air mass flow ratio
($\dot{m}_{wct}/\dot{m}_{wct,air}$). As can be seen, the $\Me$ decreases with
$\dot{m}_{wct}/\dot{m}_{wct,air}$ values following a linear trend on log-log
scale. Following the correlation for the Merkel number of a wet cooling tower
described in \refsec{cc:modelling:wct:physical}, the parameters $c$ and $n$
obtained from the data fitting are 1.52 and 0.69, respectively.

\subsubsection{Data-driven}

In order to generate the data-driven from first-principles alternative, the most
relevant input variables identified in \refsec{cc:modelling:wct:samples} are
discretized using a fixed number of resolution steps (\textit{n} in
\reftab{cc:validation:wct:samples-params}) for each variable, within ranges
based on expected operating conditions, as defined in
\reftab{cc:validation:wct:samples-params}.

\begin{margintable}[]
    \caption{Bounds and discretization of the model \gls{wctLabel} input variables.}
    \labtab{cc:validation:wct:samples-params}
    \resizebox{\linewidth}{!}{%
    \begin{tabular}{lcccc}
        \toprule
        $\mathbf{x}$ & \textbf{Units} & \textbf{lb} & \textbf{ub} & \textbf{n} \\
        \midrule
        $T_{amb}$       & $^\circ$C            & 3   & 50   & 7 \\
        $\Delta T_{amb-in}$       & $^\circ$C            & 3   & 30    & 7  \\
        $q_{dc}$       & m$^3$/h            & 0.00   & 1.00    & 5  \\
        $\omega_{\text{dc}}$   & \%         & 11.00  & 99.18   & 10 \\
        $\omega_{\text{wct}}$  & \%         & 21.00  & 93.42   & 10 \\
        \bottomrule
    \end{tabular}
    }
\end{margintable}

\reffig{cc:validation:samples-distribution} (a) shows the generated input space
distribution. The upper plot shows the frequency distribution of the samples
while the lower one the actual values per input, where the x-axis represents
the samples and the y-axis the values for each of the input variables.

\subsubsection{Validation}

\begin{figure}
    \includegraphics[width=.9\textwidth]{figures/cc-validation-wct-regression.png}
    \savebox\captionqr{\qrcode[hyperlink,height=0.5in]{\repositoryBaseUrl/figures/cc-validation-wct-regression.html}}
    \caption[\gls{wctLabel} Models performance regression]{\gls{wctLabel} models performance comparison between the different modelling approaches.\\[1ex] \usebox\captionqr}
    \labfig{cc:validation:wct:regression}
\end{figure}

\begin{table*}[]
\caption{Summary table of the prediction results obtained with the different modelling approaches studied.}
\labtab{cc:validation:wct:results}
\resizebox{\linewidth}{!}{%
    
\begin{tabular}{cclccccccccccccccccccccccc}
\hline
\multicolumn{1}{c}{\multirow{3}{*}{\textbf{\begin{tabular}[c]{@{}c@{}}Predicted\\ variable\end{tabular}}}} & &
\multicolumn{1}{c}{\multirow{3}{*}{\textbf{\begin{tabular}[c]{@{}c@{}}Modelling\\ alternative\end{tabular}}}} & &
\multicolumn{1}{c}{\multirow{3}{*}{\textbf{\begin{tabular}[c]{@{}c@{}}Model\\ config\end{tabular}}}} & &
\multicolumn{1}{c}{\multirow{3}{*}{\textbf{\begin{tabular}[c]{@{}c@{}}Topology\end{tabular}}}} & &
\multicolumn{15}{c}{\textbf{Performance metric}} & &
\multicolumn{1}{c}{\multirow{3}{*}{\textbf{\begin{tabular}[c]{@{}c@{}}Evaluation\\ time (s)\end{tabular}}}} \\ 
\cline{9-23} \multicolumn{1}{c}{} & & \multicolumn{1}{c}{} & & & & & &
\multicolumn{3}{c}{\textbf{\begin{tabular}[c]{@{}c@{}}R$^2$\\ (-)\end{tabular}}} & &
\multicolumn{3}{c}{\textbf{\begin{tabular}[c]{@{}c@{}}RMSE\\ (s.u.)\end{tabular}}} & &
\multicolumn{3}{c}{\textbf{\begin{tabular}[c]{@{}c@{}}MAE\\ (s.u.)\end{tabular}}} & &
\multicolumn{3}{c}{\textbf{\begin{tabular}[c]{@{}c@{}}MAPE \\ (\%)\end{tabular}}} & &
\multicolumn{1}{c}{} \\
\cline{9-11} \cline{13-15} \cline{17-19} \cline{21-23}
\multicolumn{1}{c}{}  & & \multicolumn{1}{c}{}  & & \multicolumn{1}{c}{}  & & \multicolumn{1}{c}{}  & &
\multicolumn{1}{c}{T} & & \multicolumn{1}{c}{V} & &
\multicolumn{1}{c}{T} & & \multicolumn{1}{c}{V} & &
\multicolumn{1}{c}{T} & & \multicolumn{1}{c}{V} & &
\multicolumn{1}{c}{T} & & \multicolumn{1}{c}{V} & & \multicolumn{1}{c}{} \\
\cline{1-1} \cline{3-3} \cline{5-5} \cline{7-7} \cline{9-9} \cline{11-11} \cline{13-13} \cline{15-15} \cline{17-17} \cline{19-19} \cline{21-21} \cline{23-23} \cline{25-25}
\multirow{11}{*}{T$_{wct,out}$ ($^\circ$C)}
    % Include Poppe manually
    & & \textbf{Physical model} & & - & & - & & - & & \textbf{0.98} & & - & & \textbf{0.33} & & - & & \textbf{0.27} & & - & & \textbf{0.87} & & 6.288 & \\
     & & Feedforward ANN & & MIMO & & 20-2 & & 0.90 & & 0.81 & & 0.60 & & 0.97 & & 0.42 & & 0.67 & & 1.36 & & 2.36 & & 0.004 & \\ & & Cascade-forward ANN & & MIMO & & 10-10-2 & & 0.90 & & 0.82 & & 0.60 & & 0.93 & & 0.44 & & 0.65 & & 1.42 & & 2.27 & & 0.005 & \\ & & Radial basis ANN & & MIMO & & 34-2 & & 0.97 & & 0.97 & & 0.34 & & 0.41 & & 0.21 & & 0.28 & & 0.66 & & 0.94 & & 0.007 & \\ & & Feedforward ANN & & Cascade & & 20-1 & & 0.90 & & 0.82 & & 0.60 & & 0.93 & & 0.43 & & 0.65 & & 1.41 & & 2.26 & & 0.011 & \\ & & Cascade-forward ANN & & Cascade & & 10-10-1 & & 0.90 & & 0.83 & & 0.60 & & 0.92 & & 0.43 & & 0.64 & & 1.40 & & 2.24 & & 0.010 & \\ & & Radial basis ANN & & Cascade & & 92-1 & & 0.97 & & -1.44 & & 0.33 & & 3.45 & & 0.10 & & 2.12 & & 0.32 & & 7.43 & & 0.009 & \\ & & \textbf{Gaussian PR} & & Cascade & & N/A & & 0.99 & & \textbf{0.97} & & 0.20 & & \textbf{0.37} & & 0.15 & & \textbf{0.26} & & 0.47 & & \textbf{0.89} & & 0.001 & \\ & & Random forest & & Cascade & & N/A & & 0.75 & & 0.30 & & 0.96 & & 1.85 & & 0.60 & & 1.46 & & 2.03 & & 5.05 & & 0.078 & \\ & & Gradient boosting & & Cascade & & N/A & & 1.00 & & 0.68 & & 0.00 & & 1.24 & & 0.00 & & 0.95 & & 0.01 & & 3.29 & & 0.015 & \\ & & \textbf{Gaussian PR (FP)} & & Cascade & & N/A & & 1.00 & & \textbf{0.94} & & 0.32 & & \textbf{0.54} & & 0.15 & & \textbf{0.41} & & 0.52 & & \textbf{1.32} & & 0.105 & \\ \hline
\multirow{11}{*}{C$_{w}$ (l/h)}
    % Include Poppe manually
    & & \textbf{Physical model} & & - & & - & & - & &  \textbf{0.97} & & - & & \textbf{8.47} & & - & & \textbf{6.74} & & - & & \textbf{3.74} & & 6.288 & \\
     & & Feedforward ANN & & MIMO & & 20-2 & & 0.92 & & 0.83 & & 14.77 & & 21.58
     & & 11.98 & & 18.64 & & 9.91 & & 10.75 & & 0.004 & \\ & & Cascade-forward
     ANN & & MIMO & & 10-10-2 & & 0.92 & & 0.84 & & 15.47 & & 20.90 & & 12.51 &
     & 17.84 & & 10.48 & & 10.22 & & 0.005 & \\ & & Radial basis ANN & & MIMO &
     & 34-2 & & 0.99 & & 0.97 & & 5.58 & & 9.34 & & 3.81 & & 7.47 & & 3.23 & &
     4.68 & & 0.007 & \\ & & Feedforward ANN & & Cascade & &  20-1 & & 0.92 & &
     0.88 & & 15.00 & & 18.45 & & 11.97 & & 15.77 & & 10.20 & & 8.92 & & 0.011 &
     \\ & & Cascade-forward ANN & & Cascade & &  10-10-1 & & 0.92 & & 0.85 & &
     15.01 & & 20.34 & & 12.11 & & 17.66 & & 10.00 & & 10.18 & & 0.010 & \\ & &
     Radial basis ANN & & Cascade & &  33-1 & & 0.99 & & 0.93 & & 4.99 & & 14.28
     & & 3.45 & & 10.14 & & 2.68 & & 6.22 & & 0.009 & \\ & & \textbf{Gaussian
     PR} & & Cascade & & N/A & & 0.99 & & \textbf{0.95} & & 4.74 & &
     \textbf{12.00} & & 3.61 & & \textbf{9.96} & & 3.09 & & \textbf{6.32} & &
     0.001 & \\ & & Random forest & & Cascade & & N/A & & 0.89 & & 0.80 & &
     17.35 & & 23.23 & & 10.51 & & 18.51 & & 7.58 & & 9.73 & & 0.078 & \\ & &
     Gradient boosting & & Cascade & & N/A & & 1.00 & & 0.77 & & 0.24 & & 25.07
     & & 0.07 & & 17.21 & & 0.05 & & 9.55 & & 0.015 & \\ & & \textbf{Gaussian PR
     (FP)} & & Cascade & & N/A & & 0.98 & & \textbf{0.95} & & 10.85 & &
     \textbf{11.63} & & 4.81 & & \textbf{8.14} & & 3.74 & & \textbf{4.52} & &
     0.105 & \\ \hline
\end{tabular}%
}
\raggedright
\textcolor{darkgray}{\footnotesize\textit{s.u.} stands for \textit{same units} as the predicted variable}
\end{table*}


The results of each modelling alternative and its comparison can be visualized
in \reffig{cc:validation:wct:regression}. It shows the results obtained with the
models using the validation dataset (\reftab{cc:exp-campaigns} --
\textit{\gls{wctLabel}-val}). In \reftab{cc:validation:wct:results}, the performance
of the studied modelling approaches are included for the different performance
metrics\sidenote{Described in \refsec{intro:modelling:metrics}} where \texttt{T}
represents the performance metric value  for the training / calibration dataset
and \texttt{V} for the validation one. 

In both the figure and the table, there are two Gaussian PR models: one trained
using the experimental data (labeled as \textit{Gaussian PR}) and the second is
the surrogate model trained using the synthetic data from the physical model
(labeled as \textit{Gaussian PR (FP)}).

Comparing the physical model to the data-driven ones, it can be seen that there
are three groups: gradient boosting and random forest alternatives underperform
with R$^2$ values around 0.8 and some clear outliers in
\reffig{cc:validation:wct:regression}. \gls{annLabel} alternatives in general
provide good results, specially the radial-basis architecture. Finally, the
Gaussian Process Regression approach provides the best results among the
data-driven alternatives, with R$^2$ values above 0.95 for both output
variables. This advantage is mantained even in the case of the surrogate model with
R$^2$ values of 0.94 and 0.95 for the outlet temperature and water consumption,
respectively. The physical model provides superior performance for both output
variables: obtaining an \gls{rmseLabel} of 0.33 $^\circ$C and an R$^2$ of 0.98
for the temperature and in terms of water consumption, \gls{rmseLabel} and R$^2$
(8.5~l/h and 0.97). This explains how the surrogate model is able to outperform
many of the data-driven alternatives, as it provides a good approximation of the
superior physical model. It achieves this with a fraction of the computational cost,
vectorization support, limited need for experimental data (same as the physical
model) and covering a wide input space.

% =================================
\subsection{Dry cooler model alternatives comparison and validation}

\subsubsection{Physical model}

\begin{marginfigure}[]
    \includegraphics[]{figures/cc-dc_nusselt_regression.png}
    \caption{Air side Nusselt number vs air side Reynolds number for the air cooled heat exchanger}
    \labfig{cc:modelling:dc:nusselt}
\end{marginfigure}

The following $\dot{m}_{dc,air}$-$w_{dc}$ relation was obtained with the
\reftab{cc:exp-campaigns} -- \textit{\gls{dcLabel}-fan} dataset
\begin{equation}
    \dot{m}_{dc,air} =0.30195\cdot w_{dc}-1.02179. 
    \labeq{cc:dc:ma}
\end{equation}
The \gls{dcLabel} model uses this linear function to estimate the air mass flow rate, with
the $\omega_{dc}$ input applied in the pilot plant.

Once the air mass flow rate is established, the set of 27 experimental points
(see \reftab{cc:exp-campaigns} -- \textit{\gls{dcLabel}-cal}) is used to fit the
experimental data to \refeq{cc:Nu_corr}. To do so, Equations
\ref{eq:cc:dc_released}--\ref{eq:cc:dc_UA} were used to determine the air side heat
transfer coefficient using the inlet and outlet water temperatures, the air
inlet temperature, and the mass flow rates for both fluids as known values. 

\reffig{cc:modelling:dc:nusselt} shows the measured air side Nusselt number as a
function of the air side Reynolds number and the fitted equation as a straight
line. The correlation (using \refeq{dc:Nu}) fits 82~\% of the data points
with a deviation lower than 20~\%. 

\begin{equation}
    \Nus_a = 0.006411 \cdot \Rey_a^{0.9143} \cdot \Pra_a^{0.36}. 
    \labeq{dc:Nu}
\end{equation}

\subsubsection{Data-driven}

In order to generate the data-driven from first-principles alternative, the most
relevant input variables identified in \refsec{cc:modelling:dc:samples} are
discretized using a fixed number of resolution steps for each variable, within
ranges based on expected operating conditions, as defined in
\reftab{cc:validation:dc:samples-params}.
\reffig{cc:validation:samples-distribution} (b) visualizes the generated input
space distribution where it can be appreciated that the samples are well
distributed across the entire input space.

\begin{margintable}[]
    \caption{Bounds and discretization of the model input variables.}
    \labtab{cc:validation:dc:samples-params}
    \resizebox{\linewidth}{!}{%
    \begin{tabular}{lcccc}
        \toprule
        $\mathbf{x}$ & \textbf{Units} & \textbf{lb} & \textbf{ub} & \textbf{n} \\
        \midrule
        $T_{amb}$       & $^\circ$C            & 3   & 50   & 7 \\
        $\Delta T_{amb-dc,in}$       & $^\circ$C            & 3   & 30    & 7  \\
        $q_{dc}$       & m$^3$/h            & 6   & 24    & 7  \\
        $T_{dc,in}$   & $^\circ$C         & 25  & 45   & - \\
        $\omega_{\text{dc}}$  & \%         & 11  & 99.18   & 6 \\
        \bottomrule
    \end{tabular}
    }
\end{margintable}


\subsubsection{Validation}

\begin{table*}[]
\caption{Summary table of the prediction results obtained with the different modelling approaches studied.}
\labtab{cc:validation:dc:results}
\resizebox{\linewidth}{!}{%
    
\begin{tabular}{cclccccccccccccccccccccccc}
\hline
\multicolumn{1}{c}{\multirow{3}{*}{\textbf{\begin{tabular}[c]{@{}c@{}}Predicted\\ variable\end{tabular}}}} & &
\multicolumn{1}{c}{\multirow{3}{*}{\textbf{\begin{tabular}[c]{@{}c@{}}Modelling\\ alternative\end{tabular}}}} & &
\multicolumn{1}{c}{\multirow{3}{*}{\textbf{\begin{tabular}[c]{@{}c@{}}Model\\ config\end{tabular}}}} & &
\multicolumn{1}{c}{\multirow{3}{*}{\textbf{\begin{tabular}[c]{@{}c@{}}Topology\end{tabular}}}} & &
\multicolumn{15}{c}{\textbf{Performance metric}} & &
\multicolumn{1}{c}{\multirow{3}{*}{\textbf{\begin{tabular}[c]{@{}c@{}}Evaluation\\ time (s)\end{tabular}}}} \\ 
\cline{9-23} \multicolumn{1}{c}{} & & \multicolumn{1}{c}{} & & & & & &
\multicolumn{3}{c}{\textbf{\begin{tabular}[c]{@{}c@{}}R$^2$\\ (-)\end{tabular}}} & &
\multicolumn{3}{c}{\textbf{\begin{tabular}[c]{@{}c@{}}RMSE\\ (s.u.)\end{tabular}}} & &
\multicolumn{3}{c}{\textbf{\begin{tabular}[c]{@{}c@{}}MAE\\ (s.u.)\end{tabular}}} & &
\multicolumn{3}{c}{\textbf{\begin{tabular}[c]{@{}c@{}}MAPE \\ (\%)\end{tabular}}} & &
\multicolumn{1}{c}{} \\
\cline{9-11} \cline{13-15} \cline{17-19} \cline{21-23}
\multicolumn{1}{c}{}  & & \multicolumn{1}{c}{}  & & \multicolumn{1}{c}{}  & & \multicolumn{1}{c}{}  & &
\multicolumn{1}{c}{T} & & \multicolumn{1}{c}{V} & &
\multicolumn{1}{c}{T} & & \multicolumn{1}{c}{V} & &
\multicolumn{1}{c}{T} & & \multicolumn{1}{c}{V} & &
\multicolumn{1}{c}{T} & & \multicolumn{1}{c}{V} & & \multicolumn{1}{c}{} \\
\cline{1-1} \cline{3-3} \cline{5-5} \cline{7-7} \cline{9-9} \cline{11-11} \cline{13-13} \cline{15-15} \cline{17-17} \cline{19-19} \cline{21-21} \cline{23-23} \cline{25-25}
\multirow{7}{*}{T$_{dc,out}$ ($^\circ$C)}
    % Include physical model results manually
     & & \textbf{Physical model} & & - & & - & & - & & \textbf{0.98} & & - & & \textbf{0.50} & & - & & \textbf{0.42} & & - & & \textbf{1.28} & & 0.035 & \\
     & & Feedforward ANN & & - & & 20-1 & & 0.77 & & 0.78 & & 1.42 & & 1.62 & &
     1.13 & & 1.18 & & 3.29 & & 3.85 & & 0.005 & \\ & & Cascade-forward ANN & &
     - & & 10-10-1 & & 0.78 & & 0.85 & & 1.39 & & 1.37 & & 1.12 & & 1.02 & &
     3.23 & & 3.24 & & 0.007 & \\ & & \textbf{Gaussian PR} & & - & & N/A & &
     0.99 & & \textbf{0.99} & & 0.24 & & \textbf{0.32} & & 0.19 & &
     \textbf{0.25} & & 0.56 & & \textbf{0.77} & & 0.005 & \\ & & Random forest &
     & - & & N/A & & 0.84 & & 0.61 & & 1.19 & & 2.17 & & 0.72 & & 1.36 & & 2.05
     & & 4.69 & & 0.022 & \\ & & Gradient boosting & & - & & N/A & & 1.00 & &
     0.86 & & 0.00 & & 1.31 & & 0.00 & & 0.86 & & 0.00 & & 2.92 & & 0.035 & \\ &
     & \textbf{Gaussian PR (FP)} & & - & & N/A & & 1.00 & & \textbf{0.98} & &
     0.03 & & \textbf{0.53} & & 0.02 & & \textbf{0.44} & & 0.07 & &
     \textbf{1.35} & & 0.002 & \\ \hline \\
\end{tabular}%
}
\raggedright
\textcolor{darkgray}{\footnotesize\textit{s.u.} stands for \textit{same units} as the predicted variable}
\end{table*}

The results of the different modelling alternatives for predicting the
outlet temperature can be seen in \reftab{cc:validation:dc:results} (with the
validation dataset, \reftab{cc:exp-campaigns} -- \textit{\gls{dcLabel}-val}). The table
presents the performance of the physical and data-driven approaches in terms of
the selected performance metrics\sidenote{Described in
\refsec{intro:modelling:metrics}}.

\begin{figure}
    \savebox\captionqr{\qrcode[hyperlink,height=0.5in]{\repositoryBaseUrl/figures/cc-validation-dc-regression.html}}
    \includegraphics[width=\textwidth]{figures/cc-validation-dc-regression.png}
    \caption[\gls{dcLabel} Models performance regression]{\gls{dcLabel} models performance comparison between the different modelling approaches.\\[1ex] \usebox\captionqr}
    \labfig{cc:validation:dc:regression}
\end{figure}

Once again, two Gaussian Process Regression models are included: one trained directly with
the experimental data (\textit{Gaussian PR}) and the other acting as a surrogate
trained with the synthetic data from the physical model (\textit{Gaussian PR
(FP)}).

The physical model itself provides robust performance, with R$^2$ = 0.98 on the
validation set and the lowest error metrics among the deterministic approaches
(\gls{rmseLabel} = 0.50 $^\circ$C, \gls{maeLabel} = 0.42 $^\circ$C, and \gls{mapeLabel} =
1.28~\%). Nevertheless, the Gaussian PR trained with experimental data surpasses
this performance in all metrics: it achieves R$^2$ values of 0.99 for both
training and validation, and substantially lower error indicators
(\gls{rmseLabel} = 0.24–0.32 $^\circ$C, \gls{maeLabel} = 0.19–0.25 $^\circ$C). The surrogate
version, Gaussian PR (FP), also delivers excellent results, nearly matching the
experimental Gaussian PR model with R$^2$ values close to 1.00 and very low
errors (\gls{rmseLabel} = 0.03–0.53 $^\circ$C, \gls{maeLabel} = 0.02–0.44 $^\circ$C).
Importantly, this surrogate achieves such accuracy at an extremely low
evaluation time (0.002~s), which is one order of magnitude faster than the
physical model.

Among the machine learning alternatives, the feedforward and cascade-forward
\gls{annLabel} architectures yield reasonable results, with validation R$^2$
values between 0.78 and 0.85. However, their errors (\gls{rmseLabel} ≈ 1.37–1.62
$^\circ$C and \gls{mapeLabel} ≈ 3–4~\%) are significantly higher than those of the
Gaussian PR. Tree-based methods show mixed outcomes: random forest underperforms
with validation R$^2$ = 0.61 and high prediction errors (\gls{rmseLabel} = 2.17
$^\circ$C, \gls{mapeLabel} = 4.69~\%), while gradient boosting achieves nearly perfect
training performance but suffers from overfitting, with validation R$^2$ = 0.86
and errors similar to the ANN alternatives.

In summary, the Gaussian Process Regression models clearly outperform all other
data-driven methods, consistently reaching R$^2$ values above 0.98 and very low
error levels. The surrogate version (\textit{Gaussian PR (FP)}) combines this
predictive accuracy with extremely low computational cost, explaining its
advantage over more complex architectures. The physical model, although slightly
less accurate than Gaussian PR, still offers solid results and remains a
valuable benchmark for validating the data-driven approaches.

%================================
%================================
\subsection{Surface condenser model}
\labsec{cc:validation:condenser}

% For the surface condenser\sidenote{See
% \nrefsec{cc:modelling:surface-condenser}} a physical model is used, with
% the heat transfer coefficient as the only parameter to calibrate. Seven
% different alternative estimations of the heat transfer coefficient were
% calculated, using the data from the experimental campaign described in
% \nrefsec{cc:facility:exp-condenser}. They are as follows:

% \begin{enumerate}
%     \item Empirical correlation using the condenser flow rate ($q_c$) and the
%     vapor temperature (\(T_{v}\)) as inputs.
%     \item Empirical correlation using the cooling water inlet temperature
%     (\(T_{c,in}\)) and \(T_{v}\) as inputs.
%     \item Empirical correlation using the flow rate per condenser tube
%     (\(q_{c,tube}=q_c/n_{tubes}=q_c/24\)) and the cooling water inlet temperature.
%     \item Nominal value from the manufacturer, which equals 1.838
%     W/m$^2\,^\circ$C
%     \item Calibra\_Uexp\_original\todo{Estos qué son? Generar una nueva versión
%     de la figura una vez se seleccionen los métodos finales}
%     \item Calibra\_Uexp\_recortado
% \end{enumerate}

% \begin{figure}
%     \includegraphics[width=\textwidth]{figures/cc-validation-condenser-U.png}
%     \caption{Heat transfer coefficient calibration results}
%     \labfig{cc:validation:condenser-U}
% \end{figure}

% The results of the calibration are shown in \reffig{cc:validation:condenser-U},
% where the y-axis shows the thermal
% power obtained and the x-axis holds different bars for the different heat transfer
% coefficient estimation methods, with bars also for the experimental heat
% released by the vapor and absorbed by the coolant. As can be seen in the
% figure. The shown results are for steady-state conditions with the condenser in
% an equilibrium state (\(Q_{\text{released}} \approx Q_{\text{absorbed}}\)), and with a
% large variation in the condenser conditions (120 to 200~kW, the whole operating
% range of the condenser). The results show that the heat transfer coefficient
% obtained with the method 3 is the one that best fits the experimental data,
% with a \gls{maeLabel} of 17.6~kW and a maximum error of 33.41~kW (15~\%).

Following the same methodology as in the previous case, data from the model
calibration campaign (see \reftab{cc:exp-campaigns} -- \textit{\gls{scLabel}-cal}) was used to calculate the global heat
transfer coefficient, $U_{c}$, with \refeq{cc:sc_UA} together with the surface
condenser manufacturer's specifications from \reftab{cc:facility:sc}. $U_{c}$ is
in this case calibrated as a function of the input variables $T_{c,in}$ and
$q_{c}$, obtaining the following relation\sidenote{$p_1$=12.71, $p_2$=2.91,
$p_3$=9.5$\cdot$10$^{-3}$, $p_4$=343.29, $p_5$=-1$\cdot$10$^{-3}$ and
$p_6$=-4.83$\cdot$10$^{-2}$, where each parameter, $p_i$ has units consistent
with the variables involved.}:
\begin{equation}
   U_c= p_1\cdot T_{c,in}+p_2\cdot \dot{m}_{c,tb}+p_3\cdot \dot{m}_{c,tb}\cdot T_{c,in}+p_4+p_5\cdot \dot{m}_{c,tb}^2+p_6\cdot T_{c,in}^2,
\end{equation}
where $\dot{m}_{c,tb}$ is the water mass flow rate inside each condenser tube.


Using the previous relation and equations from \refmod{sc} and assuming that the
water leaving the shell side of the condenser is saturated at temperature $T_v$,
the surface condenser model can be solved to estimate $T_{c,out}$ and $T_v$.
Otherwise if $T_v$ is provided $T_{c,out}$ and $T_{c,in}$ can be
estimated.% In this case the inputs of the model are $T_{c,in}$,
% $\dot{m}_c$ and $\dot{m}_v$.

\subsubsection{Validation}

To validate this model, a different data set of 15 steady state tests
(\reftab{cc:exp-campaigns} -- \textit{\gls{scLabel}-val}) has been used with the
results depicted in \reffig{cc:validation:dc:regression}. The predicted outlet water
temperature in the tubes shows good agreement with the experimental
measurements, with a low \gls{maeLabel}. In contrast, the prediction error is larger for
the condensate water temperature at the shell-side outlet (\gls{maeLabel} =
1.86~$^\circ$C). This discrepancy may arise because the experimental outlet
temperature does not exactly match the vapor saturation temperature ($T_v$), but
is instead 0.1 to 1.7~$^\circ$C subcooled.

\begin{marginfigure}[*-5]
    \centering    
    \includegraphics[width=\linewidth]{cc-condenser_model_regression.png} % .85\textwidth
    \caption{Surface condenser validation}
    \labfig{cc:validation:condenser}
\end{marginfigure}


% =================================
\subsection{Complete system model validation}
\labsec{cc:validation:complete-system}

For the proposed optimization strategy in \nrefch{cc:optimization}, a model is
required that is fast, reliable, and easily scalable to different system sizes.

First-principle models, while accurate and broadly applicable, have a much
longer execution time compared to data-driven alternatives. This becomes a
critical drawback for optimization, where the model must be evaluated many
times within short time spans. In contrast, data-driven models can be executed
orders of magnitude faster and their runtime remains nearly constant
independently of the input conditions. However, they are limited to the specific
system and operating conditions on which they were trained, and achieving robust
performance typically demands significantly larger datasets.

The main strength of the physical models presented in this chapter lies in their
generality: they can predict cooler operation under a wide range of conditions
without retraining. Data-driven models, while faster and more suitable for
vectorized evaluation, trade this generality for speed, restricting their use to
the contexts for which they were developed and requiering significant data to
achieve satisfactory results.

Therefore, as combining a wet cooler and a dry cooler into a combined cooler
offers potential advantages compared to the individual systems, combining both
modelling approaches is the chosen solution to model the system. The best
performing data-driven model, the \gls{gprLabel} is calibrated using synthetic
data from the first-principle models, where physical models can be adapted to
different scales and finally the surrogate data-driven model can be generated.
This approach provides a way of having on-demand models that can be adapted to
the particular case study, while still being fast and efficient in terms of
computational resources.

The complete model of the combined cooler has been validated with a different
dataset composed of 24 tests (see \reftab{cc:exp-campaigns} --
\textit{\gls{ccLabel}-val}). The obtained outputs regression is shown in
\reffig{cc:validation:complete-model}. This figure compares the experimental
results with the predicted values. To visualize the operational characteristics
of each test, the data points are represented with the following information:
\begin{itemize}
    \item The dashed circle represents the nominal cooling power (200~kW$_\mathrm{th}$).
    \item The filled circle represents the cooling power measured in the test
    relative to the nominal value. The closer it is to the dashed circle, the
    closer the cooling power is to the nominal one.
    \item The measured cooling power is achieved using a certain percentage of
    \gls{dcLabel} and \gls{wctLabel}. These contributions are distinguished by
    green and purple, respectively. For example, if the ring is mostly green, it
    reflects that the cooling contribution from \gls{dcLabel} is predominantly
    larger than that from \gls{wctLabel}.
    \item The filling color inside the circle represents the ambient
    temperature. From low temperature (no filling) to high ambient temperature
    (dark yellow).
\end{itemize}

With this representation, it can be observed that the model provides
satisfactory results over a wide range of operating and ambient conditions. The
outlet temperatures show a \gls{maeLabel} lower than 0.97~$^\circ$C, with the
largest error occurring in $T_{wct,out}$, when the cooling power was far from
the nominal value. This may be due to the need to improve the relation
$\dot{m}_{air}-w_{wct}$ relationship at low flow rates. In the case of the water
consumption, the tendency (R$^2$=0.82) of the predicted values follows the
experimental ones, being the \gls{maeLabel} 19.4~l/h.

A summary of the models' results is shown in \reftab{cc:validation:results}.
This table includes the performance metrics of each component simulated
individually (\textit{Cnt} column) and those obtained with the complete model
(\textit{CC} column).

\begin{table}[]
\centering
\caption{Performance metrics obtained with the complete (CC) and component (Cnt) models}
\labtab{cc:validation:results}
\resizebox{.8\textwidth}{!}{%
\begin{tabular}{ccccccccccccc}
\hline
\multirow{3}{*}{\textbf{\begin{tabular}[c]{@{}c@{}}Predicted\\ variable\end{tabular}}} &  & \multicolumn{11}{c}{\textbf{Performance metric}}
\\\cline{3-13}
    &  & \multicolumn{3}{c}{\textbf{\begin{tabular}[c]{@{}c@{}}R$^2$\\ (-)\end{tabular}}} &  & \multicolumn{3}{c}{\textbf{\begin{tabular}[c]{@{}c@{}}MAE\\ (s.u.)\end{tabular}}} &  & \multicolumn{3}{c}{\textbf{\begin{tabular}[c]{@{}c@{}}MAPE\\ (\%)\end{tabular}}}
    \\\cline{3-5}\cline{7-9}\cline{11-13}
    &  & Cnt &  & CC &  & Cnt &  & CC &  & Cnt &  & CC
    \\\cline{1-1}\cline{3-3}\cline{5-5}\cline{7-7}\cline{9-9}\cline{11-11}\cline{13-13}
T$_{dc,out}$ ($^\circ$C) &  & 0.99 &  & 0.98 &  & 0.29 &  & 0.46 &  & 0.90 &  & 1.17 \\
T$_{wct,out}$ ($^\circ$C) &  & 0.92 &  & 0.94 &  & 1.01 &  & 0.97 &  & 3.01 &  & 2.72 \\
C$_{w}$ (l/h) &  & 0.87 &  & 0.82 &  & 16.55 &  & 19.40 &  & 10.42 &  & 11.03 \\
T$_{c,out}$ ($^\circ$C) &  & 0.98 &  & 0.99 &  & 0.23 &  & 0.41 &  & 1.51 &  & 1.00 \\
T$_{c,in}$ ($^\circ$C) &  & - &  & 0.99 &  & - &  & 0.53 &  & - &  & 1.52 \\ \hline
\end{tabular}%
}
% \raggedright
% \textcolor{darkgray}{\footnotesize\textit{s.u.} stands for \textit{same units} as the predicted variable}
\end{table}

\begin{marginfigure}[-12cm]
    \includegraphics[width=\linewidth]{figures/cc_model_regression.png}
    \savebox\captionqr{\qrcode[hyperlink,height=0.5in]{\repositoryBaseUrl/figures/cc_model_regression.html}}
    \caption[Combined cooler model validation]{Complete combined cooler model validation.\\[1ex]\usebox\captionqr}
    \labfig{cc:validation:complete-model}
\end{marginfigure}

\begin{figure*}
    \centering
    \subfloat[\centering \gls{wctLabel}]{
        \includegraphics[width=0.9\linewidth]{figures/cc-validation-samples_wct.png}
    }
    \\[1ex] % line break between subfigures, adjust spacing
    \subfloat[\centering \gls{dcLabel}]{
        \includegraphics[width=0.75\linewidth]{figures/cc-validation-samples_dc.png}
    }
    \caption{Samples distribution visualization for synthetic dataset generation.}
    \labfig{cc:validation:samples-distribution}
\end{figure*}


% Optimization validation
%================================
%================================
\section{Control and optimization results}
\labsec{cc:validation:optimization}

Once the models of the main components of the system have been validated, the
next step is to validate the optimization strategies proposed in
\nrefsec{cc:modelling:optimization}. First, an optimization algorithm is chosen
by comparing different alternatives in \nrefsec{cc:validation:algorithm} for
each optimization alternative (\ie static and horizon). Then, the two proposed
variants for the combined cooler are compared in simulation for one
representative operation period in the simulated pilot plant in order to see
which one performs better in
\nrefsec{cc:validation:optimization:static-vs-horizon}. Finally, the proposed
horizon methodology is tested in the real facility, where planned changes are
introduced in the operation schedule, in order to validate how the optimization
strategy adapts to changing conditions.

%================================
\subsection{Choosing an optimization algorithm}
\labsec{cc:validation:algorithm}

\subsubsection{Static problems}

For every static optimization problem (\refprob{cc:dc}, \refprob{cc:wct},
\refprob{cc:static}) three different algorithms are tested: \gls{seacstrLabel},
\gls{ihsLabel} and \gls{decstrLabel}. For each alternative the same number of
objective function evaluations are given (800) but they are distributed
differently depending on the algorithm:
\begin{itemize}
    \item \gls{seacstrLabel} and \gls{decstrLabel} make use of the
    \gls{cstrLabel} wrapper algorithm, which allows them to the constrained
    problems. 10 iterations are performed for this wrapper algorithm, leaving
    80 iterations to spare for the inner algorithm.
    
    \item For all alternatives, three values are tested for the initial
    population size: 50, 100 and 400 individuals\sidenote{The initial
    population fitness evaluation is not counted for the budget of
    objective function evaluations}. 
    
    \item Depending on the algorithm only one individual is evolved
    (\gls{ihsLabel} and \gls{seaLabel}) or the whole population (\gls{deLabel}).
    This means that 800 generations are available for \gls{ihsLabel}, 80
    generations for \gls{seacstrLabel} and for \gls{decstrLabel}, 1 generation
    is available for the population of 50 individuals, while only the initial
    generation is for the population of 100 and 400 individuals.
\end{itemize}

\begin{table}[]
\caption{Static optimization algorithm comparison results}
\labtab{cc:validation:static-algo-comparison}
\resizebox{\linewidth}{!}{%
\begin{tabular}{llllclclccccccccc}
\hline
\multirow{2}{*}{\textbf{System}} &  & \multicolumn{1}{c}{\multirow{2}{*}{\textbf{Algorithm}}} &  & \multicolumn{5}{c}{\textbf{Parameters}}                                                                                             &  & \multicolumn{7}{c}{\textbf{Average fitness per obj. fun. evaluations}}                                                                 \\ \cline{5-9} \cline{11-17} 
                                 &  & \multicolumn{1}{c}{}                                    &  & \textbf{\begin{tabular}[c]{@{}c@{}}pop\\ size\end{tabular}} &  & \textbf{gen}     &  & \textbf{\begin{tabular}[c]{@{}c@{}}wrapper\\ algo\\ iters\end{tabular}} &  & \multicolumn{1}{c}{\textbf{0}} & \textbf{} & \multicolumn{1}{c}{\textbf{50}} & \textbf{} & \multicolumn{1}{c}{\textbf{150}} & \textbf{} & \multicolumn{1}{c}{\textbf{800}} \\ \cline{1-1} \cline{3-3} \cline{5-5} \cline{7-7} \cline{9-9} \cline{11-11} \cline{13-13} \cline{15-15} \cline{17-17} 


\multirow{9}{*}{\textbf{\protect\gls{dcLabel}}}       &  & \multirow{3}{*}{\protect\gls{ihsLabel}}                              &  & 50                                       &  & 800 &  & N/A                                                    &  & $1.28\pm0.82$             &           & $1.05\pm0.29$              &           & $0.80\pm0.10$               &           & $0.77\pm0.09$               \\
                                 &  &                                                         &  & 100                                       &  & 800 &  & N/A                                                    &  & $0.92\pm0.18$             &           & $0.87\pm0.14$              &           & $0.81\pm0.11$               &           & $0.77\pm0.10$               \\
                                 &  &                                                         &  & 400                                       &  & 800 &  & N/A                                                    &  & $0.81\pm0.11$             &           & $0.80\pm0.11$              &           & $0.79\pm0.10$               &           & $0.77\pm0.10$               \\ \cline{3-3} \cline{5-9} \cline{11-17} 
                                 &  & \multirow{3}{*}{\protect\gls{seacstrLabel}}                              &  & 50                                       &  & 80 &  & 10                                                    &  & $1.19\pm0.28$             &           & $0.95\pm0.11$              &           & $0.79\pm0.10$               &           & $0.77\pm0.09$               \\ 
                                 &  &                                                         &  & 100                                       &  & 80 &  & 10                                                    &  & $0.92\pm0.13$             &           & $0.86\pm0.10$              &           & $0.80\pm0.10$               &           & $0.77\pm0.09$               \\
                                 &  &                                                         &  & 400                                       &  & 80 &  & 10                                                    &  & $0.82\pm0.10$             &           & $0.80\pm0.10$              &           & $0.78\pm0.10$               &           & $0.77\pm0.09$               \\ \cline{3-3} \cline{5-9} \cline{11-17} 
                                 &  & \multirow{3}{*}{\protect\gls{decstrLabel}}                              &  & 50                                       &  & 1 &  & 10                                                    &  & $1.06\pm0.40$             &           & $0.97\pm0.18$              &           & $0.83\pm0.10$               &           & $1.04\pm1.04$               \\ 
                                 &  &                                                         &  & 100                                       &  & 0 &  & 10                                                    &  & $0.95\pm0.16$             &           & $0.95\pm0.16$              &           & $0.95\pm0.95$               &           & $0.95\pm0.95$               \\
                                 &  &                                                         &  & 400                                       &  & 0 &  & 10                                                    &  & $0.83\pm0.10$             &           & $0.83\pm0.10$              &           & $0.83\pm0.10$               &           & $0.83\pm0.83$               \\ \hline


\multirow{9}{*}{\textbf{\protect\gls{wctLabel}}}       &  & \multirow{3}{*}{\protect\gls{ihsLabel}}                              &  & 50                                       &  & 800 &  & N/A                                                    &  & $0.24\pm0.08$             &           & $0.18\pm0.04$              &           & $0.10\pm0.00$               &           & $0.07\pm0.00$               \\
                                 &  &                                                         &  & 100                                       &  & 800 &  & N/A                                                    &  & $0.12\pm0.02$             &           & $0.11\pm0.01$              &           & $0.08\pm0.00$               &           & $0.07\pm0.00$               \\
                                 &  &                                                         &  & 400                                       &  & 800 &  & N/A                                                    &  & $0.07\pm0.00$             &           & $0.07\pm0.00$              &           & $0.07\pm0.00$               &           & $0.07\pm0.00$               \\ \cline{3-3} \cline{5-9} \cline{11-17} 
                                 &  & \multirow{3}{*}{\protect\gls{seacstrLabel}}                              &  & 50                                       &  & 80 &  & 10                                                    &  & $0.25\pm0.04$             &           & $0.16\pm0.01$              &           & $0.07\pm0.00$               &           & $0.06\pm0.00$               \\ 
                                 &  &                                                         &  & 100                                       &  & 80 &  & 10                                                    &  & $0.17\pm0.03$             &           & $0.11\pm0.00$              &           & $0.07\pm0.00$               &           & $0.06\pm0.00$               \\
                                 &  &                                                         &  & 400                                       &  & 80 &  & 10                                                    &  & $0.07\pm0.00$             &           & $0.07\pm0.00$              &           & $0.07\pm0.00$               &           & $0.06\pm0.00$               \\ \cline{3-3} \cline{5-9} \cline{11-17} 
                                 &  & \multirow{3}{*}{\protect\gls{decstrLabel}}                              &  & 50                                       &  & 1 &  & 10                                                    &  & $0.29\pm0.07$             &           & $0.17\pm0.02$              &           & $0.09\pm0.00$               &           & $0.07\pm0.07$               \\ 
                                 &  &                                                         &  & 100                                       &  & 0 &  & 10                                                    &  & $0.11\pm0.00$             &           & $0.11\pm0.00$              &           & $0.11\pm0.11$               &           & $0.11\pm0.11$               \\
                                 &  &                                                         &  & 400                                       &  & 0 &  & 10                                                    &  & $0.07\pm0.00$             &           & $0.07\pm0.00$              &           & $0.07\pm0.00$               &           & $0.07\pm0.07$               \\ \hline


\multirow{9}{*}{\textbf{\protect\gls{ccLabel}}}       &  & \multirow{3}{*}{\protect\gls{ihsLabel}}                              &  & 50                                       &  & 1000 &  & N/A                                                    &  & $0.77\pm0.12$             &           & $0.80\pm0.11$              &           & $0.77\pm0.11$               &           & $0.59\pm0.11$               \\
                                 &  &                                                         &  & 100                                       &  & 1000 &  & N/A                                                    &  & $0.70\pm0.12$             &           & $0.78\pm0.10$              &           & $0.82\pm0.15$               &           & $0.61\pm0.13$               \\
                                 &  &                                                         &  & 400                                       &  & 1000 &  & N/A                                                    &  & $0.79\pm0.19$             &           & $0.82\pm0.21$              &           & $0.80\pm0.22$               &           & $0.65\pm0.16$               \\ \cline{3-3} \cline{5-9} \cline{11-17} 
                                 &  & \multirow{3}{*}{\protect\gls{seacstrLabel}}                              &  & 50                                       &  & 100 &  & 10                                                    &  & $0.92\pm0.13$             &           & $0.86\pm0.14$              &           & $0.74\pm0.16$               &           & $0.51\pm0.10$               \\ 
                                 &  &                                                         &  & 100                                       &  & 100 &  & 10                                                    &  & $0.88\pm0.16$             &           & $0.82\pm0.16$              &           & $0.75\pm0.21$               &           & $0.62\pm0.16$               \\
                                 &  &                                                         &  & 400                                       &  & 100 &  & 10                                                    &  & $0.84\pm0.21$             &           & $0.80\pm0.18$              &           & $0.74\pm0.21$               &           & $0.69\pm0.19$               \\ \cline{3-3} \cline{5-9} \cline{11-17} 
                                 &  & \multirow{3}{*}{\protect\gls{decstrLabel}}                              &  & 50                                       &  & 2 &  & 10                                                    &  & $0.83\pm0.16$             &           & $0.79\pm0.13$              &           & $0.73\pm0.14$               &           & $0.56\pm0.13$               \\ 
                                 &  &                                                         &  & 100                                       &  & 1 &  & 10                                                    &  & $0.82\pm0.17$             &           & $0.80\pm0.13$              &           & $0.77\pm0.10$               &           & $0.64\pm0.13$               \\
                                 &  &                                                         &  & 400                                       &  & 0 &  & 10                                                    &  & $0.73\pm0.16$             &           & $0.73\pm0.16$              &           & $0.73\pm0.16$               &           & $0.73\pm0.73$               \\ \hline


\end{tabular}
}
\end{table}

\reftab{cc:validation:static-algo-comparison} shows the results obtained, in
terms of fitness at different stages in the evolution. From the results it can
be seen that for all alternatives the best performing and most consistent
algorithm is \textbf{\gls{ihsLabel}}. It converges quickly, provides stable
results across different population sizes, and consistently reaches the best
fitness values within the evaluation budget. The \gls{seacstrLabel} also shows
competitive results, especially for small populations, but tends to converge
more slowly. In contrast, the \gls{decstrLabel} is limited by the small number
of generations available when larger populations are considered, leading to
poorer performance and higher variability. Overall, the results indicate that
the incremental evolution of a single individual, as in \gls{ihsLabel}, is
better suited to the limited evaluation budget imposed in these static
optimization problems.

\subsubsection{Horizon optimization. Path selection}

A methodology similar to the static comparison is used. This time the
algorithms evaluated are: \gls{gacoLabel}, \fullgls{ihsLabel}, \gls{sgaLabel} and
\gls{psoLabel}. Three different population sizes are tested (80, 150 and 1000)
if the particular algorithm evolves more than one individual; the number of
generations is calculated accordingly so that all alternatives have the same
budget of objective function evaluations, equal to 200k
evaluations\sidenote{Only up to 50k evaluations is shown in the figure for
clarity}. The results are visualized in
\reffig{cc:validation:path_explorer_algo_comp}, where there are different plots
for different dates, the y-axis represents the fitness and the x-axis shows the
number of objective function evaluations. The results show that consistently
the \gls{sgaLabel} outperforms the alternatives, and particularly, the smaller
population size (80) configuration followed very closely by the 150 population
size configuration.

\begin{figure*}[h!]
    \includegraphics[]{figures/cc-validation-path_explorer_algo_comp.png}
    \savebox\captionqr{\qrcode[hyperlink,height=0.5in]{\repositoryBaseUrl/figures/cc-validation-path_explorer_algo_comp.html}}
    \caption[Horizon optimization -- path selection subproblem. Fitness
    evolution comparison]{Horizon optimization --
    path selection subproblem. Fitness evolution comparison for different
    algorithms in four different
    dates.\hspace{1ex}\usebox\captionqr}
    \labfig{cc:validation:path_explorer_algo_comp}
\end{figure*}
.

%================================
\section{Implementation results}[Results]
\labsec{cc:simulation:optimization-results}

\reffig{cc:simulation:results-multiple-days} shows the optimization
results\sidenote{in simulation and at pilot-scale} for the horizon optimization
strategy\sidenote{This results figure structure is going to be repeated several
times in the next sections.}. This is for the same three days in different
months: 15-February, 15-June, 15-September, that is, different periods of the
year. This means that three different environments are available, respectively:
cold\sidenote{\reffig{cc:simulation:results-multiple-days} - \textit{Weather
conditions}} and wet\sidenote{\ie plenty of water resource available,
\reffig{cc:simulation:results-multiple-days} - \textit{Resources context}},
moderate temperature and water availability; and finally hot and dry.

The cooling costs obtained in each scenario are displayed in
\reffig{cc:simulation:results-multiple-days} - \textit{Operational costs}. In
yellow costs associated with electricity use while green for water associated
costs. 

In terms of hydraulic distribution, it can be observed that when the water
resource is abundant (\ie 15-February) the system relies mostly in the wet
cooler and when combined with the dry cooler, is the only scenario where the
parallel configuration is used. Both in the moderate (15-June) and hot-dry
(15-September) scenarios, the system operates mostly in series, or using
exclusively the dry cooler. 

At the bottom of the figure, in the \textit{Components} section, the cooling
power provided by each cooler ($\dot{Q}_{wct}$ / $\dot{Q}_{dc}$) is shown with
respect to the total ($\dot{Q}_{c}$). It can be observed that in the cold and wet
season the system relies mostly on the wet cooler with some assist from the dry
cooler also boosted by the low ambient temperatures. In the moderate scenario
the roles reverse, with the dry cooler providing most of the cooling power
while the wet cooler is used only occasionally, at the central hours of the day
where the ambient temperature is higher and the dry cooler is not able to cope with
the full load. Finally, in the hot and dry scenario, the system tries to rely
exclusively on the dry cooler. However, at the peak hours of the day, the dry
cooler is not able to provide the full cooling load, and the wet cooler is
used to supplement the cooling power (11:00 -- 16:00).

Comparing the costs, in the cold and wet season (15-Feb) not zero but orders of magnitude
cheaper cooling is obtained compared to the other two scenarios since the
thermal load is lower and the wet system can be broadly used. In the moderately
hot and dry season significant higher costs can be observed due to the extended
use of the more expensive dry cooler. This is still cheaper than the hot and dry
season, where on top of the already more expensive dry cooler cooling, the
sparse wet cooler use and the lack of water resource significantly increase the
cooling costs by 50~\% compared to the moderate scenario.

% {%
% \newgeometry{margin=0.5cm}
% \begin{figure}
%     \savebox\captionqr{\qrcode[hyperlink,height=1cm]{\repositoryBaseUrl/figures/cc-horizon-optimization-results-multiple-days.zip}}
%     \caption{\rotatebox{90}{A rotated figure \hspace{1ex}\usebox\captionqr}}
%     \makebox[\textwidth][c]{\includegraphics[angle=90,width=0.65\paperwidth]{figures/cc-horizon-optimization-results-multiple-days.png}}%
% \end{figure}%
% \restoregeometry
% }

% \newgeometry{margin=0.5cm}
% \begin{sidewaysfigure}
%     \centering
%     \includegraphics[width=0.85\paperheight]{cc-horizon-optimization-results-multiple-days.png}
%     \\[3ex]

%     \caption[Horizon optimization results]{Horizon optimization results in three different seasons}

%     \captionlistentry{Extra caption text} % does not interfere
%     \par\smallskip
%     \begin{minipage}{\textwidth}
%       \centering
%       \begin{tabular}{m{0.3\textwidth} c}
%         Degrading water availability from left (cold weather and abundant water) to right (hot and water scarcity) &
%         \usebox\captionqr
%       \end{tabular}
%     \end{minipage}
% \end{sidewaysfigure}
% \restoregeometry

\newgeometry{margin=0.5cm}
\begin{figure}
    \centering
    \savebox\captionqr{\qrcode[hyperlink,height=1cm]{\repositoryBaseUrl/figures/cc-horizon-optimization-results-multiple-days.zip}}

    \makebox[\textwidth][c]{%
      \includegraphics[angle=90,height=0.9\paperheight]{cc-horizon-optimization-results-multiple-days.png}%
    }

    \captionof{figure}{Horizon optimization results in three different seasons
    \hspace{1ex}\usebox\captionqr}
    \labfig{cc:simulation:results-multiple-days}
\end{figure}
\restoregeometry
\pagelayout{margin} % Restore margins to class's default

\subsubsection{Comparing the static and horizon optimization strategies}
\labsec{cc:validation:optimization:static-vs-horizon}


\begin{figure*}[h!]
    \includegraphics[]{figures/cc-validation-comp_static_horizon.png}
    \savebox\captionqr{\qrcode[hyperlink,height=0.5in]{\repositoryBaseUrl/figures/cc-validation-comp_static_horizon.html}}
    \caption[Detailed simulation results comparison: horizon vs static.
    \gls{ccLabel}--horizon]{Simulation results for the horizon optimization
    compared to the static alternative. The horizon-based strategy is shown with
    solid lines (and filled areas where relevant), while the static strategy is
    represented with dotted lines in the same colors as the corresponding
    variables\\[1ex]\usebox\captionqr}
    \labfig{cc:validation:static-vs-horizon}
\end{figure*}


\reffig{cc:validation:static-vs-horizon} shows the results obtained for a range
of days (May 3 to May 21) with the two optimization alternatives: static and
horizon, presented in Sections \ref{sec:cc:optimization:static:cc} and
\ref{sec:cc:optimization:horizon}, respectively. Both are evaluated given the same
environment, which is shown at the top of the figure. The horizon strategy is
depicted with solid lines and/or filled areas. In comparison, the static results
are included using dotted lines with the same color for the compared variable.
The results show that the horizon strategy consistently outperforms the static
one in terms of cost (\reffig{cc:validation:static-vs-horizon} --
\textit{Cumulative cost}), with a cost advantage of more than 20~\% and often
above 50~\%. This can be explained by how each strategy manages the water
resource\sidenote{See \reffig{cc:validation:static-vs-horizon} -
\textit{Resources context -- }$V_{available}$}: the static strategy tends to use
a lot of water at the beginning of the day, when the ambient temperature is low,
and then as the day progresses and the ambient temperature rises, it has less
water available and is forced to use the alternative, more expensive water
source. In contrast, the horizon strategy manages the water resource more
efficiently, using it when it is most needed and preserving it for the times
when it will be most beneficial.


%================================
\subsection{Validation at pilot plant}

A hierarchical control\sidenote{See \refsec{intro:hierarchical-control}}
strategy has been implemented in order to validate the optimization strategy in
the real facility. \reffig{cc:validation:optimization:diagram} shows a diagram
of the methodology, where the left side represents the upper layer with the
proposed shrinking horizon optimization\sidenote{See
\nrefsec{optimization:horizon}} and the right side shows the low-level
regulatory control layer, which directly interfaces with the actuators and
sensors of the facility. 

\begin{figure*}
    \includegraphics[]{figures/cc-validation-diagram.png}
    \caption{Implementation of the optimization strategy in the real facility.
    Hierarchical control}
    \labfig{cc:validation:implementation:diagram}
\end{figure*}

\begin{margintable}[]
    \caption{Box-bounds for the decision variables.}
    \labtab{cc:validation:optimization-bounds}
    \begin{tabular}{lccc}
        \toprule
        $\mathbf{x}$ & \textbf{Units} & \textbf{lb} & \textbf{ub} \\
        \midrule
        $q_c$       & m$^3$/h            & 5.22   & 24.15   \\
        $R_p$       & --            & 0.00   & 1.00    \\
        $R_s$       & --            & 0.00   & 1.00    \\
        $\omega_{\text{dc}}$   & \%         & 11.00  & 99.18   \\
        $\omega_{\text{wct}}$  & \%         & 21.00  & 93.42   \\
        \bottomrule
    \end{tabular}
\end{margintable}

\textbf{Environment}. The optimization environment was generated using weather
forecasts obtained from the \href{https://openweathermap.org/api}{OpenWeather
API}~\sidecite{openweather_current_2025}. Electricity cost data were taken from
the 2022 Spanish market pool~\sidecite{REE_website}. The water cost was defined
as $P_{w,s1}=0.03$~\euro/m$^3$ for source~1, and $P_{w,s2}=80\times P_{e} =
8\pm5$~\euro/m$^3$ for source~2.

The thermal load profile was created by fixing the vapor temperature at
$T_v=45,^\circ$C and generating an arbitrary cooling power based on the heat
available from the flat-plate collector field—the system's heat source for the
selected day. Finally, the initial available water volume was set to
$V_{avail,0}=0.4$~m$^3$, after which it was updated dynamically according to the
system's real-time consumption data.

\textbf{Optimization layer}. The optimization algorithm is run every 30 minutes,
and generates a new set of results for the remaining operation time. The results
of the optimization are then passed to the regulatory control layer by setting
them as setpoints for the low-level control. The box-bounds for the decision
variables are shown in \reftab{cc:validation:optimization-bounds}. Since the
low-level control layer actuates over the coolers fan speeds, instead of using
$\omega_{dc}$ and $\omega_{wct}$ directly as setpoints, the would be obtained
outlet temperatures values ($T_{dc,out}$ and $T_{wct,out}$) are used instead.
This makes the strategy more robust, since as long as the outlet temperatures
from the coolers are kept, and they will be since the fan speed is continuously
regulated to achieve that, the correct condenser operation can be guaranteed,
despite possible deviations. 

The optimization algorithm is executed every 30~minutes and generates a new set
of results for the remaining operation period. These results are then passed to
the regulatory control layer, where they serve as setpoints for the low-level
control. The box constraints for the decision variables are provided in
\reftab{cc:validation:optimization-bounds}.

\begin{table}
\caption{Low-level control loops}
\centering
\labtab{cc:validation:control}
\resizebox{\textwidth}{!}{%
\begin{tabular}{cccccccc}
\hline
\multirow{2}{*}{\textbf{Controller}} & \multicolumn{2}{c}{\textbf{Control signal}} & \multicolumn{2}{c}{\textbf{Controlled signal}} & \multicolumn{2}{c}{\textbf{Controller parameters}} \\
\cline{2-7}
& Variable & P\&ID & Variable & P\&ID & $K_p$ & $T_i$ (s)\\
\hline
TC-01 & $\omega_{dc}$ & SC-02, SC-03 & $T_{dc,out}$ & TT-03 & [-16.8, -3] \%/$^\circ$C  & [30.2, 155.7]\\
TC-02 & $\omega_{wct}$ & SC-01  & $T_{wct,out}$ & TT-06 & -5.9 \%/$^\circ$C & 78.0\\ 
FC-01 & $\omega_c$ & SC-04  & $q_{c}$ & FT-06 & 0.9 \%$\cdot (h\cdot$m$^{3})$ & 1.3\\ 
FC-02 & V$_p$ & ZC-02  & $q_{dc}$ & FT-02 & -1.18 \%$\cdot h\cdot$m$^{-3}$ & 4.3\\ 
FC-03 & V$_s$ & ZC-01  & $q_{dc-wct}$ & \textit{f}(FT-01,FT-02, FT-03) & 0.7 \%$\cdot h\cdot$m$^{-3}$ & 4.0\\ \hline
\end{tabular}
}
\end{table}

\textbf{Control layer}. It is a regulatory layer with five control loops (see
\reftab{cc:validation:control}). The purpose of this layer is to track the
setpoints calculated by the upper layer for the five controlled variables and
maintain them near steady-state conditions around these references, even in the
presence of disturbances such as variations in temperature or flow rate.
Classical feedback loops with \gls{piLabel} controllers have been used in this
regulatory layer, which were tuned using the improved SIMC technique~
\sidecite{skogestad_simc_2012}. \reftab{cc:validation:control} shows the
proportional gain, $K_p$, and the integral time, $K_i$, for each control loop.
The ideal configuration of the \gls{piLabel} controller has been
implemented\sidenote{\ie $C(s) = K_p(1 + 1/(T_i·s))$}, including anti-windup
mechanism and a sample time of one second. In the special case of the
\gls{acheLabel}, due to its strong nonlinear dynamics, a Gain Scheduling
scheme~\sidecite{hagglund_advanced_2006} has been implemented. For this purpose,
a 3$\times$3 matrix of regions has been defined to account for three levels: low,
medium and high values in the three main variables involved
($q_{dc}$,$\omega_{dc}$ and $\Delta T_{in-amb}$). 

The optimization does not provide $\omega_{dc}$ and $\omega_{wct}$ as setpoints.
Instead, it provides the expected corresponding outlet temperature values. This
way the low-level control layer directly adjusts the fan speeds of the coolers
in order to regulate the outlet temperatures ($T_{dc,out}$ and $T_{wct,out}$) to
the setpoints provided. This approach increases the robustness of the control
strategy: as long as the outlet temperatures of the coolers are maintained—and
they are, since fan speeds are continuously regulated to achieve them—the proper
operation of the condenser can be ensured, even in the presence of potential
deviations due to modelling inaccuracies, external disturbances or uncertainty
in the forecasted weather conditions.

\textbf{Experimental results}. In order to validate the optimization strategy,
several tests were performed over different days. In particular
\reffig{cc:validation:exp-results} visualizes one test carried out in the 1st
August to analyze in this section. The objective of the test was twofold. For
the first part of the test a set operation plan was established: 

\begin{itemize}
    \item $\dot{Q}$ = Ramp up from 150 to 200 kW from 08:40 until 10:00, and
    hold the 200 kW value until the end of the test (13:00).
    \item $T_v$ = 45 $^\circ$C. Held constant throughout the experiment (08:40 -- 13:00)
\end{itemize}

The objective was to validate that the optimized operation based on the provided
predictions was effectively able to correctly manage resources and cool the
thermal load with the predicted associated consumptions. During operation of a
CSP plant changes to the operation plan can arise in response to changes in
electricity market dynamics, or other unforeseen environment circumstances. At
10:20 a change is introduced in the operation plan to simulate this behavior,
the thermal load was ramped down with a similar (inverse) profile to the initial
one. This allows to verify the adaptability of the proposed strategy to changing
conditions and is the second objective of the test.

The operation strategy was as follows:

\begin{itemize}
    \item Before the test and while the system starts up by generating vacuum in
    the surface condenser, the optimization layer was evaluated to have an
    initial perspective on the day operation and expected consumptions. If the
    operator was satisfied the provided values were used as reference and
    manually set to bring the system into stable operation after gradually
    increasing the thermal load.
    
    \item The optimization sample time was 20 minutes, it takes around that time
    to compute and is evaluated every 40 minutes.
    
    \item The thermal load was designed to change every 40 minutes, this means
    that for each optimization evaluation, two setpoint changes are provided to
    the regulatory control layer per optimization evaluation and thus
    predictions must be valid for those at least 40 minutes.
    
    \item For every optimization layer evaluation, first the environment is
    updated and then is provided as input to the optimization evaluation (see
    \reffig{cc:validation:implementation:diagram}).
    
    \item The low-level control layer has available the operation strategy for
    the whole horizon provided by the upper layer and following its schedule
    updates its setpoints.
\end{itemize}


%% Descripción de la figura %%
\reffig{cc:validation:exp-results} is divided in several sections. In general
solid lines represent measured (experimental) values, while the thin-dashed
equivalent (same color) is the predicted value by the upper-optimization layer.
This predicted value is provided by the latest evaluated optimization. The upper
section of the figure displays the environment evolution (weather conditions,
load conditions and resources context). They are followed by a comparison
between predicted and actual results for: (a) distribution between cooling
systems, in terms of flows (\textit{hydraulic distribution}) and in terms of the
assigned cooling power (\textit{cooling power distribution}) and (b) individual
cooler outputs in terms of temperature profile and water consumption in the wet
cooler case. Finally, the right group of plots shows the low-level control layer
performance for each control loop: coolers outlet temperature and flows.

In \reffig{cc:validation:exp-results} -- \textit{Hydraulic Distribution}, several sequentially added bars
are shown. The first bar corresponds to the experimental value, while the
remaining bars represent the predicted hydraulic distributions from successive
optimization evaluations: the second bar comes from the first evaluation, the
third from the second, and so on.

From the results, a few observations can be made:

\begin{itemize}
    \item Overall a very good agreement between the optimization layer predicted
    operation and the experimental values can be observed. It can be seen than
    as long as the environment does not change, the generated operation strategy
    is valid for hours. Particularly, the initial evaluation at XX:XX. In
    another test (not shown) where the thermal load does not change throughout
    the day the initial optimization held valid until the end of operation. But
    as mentioned, in this particular test a planned change in the thermal load
    profile is introduced at 10:30, which is then taken into account in the
    optimization layer re-evaluation at 10:40, and thus a new adapted operation
    strategy to the new scenario is generated and successfully applied.
    
    \item The dry system is very sensitive to the ambient temperature when
    operating in its limits. Less than half-degree prediction error in the
    ambient temperature (0.4 $^\circ$C between 09:04 and 09:17) translates in a
    15\% difference between the expected and the actual fan speed.
    
    \item When both systems are operating, if the dry cooler falls short in its
    cooling allocation, the wet cooler can compensate for the dry cooler
    shortcoming on its cooling allocation. However, as can be seen at the
    beginning of the test (09:04 -- 09:17), when only the dry cooler is used and
    does so in its limits (\reffig{cc:validation:exp-results} - \textit{\gls{dcLabel} outlet temperature loop - Control
    signal}) it can happen that the load is undercooled resulting in a higher
    condenser pressure (in terms of temperature, +1-2 $^\circ$C can be observed,
    which would translate in a penalty in the power produced by the turbine). A
    low-level supervisory controller should be set in place to prevent this.
    
    \item To avoid using the alternative more expensive alternative water
    source, the optimization prioritized the use of the dry cooling (as far as
    being dry-only as long as the ambient temperature and demanded thermal load
    allowed it, up until 09:20) to conserve water until the end of operation. After the
    operation plan change, the lower expected load gives more room for
    adjustment and the optimization increases the load through the wet system
    from 0-40\% to about 50\% (\reffig{cc:validation:exp-results} - \textit{Hydraulic distribution and Cooling
    power distribution}).
    
    \item The restricted availability of the water resource, means that the
    optimization strategy always prioritizes water savings either by dry-only
    operation, or combined operation using a series configuration, at no point
    the parallel configuration is used despite progressively increased
    electricity cost (\textit{Resources context} -- $P_e$).
    
    \item The good agreement between upper and lower layer, means that the upper
    layer predicted controlled variables values could be used by the low-level
    control, for example, in a static feed-forward action.
    
    \item From the initial optimization evaluation, the low-level control layer
    has available an operation strategy for the whole horizon. This makes the
    strategy robust in the case the optimization is not evaluated again, or not
    evaluated on time.
\end{itemize}

\newgeometry{margin=0.5cm}
\begin{figure}
    \centering
    \savebox\captionqr{\qrcode[hyperlink,height=1cm]{\repositoryBaseUrl/figures/cc_experimental_results_20250801.zip}}

    \makebox[\textwidth][c]{%
      \includegraphics[angle=90,height=0.9\paperheight]{cc_experimental_results_20250801.png}%
    }

    \captionof{figure}{Horizon optimization strategy results. Experimental validation
    results at pilot plant.
    \hspace{1ex}\usebox\captionqr}
    \labfig{cc:validation:exp-results}
\end{figure}
\restoregeometry
\pagelayout{margin} % Restore margins to class's default