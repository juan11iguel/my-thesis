\setchapterpreamble[u]{\margintoc}
\chapter{Validation} % Optimization strategy experimental validation
\labch{solhycool:validation}

% ====================================
% ====================================
\section{Modelling}

% ================================
\subsection{Wet cooler}

With the aim of comparing two modelling strategies established to predict the
outlet water temperature and the water consumption of WCT (physical
equation-based and ANN-based models), three different experimental campaigns
have been carried out at the pilot plant of combined cooling systems located at
PSA (see a detailed description in Section \ref{sec:description}): the first
one for the calibration of the physical equation-based model, the second one
for tuning the ANN-based model and the last one for the validation and
comparison of the two modelling strategies. These strategies and the
experimental campaigns are exhaustively detailed in Sections
\ref{sec:modelling} and \ref{Sec:Exp}, respectively.  

Fig.~\reffig{cc:validation:modelling:method} schematically shows the procedure followed for the
calibration, tuning and validation of the two modelling strategies established
as well as for the comparison between them. 

\begin{figure}
	\includegraphics[width=\textwidth]{figures/Procedimiento_v2.png}
	\caption{Calibration, tuning, validation and comparison procedure}
	\labfig{cc:validation:modelling:method}
\end{figure}

As previously mentioned, three experimental campaigns have been performed,
shown in  Fig. ~\reffig{cc:validation:modelling:method} as \textit{Exp 1}, \textit{Exp 2}, and
\textit{Exp 3}. \textit{Exp 1} corresponds to the Poppe model calibration
campaign and it was designed for the calibration of the first principles model.
The aims of such campaign was to fit a function (mapping) that relates the air
mass flow rate at the outlet of the tower, $\dot{m}_a$, with the frequency of
the fan, $f_{fan}$, and to calibrate a WCT performance coefficient: the Merkel
number, $\Me$. Exp 2, which corresponds to the ANN tuning experimental
campaign, is a set of data obtained over several years of operation in a wide
range of operating and ambient conditions that has been used for tuning the ANN
model. Finally, in the validation and comparison experimental campaign (Exp 3),
new data, not included in the other two campaigns, has been collected by
applying a design of experiments in order to validate and compare the proposed
modelling strategies.

\begin{enumerate}
	\item FP model
	\item DB from experimental campaign
	\item DB from FP
	\item Comparison
\end{enumerate}
Tabla tocha añadiendo casos (GPR, DB from FP)

On the other hand, the robustness and reliability of the models have been
evaluated by a sensitivity analysis.


% =================================
\subsection{Dry cooler}
\begin{enumerate}
	\item FP model
	\item DB from experimental campaign
	\item DB from FP
	\item Comparison
\end{enumerate}
Tabla tocha añadiendo casos (GPR, DB from FP)

% =================================
\subsection{Complete system}
\begin{enumerate}
	\item solo DB from FP
\end{enumerate}

\section{Control and optimization results}


%================================
\subsection{Regular operation}
\labsec{solhycool:validation:regular-operation}

%================================
\subsection[Planned operation changes]{Planned changes in operation}
\labsec{solhycool:validation:planned-operation}

%================================
\subsection[Unanticipated operation changes]{Unanticipated operational changes}
\labsec{solhycool:validation:unanticipated-operation}