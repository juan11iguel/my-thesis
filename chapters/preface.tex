\chapter*{Preface}
\addcontentsline{toc}{chapter}{Preface}

% Ejemplo sobre el que basarse: Sokolowski, John A., and Catherine M. Banks. Principles of Modeling and Simulation: A Multidisciplinary Approach. John Wiley & Sons, 2011.

The present manuscript is the result of a PhD thesis research work carried out
at the Plataforma Solar de Almería, Spain, under the supervision of Dr.~Lidia
Roca and Dr.~Patricia Palenzuela and is ascribed to the Computer Science
Doctorate Program at the University of Almería. The research was funded by a
scholarship from CIEMAT, a public research organization attached to the Ministry
of Science, Innovation and Universities.

% Participación en proyectos de investigación
The research work was developed within the framework of several national and
international research projects, including the European Union's Horizon 2020
research and innovation programme \textit{SFERA-III -- Solar Facilities for the
European Research Area} (823802) and \textit{Water Mining -- Next generation
water-smart management systems} (869474), as well as the national project
\textit{SOLhycool -- Hybrid cooling solutions for water saving in solar thermal
applications} (PID2021-126452OA-I00).

Different parts of the research work to be presented in the following, were
developed during international stays. In a combined short-stay at the Cyprus
Insitute (Nicosia, Cyprus -- 2023) and attendance to the EDS conference in
Limassol, it was matured and presented the initial \textit{Proposal for a
standard methodology for performance evaluation in multi-effect distillation
processes} under the supervision of Dr.~Marios Georgiou (EEWRC). A year later
(February -- June 2024), the main research stay at the Universidade Federal de
Santa Catarina (UFSC), in Florianópolis (Brazil) took place under the
supervision of Dr.~Julio E. Normey-Rico. There it was completed the
\textit{Hybrid model of a solar desalination system, composed of finite state
machines coupled to data-driven and first-principles models}. Finally, at the
end of the contractual relationship with CIEMAT, in 2025, a one-month stay at
the Technische Universität Chemnitz financed by the Erasmus program served to
advance the work in \textit{Evaluation and comparison of annual simulations of
different cooling alternatives for a case study CSP plant} which would culminate
with its presentation at the SolarPACES conference in September 2025.

This manuscript has been prepared with an intention of making it accessible to a
non-expert audience, however, it is primarily aimed at researchers and
professionals in the fields of renewable energy and water treatment, with the
technical parts of the document delving into thermodynamic, mathematical
modelling and optimization concepts. The content is structured to provide a
comprehensive understanding of the topics discussed, while also being
approachable for those who may not have a deep technical background in these
areas.

\bigskip

The text is divided into four parts where each contains a number of chapters.

\medskip

\textsc{Part One} introduces the context and motivation of the thesis, the
research plan, including the main contributions of this research work, and ends
with an introduction of the main research topics used to develop this research
work. This introductory part is then followed by two parts detailing the main
contributions, where each part is a complete unit: it describes the problem,
presents the proposed solutions and analyses the obtained results. 

\textsc{Part Two} is devoted to the cooling of the power block in \gls{cspLabel}
plants, with a focus on the optimal management of the water resource through the
modelling and optimization of alternative combined cooling systems. 
% The work also includes the experimental validation of the proposed solution in a
% pilot plant.

\textsc{Part Three} centers around thermal desalination processes, particularly
multi-effect distillation. First by analyzing the separation process from a
thermodynamic and qualitative perspective in order to standardize its evaluation
and later in the part, the process is integrated with a variable energy source:
solar thermal energy and its operation is modelled and optimized to manage the
solar resource, maximizing fresh water generation and advancing the state of
research in this area.

\textsc{Part Four} completes the manuscript with a recapitulation of the main
conclusions and findings, as well as a discussion on potential future research
directions in the studied topics and presents the derived scientific contributions.


\chapter*{Summary}
\addcontentsline{toc}{chapter}{Summary}

\glsresetall

This research encompasses two complementary studies on two intrinsically linked
resources: water and energy. The first part focuses on the efficient management
of water resources for power generation, while the second explores the efficient
use of solar energy for clean water production.

\bigskip

\textbf{\textsc{\gls{cspLabel}}} is poised to be a crucial contributor to the
energy
transition away from fossil fuels. The first phase of this transition is well
underway, driven by the massive deployment of low-cost and non-dispatchable
renewable technologies such as wind and solar photovoltaics. However, the second
and more challenging phase, which involves achieving large-scale dispatchable
renewable generation, is still ahead. \gls{cspLabel} stands out as a renewable
and scalable dispatchable technology with the potential to outcompete
combined-cycle and coal-fired power plants. 

One of the key challenges in \gls{cspLabel} systems lies in cooling the power
block, which is typically associated with high water consumption. The first part
of this research is therefore dedicated to the efficient management of water
resources in \gls{cspLabel} plants. An optimal water management strategy is
proposed for \gls{cspLabel} systems that integrate novel combined ---dry and
wet--- cooling configurations. To this end, the annual performance of different
cooling alternatives is evaluated for a commercial 50~MW$_\text{e}$
\gls{cspLabel} plant, Andasol-II, located in southern Spain. Specifically, three
cooling systems, all modelled and validated, are compared: the plant's existing
\gls{wctLabel}, and two variants of a novel \gls{ccLabel} system with dry cooler
capacities of 75~\% and 100~\% of the nominal thermal load of the \gls{wctLabel}
system. 

For each alternative, plant operation is optimized under the same water-scarcity
scenario using a multi-stage optimization framework. This framework minimizes
the cooling cost, encompassing both electricity and water expenses, while
ensuring that the cooling demand is satisfied. The key challenge lies in
effectively managing the limited water resource. Results show that the
integration of the \gls{ccLabel} can reduce specific cooling costs by up to
80~\% and annual water consumption by about 48~\%, with 38~\% savings during the
driest months. These benefits arise primarily from reduced reliance on costly
alternative water sources. The \gls{ccLabel} alternatives also provide more
stable and cost-effective operation throughout the year compared to the
\gls{wctLabel}, which is highly sensitive to water availability. 

\bigskip

\textbf{\textsc{Thermal desalination}}, particularly \gls{medLabel}, can play an
important role in mitigating water scarcity. These systems use thermal and
electrical energy to separate seawater or contaminated feedwater into fresh
water and concentrated brine. Although it may not become the dominant
desalination technology, it is well suited for specific applications. Its
competitiveness can be improved by leveraging opportunities such as brine mining
or industrial wastewater treatment, which enhance both economic feasibility and
environmental benefits.

To expand their applicability, \gls{medLabel} systems must improve in two
directions: (i) by enhancing efficiency through wider operating ranges, or (ii)
by adapting to low-temperature applications in which their heat demands can be
partially or fully met using low-exergy sources such as waste heat or solar
thermal energy. However, the true cost of thermal energy, and consequently the
performance of such systems, is often difficult to quantify, particularly
because there is no unified set of criteria for evaluating their performance.

To address this challenge, this research proposes a standardized methodology for
evaluating the performance of \gls{medLabel} processes, which can also be
extended to other thermal separation technologies. The method covers key aspects
such as instrumentation requirements, process control, and the suitability of
performance metrics. It also includes uncertainty quantification and an
algorithm for automatic steady-state detection. The proposed approach enhances
the reliability and robustness of experimental evaluations under variable
conditions. Experimental results confirm that the methodology is both reliable
and consistent, enabling fair comparisons of \gls{medLabel} systems across
different operating scenarios.

The experimental campaign includes evaluations at high \glspl{tbtLabel}. Results
analyzed using several performance metrics and scale formation risks demonstrate
that the \gls{medLabel} process can operate at high \glspl{tbtLabel} without
significant scaling and can achieve higher concentrations. However, no
substantial improvements in thermal performance or reconcentration capacity are
observed unless specific design modifications are introduced.

\medskip

Finally, a novel operational strategy is proposed to enable the seamless,
autonomous, and optimal integration of a solar-driven \gls{medLabel} system. The
method explicitly determines when to start and stop each subsystem while
considering a two-day prediction horizon. This allows the optimization to
account not only for immediate performance but also for the effect of present
decisions on future production. The approach is based on an experimentally
validated system model that includes the electrical consumption of each
component, combined with the most comprehensive data-driven \gls{medLabel} model
currently available in the literature. The control architecture follows a
hierarchical three-layer structure, in which the upper layer solves a
\gls{minlpLabel} economic optimization problem aimed at maximizing
water production while minimizing operating costs. Results from a week-long
system simulation are compared with two alternative strategies, a baseline
rule-based operation and a fixed-schedule optimized operation. The proposed
method significantly increases system performance by 32~\% relative to the
heuristic baseline and by 21~\% relative to the fixed-schedule strategy. These
gains arise from the proposed method ability to fully exploit the temporal
flexibility provided by the thermal storage while accounting for operational
costs.



\chapter*{Resumen}
\addcontentsline{toc}{chapter}{Resumen}

% \glsresetall

Esta investigación abarca dos estudios complementarios sobre dos recursos
intrínsecamente vinculados: el agua y la energía. La primera parte se centra en
la gestión eficiente de los recursos hídricos para la generación eléctrica,
mientras que la segunda explora el uso eficiente de la energía solar para la
producción de agua limpia.

\bigskip

\textbf{\textsc{La Energía Solar de Concentración}} (\gls{cspLabel}) está
destinada a
desempeñar un papel crucial en la transición energética hacia el abandono de los
combustibles fósiles. La primera fase de esta transición está bien encaminada,
impulsada por el despliegue masivo de tecnologías renovables de bajo coste y no
gestionables, como la eólica y la solar fotovoltaica. Sin embargo, la segunda y
más desafiante fase, que implica lograr una generación renovable gestionable a
gran escala, no se ha alcanzado. La \gls{cspLabel} se posiciona como una
tecnología renovable y escalable con capacidad gestionable y el potencial de
superar a las plantas de ciclo combinado y de carbón. 

Uno de los principales desafíos en los sistemas \gls{cspLabel} radica en la
refrigeración del bloque de potencia, la cual suele asociarse con un alto
consumo de agua. Por ello, la primera parte de esta investigación se dedica a la
gestión eficiente de los recursos hídricos en plantas \gls{cspLabel}. Se propone
una estrategia óptima de gestión del agua que integra configuraciones de
refrigeración combinada novedosas--- integrando sistemas húmedos y secos. Con
este fin, se evalúa el rendimiento anual de diferentes alternativas de
refrigeración para una planta comercial de 50~MW$_\text{e}$ \gls{cspLabel},
Andasol-II, ubicada en el sur de España. En concreto, se comparan tres sistemas
de refrigeración, todos ellos modelados y validados: el sistema húmedo
\gls{wctLabel} existente en la planta, y dos variantes de un novedoso sistema de
refrigeración combinada (\gls{ccLabel}) con capacidades de enfriador seco del
75~\% y del 100~\% de la carga térmica nominal del sistema \gls{wctLabel}. 

Para cada alternativa, la operación de la planta se optimiza bajo el mismo
escenario de escasez de agua utilizando un método de optimización multi-etapa.
Este método busca minimizar el coste de refrigeración, que incluye gastos
de electricidad y agua, garantizando al mismo tiempo que se satisfaga la demanda
de refrigeración dadas unas condiciones de entorno. El desafío clave radica en
gestionar eficazmente el recurso hídrico limitado. Los resultados muestran que
la integración del \gls{ccLabel} puede reducir los costes específicos de
refrigeración hasta en un 80~\% y el consumo anual de agua en alrededor del
48~\%, con un ahorro del 38~\% durante los meses más secos. Estos beneficios
surgen principalmente de la menor dependencia de fuentes de agua alternativas y
su alto coste asociado. Las alternativas \gls{ccLabel} también ofrecen una
operación más estable y rentable a lo largo del año en comparación con la
\gls{wctLabel}, que es altamente sensible a la disponibilidad de agua. 

\bigskip

\textbf{\textsc{La desalación térmica}}, en particular la Destilación Multiefecto
(\gls{medLabel}), puede desempeñar un papel importante en la mitigación de la
escasez de agua. Estos sistemas utilizan energía térmica y eléctrica para
separar agua de mar o agua contaminada en agua dulce y salmuera concentrada.
Aunque no se posicione como la tecnología de desalación dominante, es muy
adecuada para otras aplicaciones. Su competitividad puede mejorarse aprovechando
oportunidades como la minería de salmuera o el tratamiento de aguas residuales
industriales, que mejoran tanto la viabilidad económica como los beneficios
ambientales.

Para ampliar su aplicabilidad, los sistemas \gls{medLabel} deben mejorar en dos
direcciones: (i) aumentando su eficiencia mediante rangos de operación más
amplios, o (ii) adaptándose a aplicaciones de baja temperatura en las que sus
demandas térmicas puedan satisfacerse parcial o totalmente mediante fuentes de
baja exergía, como el calor residual o la energía solar térmica. Sin embargo, el
coste real de la energía térmica y, en consecuencia, el rendimiento de estos
sistemas, suele ser difícil de cuantificar, en particular porque no existe un
conjunto unificado de criterios para evaluar su desempeño.


Para abordar este desafío, esta investigación propone una metodología
estandarizada para evaluar el rendimiento de los procesos \gls{medLabel}, la
cual también puede extenderse a otras tecnologías de separación térmica. El
método abarca aspectos clave como los requisitos de instrumentación, el control
del proceso y la idoneidad de las métricas de rendimiento. También incluye la
cuantificación de incertidumbre y un algoritmo para la detección automática del
estado estacionario. El enfoque propuesto mejora la fiabilidad y la robustez de
la evaluación de la operación bajo condiciones variables. Los resultados
experimentales, obtenidos en una planta piloto real, confirman que la
metodología es confiable y consistente, lo que permite comparaciones justas
entre sistemas \gls{medLabel} bajo diferentes escenarios operativos. 

La campaña experimental incluye evaluaciones a altas temperaturas de salmuera en
el primer efecto (\gls{tbtLabel}). Los resultados,
analizados mediante diversas métricas de rendimiento y riesgos de formación de
incrustaciones, demuestran que el proceso \gls{medLabel} puede operar a altas
\glspl{tbtLabel} sin incrustaciones significativas en las superficies de
intercambio y puede alcanzar mayores concentraciones. Sin embargo, no se
observan mejoras sustanciales en el rendimiento térmico o la capacidad de
reconcentración, a menos que se introduzcan modificaciones específicas en el
diseño. 

\medskip

Finalmente, se propone una estrategia de oepración novedosa que permite la
integración autónoma y óptima de un sistema \gls{medLabel} alimentado con
energía solar. El método determina explícitamente cuándo iniciar y detener cada
subsistema, considerando un horizonte de predicción de dos días. Esto permite
que la optimización tenga en cuenta no solo el impacto inmediato de las
decisiones presentes, sino también el efecto sobre la producción futura. El
enfoque se basa en un modelo del sistema validado experimentalmente que incluye
el consumo eléctrico de cada componente, combinado con el modelo \gls{medLabel}
basado en datos más completo disponible actualmente en la literatura. La
arquitectura de control sigue una estructura jerárquica de tres niveles, en la
que la capa superior, la capa de planificación de la operación, resuelve un
problema de optimización económica de programación no lineal mixto
(\gls{minlpLabel} por sus siglas en inglés) destinado a maximizar la producción
de agua mientras se minimizan los costes de operación. Los resultados de una
simulación del sistema durante una semana se comparan con dos estrategias
alternativas: una operación simple basada en reglas y una operación optimizada
con un plan de operación fijo. El método propuesto aumenta significativamente el
rendimiento del sistema en un 32~\% en relación con la alternativa heurística y
en un 21~\% comparado con la estrategia de plan de operación fijo. Estas
ganancias se deben a la capacidad del método propuesto para aprovechar al máximo
la flexibilidad temporal que ofrece el almacenamiento térmico, al tiempo que se
tienen en cuenta los costes de operación.

