\chapter*{Acknowledgements}
\addcontentsline{toc}{chapter}{Acknowledgements}

Test test test
% To my family, if you did  not ask about how was I doing with the thesis, every
% single time you saw me, I might have forgotten I had to actually finish it.

\begin{flushright}
	\textit{Federico Marotta}
\end{flushright}


\chapter*{Summary}
\addcontentsline{toc}{chapter}{Summary} % Add the preface to the table of contents as a chapter

I am of the opinion that every \LaTeX\xspace geek, at least once during 
his life, feels the need to create his or her own class: this is what 
happened to me and here is the result, which, however, should be seen as 
a work still in progress. Actually, this class is not completely 
original, but it is a blend of all the best ideas that I have found in a 
number of guides, tutorials, blogs and tex.stackexchange.com posts. In 
particular, the main ideas come from two sources:

\begin{itemize}
	\item \href{https://3d.bk.tudelft.nl/ken/en/}{Ken Arroyo Ohori}'s 
	\href{https://3d.bk.tudelft.nl/ken/en/nl/ken/en/2016/04/17/a-1.5-column-layout-in-latex.html}{Doctoral 
	Thesis}, which served, with the author's permission, as a backbone 
	for the implementation of this class;
	\item The 
		\href{https://github.com/Tufte-LaTeX/tufte-latex}{Tufte-Latex 
			Class}, which was a model for the style.
\end{itemize}

The first chapter of this book is introductory and covers the most
essential features of the class. Next, there is a bunch of chapters 
devoted to all the commands and environments that you may use in writing 
a book; in particular, it will be explained how to add notes, figures 
and tables, and references. The second part deals with the page layout 
and design, as well as additional features like coloured boxes and 
theorem environments.

I started writing this class as an experiment, and as such it should be 
regarded. Since it has always been intended for my personal use, it may
not be perfect but I find it quite satisfactory for the use I want to 
make of it. I share this work in the hope that someone might find here 
the inspiration for writing his or her own class.

\chapter*{Resumen}
\addcontentsline{toc}{chapter}{Resumen}



\chapter*{How to read this thesis}
\addcontentsline{toc}{chapter}{How to read this thesis}


% I believe that a document of this size will be more often viewed on computer
% than on paper. This how I personally view most papers I read anyway.
% As such, I believe that the document should be adapted for such a medium, unlike
% most \LaTeX{} documents.
% While trying to make my own kind of document, I stumbled upon the great
% \href{https://github.com/fmarotta/kaobook/}{kaobook} class which corresponds
% almost exactly to what I was looking for: citations, notes, reminders can be put
% in a large margin.

% Citations like~\sidecite{KOMAScriptDoc} will go in the
% margin, as well as in the \refbib.
% The margin also contain side notes\sidenote{
%   No more going back and forth between the end and top of the page.
% }, replacing the usual footnotes.

% I will also use margin notes that are not anchored in the main document but are
% usually relevant to the paragraph or figure on their left.
% \marginnote[-0.4cm]{
%   Margin notes are not placed entirely automatically. I tend to adjust the
%   height so it is at the right place.
% }

% \begin{theorem}
%   Theorems and definitions will be in these yellow boxes.
% \end{theorem}

% On the other hand, I will sometimes refer to concepts for the first time without
% taking the time to introduce them because they are secondary to the main point.
% In some cases those will be accompanied by a short definition on the side, in
% a yellowish box.

\chapter*{About the author}
\addcontentsline{toc}{chapter}{About the author}

% Un payaso\\ \flushright{-- Lidia Roca, probablemente}