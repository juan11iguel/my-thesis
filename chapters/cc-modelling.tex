\setchapterpreamble[u]{\margintoc}
\chapter{Modelling of a combined cooling system}
\labch{cc:modelling}

\tldrbox{ This chapter describes the steady-state modelling of the different
    components of a combined cooling system, mainly a \gls{wctLabel} and a
    \gls{dcLabel}. Different alternatives are presented: from physical models to
    data-driven approaches, including the generation of surrogate data-driven
    models trained using synthetic data from a physical model. Models are also
    developed for the other components of the system and finally it is shown how
    they are integrated into a complete system model. The complete system model
    interface is defined at \refmod{cc}. \\[1ex]

    \begin{minipage}{\linewidth}
        \centering
        \includegraphics[width=\linewidth]{figures/cc-modelling-complete-model-diagram-with-bg.png}\\
        \small \textit{System block diagram including all relevant variables.}
    \end{minipage}    
}

\section*{Introduction}

In order to study the potential advantages of making use of a combined cooling
system, it is first necessary to develop the modelling of its components. Since
the objective is performance prediction, this chapter focuses on the steady
state modelling of the combined cooler main components, \ie the \gls{wctLabel}
and the \gls{dcLabel}. More specifically, the aim is to develop two modelling
strategies: that based on physical equations and that based on black box
data-driven models. 

First, in \refsec{cc:modelling:wct} a physical
model for the \gls{wctLabel} is described followed by a methodology to generate
a synthetic dataset for surrogate data-driven models. In
\refsec{cc:modelling:dc} the same structure is followed for the
\gls{dcLabel}-type \gls{acheLabel}. Other first-principles model components are
presented in \refsec{cc:modeling:other-components}: the surface condenser,
mixers, and the electrical consumption. Finally, all components are integrated
in \refsec{cc:modelling:complete-model} to form the complete model of the
combined cooling system. Each component and the complete model sections is
completed with a model interface block. 

%===================================
%===================================
\section{Wet cooler}
\labsec{cc:modelling:wct}

The static models presented in this section have been developed to predict two
main outputs, the water temperature at the outlet of the \gls{wctLabel},
$T_{wct,out}$, and the water consumed due to evaporation losses,
$C_{w,wct}$. The inputs variables required are: the cooling water flow
rate ($q_{wct}$)\sidenote{$\dot{m}_{wct}$ in terms of mass flow rate}, the water
temperature at the inlet of the \gls{wctLabel} ($T_{wct,in}$), the ambient
temperature ($T_{amb}$), the ambient relative humidity ($HR$) and the cooler fan
speed ($\omega_{wct}$)\sidenote{In the case of the physical model, internally
the fan speed is converted into the air mass flow rate ($\dot{m}_{wct,air}$)
with an empirical correlation.}

%====================================
\subsection{Physical model} 
\labsec{cc:modelling:wct:physical}

The Poppe theory is used for the thermal performance evaluation of the
\gls{wctLabel}. This approach is preferred for applications in which an accurate
determination of the outlet air state is required. As the water consumption is
of paramount importance in \gls{cspLabel} plants, it is suitable for integration
into analysis/optimization frameworks for \gls{cspLabel} plant cooling
strategies.

In the case of cooling towers, the Merkel number is a dimensionless parameter
widely recognized as a performance key indicator. It is commonly employed in
experimental characterization, numerical simulations, and optimization studies.
This parameter can be evaluated through several theoretical approaches,
including the original Merkel
formulation~\sidecite{merkel_verdunstungskuhlung_1925}, the
effectiveness-\gls{ntuLabel} method~\sidecite{jaber_design_1989}, and the more
comprehensive Poppe model~\sidecite{poppe_berechnung_1991}.

Several investigations have compared the previously mentioned theories for
thermal performance evaluation of wet cooling towers. Some examples include the
works of~\sidecite{kloppers_critical_2005,navarro_critical_2022}. These
investigations generally conclude that the Poppe method offers a more accurate
representation of the physics of the problem, as it enables the prediction of
moist air properties and the quantification of evaporative water losses.
Consequently, the Poppe method is recommended for applications in which an
accurate determination of the outlet air state is required.

According to the Poppe theory~\cite{poppe_berechnung_1991}, the governing
equations for heat and mass transfer in the transfer region of the wet cooling
tower assuming a one dimensional problem. This is illustrated in the control volume
shown in \reffig{cc:modelling:poppe-transfer-area} where the red and
green dashed lines indicate the fill and air-side control volumes, respectively.

\vspace{1.5cm} 
\begin{figure}[htbp] 
    \begin{overpic}[width=.7\textwidth]{figures/poppe_control_volume.pdf} 
        \put(-3.3,26){$dz$}
        \put(13,65){$\dot{m}_{wct}+d\dot{m}_{wct}$} \put(14,57){$h_w+dh_w$}
        \put(15,-7){$\dot{m}_{wct}$, $h_w$}
        \put(60.5,65){$\dot{m}_{wct,air}\left(1+\omega+d\omega\right)$} \put(65,57){$h+dh$}
        \put(55,-7){$\dot{m}_{wct,air}\left(1+\omega\right)$, $ h $}
        \put(44,30){$d\dot{m}_{wct}=h_D\left(\omega_{s,w}-\omega\right)dA$}
        \put(44,22){$h_C\left(T_{w}-T\right)dA$}        
    \end{overpic} 
    \vspace{1cm} 
    \caption{Control volume in the exchange area of a wet cooling tower arrangement. Source:~\cite{serrano_wet_2024}.}
    \labfig{cc:modelling:poppe-transfer-area}
\end{figure}

The major following equations for the heat and mass transfer obtained
are~\cite{kloppers_critical_2005,navarro_critical_2022}: 

{\small
\begin{align}
& \frac{d\omega}{dT_w} = \frac{c_{p_w}\,\frac{\dot{m}_{wct}}{\dot{m}_{wct,air}}\,
(\omega_{s_w} - \omega)}
{(h_{s_w}-h)+(\lew-1)\big[(h_{s_w}-h)-(\omega_{s_w} - \omega)h_v\big]
-(\omega_{s_w} - \omega) h_w}, 
\labeq{cc:poppe_omega}\\[6pt]
& \frac{dh}{dT_w} = c_{p_w}\frac{\dot{m}_{wct}}{\dot{m}_{wct,air}}\left[1+
\frac{(\omega_{s_w} - \omega)c_{p_w}T_w}
{(h_{s_w}-h)+(\lew-1)\big[(h_{s_w}-h)-(\omega_{s_w} - \omega)h_v\big]
-(\omega_{s_w} - \omega) h_w}\right], 
\labeq{cc:poppe_h}\\[6pt]
& \frac{d\Me}{dT_w} = \frac{c_{p_w}}
{(h_{s_w}-h)+(\lew-1)\big[(h_{s_w}-h)-(\omega_{s_w} - \omega)h_v\big]
-(\omega_{s_w} - \omega) h_w},
\labeq{cc:poppe_Me}
\end{align}
}

where the evolution of key variables such as the air enthalpy
(\refeq{cc:poppe_h}), humidity ratio (\refeq{cc:poppe_omega}), and the Merkel
number (\refeq{cc:poppe_Me}) along the fill height is obtained by numerically
solving the set of differential equations using a fourth-order Runge-Kutta
algorithm\sidenote{$Le$ is the Lewis number, $\omega$ is the humidity ratio
(kg/kg), $h$ is the enthalpy and $c_p$ the specific heat.}. Secondly, the Merkel
number of a wet cooling tower is not a constant value but varies with the
operating conditions. When plotted against the water-to-air mass flow ratio
(defined as the ratio between the water and air mass flow rates within the
tower) it typically follows a straight, decreasing trend in logarithmic
coordinates. As suggested by ASHRAE~\cite{ashrae_hvac_2004a}, this behavior is
often expressed through a single correlation that depends on the water-to-air
mass flow ratio:

\begin{equation}
    \Me = c\cdot \left(\frac{\dot{m}_{wct}}{\dot{m}_{wct,air}}\right)^{-n}
\labeq{cc:me-correlation}
\end{equation}

Here, \(c\) and \(n\) are empirical coefficients that depend on the specific
design of the cooling tower. These coefficients can be determined through
experimental data fitting.


%=====================================
\subsection{Surrogate model synthetic dataset generation}[Surrogate model]

The first pair of input variables for the \gls{wctLabel} synthetic dataset generation are
the wet bulb temperature ($T_{wb}$) and the difference between this temperature
and the system inlet temperature ($\Delta T_{wb-in}$). The wet bulb temperature
is used instead of the ambient temperature or the relative humidity, because as
it can be derived from the physical model, it is the most relevant
thermodynamic variable for the wet cooling tower performance. Using both the
ambient temperature and the relative humidity would lead to a larger than
necessary input space with many duplicate samples, as the wet bulb
temperature is a function of both variables. The second pair of input
variables are the cooling water flow rate ($q_{wct}$) and, following the
reasoning from the physical model, the air to water mass flow ratio
($\dot{m}_{wct,air}/\dot{m}_{wct}$), since it is a key parameter in defining the
operating conditions of the tower. From the resulting 2D grid, valid combinations
are obtained by calculating the air mass flow rate and finding if a valid fan
speed can be obtained using an air mass flow rate to fan speed empirical
correlation.

Finally, all valid thermodynamic and operational combinations are merged into a
comprehensive sample set, enabling detailed system evaluations across a
realistic and constrained input space\sidenote{See
\refsec{intro:modelling:sample-generation}}.

\subsection{Model interface}

\begin{modelcounter}{Wet cooling tower}
    \begin{align*}
        T_{wct,out},\,&C_{e,wct},\,C_{w,wct} = \text{wct\:model}(q_{wct},\, \omega_{wct},\, T_{amb},\, HR,\, T_{wct,in};\,\theta_{wct}) \\
        & C_{e,wct} = \text{electrical\:consumption}(\omega_{wct}) \\
    \end{align*}
    \labmod{wct}
\end{modelcounter}

% \begin{modelcounter}{Wet cooling system model}
%     \begin{align*}
%         T_{wct,out},&\,C_{e},C_{w},\,T_{c,in},\,T_{c,out} = \text{wcs\:model}(q_{wct},\, \omega_{wct},\, T_{amb},\, HR,\, T_{wct,in}) \\
%         % condenser model
%         & T_{c,in},\,T_{c,out} = \text{condenser\:model}(q_c,\, \dot{m}_v,\, T_v) \\
%         % wet cooler model
%         & T_{wct,out},\,C_{w,wct} = \text{wct\:model}(q_{wct},\, \omega_{wct},\, T_{amb},\, HR,\, T_{c,out}) \\
%         % electrical consumptions
%         & C_{e,c} = \text{electrical\:consumption}(q_c) \\
%         & C_{e,wct} = \text{electrical\:consumption}(\omega_{wct}) \\
%         % totals
%         & C_{e} = C_{e,wct} + C_{e,c} \\
%         & C_{w} = C_{w,wct}
%     \end{align*}
%     \labmod{wet-system}
% \end{modelcounter}

%=====================================
%=====================================
\section{Dry cooler}
\labsec{cc:modelling:dc}

Similar to the \gls{wctLabel} model, static models are used to predict, this
time, one output: the water temperature at the outlet of the \gls{dcLabel},
$T_{dc,out}$. The inputs variables required are similar to the \gls{wctLabel}:
the cooling water flow rate ($q_{dc}$)\sidenote{$\dot{m}_{dc}$ in terms of mass
flow rate}, the water temperature at the inlet of the \gls{dcLabel}
($T_{dc,in}$), the ambient temperature ($T_{amb}$) and the cooler fan speed
($\omega_{dc}$)\sidenote{In the case of the physical model, internally the fan
speed is converted into the air mass flow rate ($\dot{m}_{dc,air}$) with an
empirical correlation.}.

%================================
\subsection{Physical model}
\labsec{cc:modelling:dc:physical}

The modelling of air-cooled heat exchangers can involve different levels of
complexity. A common simplification is to assume a constant effectiveness or
overall heat transfer coefficient. However, such models cannot capture the
influence of relevant operating conditions, such as water or air flow rates,
which significantly affect both the transferred heat and the power consumption.
For preliminary designs, it is common to rely on empirical models based on
experimental correlations available in the open
literature~\sidecite{zhang_preliminary_2024}. While this approach accounts for
operating conditions, it must be applied with caution when used outside the
range of the original correlations. Furthermore, it can be challenging to find
correlations for relatively complex geometries. To partially address these
limitations, other models adjust certain parameters through component tests
while still preserving the main physics of the problem, primarily by obtaining a
Nusselt number ($\Nus$) correlation for a specific configuration as a function
of the Reynolds ($\Rey$) and Prandtl ($\Pra$) numbers. 

The \gls{dcLabel} (\gls{acheLabel}-type) in the pilot-plant is
modelled using the standard heat exchanger equations~\sidecite{yunusa.cengel_heat_2014}:
heat transferred from the hot fluid (water), heat transferred to the cold fluid
(air), and the heat transfer expressed through the overall heat transfer
coefficient and the logarithmic mean temperature difference.

 \begin{align}  
    & \dot{Q}_{dc, released}= \dot{m}_{dc} \cdot c_{p,c} \cdot(T_{dc,in}-T_{dc,out}), \label{eq:cc:dc_released} \\ 
    & \dot{Q}_{dc, absorbed} = \dot{m}_{dc,air} \cdot c_{p,a}\cdot(T_{dc,air,out}-T_{amb}), \label{eq:cc:dc_absorbed}\\ 
    & \dot{Q}_{dc, transferred} = U_{dc}\cdot A_{dc} \cdot F \cdot LMTD, \label{eq:cc:dc_transferred} \\
    & \dot{Q}_{dc, released} = \dot{Q}_{dc, absorbed} = \dot{Q}_{dc, transferred}
\end{align} 

where \(\dot{Q}\) represents the heat transfer rate \ie, the thermal
power, \textit{F} is the correction factor for the \gls{lmtdLabel}, which is
calculated for a cross flow heat exchanger with three tube rows and three tube
passes~\sidecite{kroger2004}, and the product of the overall heat transfer
coefficient and the heat exchange area is calculated from:

\begin{equation}
    U_{dc}\cdot A_{dc} = \left( \frac{1}{A_{dc,i}\cdot h_i}+ \frac{\ln(D_{dc,tb,o}/D_{dc,tb,i})}{2 \cdot \pi \cdot k_{dc,tb} \cdot L_{dc,tb} \cdot n_{dc,tb}}+\frac{1}{A_{dc,o}\cdot h_o} \right)^{-1}, \\
    \labeq{cc:dc_UA}
\end{equation}

being $h_{o}$ the convective heat transfer coefficient for the external surface
of a single horizontal tube, $D_{tb,o}$ the outer tube diameter, $D_{tb,i}$ the
inner tube diameter, $k_{dc,tb}$ the thermal conductivity of the tube wall
material, and $h_{i}$  the convective heat transfer coefficient for the water
inside the tube\sidenote{See their values in \reftab{cc:facility:dc}}.

The convective heat transfer coefficient on the internal side (cooling water),
$h_i$, is calculated using Gnielinski's
correlation~\sidecite{volkergnielinski_new_1976}, assuming uniform flow
distribution in the tubes and negligible roughness. The convective heat transfer
coefficient on the external surface (air), $h_o$, cannot be easily calculated
from correlations in the open literature due to the geometry involved (tube
arrays with transversal plate fins). It can be determined experimentally with a
campaign covering a wide range of operating conditions to then fit the
experimental data to an equation that relates the air side Nusselt number and
the air side Reynolds number, as follows:

\begin{equation}\labeq{cc:Nu_corr}
    \Nus_a = G \cdot \Rey_a^{m} \cdot \Pra_a^{0.36}, \\
\end{equation}

where $\Nus_a=h_o\cdot L_{dc,tb}/k_{dc,tb}$, and \textit{G} and \textit{m} are
parameters to be determined by fitting to experimental data, and the exponent
for the Prandtl number, $\Pra$, is assumed to be the same value as that proposed
by Zukauskas for staggered tube banks~\sidecite{kakac}\sidenote{
The system design characteristics needed for applying this methodology for the
pilot plant are listed in \reftab{cc:facility:dc}.
}.

%=====================================
\subsection{Surrogate model synthetic dataset generation}[Surrogate model]
\labsec{cc:modelling:dc:samples}
% de muestreos seguida y mostrar la gráfica de resultados

Similar to the wet cooling tower case, setting absolute values for both the
inlet temperature and the environment temperature will lead to many unfeasible
combinations ($T_{dc,in} \le T_{db}$). So instead, values are generated for the
temperature difference, therefore, a 2D grid is constructed using combinations
of ambient/dry-bulb temperature ($T_{amb}$) and the difference between inlet
and ambient temperature ($\Delta T_{amb-in}$). For each valid temperature
pair ($T_{amb}$, $T_{dc,in}$), additional independent variables ($q_{dc}$,
$\omega_{dc}$) are combined via a cartesian product, resulting in a full
multidimensional grid of plausible operating points. This systematic procedure
ensures a dense and uniform sampling across all relevant input dimensions.
Finally, infeasible combinations are filtered based on physical constraints.


%incluir figura con distribución de muestras generadas

%================================
\subsection{Model interface}

\begin{modelcounter}{Dry cooler}
    \begin{align*}
        T_{dc,out} &= \text{dc\:model}(q_{dc},\, \omega_{\text{dc}},\, T_{\text{amb}},\, T_{dc,in};\,\theta_{dc}) \\
        & C_{e,dc} = \text{electrical\:consumption}(\omega_{dc}) \\
    \end{align*}
    \labmod{dc}
\end{modelcounter}
% \marginnote[*-2]{The condenser area ($A$) is a constant parameter}

% \begin{modelcounter}{Dry cooling system model}
%     \begin{align*}
%         T_{dc,out},&\,C_{e},\,T_{c,in},\,T_{c,out} = \text{dcs\:model}(q_{dc},\, \omega_{dc},\, T_{amb},\, T_{dc,in}) \\
%         % condenser model
%         & T_{c,in},\,T_{c,out} = \text{condenser\:model}(q_c,\, \dot{m}_v,\, T_v) \\
%         % dc model
%         & T_{dc,out} = \text{dc\:model}(q_{dc},\, \omega_{\text{dc}},\, T_{\text{amb}},\, T_{c,out}) \\
%         % electrical consumptions
%         & C_{e,c} = \text{electrical\:consumption}(q_c) \\
%         & C_{e,dc} = \text{electrical\:consumption}(\omega_{dc}) \\
%         % totals
%         & C_{e} = C_{e,dc} + C_{e,c} \\
%     \end{align*}
%     \labmod{dc-system}
% \end{modelcounter}


%=====================================
%=====================================
\section{Other components and outputs}
\labsec{cc:modelling:other-components}

%================================
\subsection{Surface condenser}
\labsec{cc:modelling:condenser}

The surface condenser is a heat exchanger that condenses steam into water,
assuming that all the vapor that enters the condenser (at saturated conditions),
leaves it as saturated liquid, it can be modelled similar to the dry cooler by
applying the first law of thermodynamics, which states that the heat lost by the
steam (\textit{released}) is equal to the heat gained by the cooling water
(\textit{absorbed}), and equal to the heat transferred by the condenser heat
transfer surfaces (\textit{transferred}).

\begin{modelcounter}{Surface condenser}
    \begin{align*}
        T_{c,in},\,T_{c,out}& = \text{condenser model}(\dot{m}_{c}, T_{v},\dot{m}_{v};\,\theta_c) \\
        & \dot{Q_{c}} = \dot{m}_v \cdot (h_{sat.vap}-h_{sat.liq}) \\ 
        & \dot{Q}_{c} = \dot{m}_c \cdot c_p  \cdot(T_{c,out}-T_{c,in}) \\ 
        & \dot{Q}_{c} = U_c\cdot A_c \cdot LMTD \\
        & LMTD = \frac{T_{c,out}-T_{c,in}}{\ln\left(\frac{T_{v}-T_{c,in}}{T_{v}-T_{c,out}}\right)} \\
    \end{align*}
    \labmod{sc}
\end{modelcounter}

being $U_c$ the overall heat transfer coefficient. Which can be determined in a
variety of ways. One of them is given by:
\begin{equation}
    U_{c}\cdot A_c = \left( \frac{1}{A_{c,i}\cdot h_i}+ \frac{R_{c,if}}{A_{c,i}} + \frac{\ln(D_{c,tb,o}/D_{c,tb,i})}{2 \cdot \pi \cdot k_{c,tb} \cdot L_{c,tb} \cdot n_{c,tb}}+\frac{1}{A_{c,o}\cdot h_o} + \frac{R_{c,of}}{A_{c,o}} \right)^{-1}, \\
    \labeq{cc:sc_UA}
\end{equation}

where $R_{c,if}$ and $R_{c,of}$ are the fouling resistances (inside and outside,
respectively). The shell-side heat transfer coefficient $h_{o}$ is estimated
once again using the Nusselt method~\cite{Serth2007} for laminar-flow condensation over a
horizontal tube bundle ($0 < \Rey < 30 \times 10^{6}$) including the Kern
correction for condensate inundation, while the tube-side convective heat
transfer coefficient for the water flow inside the tube bundle, $h_{i}$, is
approximated using the Petukhov-Kirillov-Popov correlation for fully-developed
turbulent flow through smooth circular tubes~\sidecite{rohsenow_handbook_1998}, valid within
$0.5 < \Pra < 10^{6}$ and $4000 < \Rey < 5 \times 10^{6}$.

%================================
\subsection{Electrical consumption}

Electrical consumption is modelled with polynomial regressions of order 3 from
experimental data:

\begin{modelcounter}{Electrical consumption}
    \begin{align*}
        C_{e} &= \text{electrical consumption model}(x;\, \theta) \\
        & C_{e} = p_1 \cdot x^3 + p_2 \cdot x^2 + p_3 \cdot x + p_4
    \end{align*}
    \labmod{electrical-consumption}
\end{modelcounter}

where \(C_e\) represents the electrical consumption, and \(x\) is the input
variable (e.g., the recirculated cooling water flow rate, particular cooler fan
speed, etc.). The coefficients \(p_i\) correspond to a polynomial regression
and must be calibrated individually for each component.

\subsection{Mixers}

The mixers outlet flow ($q_{mix,out,i}$) and temperature ($T_{mix,out,i}$) can
be determined with a simple mass and energy balances from its inlets streams
($q_{mix,in},\,T_{mix,in}$):


% \modeldefinitionbox{Mixer model}{
\begin{modelcounter}{Mixer model}
    \begin{align*}
        q_{mix,out},\,&T_{mix,out} = \text{mixer model}(q_{mix,in,1},\,T_{mix,in,1},\,q_{mix,in,2},\,T_{mix,in,2}) \\
        & q_{mix,out} = q_{mix,in,1} + q_{mix,in,2} \\
        & T_{mix,out} = T_{mix,in,1} \cdot \frac{c_p(T_{mix,in,1})}{c_p(T_{out,i})}\frac{q_{mix,in,1}}{q_{mix,out,i}} + \nonumber \\ &\qquad\qquad T_{mix,in,2} \cdot \frac{c_p(T_{mix,in,2})}{c_p(T_{out,i})}\frac{q_{mix,in,2}}{q_{mix,out,i}}
    \end{align*}
    \labmod{mixer}
\end{modelcounter}
% }

where $c_p(\cdot)$ is the specific heat, which can be assumed to be the same for the
mixing temperature differences of this type of system.

\subsection{Valves}

The two divergent three-way valves are modeled using the distribution ratio $R$
as inputs, so the mass flow rate at the outlets of each valve can be easily
estimated as $q_{out,1}=R\cdot q_{in}$ and conversely $q_{out,2}=(1-R)\cdot
q_{in}$. Where $R$ is a value in the range [0,1], which implicitly includes the
non-linearities of the valve.

%=====================================
%=====================================
\section{Complete system}
\labsec{cc:modelling:complete-model}

\begin{figure}
    \includegraphics[width=\textwidth]{figures/cc-modelling-complete-model-diagram.png}
    \caption{Complete model diagram of the combined cooling system}
    \labfig{cc:modelling:complete-model-diagram}
\end{figure}

The complete model of the combined cooling system integrates the models of the
\gls{wctLabel} and \gls{dcLabel}, along with the surface condenser and the
mixers, as defined in \nrefmod{cc}, where $\theta$ represents the parameters of
the different components (condenser area, heat transfer coefficients, number of
tubes in the \gls{acheLabel}, etc.)\sidenote{The electrical consumption for the
cooling water recirculation depends on the hydraulic circuit configuration,
since the dry cooler and the wet cooler offer different circulation
resistances}. The full diagram, including all variables, is shown in
\reffig{cc:modelling:complete-model-diagram}.

The system can be solved with the provided inputs ($q_{c}$, $R_{p}$, $R_{s}$,
$\omega_{dc}$, $\omega_{wct}$, $T_{amb}$, $HR$, $T_{v}$, $\dot{m}_{v}$) in the
following way: First, flows across the different paths can be determined
provided the recirculation flow rate ($q_{c}$) and the valve positions ($R_{p},
R_{s}$). It follows the evaluation of the condenser model, providing the only
remaining input to fully determine the dry cooler, its inlet temperature
($T_{dc,in}==T_{c,out}$). Then, the temperature inlet and outlet conditions for
the wet cooler can be obtained by evaluating the mixers with the now known dry
cooler outlet conditions ($T_{dc,out}$), together with the combined cooler
outlet temperature, which equals the condenser inlet temperature
($T_{cc,out}=T_{c,in}$). It only remains to evaluate the wet cooling tower,
where given its inlet temperature ($T_{wct,in}$, obtained from the first mixer),
and its fan speed ($\omega_{wct}$), the outlet temperature ($T_{wct,out}$) and
the water consumption ($C_{w,wct}$) can be determined. With the outlet
temperature requires matching the obtained value from the second mixer.

If no convergence is achieved, the process can either be repeated iteratively
changing the condenser vapor temperature ($T_v$) until the system converges to a
valid equilibrium steady-state. Or alternatively, a flag can be set to indicate
that the current input combination is not feasible.

\begin{modelcounter}{Combined cooling system}
    \begin{align*}
        T_{cc,out},\,C_{e}&,\,C_{w},\,T_{c,in},\,T_{c,out} = \text{ccs\:\:model}(q_{c}, R_{p}, R_{s}, \omega_{dc}, \omega_{wct},T_{amb},HR,T_{v},\dot{m}_{v};\,\theta) \\
        & T_{cc,in}=T_{c,out} \\
        & T_{dc,in}=T_{cc,in} \\
        % three-way valves
        & q_{dc} = q_{c} \cdot (1-R_{p}) \\
        & q_{wct,p} = q_{c} \cdot R_{p} \\
        & q_{wct,s} = q_{dc} \cdot R_{s} \\
        % condenser model
        & T_{c,in},\,T_{c,out} = \text{condenser\:model}(q_c,\, \dot{m}_v,\, T_v) \\
        % dc model
        & T_{dc,out},\,C_{e,dc} = \text{dc\:model}(q_{dc},\, \omega_{\text{dc}},\, T_{\text{amb}},\, T_{dc,in};\,\theta_{dc}) \\
        % first mixer
        & q_{wct},\,T_{wct,in} = \text{mixer\:model}(q_{wct,p},\,T_{T{cc,in}},\, q_{wct,s},\, T_{dc,out}) \\
        % final mixer
        & q_{cc},\,T_{cc,out} = \text{mixer\:model}(q_{wct},\, T_{wct,out},\, q_{dc},\, T_{dc,out}) \\
        % wct model
        & T_{wct,out},\,C_{e,wct},\,C_{w,wct} = \text{wct\:model}(q_{wct},\, \omega_{wct},\, T_{amb},\, HR,\, T_{wct,in};\,\theta_{wct}) \\
        % electrical consumptions
        & C_{e,c} = \text{electrical\:consumption}(q_c) \\
        & C_{e,dc} = \text{electrical\:consumption}(\omega_{dc}) \\
        & C_{e,wct} = \text{electrical\:consumption}(\omega_{wct}) \\
        % totals
        & C_{e} = C_{e,dc} + C_{e,wct} + C_{e,c} \\
        & C_{w} = C_{w,wct}
    \end{align*}
    \labmod{cc}
\end{modelcounter}

\begin{marginfigure}[*-20]
    \centering
    \includegraphics[]{figures/cc-modelling-wct-io-diagram.png}
    % {\footnotesize \textbf{(a)} \gls{mimoLabel} configuration\\}
    
    \vspace{1ex}

    \includegraphics[]{figures/cc-modelling-dc-io-diagram.png}
    % {\footnotesize \textbf{(a)} \gls{mimoLabel} configuration\\}
    
    \vspace{1ex}
    
    \includegraphics[]{figures/cc-modelling-c-io-diagram.png}
    % {\footnotesize \textbf{(b)} Cascade configuration}
    
    \caption{Inputs-outputs block diagram of the main model components}
    \labfig{intro:modelling:ann-model-configuration}
\end{marginfigure}
