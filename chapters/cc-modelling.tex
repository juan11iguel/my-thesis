\setchapterpreamble[u]{\margintoc}
\chapter{Modelling of a combined cooling system}
\labch{solhycool:modelling}

\tldrbox{
    This chapter describes the steady-state modelling of the different
    components of a combined cooling system, mainly a \gls{wctLabel} and a
    \gls{dcLabel}. Different alternatives are presented: from physical models
    to data-driven approaches, including the generation of samples for
    data-driven models trained using data from a physical model. Models are
    also developed for the other components of the system and finally it is
    shown how they are integrated into a complete system model. The complete
    system model interface is defined at \refmod{mod:cc} and a block diagram is
    presented in \reffig{cc:modelling:complete-model-diagram} including all
    relevant variables.
}

\section*{Introduction}

% Estructura a seguir:

% Estado del arte + Contribución del capítulo + Estructura del capítulo

% Introducción de artículo: Wet cooling tower performance prediction in CSP
% plants: A comparison between artificial neural networks and Poppe’s model
In order to study the potential advantages of making use of a combined
cooling\todo{ Ahora mismo esta introducción es demasiado parecida al TL;DR, hay
que distinguirla } system, it is first necessary to develop the modelling of
its components. Since the objective is performance prediction, this chapter
focuses on the steady state modelling of the combined cooler main components,
\ie the \gls{wctLabel} and the \gls{dcLabel}. More specifically, the aim is to
compare two modelling strategies: that based on physical equations
(\refsec{intro:modelling:first-principle}) and that based on black box models
(\refsec{intro:modelling:data-driven}) such as \glspl[format=long]{annLabel},
in order to see which one is more suitable for its integration in the
optimization of the complete process. 

This chapter presents a comparison between the two modelling approaches, at
steady state and with a focus on optimization applications, in terms of
predictive capabilities, experimental and instrumentation requirements,
execution time, implementation and scalability. A sensitivity analysis is
performed to further analyze and compare each case study. It also presents and
evaluates all relevant aspects of interest in the development of such models,
specifically for \gls{annLabel}s, model configuration, architecture and
topology are discussed. Other system components are also described in
\nrefsec{cc:modelling:other-components} and finally their integration is
discussed in \nrefsec{cc:modelling:complete-model}.


%===================================
%===================================
\section{Wet cooler}
\labsec{cc:modelling:wct}

% Estado del arte en modelado \gls{wctLabel}
%================================
% \subsection{Wet cooling tower modelling}

In the case of the models based on physical equations, the analysis of wet
cooling towers has its origin in~\sidecite{merkel_verdunstungskuhlung_1925}, in
which the theory for their performance evaluation was developed. Merkel
proposed a model based on several assumptions to simplify the heat and mass
transfer equations to a simple hand calculation. However, these assumptions
mean that Merkel's method does not reliably represent the physics of the heat
and mass transfer process in a cooling tower. This was already stated by
Bourillot~\sidecite{bourillot_hypotheses_1983} who concluded that the Merkel
method is simple to use and can correctly predict cold water temperature when
an appropriate value of the coefficient of evaporation is used. However, it is
insufficient for the estimation of the characteristics of the warm air leaving
the fill and for the calculation of changes in the water flow rate due to
evaporation. Jaber and Webb~\sidecite{jaber_design_1989} developed the
equations necessary to apply the effectiveness-NTU\sidenote{The
effectiveness-NTU method estimates how well a heat exchanger transfers heat by
comparing the actual heat transfer to the maximum possible, using a parameter,
\gls{ntuLabel}, that reflects its size and flow characteristics.} method
directly to counterflow or crossflow cooling towers. This approach is
particularly useful in the latter case and simpler compared to a more
conventional numerical procedure. Notice that the effectiveness-\gls{ntuLabel}
method is based on the same simplifying assumptions as the Merkel method. On
the other hand, Poppe and Rögener~\sidecite{poppe_berechnung_1991} developed
the Poppe method. They derived the governing equations for heat and mass
transfer in a wet cooling tower and did not make any simplifying assumptions as
in the Merkel theory, which makes it a very precise model. As a matter of fact,
predictions from the Poppe formulation have resulted in values of evaporated
water flow rate that are in good agreement with full scale cooling tower test
results~\sidecite{kloppers_critical_2005}. This model has already been used for
the evaluation of the thermal performance of solar power plants using different
condensation systems (wet, dry and hybrid system), as can be found in Cutillas
et al.~\sidecite{cutillas_energetic_2021}. 

In the case of black box models, numerous authors in the literature have
designed \gls{annLabel} models for \gls{wctLabel} with different objectives,
such as performance prediction, simulation and optimization. One of the first
works in this area is the one described in~\sidecite{hosoz_performance_2007}
where an \gls{annLabel} model was developed to predict the performance of a
forced-counter flow cooling tower at lab scale. In this case, the input
variables were the dry bulb temperature, the relative humidity of the air
stream entering the tower, the temperature of the water entering the tower, the
air volume flow rate and the cooling water mass flow rate. The outputs of this
model were the heat rejection rate at the tower, the mass flow rate of water
evaporated, the temperature of the cooling water at the tower outlet, the dry
bulb temperature and the relative humidity of the air at the outlet of the
tower. The results obtained with a 5-5-5\sidenote{The notation $n_1$-...-$n_l$
represents the architecture of the \gls{annLabel} model, where $l$ is the
number of layers and $n_i$ are the nodes in each one of the layers.}
\gls{annLabel} demonstrated that wet cooling towers at lab-scale can be
modelled using \gls{annLabel}s with a high degree of accuracy. There are also
\gls{annLabel} models for Natural Draft Counter-flow Wet Cooling Towers
(ND\gls{wctLabel}) at lab-scale, such as the one proposed by
\sidecite{gao_artificial_2013}. In this case, the authors used a 4-8-6
\gls{annLabel} structure and considered some additional variables, such as air
gravity, wind velocity, heat transfer coefficients and efficiency as outputs.
All these works can be useful to validate the model development methodology but
may fail predicting the performance of \gls{wctLabel} at larger scale. In this
sense, special attention deserves the study carried out by
\sidecite{song_novel_2021} where an 8-14-2 \gls{annLabel} model was proposed to
predict the performance (the cooling number and the evaporative loss
proportion) of ND\gls{wctLabel}s at commercial scale. The model is based on 638
sets of field experimental data collected from 36 diverse ND\gls{wctLabel}s
used in power plants. It is a very challenging work since it covers samples
from a wide range of tower sizes and capacities being the \gls{mreLabel} below
5~\%. 

From the literature review, it can be stated that there are works based on
Poppe and \gls{annLabel} models that evaluate the main output variables of
\gls{wctLabel}s. Nevertheless, to the author knowledge, there are no studies
focused on the comparison between both modelling strategies. Also lacking is a
comprehensive analysis of the different aspects that affect the models
development and performance.


The static models presented in this section have been developed to predict two
main outputs, the water temperature at the outlet of the \gls{wctLabel},
$T_{w,o}$, and the water consumed due to evaporation losses,
$\dot{m}_{w,lost}$. The inputs variables required by both modelling approaches,
Poppe model and \gls{annLabel} models, are: the cooling water flow rate
($\dot{m}_w$), the water temperature at the inlet of the \gls{wctLabel}
($T_{w,i}$), the ambient temperature ($T_{\infty}$), the ambient relative
humidity ($\phi_{\infty}$) and the frequency percentage of the fan ($f_{fan}$)
(or its equivalence in air mass flow rate\sidenote{\gls{annLabel} uses as input
$f_{fan}$ whereas Poppe's model uses $\dot{m}_a$.}, $\dot{m}_a$).

%====================================
\subsection{Poppe model} 
\labsec{cc:modelling:wct:poppe}

The well-known Merkel number is accepted as the performance coefficient of a
wet cooling tower \sidecite{navarro_critical_2022}. This dimensionless number is
defined in \refeq{Me}, and it measures the degree of difficulty of the
mass transfer processes occurring in the exchange area of a wet cooling
tower.

\begin{equation} 
    Me = \frac{h_D a_v V}{\dot{m}_w}, 
    \labeq{Me}
\end{equation}

where $h_D$ is the mass transfer coefficient, $a_V$ is the surface area of
exchange per unit of volume and $V$ is the volume of the transfer region.

The Merkel number can be calculated using the Merkel and Poppe theories for
the performance evaluation of cooling towers. On the one hand, the Merkel
theory \sidecite{merkel_verdunstungskuhlung_1925} relies on several critical assumptions, such as the
Lewis factor (Le) being equal to 1, the air exiting the tower being saturated
with water vapour and it neglects the reduction of water flow rate by
evaporation in the energy balance. On the other hand, the Poppe theory
\sidecite{poppe_berechnung_1991}, which is the one used in this work, do not consider
simplifying assumptions, thus being the one most usually preferred. In this
theory, the authors derived the governing equations for heat and mass
transfer in the transfer region of the wet cooling tower (control volume
shown in \reffig{cc:modelling:poppe-transfer-area}) assuming a one dimensional problem. In
this figure, the red and green dashed lines indicate the fill and air-side
control volumes, respectively.

\vspace{1.5cm} 
\begin{figure}[htbp] 
    \begin{overpic}[width=10cm]{figures/poppe_control_volume.pdf} 
        \put(-3.3,26){$dz$}
        \put(13,65){$\dot{m}_w+d\dot{m}_w$} \put(14,57){$h_w+dh_w$}
        \put(15,-7){$\dot{m}_w$, $h_w$}
        \put(60.5,65){$\dot{m}_a\left(1+\omega+d\omega\right)$} \put(65,57){$h+dh$}
        \put(55,-7){$\dot{m}_a\left(1+\omega\right)$, $ h $}
        \put(44,30){$d\dot{m}_w=h_D\left(\omega_{s,w}-\omega\right)dA$}
        \put(44,22){$h_C\left(T_{w}-T\right)dA$}        
    \end{overpic} 
    \vspace{1cm} 
    \caption {Control volume in the exchange area of a wet cooling tower arrangement.}
    \labfig{cc:modelling:poppe-transfer-area}
\end{figure}

Following the detailed derivation process and simplification of the
previously-mentioned governing equations described in
\cite{navarro_critical_2022}, the major following equations for the heat and mass
transfer obtained, according to the Poppe theory, are: 

\begin{eqnarray}
    & & 
    \frac{d\omega}{dT_w} = \frac{c_{p_w}
    \frac{\dot{m}_w}{\dot{m}_a}\left(\omega_{s,w} - \omega \right)
    }{\left(h_{s,w}-h\right)+\left(\lew-1
    \right)\left[\left(h_{s,w}-h\right)-\left(\omega_{s,w} - \omega
    \right)h_v\right] -\left(\omega_{s,w} - \omega \right)h_w } \\
    & & 
    \frac{dh}{dT_w} = c_{p_w}\frac{\dot{m}_w}{\dot{m}_a}
    \left[1+\frac{\left(\omega_{s,w} - \omega  \right)c_{p_w}
    T_w}{\left(h_{s,w}-h\right)+\left(\lew-1
    \right)\left[\left(h_{s,w}-h\right)-\left(\omega_{s,w} - \omega
    \right)h_v\right] -\left(\omega_{s,w} - \omega \right)h_w}\right] \\
    & & 
    \frac{d\Me}{dT_w} = \frac{c_{p_w}  }{\left(h_{s,w}-h\right)+\left(\lew-1
    \right)\left[\left(h_{s,w}-h\right)-\left(\omega_{s,w} - \omega
    \right)h_v\right] -\left(\omega_{s,w} - \omega \right)h_w }, \label{eq:Me1}
\end{eqnarray}

where the quantity referred to as $\Me $ in Eq. \ref{eq:Me1}, is the Merkel
number calculated according to the Poppe theory. The above described
governing equations can be solved by the fourth order Runge-Kutta method to
provide the evolution of the air humidity ratio, air enthalpy and Merkel
number inside the transfer area of the cooling tower (fill). Once these
profiles are known, the amount of water lost due evaporation can be
calculated as per Eq. \refeq{mwlost}. Refer to \sidecite{navarro_critical_2022}
for additional information concerning the calculation procedure.

\begin{equation} 
    Me = \frac{h_D a_v V}{\dot{m}_w}, % \label{eq:Me} %
\end{equation}

\begin{equation}
    \dot{m}_{w,lost}=\dot{m}_a (\omega_{a,o}-\omega_{a,i})
    \labeq{mwlost} 
\end{equation}


It is important to mention that the Merkel number varies with the operation
conditions and its value can be obtained using a correlation with the
water-to-air mass flow ratio as an independent variable. One of the proposed
correlations in ASHRAE~\sidecite[*-2]{ashrae_hvac_2004} is:

\begin{equation}
    \Me = c\left({\dot{m}_w}/{\dot{m}_a}\right)^{-n}
    \labeq{cc:Me-corr}
\end{equation}

where the constants $c$ and $n$ can be obtained from the fitting of
 experimental data\sidenote[][*-1]{See \nrefsec{cc:validation:wct}}. 

%=====================================
\subsection{Samples generation for first-principles to data-driven models}

The first pair of input variables for the \gls{wctLabel} sample generation are
the wet bulb temperature ($T_{wb}$) and the difference between this temperature
and the system inlet temperature ($\Delta T_{wb-in}$). The wet bulb temperature
is used instead of the ambient temperature or the relative humidity, because as
it can be derived from the physical model, it is the most relevant
thermodynamic variable for the wet cooling tower performance. Using both the
ambient temperature and the relative humidity would lead to a larger than
necessary input space with many duplicate samples, as the wet bulb
temperature is a function of both variables. The second pair of input
variables are the cooling water flow rate ($q_{wct}$) and, following the
reasoning from the physical model, the air to water mass flow ratio
($\dot{m}_{a}/\dot{m}_{wct}$), since it is a key parameter in defining the
operating conditions of the tower. From the resulting 2D grid, valid combinations
are obtained by calculating the air mass flow rate and finding if a valid fan
speed can be obtained using an air mass flow rate to fan speed empirical
correlation.

Finally, all valid thermodynamic and operational combinations are merged into a
comprehensive sample set, enabling detailed system evaluations across a
realistic and constrained input space.

\subsection{Model interface}

\begin{modelcounter}{Wet cooling tower}
    \begin{equation*}
        T_{wct,out},\,C_{w,wct} = \text{wct\:model}(q_{wct},\, \omega_{wct},\, T_{amb},\, HR,\, T_{wct,in})
    \end{equation*}
    \labmod{wct}
\end{modelcounter}

\begin{modelcounter}{Wet cooling system model}
    \begin{align*}
        T_{wct,out},&\,C_{e},C_{w},\,T_{c,in},\,T_{c,out} = \text{wcs\:model}(q_{wct},\, \omega_{wct},\, T_{amb},\, HR,\, T_{wct,in}) \\
        % condenser model
        & T_{c,in},\,T_{c,out} = \text{condenser\:model}(q_c,\, \dot{m}_v,\, T_v) \\
        % wet cooler model
        & T_{wct,out},\,C_{w,wct} = \text{wct\:model}(q_{wct},\, \omega_{wct},\, T_{amb},\, HR,\, T_{c,out}) \\
        % electrical consumptions
        & C_{e,c} = \text{electrical\:consumption}(q_c) \\
        & C_{e,wct} = \text{electrical\:consumption}(\omega_{wct}) \\
        % totals
        & C_{e} = C_{e,wct} + C_{e,c} \\
        & C_{w} = C_{w,wct}
    \end{align*}
    \labmod{wet-system}
\end{modelcounter}


%=====================================
%=====================================
\section{Dry cooler}



%================================
\subsection{Physical model}

a\todo{Pendiente de basarse en el artículo del modelo físico del DC con Elxe}


%=====================================
\subsection{Samples generation for first-principles to data-driven models}
\labsec{cc:modelling:dc:samples}
% de muestreos seguida y mostrar la gráfica de resultados

Similar to the wet cooling tower case, setting absolute values for both the
inlet temperature and the environment temperature will lead to many unfeasible
combinations ($T_{dc,in} \le T_{db}$). So instead, values are generated for the
temperature difference, therefore, a 2D grid is constructed using combinations
of ambient/dry-bulb temperature ($T_{amb}$) and the difference between inlet
and ambient temperature (($\Delta T_{amb-in}$)). For each valid temperature
pair ($T_{amb}$, $T_{dc,in}$), additional independent variables ($q_{dc}$,
$\omega_{dc}$) are combined via a Cartesian product, resulting in a full
multidimensional grid of plausible operating points. This systematic procedure
ensures a dense and uniform sampling across all relevant input dimensions.
Finally, infeasible combinations are filtered based on physical constraints.


%incluir figura con distribución de muestras generadas

%================================
\subsection{Model interface}

\begin{modelcounter}{Dry cooler}
    \begin{equation*}
        T_{dc,out} = \text{dc\:model}(q_{dc},\, \omega_{\text{dc}},\, T_{\text{amb}},\, T_{dc,in})
    \end{equation*}
    \labmod{dc}
\end{modelcounter}
% \marginnote[*-2]{The condenser area ($A$) is a constant parameter}

\begin{modelcounter}{Dry cooling system model}
    \begin{align*}
        T_{dc,out},&\,C_{e},\,T_{c,in},\,T_{c,out} = \text{dcs\:model}(q_{dc},\, \omega_{dc},\, T_{amb},\, T_{dc,in}) \\
        % condenser model
        & T_{c,in},\,T_{c,out} = \text{condenser\:model}(q_c,\, \dot{m}_v,\, T_v) \\
        % dc model
        & T_{dc,out} = \text{dc\:model}(q_{dc},\, \omega_{\text{dc}},\, T_{\text{amb}},\, T_{c,out}) \\
        % electrical consumptions
        & C_{e,c} = \text{electrical\:consumption}(q_c) \\
        & C_{e,dc} = \text{electrical\:consumption}(\omega_{dc}) \\
        % totals
        & C_{e} = C_{e,dc} + C_{e,c} \\
    \end{align*}
    \labmod{dc-system}
\end{modelcounter}


%=====================================
%=====================================
\section{Other components}
\labsec{cc:modelling:other-components}


%================================
\subsection{Electrical consumption}

Electrical consumption is modelled with polynomial regressions of order 3 from
experimental data:

\begin{modelcounter}{Electrical consumption}
    \begin{align*}
        C_{e} &= \mathrm{electrical\:consumption\:model}(x) \\
        & C_{e} = p_1 \cdot x^3 + p_2 \cdot x^2 + p_3 \cdot x + p_4
    \end{align*}
    \labmod{electrical-consumption}
\end{modelcounter}

where \(C_e\) represents the electrical consumption, and \(x\) is the input
variable (e.g., the recirculated cooling water flow rate, particular cooler fan
speed, etc.). The coefficients \(p_i\) correspond to a polynomial regression
and must be calibrated individually for each component.


%================================
\subsection{Surface condenser}
\labsec{cc:modelling:condenser}

The surface condenser is a heat exchanger that condenses steam into water,
assuming that all the vapor that enters the condenser (at saturated
conditions), leaves it as saturated liquid, it can be modelled by applying the
first law of thermodynamics, which states that the heat lost by the steam
(\textit{released}) is equal to the heat gained by the cooling water
(\textit{absorbed}), and equal to the heat transferred by the condenser heat
transfer surfaces (\textit{transferred}).

\begin{modelcounter}{Surface condenser}
    \begin{align*}
        T_{c,in},\,T_{c,out}& = \mathrm{condenser\:model}(\dot{m}_{c}, T_{v},\dot{m}_{v}) \\
        & LMTD = \frac{T_{c,out}-T_{c,in}}{\ln\left(\frac{T_{v}-T_{c,in}}{T_{v}-T_{c,out}}\right)} \\
        & \dot{Q}_{released} = \dot{m}_v \cdot (h_{sat.vap}-h_{sat.liq}) \\
        & \dot{Q}_{absorbed} = \dot{m}_c \cdot c_p(T_{c,out}-T_{c,in}) \\
        & \dot{Q}_{transferred} = U\cdot A \cdot LMTD \\
        & U = \ldots
    \end{align*}
    \labmod{sc}
\end{modelcounter}
\marginnote[*-2]{The condenser area ($A$) is a constant parameter}


where \(T_{c,in}\) and \(T_{c,out}\) are the cooling water inlet and outlet
temperatures, respectively, \(\dot{m}_c\) the cooling water mass flow rate,
\(T_v\) vapour temperature and \(\dot{m}_v\) its mass flow rate and
\(h_{sat,vap}\) and \(h_{sat,liq}\) are the specific enthalpies of the steam at
the inlet and outlet of the condenser, respectively. \(\dot{Q}\) represents the
heat transfer rate \ie the thermal power.



% TODO: Añadir unidades al lado como una notación al margen de cada bloque de modelo
% \begin{modelcounter}{Condenser}
%     \begin{align*}
%         C_{e,c} &= \text{recirculation\:consumption}(q_c) \\
%         & C_{e,c} = p_1 \cdot q_c^3 + p_2 \cdot q_c^2 + p_3 \cdot q_c + p_4
%     \end{align*}
% \end{modelcounter}
    % q (m³/h) -> P_pump (kW)
    % f(x) = p1*x^3 + p2*x^2 + p3*x + p4
    %        p1 =    0.1461;
    %        p2 =    5.763;
    %        p3 =    -38.32;
    %        p4 =    227.8;
    % Ce_c =max((p1.*qc.^3 + p2.*qc.^2 + p3.*qc + p4)*1e-3, 0); %kW

\subsection{Mixers}

The mixers outlet flow ($q_{mix,out,i}$) and temperature ($T_{mix,out,i}$) can
be determined with a simple mass and energy balances from its inlets streams
($q_{mix,in},\,T_{mix,in}$):


% \modeldefinitionbox{Mixer model}{
\begin{modelcounter}{Mixer model}
    \begin{align}
        q_{mix,out},\,&T_{mix,out} = \text{mixer\:model}(q_{mix,in,1},\,T_{mix,in,1},\,q_{mix,in,2},\,T_{mix,in,2}) \\
        & q_{mix,out} = q_{mix,in,1} + q_{mix,in,2} \\
        & T_{mix,out} = T_{mix,in,1} \cdot \frac{c_p(T_{mix,in,1})}{c_p(T_{out,i})}\frac{q_{mix,in,1}}{q_{mix,out,i}} + \nonumber \\ &\qquad T_{mix,in,2} \cdot \frac{c_p(T_{mix,in,2})}{c_p(T_{out,i})}\frac{q_{mix,in,2}}{q_{mix,out,i}}
    \end{align}
    \labmod{mixer}
\end{modelcounter}
% }

where $c_p(\cdot)$ is the specific heat, which can be assumed to be the same for the
mixing temperature differences of this type of system.

%=====================================
%=====================================
\section{Complete system}
\labsec{cc:modelling:complete-model}

The complete model of the combined cooling system integrates the models of the
\gls{wctLabel} and \gls{dcLabel}, along with the surface condenser and the
mixers, as defined in \nrefmod{cc}\sidenote{Although the electrical consumption
for cooling water recirculation is attributed to the condenser in this model,
other components—particularly the hydraulic circuit and the dry cooler—also
contribute significantly to circulation resistance}. The full diagram,
including all variables, is shown in
\reffig{cc:modelling:complete-model-diagram}.

To solve the system, the condenser model is evaluated first, providing the
inlet temperature for the dry cooler. Once the dry cooler is solved, the
resulting temperatures allow for solving the wet cooling tower. Finally, the
mixers are evaluated to determine the final outlet temperature of the combined
cooler, which should match the condenser’s inlet temperature.

\begin{marginfigure}[]
    \includegraphics[]{figures/cc-modelling-wct-io-diagram.png}
    % {\footnotesize \textbf{(a)} \gls{mimoLabel} configuration\\}
    
    \vspace{1ex}

    \includegraphics[]{figures/cc-modelling-dc-io-diagram.png}
    % {\footnotesize \textbf{(a)} \gls{mimoLabel} configuration\\}
    
    \vspace{1ex}
    
    \includegraphics[]{figures/cc-modelling-c-io-diagram.png}
    % {\footnotesize \textbf{(b)} Cascade configuration}
    
    \caption{Inputs-outputs block diagram of the main model components}
    \labfig{intro:modelling:ann-model-configuration}
\end{marginfigure}


\begin{modelcounter}{Combined cooling system}
    \begin{align*}
        T_{cc,out},\,C_{e}&,\,C_{w},\,T_{c,in},\,T_{c,out} = \text{ccs\:\:model}(q_{c}, R_{p}, R_{s}, \omega_{dc}, \omega_{wct},T_{amb},HR_i,T_{v},\dot{m}_{v}) \\
        & T_{cc,in}=T_{c,out} \\
        & T_{dc,in}=T_{cc,in} \\
        % three-way valves
        & q_{dc} = q_{c} \cdot (1-R_{p}) \\
        & q_{wct,p} = q_{c} \cdot R_{p} \\
        & q_{wct,s} = q_{dc} \cdot R_{s} \\
        % dc model
        & T_{dc,out},\,C_{e,dc} = \text{dc\:model}(q_{dc},\, \omega_{\text{dc}},\, T_{\text{amb}},\, T_{dc,in}) \\
        % first mixer
        & q_{wct},\,T_{wct,in} = \text{mixer\:model}(q_{wct,p},\,T_{T{cc,in}},\, q_{wct,s},\, T_{dc,out}) \\
        % wct model
        & T_{wct,out},\,C_{e,wct},\,C_{w,wct} = \text{wct\:model}(q_{wct},\, \omega_{wct},\, T_{amb},\, HR,\, T_{wct,in}) \\
        % condenser model
        & T_{c,in},\,T_{c,out} = \text{condenser\:model}(q_c,\, \dot{m}_v,\, T_v) \\
        % final mixer
        & q_{cc},\,T_{cc,out} = \text{mixer\:model}(q_{wct},\, T_{wct,out},\, q_{dc},\, T_{dc,out}) \\
        % electrical consumptions
        & C_{e,c} = \text{electrical\:consumption}(q_c) \\
        & C_{e,dc} = \text{electrical\:consumption}(\omega_{dc}) \\
        & C_{e,wct} = \text{electrical\:consumption}(\omega_{wct}) \\
        % totals
        & C_{e} = C_{e,dc} + C_{e,wct} + C_{e,c} \\
        & C_{w} = C_{w,wct}
    \end{align*}
    \labmod{cc}
\end{modelcounter}
% }

\begin{figure}
    \includegraphics[width=\textwidth]{figures/cc-modelling-complete-model-diagram.png}
    \caption{Complete model diagram of the combined cooling system}
    \labfig{cc:modelling:complete-model-diagram}
\end{figure}
