\setchapterpreamble[u]{\margintoc}
\chapter{Method for optimal coupling and operation of a solar driven \gls{medLabel} system}
\labch{solarmed:optimization}

\tldrbox{
    This chapter describes a method to develop an operational strategy enabling
    the seamless integration of a solar driven \gls{medLabel} system in an
    autonomous and optimal manner, including decisions on when to start or stop
    each subsystem and how to regulate them during operation.

    % The method is based on a hierarchical control approach consisting of three layers: operation plan, operation optimization, and operation regulation. The operation plan layer makes decisions for the operation state, the operation optimization layer sets the setpoints given to the continuous process variables that are to be followed by the low-level regulatory control layer.    
}

%===================================
%===================================
\section{Introduction}

% Estado del arte de la optimización de procesos de desalación térmica

% Contribución

% Estructura del capítulo

%===================================
%===================================
\section{Problem description}
\labsec{solarmed:optimization:problem-description}

Two decision variables are defined to manipulate the discrete state of
each subsystem defined in \refsec{solarmed:modelling:discrete}: $med_{mode}$ and $sfts_{mode}$. These binary (\ie integer) variables establish
whether the particular subsystem is active (=1) or inactive (=0). This is
directly related to the operation state of the particular subsystem\sidenote{As
defined in Tables~\ref{tab:solarmed:modelling:sfts_states} and
\ref{tab:solarmed:modelling:med_fsm_states}}. Once the values for these
decision variables are provided, the low-level control layer is in charge of
safely transitioning between operation states\sidenote{\eg $med_{mode}: 0 \rightarrow
1$, med state: \textit{off} $\rightarrow$ \textit{generating vacuum}
$\rightarrow$ \textit{starting-up} $\rightarrow$ \textit{active}}. This is
accounted for in the models by the integrated finite-state machines as
explained in \refsec{solarmed:modelling:discrete}.

The problem is defined as follows:\sidenote{In general $q$ represents flow
rates while $T$ are temperatures. \reffig{solarmed:process-diagram} can be consulted
for subscript reference.}\todo{en el bloque de definición del modelo
completo, incluir $T_{sf,out} \le \overline{T_{sf,out}}$ para determinar la
variable de salida \textit{valid operation}}

\marginnote[*5]{$\forall i = 1 \ldots n_{steps}$ is a notation to indicate that
    a condition must be held at every step $i$ in the optimization horizon
    ($n_{steps}$).\\ Bold variables represent vectors.}

\problemdefinitionbox{\gls{solarmedLabel}}{
    \begin{equation*}
        \min_{\mathbf{x},\, \mathbf{e};\, \boldsymbol{\theta}} \quad J = f(\mathbf{x}, \mathbf{e}; \boldsymbol{\theta}) = \sum_{i=1}^{n_{steps}} \left( J_{e,i}-J_{w,i} \right)
    \end{equation*}

    \textbf{with}:
    \begin{align*}
        \quad for\: i &= 1 \ldots n_{steps}: \\
        & \quad J_{w,i}=q_{d,i} \cdot P_{w,i}\:\text{if \textit{valid operation} else 0} \\
        & \quad J_{e,i} = C_{e,i} \cdot P_{e,i} \\
        & \quad q_{d,i},\,C_{e,i},\,\text{valid operation}=\text{solarmed model}(x_{c,i}, x_{p,i}, ...)
    \end{align*}
    \begin{itemize}
        \item Decision variables
        \[
        \mathbf{x} = [\mathbf{med_{mode}},\,\mathbf{sfts_{mode}},\,\mathbf{q_{sf}},\,\mathbf{q_{ts,src}},\,\mathbf{q_{med,s}},\,\mathbf{q_{med,f}},\,\mathbf{T_{med,s,in}},\,\mathbf{T_{med,c,out}}]
        \]
        where $\mathbf{x}_{nx \times \sum{n_{updates,xi}}}=[x_{1,i},\allowbreak\;
        \ldots,x_{1,n_{updates,x_{1}}},\allowbreak\; \ldots,x_{n_x, n_{updates,
        x_{nx}}}]$
        % where $\mathbf{x}=[x_{1,1},\,\ldots\, x_{1,n_{steps}},\, \ldots,\,
        % x_{n_{x},n_{steps}}]$
        
        \item Environment variables
        \[
        \mathbf{e} = [\mathbf{I},\,\mathbf{T_{\text{amb}}},\, \mathbf{P_e},\, \mathbf{P_{w}}]
        \]
        where $e=[e_{1,1},\,\ldots\, e_{1,n_{steps}},\, \ldots,\, e_{n_{e},n_{steps}}]$

    \end{itemize}

    \textbf{subject to}:
    \begin{itemize}
        
        \item Box-bounds
        \begin{itemize}
            \item $\mathbf{med_{mode}} \in [0,1] \subset \mathbb{Z}$
            \item $\mathbf{sfts_{mode}} \in [0,1] \subset \mathbb{Z}$
            \item $\mathbf{q_{sf}} \in [\underline{q_{sf}}, \overline{q_{sf}}] \subset \mathbb{R}$
            \item $\mathbf{q_{ts,src}} \in [\underline{q_{ts,src}}, \overline{q_{ts,src}}] \subset \mathbb{R}$
            \item $\mathbf{q_{med,s}} \in [\underline{q_{med,s}}, \overline{q_{med,s}}] \subset \mathbb{R}$
            \item $\mathbf{q_{med,f}} \in [\underline{q_{med,f}}, \overline{q_{med,f}}] \subset \mathbb{R}$
            \item $\mathbf{T_{med,s,in}} \in [\underline{T_{med,s,in}}, \overline{T_{med,s,in}}] \subset \mathbb{R}$
            \item $\mathbf{T_{med,c,out}} \in [\underline{T_{med,c,out}}, \overline{T_{med,c,out}}] \subset \mathbb{R}$
        \end{itemize}
    \end{itemize}

    \textit{valid operation} conditions, $\forall i = 1 \ldots n_{steps}$:
    \begin{itemize}
        \item $T_{sf,out} \le \overline{T_{sf,out}}$
    \end{itemize}
}

Where the objective is to minimize the cumulative cost of operation ($J$).
Fresh water ($q_{med,d}$) sold ($J_w$) at price $P_w$ is the negative term
while electrical consumptions ($C_e$) at price $P_e$ make up the positive cost
term ($J_e$). The benefit ($B$) of operation is simply the inverse of the cost
of operation.

%================================
%================================
\subsection{Implementation discussion}
\labsec{solarmed:optimization:implementation}

%==============================
\subsubsection{On the constraint handling}
\labsec{solarmed:optimization:constraints-discussion}

% % The reader might notice that no constraints are explicitly defined in the
% % problem definition. This is because the constraints are implicitly defined in
% % the model equations, which are used to evaluate the objective function. This
% % design decision is motivated to avoid the need for a constraint-handling
% % capable optimization algorithm, limiting the choice for an already complex
% % \gls{minlpLabel} problem.\sidenote{See
% % \nrefsec{intro:optimization:constraints} for a more detailed discussion on the
% % topic}. Specifically, two aspects demand further consideration:

% \begin{enumerate}
%     \item The decision value for the \gls{medLabel} outlet condenser
%     temperature ($T_{med,c,out}$) is not a direct input to the system, but rather a
%     setpoint to be followed by a low-level control loop, by manipulating the cooling
%     water flow rate ($q_{med,c}$). This input might saturate and thus not be
%     able to achieve the desired setpoint. In this case, a new value for the
%     decision variable is computed, which is the maximum value that can be
%     achieved. 
    
%     Here, in order to ensure \texttt{valid operation} the decision
%     variable is manipulated.

%     \item For the solar field, in order to not constantly interrupt the
%     evaluation due to the solar field temperature going above
%     $\overline{T_{sf,out}}$ (\ie 120 $^\circ$C), the model saturates this
%     temperature when going above and sets a flag. The problem with the approach
% i  s that when there is low energy demand from the
%     load, and likely because it favors energy transfer in the heat exchanger
%     \sidenote{greater temperature difference in primary side instead of greater mass
%     flow rate with increased cost and diminished available energy increase},
%     the optimizer tends to minimize the solar field flow, and systematically
%     lets the solar field outlet temperature reach the limit. To avoid this
%     situation, the positive term of the objective function is nullified in
%     iterations where the constraint is not met.
    
%     Here, in order to ensure \texttt{valid operation} the fitness function is
%     manipulated to de-incentivize decision variable values that lead to
%     unfeasible operation. 

% \end{enumerate}

% \subsubsection{On the prediction horizon}
% \labsec{solarmed:optimization:decision_variables-discussion}

% The problem is designed as an optimization problem with a shrinking
% horizon\marginreminder{An optimization where the horizon end is fixed, and as
% time progresses, the start of the horizon moves forward. See
% \nrefsec{intro:optimization}}. The horizon size should be large enough so that
% decisions on how to operate the system are made with perspective, taking into
% account how they will affect the system in the future, but not so large that
% current decisions have no impact on the far future, and making the
% the problem dimensionality become unmanageable. 

% For this case study, this parameter should be chosen based
% on the hours of capacity of the thermal storage to operate the \gls{medLabel} system.

% The thermal storage capacity is XXX which allows the system to operate with no
% supply from the solar field for up to XX hours. This means that depending on
% the charge state of the thermal storage, the system could start operation
% independently of the irradiance conditions, or operate at different levels of
% temperature. Considering this, the optimization horizon
% in time units, chosen was 36 hours. This means that if the problem is evaluated at
% 5:00 on day 1, the fitness function is evaluated until 19:00 of day 2 \ie
% including the end of operation for day 2. 

% \subsubsection{On the decision variables update frequency}
% \labsec{solarmed:optimization:decision_variables-discussion}

% If a fixed decision variable update frequency is chosen, the number of decision
% variables for a large horizon like the one chosen, can become unmanageable,
% specially for the binary decision variables\sidenote{Further discussed in the
% next section}. Instead, a new design parameter is introduced: the number of
% decision variable updates ($n_{updates, x_i}$) for each decision variable in
% the optimization problem.

% Thus, the decision vector is formed by each individual decision variable repeated as
% many times as updates for it, $X_{nx \times
% \sum{n_{updates,xi}}}=[x_{1,k},\allowbreak\;
% \ldots,x_{1,n_{updates,x_{1}}},\allowbreak\; \ldots,x_{n_x, n_{updates,
% x_{nx}}}]$. The number of updates of the decision variable ($n_{updates, x_i}
% \in [1,n_{steps}]$) can be chosen individually. More updates are assigned to
% variables regulating faster dynamics ($q_{sf}$, $q_{ts,src}$), and these
% updates of the decision variables are evenly distributed throughout the active
% period of the subsystem within the horizon. 


%===================================
%===================================
\subsubsection{On solving the optimization problem}[On solving the problem]
\labsec{solarmed:optimization:solving-discussion}

% Discusión sobre la resolución del problema
The implementation challenge of a control scheme for this MINLP, is shown in
\cite{josedomingo}. Since the decision variable update at any given step, can
be approached, from many different previous decision variables updates in past
steps, the range of possible operation modes grows exponentially...

Enabling the coupling between the variable solar energy source and the flexible
MED system, can be achieved in several ways. Here we are dealing with a Mixed
Integer Non Linear Problem (MINLP).

One of the most important considerations when attempting to solve the
optimization problem, is whether the paths to take (operating modes evolution
in the prediction horizon) are pre-computed or if on the other hand, it's left
to the optimization algorithm to explore the tree and generate its own
trajectories.

An alternative to tackling the problem in its full complexity (both integer and
continuous components), is to solve it in two steps. Where in the first one a
simplified version of the MINLP is first solved using one of the
afore-mentioned HGO alternative, so that the integer variables are set, and
then a second evaluation is solved where only the non-linear problem is solved
through Local optimization. The following solvers are tested:


%===================================
\section[]{Proposed optimization strategy}[Proposed strategy]
\labsec{solarmed:optimization:strategy}

The behavior of the \textit{SolarMED} process is controlled by acting on two
components, a discrete and a continuous one:
\begin{itemize}
    \item Operation state. It defines the discrete state of the system. The
    complete system is divided into two subsystems: the heat generation and
    storage subsystem and the separation subsystem. For the solar field and
    thermal storage subsystem (sfts), the states are simply defined based on
    whether water is being recirculated in each circuit, while for the MED
    various discrete modes can be found: \textit{off}, \textit{generating
    vacuum}, \textit{startup}, \textit{shutdown}, and \textit{idle} operation
    depending on the sequence of values of several process variables.
    \item Process variables. Regulates the continuous-dynamic behavior of the
    process. Specifically, two recirculation flow rates for the sfts subsystem,
    five flow and temperature variables for the MED.
\end{itemize}

The goal is to design an operational strategy that enables the seamless
integration of both subsystems in an autonomous and optimal manner, including
decisions on when to start or stop each subsystem and how to regulate them
during operation. Therefore, considering the whole system as a Mixed Integer
Non-Linear Problem (MINLP). \marginnote{See
\nrefsec{intro:optimization:minlp_problems}}

In order to achieve this objective, a hierarchical control approach was chosen
consisting on three-layers: operation plan, operation optimization, and
operation regulation. This scheme was chosen for two main reasons. On the one
hand, the time scales of the different aspects of the operation of the system
(operation mode changes, process variables setpoint changes, regulatory
control, respectively) can differ substantially. Secondly, it allows to
abstract process complexity from the more computationally demanding upper
layers by allocating it into the downstream layers: The operation plan layer
makes decisions for the operation state, the operation optimization layer sets
the setpoints given to the continuous process variables that are to be followed
by the low-level regulatory control layer.

\begin{marginfigure}[-5.5cm]
    \includegraphics[]{figures/solarmed-optimization-architecture.png}
    \caption{Proposed optimization strategy architecture}
    \labfig{solarmed:optimization:architecture}
\end{marginfigure}

The optimization problem definition is similar between the operation plan and
the operation optimization layers, the only difference being that the operation
plan layer evaluates $n_{problems}$ combinations of $med_{mode}$ and
$sfts_{mode}$ and the evaluation result are the values of these two binary
decision variables. Then, in the operation optimization layer, these binary
variables can be moved to the environment, and thus instead of evaluating a
library of candidate problems, only one is. The periodic evaluation result of
this layer establishes the setpoint values of the continuous process variables.
In this report, focus is given to the upper operation plan layer, which is
detailed below.

\section{Operation plan layer description}

As mentioned, this layer establishes the operation modes, \ie the integer part
of the \gls{minlpLabel} problem. Two evaluations of this layer are performed at different
moments in time in order to establish when to start the subsystems and when to
stop them.


\begin{figure*}
    \centering
    \subfloat[\centering Computation timeline]{{\includegraphics[width=0.48\linewidth]{figures/solarmed-optimization-timeline.png}}}%
    \hspace{0.01\linewidth}
    \subfloat[\centering Operation mode updates time distribution]{{\includegraphics[width=0.48\linewidth]{figures/solarmed-optimization-updates_distribution.png}}}%
    \caption[Operation plan layer computation and updates distribution]{Operation plan layer computation and updates distribution. The yellow line represents the irradiance illustrating the solar day.}%
    \labfig{solarmed:optimization:operation_plan}
\end{figure*}

The complexity of producing good values for the integer decision variables
arises from their combinatory nature. Each value of the integer decision
variables ($med_{mode}$, $sfts_{mode}$) along the prediction horizon produces a
different solution space and is associated with an exponentially growing number
of possible combinations as a function of the number of integer variables
($n_{xi}$) and the number of integer decision variable updates ($n_{updates,
x_i}$). To make this problem solvable in a reasonable amount of time, a
compromise fixed number of combinations, $n_{problems}$, are studied. This
approach allows to move the integer decision variables to the environment and
simplify the problem, at the cost of requiring to evaluate several candidate
problems (MINLP $\rightarrow$ nNLP). In order to maintain the solution close to
the optimal one, decision updates are assigned at strategic time instants.

\subsection{Update times generation}

Since the case study system is fundamentally a solar process, the operation is
strongly dependent on the irradiance availability, and thus operation changes
are likely to take place at the start and end of the solar day. Depending on
the number of updates available (DoF in Fig. \ref{fig:updates_distribution})
for when to start and stop the operation, the operation mode updates are
distributed temporally, as shown in Fig. \ref{fig:updates_distribution} at some
update times per subsystem. These update times are dependent on the solar
irradiance profile and are bounded by lower- and upper-level thresholds,
depending on the plan action (start-up or shutdown). They are named early-late
start or early-late stop thresholds, respectively. Depending on the number of
updates for the particular action, additional thresholds can be added.

\subsection{Candidate problems generation}

\subsubsection{Start-up.} Evaluated before the operation starts (see
Fig. \ref{fig:computation_time}), it has plenty of time to be computed since it
can start hours in advance. The main objective is to establish the operation
start of the subsystems and an estimation of when to stop,and computed before
the first potential operation mode change. Given the computation time of the
objective function for the available computing hardware, it has been determined
that in the order of $n_{problems}\sim 100$ can be evaluated. Given this
constraint, updates are allocated as follows:

\subsubsection{Shutdown.} Computed some time before shutting down the system for day 1 (see Fig. \ref{fig:computation_time}). The objective is to establish accurately when to do so. It uses the latest available state and has available updates for the operation start/end of the second day, in order to consider how it will be affected by the day 1 shutdown decision.


\begin{table}[]
\centering
\caption{Operation plan. Start-up (1) and shutdown (2) degrees of freedom for changes in the operation state.}
\labtab{solarmed:optimization:dof}
\resizebox{\textwidth}{!}{%
\begin{tabular}{lllllllllllll}
\cline{3-13}
 &  & \multicolumn{1}{c}{\multirow{3}{*}{\textbf{Subsystem}}} &  & \multicolumn{7}{l}{\textbf{Degrees of freedom}} &  & \multirow{3}{*}{\textbf{nproblems}} \\ \cline{5-7} \cline{9-11}
 &  & \multicolumn{1}{c}{} &  & \multicolumn{3}{l}{Day 1} &  & \multicolumn{3}{l}{Day 2} &  &  \\ \cline{5-7} \cline{9-11}
 &  & \multicolumn{1}{c}{} &  & Start &  & Stop &  & Start &  & Stop &  &  \\ \cline{3-3} \cline{5-5} \cline{7-7} \cline{9-9} \cline{11-11} \cline{13-13} 
\multicolumn{1}{l|}{\multirow{2}{*}{Evaluation: Start-up (1)}} &  & sfts &  & 3 &  & 3 &  & 1 &  & 1 &  & \multirow{2}{*}{101} \\
\multicolumn{1}{l|}{} &  & med &  & 3 &  & 3 &  & 1 &  & 1 &  &  \\
 &  &  &  &  &  &  &  &  &  &  &  &  \\
\multicolumn{1}{l|}{\multirow{2}{*}{Evaluation: Shutdown (2)}} &  & sfts &  & 3 &  & 2 &  & 2 &  & 2 &  & \multirow{2}{*}{144} \\
\multicolumn{1}{l|}{} &  & med &  & 3 &  & 2 &  & 2 &  & 2 &  &  \\ \cline{3-13} 
\end{tabular}%
}
\end{table}

% \subsection{Candidate problems generation}


% \section{Operation optimization layer description}