\myepigraphhead[500]{Should I kill myself or have a cup of coffee?}{Albert Camus}
\appendix % From here onwards, chapters are numbered with letters, as is the appendix convention
\setchapterstyle{lines}

\chapter{\gls{medLabel} Performance Evaluation}

\section{Uncertainty estimation through first-order Taylor series approximation}
\labsec{appendix:uncertainty}

In this appendix, expressions for the uncertainty estimation of energetic and separation metrics are shown. %In addition, in order to prove that the simpler approach (first-order Taylor expansion) is valid for these metrics, a comparison of the uncertainty propagation results using the first-order Taylor expansion with respect to the Monte Carlo alternative has been performed for the energetic metrics (see Fig.~\ref{fig:un_prop_comparison}). As can be observed, both follow a very similar normal distribution, so it can be stated that the use of the first-order Taylor expansion is valid for these metrics.

%\begin{figure}
    %\centering
    %\includegraphics[width=.7\textwidth]{fig/%uncertainty_propagation_comparison.png}
    %\caption{Probability function distribution results for some metrics with %Monte Carlo and first-order Taylor expansion approaches}
 %   \label{fig:un_prop_comparison}
%\end{figure}


\subsection{Specific Thermal Energy Consumption (STEC)}

\begin{align}
    \Delta STEC &= \Bigg(
    \left( \left| \frac{\delta STEC}{\delta \dot{m}_{d}} \right| \Delta \dot{m}_{d} \right)^{2}
    + \left( \left| \frac{\delta STEC}{\delta \dot{m}_{s}} \right| \Delta \dot{m}_{s} \right)^{2} \nonumber \\
    &\qquad
    + \left( \left| \frac{\delta STEC}{\delta T_{s}} \right| \Delta T_{s} \right)^{2}
    \Bigg)^{1/2}
    \labeq{STEC_uncertainty}
\end{align}
    
\marginnote{where: $T_s=T_{s,in}-T_{s,out} \rightarrow \Delta T_s= \Delta
(T_{s,in} - T_{s,out})$. If they are the same sensor and were calibrated at the
same time using the same calibration standard}

Calculating the partial derivatives, the expression is obtained:

\begin{align}
    STEC \pm \Delta STEC \: \left[ \mathrm{\frac{kWh_{th}}{m^{3}}} \right]
    &= \frac{\dot{m}_{s} c_p T_{s}}{\dot{m}_{d}} 
    \pm \frac{c_{p}}{\dot{m}_{d}} \nonumber \\ 
    &\quad \times 
    \Bigl[ 
    (T_{s}\, \Delta \dot{m}_{s})^{2}
    + (\dot{m}_{s}\, \Delta T_{s})^{2} 
    + (\dot{m}_{s} T_{s} \dot{m}_{d}^{-1} \Delta \dot{m}_{d})^{2}
    \Bigr]^{1/2}
    \labeq{STEC_complete}
\end{align}


% \begin{multicols}{2}
% \begin{align}
% V_{\V{DABC}}
% &=\frac{1}{6}\, \V{DA} \cdot \V{DB} \cdot \V{DC} \cdot \nonumber  \\
% & \sqrt{\splitfrac{
% 1+2\cos \widehat{\V{ADB}} \cdot \cos \widehat{\V{BDC}} \cdot \cos \widehat{\V{ADC}}}{
% -\cos^2 \widehat{\V{ADB}} -\cos^2 \widehat{\V{ADC}}  -\cos^2 \widehat{\V{BDC}}}\,} 
% \end{align}
% \end{multicols}

% \begin{multicols}{2}
% \begin{align}
%     F(x) &= \left(\frac{1}{2\pi}\int_{-\infty}^{\infty}e^{-j\omega x}\,d\omega \right. \nonumber \\
%     &= \frac{1}{2\pi}\int_{-\infty}^{\infty}\cos(\omega x) - j\sin(\omega x)\,d\omega \nonumber \\
%     &= \left.\frac{1}{2\pi}\int_{-\infty}^{\infty}\left(\cos(\omega x) - j\sin(\omega x)\right)\,d\omega \right)
% \end{align}
% \end{multicols}

\subsection{Specific Electrical Energy Consumption (SEEC)}    

\begin{align}
    SEEC &\pm \Delta SEEC \: \left[ \mathrm{\frac{kWh_{e}}{m^{3}}} \right] = \nonumber \\
    &\quad \frac{\sum\limits_{i=1}^{N} E_{i}}{\dot{m}_{d}} 
    \pm
    \Biggl[
    \Biggl( \frac{\sum\limits_{i=1}^{N} \Delta E_{i}}{\dot{m}_{d}} \Biggr)^{2}
    +
    \Biggl( \frac{\sum\limits_{i=1}^{N} E_{i}}{\dot{m}_{d}^{2}} \, \Delta \dot{m}_{d} \Biggr)^{2}
    \Biggr]^{1/2}
    \labeq{SEEC_complete}
\end{align}

\subsection{Performance Ratio}
% %%
% \begin{equation}
% \Delta PR = \frac{\delta PR}{\delta \dot{m}_{d}}\Delta \dot{m}_{d} +
% \frac{\delta PR}{\dot{m}_{s}}\Delta\dot{m}_{s} + \frac{\delta PR}{\delta T_{s}}\Delta T_{s}
% \end{equation}
% %%

% \begin{multicols}{2}
% \begin{align}
% PR \pm \Delta PR &= \frac{\dot{m}_{d}}{\dot{m}_{s}·T_s}·\frac{\Delta h_{ref}}{c_{p}} \pm \nonumber \\
% & \frac{\Delta h_{ref}}{c_{p}}\frac{1}{\dot{m}_{s}·T_s} \left( \Delta \dot{m}_{d}^2 + \left( \frac{\dot{m}_d}{\dot{m}_s} \Delta \dot{m}_s \right)^2 + \left( \frac{\dot{m}_d}{T_s} \Delta T_s \right)^2 \right)^{1/2}
% \end{align}
% \end{multicols}

% Updated to use enthalpy insetad of specific heat
\begin{align}
    PR \pm \Delta PR &= \frac{\dot{m}_{d} \, \Delta h_{ref}}{\dot{m}_{s} \, \Delta h_{s}} \pm \nonumber \\
    &\quad
    \frac{\Delta h_{ref}}{\dot{m}_{s} \, \Delta h_{s}}
    \Biggl[
    (\Delta \dot{m}_{d})^{2}
    +
    \dot{m}_{d}^{2}
    \Biggl(
    \left( \frac{\Delta \dot{m}_{s}}{\dot{m}_{s}} \right)^{2}
    +
    \left( \frac{\Delta (\Delta h_{s})}{\Delta h_{s}} \right)^{2}
    \Biggr)
    \Biggr]^{1/2}
    \labeq{PR_complete}
\end{align}


\subsection{Waste heat performance ratio}

\begin{align}
    PR_{WH} \pm \Delta PR_{WH} &= \frac{\dot{m}_{d}}{\dot{m}_{s} \, T} \, \frac{\Delta h_{ref}}{c_{p}} 
    \pm 
    \frac{\Delta h_{ref}}{c_{p}} \, \frac{1}{\dot{m}_{s} \, T} \nonumber \\
    &\quad
    \times
    \Biggl[
    (\Delta \dot{m}_{d})^{2}
    +
    \dot{m}_{d}^{2}
    \Biggl(
    \frac{\Delta \dot{m}_{s}}{\dot{m}_{s}}
    +
    \frac{\Delta T}{T}
    \Biggr)^{2}
    \Biggr]^{1/2}
    \labeq{PR_WH_complete}
\end{align}
\marginnote{where: $T=T_{s,in}-T_{c,in} \rightarrow \Delta T= \Delta T_{s,in} + \Delta T_{c,in}$}

% \begin{multicols}{2}
% \begin{align}
% PR_{WH} \pm \Delta PR_{WH} &= \frac{\dot{m}_{d}}{\dot{m}_{s}·T}·\frac{\Delta h_{ref}}{c_{p}} \pm \frac{\Delta h_{ref}}{c_{p}}\frac{1}{\dot{m}_{s}·T} \nonumber \\
% & \left( \Delta \dot{m}_{d}^2 + \left(\frac{\dot{m}_d}{\dot{m}_s} \Delta \dot{m}_s\right)^2 + \left(\frac{\dot{m}_d}{T} \Delta T\right)^2 \right)^{1/2}
% \end{align}
% \end{multicols}


\subsection{Recovery ratio}

\begin{equation}
RR \pm \Delta RR = \frac{\dot{m}_{d}}{\dot{m}_{f}} \pm \left( \left(\frac{1}{\dot{m}_{f}}\Delta\dot{m}_{d}\right)^2 + \left(\frac{\dot{m}_{d}}{\dot{m}_{f}^2}\Delta\dot{m}_{f}\right)^2 \right)^{1/2}
\end{equation}    

\subsection{Concentration factor}

\begin{equation}
    CF \pm \Delta CF = \frac{\dot{m}_f}{\dot{m}_b} \pm \frac{1}{\dot{m}_b} \left( \left(1+\frac{\dot{m}_f}{\dot{m}_b}\right)^2·\Delta \dot{m}_f^2 + \left( \dot{m}_f·\Delta \dot{m}_b \right)^2 \right)^{1/2}
\end{equation}

\marginnote{where: $\dot{m}_b=\dot{m}_f-\dot{m}_d \rightarrow \Delta \dot{m}_b= \Delta \dot{m}_f+\Delta \dot{m}_d$}

% \appendix

\section{Exergy calculations}
\labsec{appendix:exergy} 

Exergy consists on two components: thermomechanical exergy and chemical exergy.
When performing an exergetic analysis, the balances of the exergy flows of
interest are calculated given the control volume presented in
Fig.~\ref{fig:control_volume}. Any external stream entering the control volume
is considered an exergy input ($\dot{E}x_{in}$), while any stream leaving it is
considered an exergy output ($\dot{E}x_{out}$). A general expression to
determine the specific exergy flow ($\dot{e}_x$) of a stream is given in
\refeq{exergy}, where the first two summands represent the thermomechanical
component and the last the chemical component.

% Expresión general de exergía
\begin{equation}
    \dot{e}_x=(h-h^*)-T_0(s-s^*)+\sum\limits_{i=1}^{n}w_i(\mu_i^*-\mu_i^0).
    \labeq{exergy}
\end{equation}

In \refeq{exergy}, the variables $h$, $s$, $\mu$, and $w$ represent the
specific enthalpy, specific entropy, chemical potential, and mass fraction,
respectively. The properties denoted with an asterisk in the equation are
calculated at the restricted dead state conditions (when the temperature and
pressure of the system change to match the temperature and pressure of the
environment). On the other hand, properties labeled with a superscript of ``0"
are determined at the global dead state (when the concentration is also changed
to match that of the environment). The subscript $i$ represents a species (NaCl,
H$_2$O and others if considered). Notice that the chemical exergy component in
this work has been calculated by two approaches: empirical correlations (i) and
modelling seawater as an electrolyte for a solution of NaCl with the same
concentration as the feedwater salinity (ii). In the latter case, the required
activity coefficients have been determined by Pitzer equations~\sidecite{pitzer_thermodynamics_1973}. 

Finally, in order to calculate the specific exergy flows, libraries in MATLAB
~\sidecite{sharqawy_thermophysical_2010,nayar_thermophysical_2016} and Python
~\sidecite{romera_jjgomera_2021a} are available. However, they are limited to 120
kg/kg of concentration. For higher values, the approach used is the modelling of
seawater as an electrolyte and in this case the chemical exergy flows are
determined by the activity coefficients using a free and open source tool~\sidecite{marcellos_pyequion_2021}.


% Finally, in order to calculate the specific exergy flows, it should be mentioned that a library for the determination of the thermophysical properties of seawater is available at \href{http://web.mit.edu/seawater/}{MIT webpage}, which is based on the work shown in \cite{sharqawy_thermophysical_2010} and \cite{nayar_thermophysical_2016}. It includes functions to calculate the the specific exergy flows and it is available in MATLAB, Engineering Equation Solver and Excel. There is also an implementation of the IAPWS R13-08 seawater properties standard \cite{IAPWSR13-08_2008} in Python, which includes the calculation of the specific enthalpy, entropy and chemical potential of seawater \cite{romera_jjgomeraiapws_2021}. The chemical exergy calculations in the above mentioned libraries are limited to 120 kg/kg of concentration. For higher values of concentration, the approach used is the modelling of seawater as an electrolyte and the chemical exergy flows are determined using the activity coefficients \cite{fitzsimons_exergy_2015,thiel_energy_2015,chung_thermodynamic_2017}. A free and open source implementation for obtaining such coefficients is available in \cite{marcellos_pyequion_2021} for the Python programming language.


%\textbf{Modelling of seawater} has been a topic of discussion in the scientific literature \cite{sharqawy_formulation_2010,mistry_effect_2012} in terms of the results obtained by the different models (if they are similar or not) and which one is the most appropriate to use. Fitzsimons et al \cite{fitzsimons_exergy_2015} performed a thorough comparison of the various approaches:

%\begin{enumerate}[(i)]
    %\item Water as an ideal mixture. Not recommended since seawater is not an ideal mixture and results in significant errors.
    %\item Thermodynamic properties of seawater using empirical correlations. Specific solution to the problem with good results. However, limited in concentration.
    %\item Water as an electrolyte. Gives good results and does not have the concentration limitations. In order to calculate the required activity coefficients, Pitzer equations are recommended \cite{pitzer_thermodynamics_1973}. This approached has been used in \cite{thiel_energy_2015} and \cite{chung_thermodynamic_2017} to successfully model seawater with higher concentrations up to saturation.
%\end{enumerate}

%\begin{figure}
    %\centering
    %\includegraphics[width=.75\textwidth]{fig/exergetic_calculation_comparison.png}
    %\caption{Comparison of two methods to obtain exergetic metrics: seawater physical properties and water as an electrolyte}
    %\label{fig:exergy_chemical_comparison}
%\end{figure}

%In this work the chemical exergy component was calculated by two approaches: empirical correlations (ii) and modelling seawater as an electrolyte for a solution of NaCl with the same concentration as the feedwater salinity (iii). A comparison of both approaches is shown in Fig.~\ref{fig:exergy_chemical_comparison} when used to determine the Second Law efficiency, it can be seen that very similar results are obtained. \\

\textbf{Least and minimum least work of separation}. To determine how efficient
a desalination plant is at separating fresh water from seawater, it is compared
to the thermodynamic minimum. This is the least work required to accomplish the
separation and is only achievable with an ideal reversible separator (without
entropy generation). It has been analyzed and presented in different ways in the
literature~\sidecite{spiegler_energetics_2001,sharqawy_exergy_2011,thiel_energy_2015,lienhard_thermodynamics_2017}.
A general expression is shown in \refeq{Wleast} in terms of the Gibbs free
energy ($g$).

\begin{equation}
    \labeq{Wleast}
    \dot{W}_{least} = \dot{m}_d · g_d + \dot{m}_b · g_b - \dot{m}_{f} · g_{f}.
\end{equation}

If it is normalized to the distillate production and the flows expressed in
terms of the recovery ratio according to \refeq{RR}, the expression
becomes:

\begin{equation}
    \labeq{Wleast_rr}
    \frac{\dot{W}_{least}}{\dot{m}_d} = g_d + \frac{1-RR}{RR} · g_b - \frac{1}{RR} · g_f.
\end{equation}

As can be seen in \refeq{Wleast_rr}, the least work of separation depends
on how much pure water is extracted per unit of feed (RR), and as proven
in~\sidecite{mistry_entropy_2011}, the higher the RR, the higher the least
energy required to produce the separation. In this context, the minimum least
work of separation ($W_{least}^{min}$) is determined when RR $\rightarrow$ 0. 
% \\ When the pure water is the desired product (traditional desalination), it is more
% appropriate to set the reference ideal system to  while~\cite{thielEnergyConsumptionDesalinating2015}:

%In the particular case of this paper, the mentioned open source library \cite{marcellos_pyequion_2021} was used to obtain the activity coefficients for a solution of NaCl with the same concentration as the feedwater salinity

\section{Separation metrics calculation}\labsec{appendix:rr_max}

The molality of sodium chloride at saturation (see \refeq{rr_max}) is
determined using the following correlation that was established
by~\sidecite{pinho_solubility_2005} in terms of mass fraction and it is valid
for a temperature range between 25 and 80~$^\circ$C:

$$ w_{NaCl,sat}= a + b·T + c·T^2 + d·T^3 \: \left[ \mathrm{100g_{NaCl}/g_w} \right] $$
\marginnote{where:
\begin{itemize}
    \item $a=5.671·10^1$
    \item $b=-2.713·10^{-1}$
    \item $c=7.598·10^{-4}$
    \item $d=-6.373·10^{-7}$
\end{itemize}
}

Likewise, the following conversion formula between mass fraction and molality
can be used:

$$ b_{NaCl,sat} = \frac{ w_{NaCl,sat}/100 }{ M_{NaCl}\left(1-w_{NaCl,sat}/100\right) } \: \left[ \mathrm{ mol_{NaCl}/g_w} \right] $$

Where $M_{NaCl}$ is the molecular weight of NaCl in g/mol. 


\newpage

\section{Control system and steady state identification parameters}\labsec{appendix:params}

This appendix section provides reference tables outlining parameters used in the
algorithms discussed in this document. \reftab{solarmed:std:params_control}
summarizes the parameter values for the \gls{pidLabel}-based process control,
while \reftab{solarmed:std:params_ssi} details those for steady-state detection.
For the first, $K_p$, $K_i$, and $K_d$ are the proportional, integral, and
derivative gains, respectively (see \nrefsec{intro:control:pid}). In the latter,
$\gamma_a$ represents the wavelet transform threshold, $\gamma_d$ the derivative
threshold and finally $T_{ss}$ the time window duration. The algorithm they are
used in is described in \nrefsec{solarmed:std:monitoring}. In both tables, $T_s$
represents the sample time.

% tables/table_params_ssi.tgn
\begin{table}[t]
\centering
\caption{Parameters for the steady-state detection algorithm, where
\textit{s.u.} represents that the parameter has the same units to the related
variable}
\labtab{solarmed:std:params_ssi}
\resizebox{.5\textwidth}{!}{%
\begin{tabular}{@{}clccccccccc@{}}
\toprule
\multirow{2}{*}{\textbf{Parameter}} &  & \multicolumn{9}{c}{\textbf{Variable}} \\ \cmidrule(l){3-11} 
 &  & $P_{v,c}$ &  & $P_{v,1}$ &  & $\dot{m}_d$ &  & $\dot{m}_s$ &  & $\dot{m}_f$ \\ \cmidrule(r){1-1} \cmidrule(lr){3-3} \cmidrule(lr){5-5} \cmidrule(lr){7-7} \cmidrule(lr){9-9} \cmidrule(l){11-11} 
$\gamma_a$ [v.u.] &  & 0.05 &  & 0.05 &  & 0.1 &  & 0.3 &  & 0.2 \\
$\gamma_d$ [v.u./s] &  & 0.002 &  & 0.03 &  & 0.001 &  & 0.02 &  & 0.001 \\
Ts [s] &  & \multicolumn{9}{c}{1} \\
Tss [s] &  & \multicolumn{9}{c}{600} \\ \bottomrule
\end{tabular}%
}
\end{table}

% tables/table_params_control.tgn
\begin{table}[t]
\centering
\caption{Parameters for the \gls{pidLabel} based process control, where \textit{i.u.} represents the input variable units, and \textit{o.u.} the output units.}
\labtab{solarmed:std:params_control}
\resizebox{\textwidth}{!}{%
\begin{tabular}{@{}clccccccccclc@{}}
\toprule
\multirow{2}{*}{\textbf{Parameter}} &  & \multicolumn{11}{c}{\textbf{Subsystem}} \\ \cmidrule(l){3-13} 
 &  & Brine level &  & Distillate level &  & \begin{tabular}[c]{@{}c@{}}Condenser outlet \\ temperature\end{tabular} &  & \begin{tabular}[c]{@{}c@{}}Heat source \\ temperature\end{tabular} &  & \begin{tabular}[c]{@{}c@{}}Heat source\\ flow\end{tabular} &  & Feedwater flow \\ \cmidrule(r){1-1} \cmidrule(lr){3-3} \cmidrule(lr){5-5} \cmidrule(lr){7-7} \cmidrule(lr){9-9} \cmidrule(lr){11-11} \cmidrule(l){13-13} 
Kp [i.u./o.u.] &  & -0.01 &  & -0.05 &  & -1.7526 &  & 1 &  & 5 &  & 4 \\
Ki[i.u./(o.u.·s)] &  & -0.02 &  & -0.005 &  & -0.0322 &  & 0.2 &  & 1 &  & 1 \\
Kd [i.u./(o.u./s)] &  & 0 &  & 0 &  & 0 &  & 0.5 &  & 0.8 &  & 0 \\
Ts [s] &  & 5 &  & 3 &  & 5 &  & 2 &  & 1 &  & 1 \\
Configuration &  & \multicolumn{9}{c}{Parallel configuration} &  & \multicolumn{1}{l}{} \\ \bottomrule
\end{tabular}%
}
\end{table}




%===================================
%===================================
\chapter{\gls{medLabel} First-Principles Model}
\labch{appendix:med-model}

% \section{Multi-effect distillation first-principles model}
\tldrbox{ A first-principles model of a \gls{medLabel} plant is presented in
    this appendix. It is based on thermodynamic equations and mass and energy
    balances and can be used in two modes depending on the application. }

This model simulates thermal and mass transfer processes in a
\gls{medLabel} plant, such as the one at \gls{psaLabel}. The MED
process consists of a series of effects (evaporators) and preheaters connected
in sequence. In each effect, seawater partially evaporates under decreasing
pressure and temperature conditions, while in the preheaters, the feed water is
gradually warmed using the condensation heat from the vapor produced in the
effects.

The model is based on several assumptions to simplify the calculations:
\begin{itemize}
    \item Steady-state operation.
    \item Negligible heat losses to the environment.
    \item Isothermal physical properties have been considered for all cases.
\end{itemize}
    
And is based on several works found in the
literature~\sidecite{palenzuela_steady_2014,mistry_improved_2013,el-dessouky_steadystate_1998}
but extends them by including more detailed calculations for the different heat
transfer modes (boiling, flashing), considering not-constant \gls{neaLabel} and
\gls{bpeLabel} effects, and considering the flashing process of the distillate.
It works both at nominal and partial load conditions.

To solve the model, an iterative process is followed where the model proceeds
effect by effect, starting from the first stage. For each effect, it uses
nonlinear solvers to solve a system of non linear equations (see
Figures~\ref{ap:med-model:effect}--\ref{ap:med-model:preheater}) ensuring
consistent heat exchange and energy balances. The preheaters are solved in a
similar manner. Throughout the process, mass and energy conservation are
verified. After completing a cell, the model updates the inlet conditions for
the next effect taking into consideration the distillate vapor lost in the
preheater-effect distribution line, temperature losses and more importantly, the
plant's condensate distribution layout (see \reffig{ap:med-model:distribution}).

The model can be used in two modes depending on the application:
\textit{calibration mode} and \textit{simulation mode}. Both modes share the
same equations and structure, but differ in the inputs required and the
parameters used.

\section{Nomenclature}

\annotation{Nomenclature inconsistency}{Flows (either mass or volumetric) are
represented with a capital $M$, different to the rest of the manuscript where
lowercase $\dot{m}$ is used for mass flows or $q$ for volumetric flows.}


% Nomenclatura general
\begin{figure}[h!]
    \includegraphics[]{med-modelling-general.png}
    \caption{Overall schematic of the \gls{medLabel} model with inputs, outputs,
    main variables and components}
    \labfig{ap:med-model:general}
\end{figure}

\textbf{Plant parameters:}
\begin{itemize}
    \item Number of effects and preheaters ($N_{ef}$, $N$)
    \item Mixer distribution ratios.  ($\mathbf{Y} \in \mathbb{R}^3$)
    \item Effect and preheater areas ($\mathbf{A}_{ef} \in \mathbb{R}^N$, $\mathbf{A}_{ph} \in \mathbb{R}^N$) [m$^2$]
    \item Condenser area ($A_c$) [m$^2$]
    \item $m$, $m_2$: indices for effects without/with mixer distillate [\,]\\
\end{itemize}

\textbf{Model parameters:}
\begin{itemize}
    \item Overall effect heat transfer coefficients ($\mathbf{U}_{ef} \in \mathbb{R}^N$) [kW/m$^2$K]
    \item Overall preheater heat transfer coefficients ($\mathbf{U}_{ph} \in \mathbb{R}^N$) [kW/m$^2$K]
    \item Preheater-effect distribution line vapor-loss factor due to condensation ($\mathbf{\Delta M}_v \in \mathbb{R}^2$) [\,]\\
\end{itemize}


\textbf{Flows [kg/s]:}
\begin{itemize}
    \item $M_s$: Heat source
    \item $M_cw$: Cooling water
    \item $M_f$: Feedwater
    \item $M_{prod}$: total distillate production
    \item $M_{brine}$: total brine discharge
    \item $M_{v,in}$, $M_{v,out}$: vapor mass flow into / out of effect
    \item $M_{b,in}$, $M_{b,out}$: brine mass flow into / out of effect 
    \item $M_{gb}$, $M_{gf}$: vapor generated by boiling / flashing 
    \item $M_{dest}$: mixed distillates entering the effect ; $M_{dest,f}$: fraction that flashes 
    \item $M_{d,in}$: distillate from previous effect
    \item $M_{vh}$: distillate from previous preheater 
    \item $M_{d,out}$: distillate out of effect
    \item $M_{da}$, $M_{db}$: distillate split from mixer to effect / bypass
    \item $M_{mix,in}$: distillate in distribution line
    \item $M_{mix,out}$: out distillate from effect to distribution line
    \item $M_{bb}$: non-flashing brine\\
    % \item $M_{df}$: distillate flashed in condenser
\end{itemize}

\textbf{Temperatures [$^\circ$C]} (some omitted when equivalent to flows):
\begin{itemize}
    \item $T_{s,in}$, $T_{s,out}$: heat source in/out first effect
    \item $T_{cw,in}$, $T_{cw,out}$: cooling water in/out condenser
    \item $T_f$: preheated feedwater
    % \item $T_{v,in}$, $T_{v,out}$: vapor in/out effect
    % \item $T_{b,in}$, $T_{b,out}$: brine in/out effect
    % \item $T_{d,in}$, $T_{d,out}$: distillate in/out effect
    % \item $T_{dest}$, $T_{dest,out}$, $T_{dest,in}$: mixed distillate temperature (inside / out / in)
    \item $T_{mix,in}$, $T_{mix,out}$, $T_{mixx}$: mixer in/out and after subcooling in effect
    \item $T_{vv}$, $T_{dd}$: preheater and previous-effect distillate temperatures after subcooling
    \item $T_{ph,in}$, $T_{ph,out}$: preheater in/out
\end{itemize}

\textbf{Concentrations and pressures:}
\begin{itemize}
    \item $X_{b,in}$, $X_{b,out}$, $X_{bb}$: brine salinity in/out and non-flashing brine [g/kg]
    \item $X_f$: feedwater salinity [g/kg]
    \item $P_s$: source pressure at first effect [bar]
\end{itemize}

\textbf{Heat contributions [W]:}
\begin{itemize}
    \item $\dot{Q}_{v\,ant}$: from previous-effect vapor
    \item $\dot{Q}_{ph\,ant}$: from preheater distillate
    \item $\dot{Q}_{ef\,ant}$: from previous-effect distillate
    \item $\dot{Q}_{mix}$: from distribution-line mixer
    \item $\dot{Q}_{\delta}$: from condensed vapor along path
    \item $\dot{Q}_{dest}$: total from distillate streams
    \item $\dot{Q}_{ext\,source}$: from external heat source
\end{itemize}

\textbf{Auxiliary components and losses:}
\begin{itemize}
    \item Demister: $v_{vap}$ vapor velocity [m/s]; $h_{dem}$ height [m]; $w_{dem}$ thickness [m]; $l_{dem}$ length [m];
          $\rho_{dem}$ packing density [kg/m$^3$]; $D_{dem}$ wire diameter [mm]
    \item Piping (preheater–effect): $L_{tub}$ length [m]; $D_{tub}$ internal diameter [mm]
\end{itemize}

%================================
\subsection{Calibration mode}

\begin{marginfigure}[-8.5cm]
    \includegraphics[]{med-modelling-calibration-interface.png}
    \caption{\gls{medLabel} model \textit{calibration mode} diagram with inputs and outputs}
    \labfig{appendix:med-model:calibration-interface}
\end{marginfigure}

In the \textit{calibration mode} (see
\reffig{appendix:med-model:calibration-interface}), the model is used to obtain
different detailed parameters/outputs of interest that cannot be measured
directly, such as the heat transfer coefficients, the different heat transfer
modes contribution (boiling, flashing, etc), per effect brine concentration, per
effect distillate production, \etc. The computed parameters in this mode can
then be used to generate models for these parameters. For this purpose, the
model requires an extended set of inputs, including measured temperatures or
pressures per effect and preheater\sidenote[][*10]{Or a limited set of them, which the
rest being estimated by interpolation or other methods}.

To solve it, an initial guess of $\Delta M_{v}$ is provided, and then the model
iteratively solves for the heat transfer coefficients ($U_{ef}$ and $U_{ph}$)
and outlet conditions until the final condenser. The total distillate produced
is compared with the measured value, and $\Delta M_{v}$ is adjusted accordingly.
This process continues until the calculated distillate matches the measured
value within a specified tolerance.

This mode can be used to identify loss of performance, fouling and other issues.
Evaluated over time, it can provide trends and be integrated into predictive
maintenance strategies. It can also be used to generate data-driven models for the
different parameters, which can then be used in the \textit{simulation mode}.
Also, it provides more detailed outputs that can be used for further analysis
by the O\&M team.


\textbf{Outputs}
\begin{itemize}
    \item Model parameters ($U_{ef}$, $U_{ph}$, $\Delta M_v$)
    \item Detailed per effect and preheater outputs (temperatures, pressures,
    flows: condensate, flashed condensate, vapor produced by boiling, flashing,
    etc; concentration, heat transfer rates and more)
    \item Global outputs: Total distillate produced, brine concentration, heat
    source outlet temperature, etc.
\end{itemize}

%================================
\subsection{Simulation mode}

\begin{marginfigure}[-5cm]
    \includegraphics[]{med-modelling-simulation-interface.png}
    \caption{\gls{medLabel} model \textit{simulation mode} diagram with inputs and outputs}
    \labfig{appendix:med-model:calibration-interface}
\end{marginfigure}

In this variant, the model is used to simulate the plant behavior, but given
fewer inputs compared to the previous variant. It uses pre-trained models for
the different parameters (\ie heat transfer coefficients of the different
effects and preheaters, vapor loss). Either empirical correlations from the
literature or a data-driven model trained using the outputs from the
\textit{calibration mode} evaluated for a long enough experimental campaign can
be used.

The model uses models (either pre-trained data-driven models,
empirical correlations or physical models) for $U_{ef}$, $U_{ph}$, and $\Delta
M_v$. Thus, the iterative process to estimate $\Delta M_v$ is skipped. Also,
during the sequential calculation of cell, the pre-trained models are given the
current operating conditions and return the required parameters to solve each
effect and preheater.

This mode provices a detailed operation steady state model of an \gls{medLabel}
plant with minimum assumptions and that does not require outputs of the plant as
inputs.


\section{Implementation}

% Nomenclatura celda
\begin{figure*}[!htpb]
    \includegraphics[]{med-modelling-cell.png}
    \caption{Detailed schematic of a single cell in the \gls{medLabel}
    containing the effect or evaporator (left) and the preheater (right)}
    \labfig{ap:med-model:cell}
\end{figure*}


% Detalles efecto
\begin{figure*}[!htpb]
    \includegraphics[]{med-modelling-effect.png}
    \caption{Detailed schematic of a single cell in the \gls{medLabel} with
    the effect's equations.}
    \labfig{ap:med-model:effect}
\end{figure*}

\begin{figure*}[!htpb]
    \includegraphics[]{med-modelling-effect2.png}
    \caption{Detailed schematic of a single cell in the \gls{medLabel} with
    the energy source side equations and internal effect condensate flashing.}
    \labfig{ap:med-model:effect2}
\end{figure*}

% Detalle precalentador
\begin{figure*}[!htpb]
    \includegraphics[]{med-modelling-preheater.png}
    \caption{Detailed schematic of a single cell in the \gls{medLabel} with
    the preheater's equations.}
    \labfig{ap:med-model:preheater}
\end{figure*}

% \clearpage
% Detalle distribución
\begin{figure*}[!htpb]
    \includegraphics[]{med-modelling-distribution.png}
    \caption{Detailed schematic of a single cell in the \gls{medLabel} with
    distribution lines for the energy source side and generated steam. Also,
    auxiliary elements like demister and preheater-effect distribution line geometry.}
    \labfig{ap:med-model:distribution}
\end{figure*}

\FloatBarrier
% \begin{figure}
%     \includegraphics[width=\textwidth]{med-fp-model-calibration-mode-validation.png}
%     \caption{\gls{medLabel} first-principles model \textit{calibration mode} validation}
%     \labfig{ap:med-model:validation:calibration}
% \end{figure}


% \begin{figure}
%     \includegraphics[width=\textwidth]{med-fp-model-simulation-mode-validation.png}
%     \caption{\gls{medLabel} first-principles model \textit{simulation mode} validation}
%     \labfig{ap:med-model:validation:simulation}
% \end{figure}

\section{Validation}

Using the same dataset presented in \nrefsec{solarmed:modelling:med} --
Palenzuela~\etal~\sidecite{palenzuela_experimental_2016},
the \textit{calibration mode} of the model is evaluated, with the results shown in
\reffig{ap:med-model:validation}~(a). Then it was divided in training
and validation set and the training set was used to generate data-driven models
for the different parameters required in the \textit{simulation mode}. The
\textit{simulation mode} was then evaluated using the validation set and the
results are presented in \reffig{ap:med-model:validation}~(b).

\begin{figure}[!b]
    \includegraphics[width=\textwidth]{med-fp-model-heat-transfer-coeff-comp.png}
    \caption{Heat transfer coefficients comparison between \textit{calibration
    mode} and \textit{simulation mode} for the validation set}
    \labfig{ap:med-model:heat-transfer-coeffs-comp}
\end{figure}

The results demonstrate that both variants of the model can accurately predict
the output variables such as the plant's distillate production. As expected, the
calibration model performs better since it uses more detailed inputs, but the
simulation model also shows good accuracy. Finally, a comparison of both modes
obtained heat transfer coefficients is shown in
\reffig{ap:med-model:heat-transfer-coeffs-comp}. They show similar trends
except for part of the experimental dataset where higher discrepancies are
observed for the latter effects (effects 11--14). This can probably be explained
by the fact that the sequential calculation of the model coupled to its high
non-linearity means that error is accumulated, where small deviations in the
first effects can lead to larger errors in the latter ones. Nonetheless, both
models overall provide similar results and trends.


\begin{figure*}[]
    \centering
    \subfloat[\centering  \textit{Calibration mode}]{{\includegraphics[width=0.85\linewidth]{med-fp-model-calibration-mode-validation.png}}}%
    % \hspace{0.01\linewidth}
    \vspace{1ex}
    \subfloat[\centering \textit{Simulation mode}]{{\includegraphics[width=0.85\linewidth]{med-fp-model-simulation-mode-validation.png}}}%
    \caption[]{\gls{medLabel} first-principles model validation}
    \labfig{ap:med-model:validation}
\end{figure*}

% \reffig{solarmed:tests-calendar} shows the operation history of the plant (starting from 2009)


% \begin{marginfigure}[-5.5cm]
%     \includegraphics[]{med_test_days.png}
%     \savebox\captionqr{\qrcode[hyperlink,height=0.4in]{\repositoryBaseUrl/figures/med_test_days.html}}
%     \caption[]{Operation history of the pilot plant.\hspace{1ex}\usebox\captionqr}
%     \labfig{solarmed:tests-calendar}
% \end{marginfigure}