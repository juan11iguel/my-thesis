\pagelayout{wide} % No margins
\addpart{Conclusions and outlook}
\pagelayout{margin} % Restore margins

\section*{Conclusions}
\addcontentsline{toc}{section}{Conclusions}


% Water in CSP
% - almacenamiento y operación nocturna
% - uso del agua cuando esté disponible
% - potencial de optimización
% - hybrid/combined cooling como solución potencial para mayoría de nuevas
%   plantas CSP
% - 





\section*{Outlook and future work}
\addcontentsline{toc}{section}{Outlook and future work}


%================================
\subsection*{Optimal water and electricity management in a combined cooling system}

\textbf{Improved Pareto front computation}. In the current optimization
    implementation, the Pareto front for each step in the optimization horizon
    is constructed using a grid search over the decision space. This approach
    can become computationally expensive, especially as the grid resolution
    increases. Additionally, the Pareto front must be recalculated from scratch
    at every step, even though the sequential steps are often very similar—cost
    parameters remain constant, and only the thermal load and weather
    conditions change, typically with small variations. A more efficient
    solution would be to use a multi-objective optimization algorithm such as
    NSGA-II~\sidecite{}, which can transfer evolved populations between
    successive evaluations, significantly reducing redundant computations.

\textbf{Better water management} In the current implementation, the primary
    water source is distributed evenly each day, so the optimization process
    uses up the entire supply daily. However, a more intelligent daily
    distribution—essentially, a new optimization problem—could improve water
    management by allocating different amounts on different days, based on
    expected weather conditions and predicted generation. This approach would
    likely be incorporated as a new layer in the hierarchical control
    structure.\sidenote{ The resulting structure would be: 1.~Water allocation,
    2.~ \gls{ccsLabel} \hyperref[sec:cc:optimization]{operation optimization},
    3.~\gls{ccLabel} regulatory control.}

\textbf{Techno-economic analysis.} The presented cooling alternatives
comparative in this thesis focus on the operation cost of the system, but to
get a better picture of the alternatives performance, a techno-economic
analysis that includes the capital cost of the system and the
expected lifetime of the components should be performed \ie considering all
costs associated with the system the plant's lifetime. This is currently being
worked on as part of \cite{SOLHycool}, where the methodology presented here in
terms of operation costs will be integrated in a techno-economic analysis for
different case studies.



%================================
\subsection*{Energy management in MED processes driven by variable energy sources}

\textbf{Alternative configurations for an MED brine concentrator}.
Configuraciones alternativas para procesos MED para aplicaciones de
concentración de salmueras: geometría variable de efectos, fuentes externas en
efectos distintos al primero, acoplamiento con MSF para efectos posteriores.

\textbf{Alternative configurations for solar-driven MED}. Configuraciones
alternativas para el proceso solar MED (almacenamiento con distintos puntos de
carga y descarga, MED con distintos puntos de fuente externa, etc. Incluir
diagrama de draw.io con las distintas configuraciones)

The layout configuration of the facility focused on realibility and simplifying operation and maintenance, not strictly on thermodynamic efficiency. The efficiency of the system could be improved:

\begin{enumerate}
    \item if direct coupling between solar field and thermal storage was used, avoiding the heat exchanger energy transfer associated losses

    \item thermal storage allowed charge and discharge from different levels, in order to take advantage of the temperature stratification and avoid fluid mixing

    \item 

    \item 
\end{enumerate}

These decisions were made to, on the one hand allow to separate the solar field and thermal storage into two distinct decoupled circuits, providing flexibility, reducing the volume of additives required (only added to the solar field circuit), and operational flexibility (other external loads can be connected to the solar field when the MED is not being operated).

In conclusion this system, although improvable, allows to validate the feasibility of the proposed approach by means of the implementation of a suitable control system, in such a way, that the ideas and techniques presented in this work, could be directly extrapolated to a commercial system just by modifying some of the decision variables to suit the particular implementation. 

\section*{Derived scientific contributions}
\addcontentsline{toc}{section}{Derived scientific contributions}

% \begingroup % Local scope for the following commands

% % Define the style for the TOC, LOF, and LOT
% %\setstretch{1} % Uncomment to modify line spacing in the ToC
% %\hypersetup{linkcolor=blue} % Uncomment to set the colour of links in the ToC
% \setlength{\textheight}{230\hscale} % Manually adjust the height of the ToC pages

% % Turn on compatibility mode for the etoc package
% \etocstandarddisplaystyle % "toc display" as if etoc was not loaded
% \etocstandardlines % "toc lines as if etoc was not loaded

% % TODO: Add \listofcontributions

% \endgroup

\begin{enumerate}
	\item Publicaciones en revista
	\item Contribuciones a congreso
	\item Coloquios doctorales
	\item Colaboraciones en proyectos de investigación
	\item Estancias de investigación
	\item Repositorios de código
	\item Repositorios de datos
	\item Herramientas interactivas
	\item Contribuciones a librerías de código abierto?
\end{enumerate}