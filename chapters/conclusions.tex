\pagelayout{wide} % No margins
\myepigraphhead[500]{Open source is like academia is meant
to be. You're paid a salary but your research
contributions are distributed to the public, freely.\\//\\El código abierto es
como el mundo académico debería de ser. Se te paga un salario, pero el fruto de
tu investigación es distribuido a la sociedad, de manera libre.}{}
\addpart{Conclusions and outlook}
\pagelayout{margin} % Restore margins


\section*{Conclusions}
\addcontentsline{toc}{section}{Conclusions}
\labsec{conclusions:conclusions}

%================================
\subsection*{Optimal water and electricity management in a combined cooling system}

This work has addressed the challenge of reducing water consumption in
\gls{cspLabel} plants through the development and assessment of a hybrid
combined cooling concept that integrates dry and wet technologies. Starting from
the broader context of concentrated solar thermal power and its dependence on
water resources, the thesis has motivated the need for more water-efficient
cooling solutions, particularly in water-scarce regions where \gls{cspLabel}
deployment is most attractive.

A detailed steady-state modelling framework was developed for the main
components of the combined cooler. Two complementary modelling strategies were
employed: (i) physics-based, first-principles models with broad applicability,
and (ii) several data-driven, black-box approaches, including artificial neural
networks, Random Forest, Gradient Boosting, and \gls{gprLabel}.

To reconcile the accuracy and generality of physical models with the
computational efficiency of data-driven surrogates, Gaussian Process Regression
models were trained on synthetic data generated by the first-principles models.
This hybrid strategy yields fast, on-demand surrogate models that can be adapted
to different plant scales while preserving high predictive accuracy. Validation
against 24 experimental tests covering a wide operating envelope demonstrated
mean absolute errors below 0.97~$^\circ$C for key temperatures and 19.4~l/h for
water consumption, confirming that the integrated model can reliably reproduce
the behavior of the combined cooling system under diverse operating and ambient
conditions. Given the obtained accuracy, the significant reductions in
computational time and experimental data requirements ---the same amount as the
physical model--- this approach has proven to be a robust and practical
alternative to purely physics-based simulations. 

Building on this modelling framework, a two-stage multi-objective optimization
strategy was proposed to minimize the daily cost of cooling ---comprising both
electricity and water costs--- under limited water availability, while
guaranteeing the required cooling duty. At each time step of a prediction
horizon, a multi-objective optimization problem is solved to obtain a Pareto
front between cost and water use. A second-stage global optimization problem
then selects an optimal path through the sequence of Pareto fronts, effectively
planning an operation schedule over the horizon. This is the first time that
an advanced optimization framework has been applied to the operation of 
combined cooling systems in concentrated solar power plants.

Experimental tests confirmed that the optimized operation strategy consistently
match the measured behavior of the pilot plant, demonstrating its effectiveness
in managing cooling resources under realistic operating conditions. The results
demonstrated that: 

\begin{itemize}
	\item The dry cooler becomes highly sensitive to ambient temperature when
	operating near its limits, whereas the wet cooler reliably compensates in
	such cases when operating in the often used series configuration.
	\item The optimization framework also showed a clear tendency to prioritize
	water savings, favoring dry-only operation or series operation of the coolers
	whenever feasible.
	\item Importantly, the optimized strategies remained valid over extended
	periods, even through planned changes in thermal load, and could therefore
	support robust operation.
	\item The close agreement between predicted and measured variables further
	indicates that upper-layer forecasts can be safely exploited in low-level
	control (\eg~in feedforward actions).
\end{itemize}

Finally, the methodology was applied to a case study of a commercial
50~MW$_\mathrm{e}$ \gls{cspLabel} plant with 8~h of storage (Andasol-II,
southern Spain). Three cooling configurations were compared under a
water-scarcity scenario using annual simulations: the existing wet-only
\gls{wctLabel} system, and the proposed \gls{ccLabel} concept with two
dry-cooler capacities (75~\% and 100~\% of the nominal wet-system load). For
each configuration, operation was optimized using the proposed multi-stage
framework. The combined cooler alternatives reduced specific cooling costs by up
to 80~\% and decreased annual water consumption by about 48~\%, with 38~\%
savings during the driest and hottest months. These improvements were mainly due
to the fact that optimal solutions are less reliant on costly alternative water
sources, which can dominate cooling costs. The \gls{ccLabel} options also
exhibited much more stable and predictable costs throughout the year compared to
the wet-only reference, which proved highly sensitive to water availability.

Overall, the analysis underscores the importance of evaluating cooling concepts
over representative annual conditions rather than relying on aggregated
averages, and confirms that water availability can become the dominant
constraint and cost driver, especially where alternative water sources are
costly. Although no optimization strategy can fully resolve the intrinsic
mismatch between peak cooling demand and water scarcity in hot-dry seasons, the
results demonstrate significant remaining potential for improved water
management through optimized operation of hybrid cooling systems. The proposed
modelling and optimization framework is generic and can be adapted to other
plant designs, locations, and resource scenarios, thereby supporting informed
design and operational decisions for more sustainable \gls{cspLabel} power
plants.


%================================
\subsection*{Energy management in MED processes driven by variable energy sources}

The second part of this thesis concerns the integration, evaluation, and optimal
operation of low-temperature \gls{medLabel} systems powered by variable energy
sources, with a particular focus on solar-thermal driven systems. As
desalination becomes increasingly important to address global freshwater
scarcity, thermal systems --- and especially low-temperature \gls{medLabel} ---
offer a robust and complementary pathway for brine concentration and resource
recovery, particularly when coupled with waste-heat or renewable energy sources.

A standardized methodology was firstly developed to evaluate \gls{medLabel}
performance under realistic and supporting highly variable operating conditions.
The method defines instrumentation requirements, key performance indicators, and
uncertainty quantification procedures, and is complemented by an automated
steady-state detection algorithm that enhances the robustness of experimental
assessments. Application of the methodology showed that reproducible performance
metrics can be obtained even when the duration of steady-state episodes differs
significantly, provided that stable operation is correctly identified. Tests at
elevated top-brine temperatures confirmed the feasibility of high-temperature
operation without significant scale formation, as corroborated with control
tests and stable heat transfer coefficients estimated from a physical model. The
results further showed that, while higher top-brine temperatures enable higher
brine concentrations, they do not necessarily improve thermal performance in the
absence of design changes (e.g.\ additional effects or modified geometry), due
to increased exergy destruction in unbalanced heat exchangers and limitations
imposed by boiling point elevation.

To reproduce system behavior under realistic solar-driven conditions, a
complete hybrid dynamic model of the \gls{solarmedLabel} installation at
\gls{psaLabel} was formulated. The model combines continuous physics-based and
data-driven sub-models for the solar field, heat exchanger, thermal storage,
three-way valve, and \gls{medLabel} plant with discrete supervisory
\glspl{fsmLabel} that model the operational logic of both the \gls{sftsLabel}
and the \gls{medfsmLabel}. Each \gls{fsmLabel} governs subsystem activation,
startup, shutdown, and transitions based on system inputs, internal rules, and
configurable parameters such as cooldown or startup durations.

The integrated hybrid model was evaluated under realistic operation
conditions---experimental data from the pilot-plant---with different prediction
horizons and showed good agreement with experimental data. For multi-hour
predictions, the model maintained \gls{mapeLabel} values below 15~\%, which is a
commendable result given the strong coupling between subsystems and the absence
of feedback within the prediction horizon. Static variables, such as distillate
production, were accurately captured and showed little sensitivity to the
prediction horizon, while dynamic variables associated with thermal storage and
solar field temperatures exhibited increasing errors with longer horizons, as
expected. Nevertheless, the final thermal states and stored energy at the end of
operation remained close to the measured values, demonstrating that the model
captures the essential dynamic behavior of the coupled system with reasonable
computational time ---in the order of single-digit seconds for multi-day
simulations.

With the model of the \gls{solarmedLabel} system, a novel hierarchical
optimization strategy was proposed to enable its autonomous and economically
optimal operation. The strategy is structured into three layers, with the upper
operation layer solving a \gls{minlpLabel} economic problem that decides when to
start or stop each subsystem and how to regulate them throughout operation,
while exploiting solar availability and thermal storage flexibility.

The proposed method was benchmarked over a week-long system simulation against
two alternative strategies: a baseline heuristic rule-based operation and a
fixed-schedule operation-only optimization solving a continuous \gls{nlpLabel}
problem. The results showed that the hierarchical \gls{minlpLabel}-based
strategy improves the economic system performance by approximately 32~\%
relative to the heuristic baseline and by 21~\% relative to the fixed-schedule
\gls{nlpLabel} strategy. These gains arise from the ability to fully exploit the
temporal flexibility provided by the thermal storage and the solar resource,
maximizing useful temperature differences. The resulting near-optimal operation
closely resembles that of a waste-heat-driven process ---focusing on maximizing
energy utilization--- highlighting that, when evaluated from the perspective of
primary energy utilization, the optimal strategy for a solar-driven
\gls{medLabel} unit differs markedly from conventional fossil-driven approaches
that focus solely on minimizing process energy consumption.

The analysis showed that achieving near-optimal performance in thermally driven
separation processes powered by variable energy sources requires the combination
of a sufficiently long optimization horizon, carefully selected decision
variables, flexible scheduling of subsystem operation, and an objective function
that accurately reflects performance relative to primary energy input. When
these elements are jointly considered, solar-thermal separator systems can be
operated in a truly optimized and autonomous manner. This can be accomplished
with the proposed optimization framework, which brings notable improvements to
the economic and energetic performance of state-of-the-art solar-driven
\gls{medLabel} systems.


% In summary, the \gls{solarmedLabel} part of this thesis has delivered: (i) a
% standardized and experimentally validated methodology for assessing
% \gls{medLabel} performance under variable conditions, (ii) a hybrid dynamic
% model capable of reproducing multi-hour system dynamics with sufficient accuracy
% for control and optimization, and (iii) a hierarchical optimization framework
% that significantly enhances the economic and energetic performance of
% solar-driven \gls{medLabel} systems state of the art. 

Together with the combined cooling contributions, these results provide a
coherent set of tools and insights for the design and operation of water- and
energy-efficient solar-thermal systems.




\section*{Outlook and future work}
\addcontentsline{toc}{section}{Outlook and future work}
\labsec{conclusions:outlook}


%================================
\subsection*{Water and electricity management in a combined cooling system}

\textbf{Improved Pareto front computation}. In the current optimization
implementation, the Pareto front for each step in the optimization horizon is
constructed using a grid search over the decision space. This approach can
become computationally expensive, especially as the grid resolution increases.
Additionally, the Pareto front must be recalculated from scratch at every step,
even though the sequential steps are often very similar; cost parameters remain
constant, and only the thermal load and weather conditions change, typically
with small variations. A more efficient solution would be to use a
multi-objective optimization algorithm, which can transfer evolved populations
between successive evaluations, significantly reducing redundant computations.

\textbf{Low-level control}. Enhance the robustness of the operation strategy by
ensuring that cooling demand is satisfied even under unexpected disturbances,
such as rapid changes in thermal load. This can be achieved by incorporating a
supervisory low-level controller that monitors the condenser pressure and,
whenever the target value is not met, increases the cooling recirculation flow
rate until the pressure returns to its setpoint. At the higher flow rate, the
cooling components naturally attempt to maintain their outlet temperatures,
thereby increasing the available cooling capacity.

\textbf{Better water management} In the current implementation, the primary
water source is distributed evenly each day, so the optimization process uses up
the entire supply daily. However, a more intelligent daily distribution
---essentially, a new optimization problem--- could improve water management by
allocating different amounts on different days, based on expected weather
conditions and predicted generation. This approach would likely be incorporated
as a new uppper layer in the hierarchical control structure.\sidenote{The
resulting structure would be: 1.~Water allocation, 2.~ \gls{ccLabel}
\hyperref[sec:cc:optimization]{operation optimization}, 3.~\gls{ccLabel}
regulatory control.}. At the higher level a simpler and more abstract model
would be considered to predict the long term behavior of the system and to
optimize it over a long time horizon, probably considering the availability and
capacity of a water reservoir.

\textbf{Analyze different combined coolers configurations and within each
configuration, different component sizes}. The cooler analyzed has a combined
dry and wet coolers which can either satisfy the nominal cooling load. Different
ratios could be analyzed and one would probably be a better fit for the
particular case study. Furthermore, the \fullgls{acheLabel} is used for the
\gls{dcLabel}, but other options could be considered and added to the
comparison, such as an \fullgls{accLabel} in parallel with a surface condenser
together with a \gls{wctLabel} or a deluged condenser.
    
This in itself is a design optimization problem that is not addressed in this
thesis. However, it is important to integrate a method like the proposed
optimization and include it in the design process to evaluate the
performance of different configurations and sizes. In the end, the decision
of the configuration and size of the cooling system should be based on a
techno-economic analysis.
    

\textbf{Techno-economic analysis.} The presented cooling alternatives
comparative in this thesis focus on the operation cost of the system, but to get
a better picture of the alternatives performance, a techno-economic analysis
that includes the capital cost of the system and the expected lifetime of the
components should be performed \ie considering all costs associated with the
system the plant's lifetime. This is currently being worked on as part of
\href{https://solhycool.psa.es/}{SOLhycool}\sidenote{\url{https://solhycool.psa.es/}},
where the methodology presented here in terms of operation costs will be
integrated in a techno-economic analysis for different case studies.



%================================
\subsection*{Energy management in \gls{medLabel} processes driven by variable
energy sources}

\textbf{Analysis in simulation of a high-\gls{tbtLabel} case study}. As
mentioned in \refch{solarmed:std}, increasing the number of effects and/or
modifying the geometry of the heat exchangers could improve the performance of
high-\gls{tbtLabel} \gls{medLabel} systems. This is not practical to test in the
experimental facility but it could be analyzed in simulation.

\textbf{Analysis and optimization in simulation of a waste-heat-driven
\gls{medLabel} case study}. It would be interesting to analyze the performance
of the proposed hierarchical optimization strategy in a waste-heat-driven
\gls{medLabel} from its integration with an industrial process. This would allow
to evaluate the performance of the strategy in a scenario with less variability
than the solar-driven case, but with different constraints and objectives, such
as minimizing the impact on the industrial process or maximizing the use of
waste heat.

\textbf{Alternative configurations for an MED brine concentrator}. Several 
non-standard \gls{medLabel} configurations could be considered for brine 
concentration applications. Examples include variable-effect geometry 
(increasing the temperature difference in effects as brine concentration 
rises to maintain balanced vapor production), the use of external heat sources 
in effects beyond the first (see \reffig{conclusions:solarmed-outlook} --
\textit{MED} -- \textit{alternative~B}), and hybridization with 
\gls{msfLabel} in later effects where \gls{bpeLabel} significantly limits the 
driving force. Depending on the application and heat-source availability, these 
configurations may offer enhanced performance or reduced cost.

\textbf{Alternative configurations for solar and/or waste heat driven MED
processes}. The current layout of the experimental \gls{solarmedLabel} facility
at \gls{psaLabel} was designed with priorities on reliability, flexibility, and
reduced operation and maintenance costs, rather than strict thermodynamic
optimization. This approach led to the separation of the solar field and thermal
storage into two independent circuits, minimizing the volume of antifreeze
additives (used only in the solar field) and increasing operational flexibility
by enabling independent pumping per solar field loop. This arrangement also
allows external loads to be connected to the solar field when the \gls{medLabel}
unit is offline.

However, the overall efficiency and cost-competitiveness of the system could be 
improved if some of these design choices were revisited with the goal of 
maximizing the performance of solar- and waste-heat-driven \gls{medLabel} 
processes, in light of the findings of this research (see 
\reffig{conclusions:solarmed-outlook}). Potential improvements include:

\begin{itemize}
	\item \texttt{SF}. By adding a small number of three-way control valves, the 
	solar field could operate in both parallel and series configurations, improving 
	its ability to adapt to variable solar input and enhancing outlet temperature 
	control. This is particularly beneficial at the beginning and end of the solar 
	day, when a series configuration would enable earlier and more sustained useful 
	heating of the thermal storage.

    \item \texttt{SF-TS}. Direct coupling of the solar field and thermal storage 
    could eliminate the energy transfer losses associated with the intermediate 
    heat exchanger.

    \item \texttt{TS}. Allowing charge and discharge at different levels of the 
    thermal storage tank would take advantage of thermal stratification while 
    avoiding fluid mixing.

    \item \texttt{TS-MED}. Directly connecting the thermal storage to the 
    \gls{medLabel} unit would reduce regulation capability but eliminate mixing 
    losses in the three-way valve.

    \item \texttt{MED}. As noted above, alternative \gls{medLabel} 
    configurations (e.g., variable geometry, multiple external heat sources at 
    different effects, or heat-source recirculation) could further improve 
    performance for brine concentration and enable better integration with solar 
    and waste-heat sources.
\end{itemize}


\begin{figure*}[h!]
	\includegraphics[width=\linewidth]{figures/solarmed-outlook.png}
	\caption{Alternative configurations for variable energy source(s) driven MED processes.}
	\labfig{conclusions:solarmed-outlook}
\end{figure*}


\newpage
\section*{Derived scientific contributions}
\addcontentsline{toc}{section}{Derived scientific contributions}
\labsec{conclusions:contributions}

% \begingroup % Local scope for the following commands

% % Define the style for the TOC, LOF, and LOT
% %\setstretch{1} % Uncomment to modify line spacing in the ToC
% %\hypersetup{linkcolor=blue} % Uncomment to set the colour of links in the ToC
% \setlength{\textheight}{230\hscale} % Manually adjust the height of the ToC pages

% % Turn on compatibility mode for the etoc package
% \etocstandarddisplaystyle % "toc display" as if etoc was not loaded
% \etocstandardlines % "toc lines as if etoc was not loaded

% % TODO: Add \listofcontributions

% \endgroup
The author has published or submitted for publication several journal articles,
contributed to conferences (national and international) and colloquiums:

\begin{kaobox}[title=Journal publications,colback=ForestGreen!15!white,colbacktitle=ForestGreen!15!white]

	\begin{itemize}
		\item J. M. Serrano, P. Navarro, J. Ruiz, P. Palenzuela, M. Lucas, and L. Roca. ``Wet Cooling Tower Performance
		Prediction in CSP Plants: A Comparison between Artificial Neural
		Networks and Poppe's Model.'' Energy, May 29, 2024, 131844.\\
		DOI: \href{https://doi.org/10.1016/j.energy.2024.131844}{https://doi.org/10.1016/j.energy.2024.131844}.
		
		\item P. Navarro, J. M. Serrano, L. Roca, P. Palenzuela, M. Lucas, and
		J. Ruiz. ``A Comparative Study on Predicting Wet Cooling Tower
		Performance in Combined Cooling Systems for Heat Rejection in CSP
		Plants.'' Applied Thermal Engineering, June 21, 2024, 123718.\\
		DOI: \href{https://doi.org/10.1016/j.applthermaleng.2024.123718}{https://doi.org/10.1016/j.applthermaleng.2024.123718}.

		\item J. M. Serrano, P. Palenzuela, J. Ruiz, P. Navarro, J. Muñoz, B. Ortega, L. Roca. ``Combined cooling for CSP plants: Modeling,
		experimental validation and optimization analysis'' Energy Conversion and Management 348 (January 2026): 120752.\\
		DOI: \href{https://doi.org/10.1016/j.enconman.2025.120752}{https://doi.org/10.1016/j.enconman.2025.120752}.
	\end{itemize}
\end{kaobox}


\begin{kaobox}[title=Contribution to conferences,colback=ForestGreen!15!white,colbacktitle=ForestGreen!15!white]

	\begin{itemize}
		\item J. M. Serrano, J. D. Gil, J. Bonilla, P. Palenzuela, and L. Roca,
		“Optimal operation of a combined cooling system” in 4th IFAC
		International Conference on Advances in Proportional-Integral-Derivative Control,
		Almería, Spain, 2024-06-12/2024-06-14.

		\item J. M. Serrano, J. D. Gil Vergel, J. Bonilla, P.
		Palenzuela, and L. Roca, “Operación óptima de un sistema de
		refrigeración combinada,” in XLIV Jornadas de Automática, Universidad
		de Zaragoza, 6, 7 y 8 de septiembre de 2023, Zaragoza, 2023rd ed., Aug.
		2023, pp. 477-482. \\DOI:
		\href{https://doi.org/10.17979/spudc.9788497498609.477}{10.17979/spudc.9788497498609.477}.

		\item P. Navarro, J. M. Serrano, J. Ruiz, M. Lucas, L. Roca, and
		P. Palenzuela. “Comparison Between an Artificial Neural
		Network and Poppe's Model for Wet Cooling Tower Performance Prediction
		in CSP Plants.” Efficiency, Cost, Optimization, Simulation and
		Environmental Impact of Energy Systems. International Conference., ECOS
		2023, June 25, 2023, 1609–20.\\DOI: \href{https://doi.org/10.52202/069564-0146}{10.52202/069564-0146}.

		\item L. Roca, J. M. Serrano, J.D. Gil, G. Zaragoza, M. Beschi, and A.
		Visioli. “Modelo de parámetros concentrados para captadores solares
		planos con reflectores.” Jornadas de Automática, no. 45 (July 2024): 45.
		\\DOI: \href{https://doi.org/10.17979/ja-cea.2024.45.10930}{10.17979/ja-cea.2024.45.10930}
	\end{itemize}
\end{kaobox}

\begin{kaobox}[title=Participation in conferences and colloquiums, colback=ForestGreen!15!white,colbacktitle=ForestGreen!15!white]

	\begin{itemize}
		\item J.M. Serrano, L. Roca, P. Palenzuela. ``Yearly Simulation of a
		Combined Cooling System Integrated into a Concentrating Solar Power
		Plant''. SolarPACES. Almería, Spain (September 2025).
		\item J.M. Serrano, P. Palenzuela and L. Roca. ``Methodology for the
		implementation of a steady state simulation model in a multi-effect
		distillation plant. Case study: PSA MED pilot plant''. Desalination for
		the Environment, Clean Water and Energy (EDS). Las Palmas de Gran
		Canaria, Spain (2022).
		\item J.M. Serrano, P. Palenzuela and L. Roca. Experimental evaluation
		of MED at high top brine temperatures with no divalent ions in feed
		water. Desalination for the Environment, Clean Water and Energy (EDS).
		Limassol, Chipre (2023).
		\item P. Palenzuela, J.M. Serrano, L. Roca, D.C Alarcón-Padilla and G.
		Zaragoza. ``Effect of increasing Top Brine Temperature on the
		performance of a solar-powered MED system''. Desalination for the
		Environment, Clean Water and Energy (EDS). Porto, Spain (2025).
		\item 3rd SFERA-III Doctoral Colloquiums:
		\begin{enumerate}
			\item Technological developments for solar multi-effect distillation
			processes. Almería, Spain (2021).
			\item Modelling and automation of a multi-effect distillation plant
			for the optimal coupling with solar energy. ETH Zurich, Switzerland (2022).
			\item Towards the optimal coupling of multi-effect distillation with
			solar energy. DLR. Cologne, Germany (2023).
		\end{enumerate}
	\end{itemize}

\end{kaobox}

As a result of the work developed in the present research work, several
repositories containing experimental datasets and open-source code have been
made publicly available. Particularly, each of the parts of the thesis has an
associated repository with the implementation of the presented results, in order
to facilitate transparency, reproducibility, and reusability of the developed
methods. Additionally, this thesis manuscript itself is also made available
together with all its associated media: 

\begin{kaobox}[title=Open datasets,colback=Violet!15!white,colbacktitle=Violet!15!white]

	\begin{itemize}
		
		\item P. Palenzuela, L. Roca, J.M. Serrano (CIEMAT-PSA). “Steady-State
		Operation Dataset of an Experimental Wet Cooling Tower Pilot Plant
		Located at Plataforma Solar de Almería”. Zenodo, June, 2024.\\
		DOI: \href{https://doi.org/10.5281/zenodo.10806201}{10.5281/zenodo.10806201}.
	
		\item P. Palenzuela, L. Roca, J.M. Serrano (CIEMAT-PSA). “Steady-State
		Operation Dataset of an Experimental Air-Cooled Heat Exchanger Located
		at Plataforma Solar de Almería”. Zenodo, December, 2025.\\
		DOI: \href{https://doi.org/10.5281/zenodo.17312369}{10.5281/zenodo.17312369}.
	
		\item P. Palenzuela, L. Roca, J.M. Serrano (CIEMAT-PSA). “Steady-State
		Operation Dataset of an Experimental Surface Condenser Located at
		Plataforma Solar de Almería”. Zenodo, December, 2025.\\
		DOI: \href{https://doi.org/10.5281/zenodo.17312530}{10.5281/zenodo.17312530}.
	
		\item P. Palenzuela, L. Roca, J.M. Serrano (CIEMAT-PSA). “Steady-State
		Operation Dataset of an Experimental Combined Cooling System Located at
		Plataforma Solar de Almería”. Zenodo, December, 2025.\\
		DOI:\href{https://doi.org/10.5281/zenodo.17312546}{10.5281/zenodo.17312546}.

	\end{itemize}
\end{kaobox}


\begin{kaobox}[title=Open-source implementation,colback=Violet!15!white,colbacktitle=Violet!15!white]

	\begin{itemize}
		\item J.M. Serrano, L. Roca., P. Navarro. ``Repository with the implementation source code for modeling,
		optimization and simulation of a combined cooling system (wet cooling
		tower, dry cooler and surface condenser) at Plataforma Solar de Almería
		as part of the SOLhycool research project''.\\
		DOI: \href{https://doi.org/10.5281/zenodo.17882447}{10.5281/zenodo.17882447}

		\item J.M. Serrano, ``Repository with the implementation and results of the modeling and optimization of a solar-driven multi-effect distillation system at Plataforma Solar de Almería''.\\
		DOI: \href{https://doi.org/10.5281/zenodo.17882753}{10.5281/zenodo.17882753}

		\item J.M. Serrano, Repository with the source code for the PhD thesis
		manuscript: ``Towards optimal resource management in solar thermal
		applications: CSP and desalination''.\\
		DOI:~\doiHref
	\end{itemize}
	
\end{kaobox}