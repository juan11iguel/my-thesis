\pagelayout{wide} % No margins
\addpart{Conclusions and outlook}
\pagelayout{margin} % Restore margins

\section*{Conclusions}
\addcontentsline{toc}{section}{Conclusions}
\labsec{conclusions:conclusions}


% Water in CSP
% - almacenamiento y operación nocturna
% - uso del agua cuando esté disponible
% - potencial de optimización
% - hybrid/combined cooling como solución potencial para mayoría de nuevas
%   plantas CSP
% - 





\section*{Outlook and future work}
\addcontentsline{toc}{section}{Outlook and future work}
\labsec{conclusions:outlook}


%================================
\subsection*{Optimal water and electricity management in a combined cooling system}

\textbf{Improved Pareto front computation}. In the current optimization
implementation, the Pareto front for each step in the optimization horizon is
constructed using a grid search over the decision space. This approach can
become computationally expensive, especially as the grid resolution increases.
Additionally, the Pareto front must be recalculated from scratch at every step,
even though the sequential steps are often very similar—cost parameters remain
constant, and only the thermal load and weather conditions change, typically
with small variations. A more efficient solution would be to use a
multi-objective optimization algorithm, which can transfer evolved populations
between successive evaluations, significantly reducing redundant computations.

\textbf{Better water management} In the current implementation, the primary
water source is distributed evenly each day, so the optimization process
uses up the entire supply daily. However, a more intelligent daily
distribution—essentially, a new optimization problem—could improve water
management by allocating different amounts on different days, based on
expected weather conditions and predicted generation. This approach would
likely be incorporated as a new uppper layer in the hierarchical control
structure.\sidenote{The resulting structure would be: 1.~Water allocation,
2.~ \gls{ccsLabel} \hyperref[sec:cc:optimization]{operation optimization},
3.~\gls{ccLabel} regulatory control.}. At the higher level a simpler and
more abstract model would be considered to predict the long term behavior of
the system and to optimize it over a long time horizon, probably considering
the availability and capacity of a water reservoir.

\textbf{Analyze different combined coolers configurations and within each
configuration, different component sizes}. The cooler analyzed has a combined
dry and wet coolers which can either satisfy the nominal cooling load. Different
ratios could be analyzed and one would probably be a better fit for the
particular case study. Furthermore, the \fullgls{acheLabel} is used for the
\gls{dcLabel}, but other options could be considered and added to the
comparison, such as an \fullgls{accLabel} in parallel with a surface condenser
together with a \gls{wctLabel} or a deluged condenser.
    
This in itself is a design optimization problem that is not addressed in this
thesis. However, it is important to integrate a method like the proposed
optimization and include it in the design process to evaluate the
performance of different configurations and sizes. In the end the decision
of cooling system configuration and size will be informed by a
techno-economic analysis.
    

\textbf{Techno-economic analysis.} The presented cooling alternatives
comparative in this thesis focus on the operation cost of the system, but to get
a better picture of the alternatives performance, a techno-economic analysis
that includes the capital cost of the system and the expected lifetime of the
components should be performed \ie considering all costs associated with the
system the plant's lifetime. This is currently being worked on as part of
\href{https://solhycool.psa.es/}{SOLHycool}\sidenote{\url{https://solhycool.psa.es/}},
where the methodology presented here in terms of operation costs will be
integrated in a techno-economic analysis for different case studies.



%================================
\subsection*{Energy management in MED processes driven by variable energy sources}

\textbf{Alternative configurations for an MED brine concentrator}.
Configuraciones alternativas para procesos MED para aplicaciones de
concentración de salmueras: geometría variable de efectos, fuentes externas en
efectos distintos al primero, acoplamiento con MSF para efectos posteriores.

\textbf{Alternative configurations for solar-driven MED}. Configuraciones
alternativas para el proceso solar MED (almacenamiento con distintos puntos de
carga y descarga, MED con distintos puntos de fuente externa, etc. Incluir
diagrama de draw.io con las distintas configuraciones) \sidecite{schar_optimization_2023}

The layout configuration of the facility focused on realibility and simplifying operation and maintenance, not strictly on thermodynamic efficiency. The efficiency of the system could be improved:

\begin{enumerate}
    \item if direct coupling between solar field and thermal storage was used, avoiding the heat exchanger energy transfer associated losses

    \item thermal storage allowed charge and discharge from different levels, in order to take advantage of the temperature stratification and avoid fluid mixing

    \item 

    \item 
\end{enumerate}

These decisions were made to, on the one hand allow to separate the solar field and thermal storage into two distinct decoupled circuits, providing flexibility, reducing the volume of additives required (only added to the solar field circuit), and operational flexibility (other external loads can be connected to the solar field when the MED is not being operated).

In conclusion this system, although improvable, allows to validate the feasibility of the proposed approach by means of the implementation of a suitable control system, in such a way, that the ideas and techniques presented in this work, could be directly extrapolated to a commercial system just by modifying some of the decision variables to suit the particular implementation. 

\section*{Derived scientific contributions}
\addcontentsline{toc}{section}{Derived scientific contributions}
\labsec{conclusions:contributions}

% \begingroup % Local scope for the following commands

% % Define the style for the TOC, LOF, and LOT
% %\setstretch{1} % Uncomment to modify line spacing in the ToC
% %\hypersetup{linkcolor=blue} % Uncomment to set the colour of links in the ToC
% \setlength{\textheight}{230\hscale} % Manually adjust the height of the ToC pages

% % Turn on compatibility mode for the etoc package
% \etocstandarddisplaystyle % "toc display" as if etoc was not loaded
% \etocstandardlines % "toc lines as if etoc was not loaded

% % TODO: Add \listofcontributions

% \endgroup
The author has published or submitted for publication several journal articles,
contributed to conferences (national and international) and colloquiums:

\begin{kaobox}[title=Journal publications,colback=Azure2!25!white,colbacktitle=Azure2!25!white]

	\begin{itemize}
		\item J. M. Serrano, P. Navarro, J. Ruiz, P. Palenzuela, Manuel Lucas, and L. Roca. “Wet Cooling Tower Performance
		Prediction in CSP Plants: A Comparison between Artificial Neural
		Networks and Poppe's Model.” Energy, May 29, 2024, 131844.\\
		DOI: \href{https://doi.org/10.1016/j.energy.2024.131844}{https://doi.org/10.1016/j.energy.2024.131844}.
		
		\item P. Navarro, J. M. Serrano, L. Roca, P. Palenzuela, M. Lucas, and
		J. Ruiz. “A Comparative Study on Predicting Wet Cooling Tower
		Performance in Combined Cooling Systems for Heat Rejection in CSP
		Plants.” Applied Thermal Engineering, June 21, 2024, 123718.\\
		DOI: \href{https://doi.org/10.1016/j.applthermaleng.2024.123718}{https://doi.org/10.1016/j.applthermaleng.2024.123718}.

		\item J. M. Serrano, P. Navarro, L. Roca, P. Bartolomé, ..., P. Palenzuela,
		M. Lucas, and J. Ruiz. “Combined cooling for CSP plants: Modeling,
		experimental validation and optimization analysis” Applied Thermal
		Engineering, December 21, 2025, 123718.\\
		DOI: \href{https://doi.org/...}{https://doi.org/...} (Under review).

		
	\end{itemize}
\end{kaobox}


\begin{kaobox}[title=Contribution to conferences,colback=ForestGreen!15!white,colbacktitle=ForestGreen!15!white]

	\begin{itemize}
		\item J. M. Serrano, J. D. Gil, J. Bonilla, P. Palenzuela, and L. Roca,
		“Optimal operation of a combined cooling system” in 4th IFAC
		International Conference on Advances in Proportional-Integral-Derivative Control,
		Almería, Spain, 2024-06-12/2024-06-14.

		\item J. M. Serrano, J. D. Gil Vergel, J. Bonilla, P.
		Palenzuela, and L. Roca, “Operación óptima de un sistema de
		refrigeración combinada,” in XLIV Jornadas de Automática, Universidad
		de Zaragoza, 6, 7 y 8 de septiembre de 2023, Zaragoza, 2023rd ed., Aug.
		2023, pp. 477-482. \\DOI:
		\href{https://doi.org/10.17979/spudc.9788497498609.477}{10.17979/spudc.9788497498609.477}.

		\item P. Navarro, J. M. Serrano, J. Ruiz, M. Lucas, L. Roca, and
		P. Palenzuela. “Comparison Between an Artificial Neural
		Network and Poppe's Model for Wet Cooling Tower Performance Prediction
		in CSP Plants.” Efficiency, Cost, Optimization, Simulation and
		Environmental Impact of Energy Systems. International Conference., ECOS
		2023, June 25, 2023, 1609–20.\\DOI: \href{https://doi.org/10.52202/069564-0146}{10.52202/069564-0146}.

		\item L. Roca, J. M. Serrano, J.D. Gil, G. Zaragoza, M. Beschi, and A.
		Visioli. “Modelo de parámetros concentrados para captadores solares
		planos con reflectores.” Jornadas de Automática, no. 45 (July 2024): 45.
		\\DOI: \href{https://doi.org/10.17979/ja-cea.2024.45.10930}{10.17979/ja-cea.2024.45.10930}
	\end{itemize}
\end{kaobox}

\begin{kaobox}[title=Participation in conferences and colloquiums, colback=Violet!15!white,colbacktitle=Violet!15!white]

	\begin{itemize}
		\item J.M. Serrano, L. Roca, P. Palenzuela. "Yearly Simulation of a
		Combined Cooling System Integrated into a Concentrating Solar Power
		Plant". SolarPACES. Almería, Spain (September 2025).
		\item J.M. Serrano, P. Palenzuela and L. Roca. "Methodology for the
		implementation of a steady state simulation model in a multi-effect
		distillation plant. Case study: PSA MED pilot plant". Desalination for
		the Environment, Clean Water and Energy (EDS). Las Palmas de Gran
		Canaria, Spain (2022).
		\item J.M. Serrano, P. Palenzuela and L. Roca. Experimental evaluation
		of MED at high top brine temperatures with no divalent ions in feed
		water. Desalination for the Environment, Clean Water and Energy (EDS).
		Limassol, Chipre (2023).
		\item Patri ponemos el EDS de Marruecos?
		\item 3rd SFERA-III Doctoral Colloquiums:
		\begin{enumerate}
			\item Technological developments for solar multi-effect distillation
			processes. Almería, Spain (2021).
			\item Modelling and automation of a multi-effect distillation plant
			for the optimal coupling with solar energy. ETH Zurich, Switzerland (2022).
			\item Towards the optimal coupling of multi-effect distillation with
			solar energy. DLR. Cologne, Germany (2023).
		\end{enumerate}
	\end{itemize}

\end{kaobox}

As a result of the work developed in the present research work, several
repositories containing experimental datasets and open-source code have been
made publicly available. Particularly, each of the parts of the thesis has an
associated repository with the implementation of the presented results, in order
to facilitate transparency, reproducibility, and reusability of the developed
methods. Additionally, this thesis manuscript itself is also made available
together with all its associated media: 

\begin{kaobox}[title=Open datasets,colback=MediumPurple2!25!white,colbacktitle=MediumPurple2!25!white]

	\begin{itemize}
		
		\item P. Palenzuela, L. Roca, J.M. Serrano (CIEMAT-PSA). “Steady-State
		Operation Dataset of an Experimental Wet Cooling Tower Pilot Plant
		Located at Plataforma Solar de Almería.” Version 1.0.0. Zenodo, June 21,
		2024.\\
		DOI: \href{https://doi.org/10.5281/zenodo.10806201}{10.5281/zenodo.10806201}.
	
		\item P. Palenzuela, L. Roca, J.M. Serrano (CIEMAT-PSA). “Steady-State
		Operation Dataset of an Experimental Air-Cooled Heat Exchanger Located
		at Plataforma Solar de Almería.” Version 1.0.0. Zenodo, December? ,
		2025.\\
		DOI: \href{https://doi.org/10.5281/zenodo.17312369}{10.5281/zenodo.17312369}
		(To be published).
	
		\item P. Palenzuela, L. Roca, J.M. Serrano (CIEMAT-PSA). “Steady-State
		Operation Dataset of an Experimental Surface Condenser Located at
		Plataforma Solar de Almería.” Version 1.0.0. Zenodo, December? , 2025.\\
		DOI: \href{https://doi.org/10.5281/zenodo.17312530}{10.5281/zenodo.17312530}
		(To be published).
	
		\item P. Palenzuela, L. Roca, J.M. Serrano (CIEMAT-PSA). “Steady-State
		Operation Dataset of an Experimental Combined Cooling System Located at
		Plataforma Solar de Almería.” Version 1.0.0. Zenodo, December? , 2025.\\
		DOI:\href{https://doi.org/10.5281/zenodo.17312546}{10.5281/zenodo.17312546}
		(To be published).

	\end{itemize}
\end{kaobox}


\begin{kaobox}[title=Open-source implementation,colback=Goldenrod!25!white,colbacktitle=Goldenrod!25!white]

	\begin{itemize}
		\item J.M. Serrano, L. Roca. ``Repository with the implementation source code for modeling,
		optimization and simulation of a combined cooling system (wet cooling
		tower, dry cooler and surface condenser) at Plataforma Solar de Almería
		as part of the SOLhycool research project''.\\
		DOI: \href{https://doi.org/10.5281/zenodo.CHANGEME}{https://doi.org/10.5281/zenodo.CHANGEME}

		\item J.M. Serrano, ``Repository with the implementation and results of the modeling and optimization of a solar-driven multi-effect distillation system at Plataforma Solar de Almería''.\\
		DOI: \href{https://doi.org/10.5281/zenodo.CHANGEME}{https://doi.org/10.5281/zenodo.CHANGEME}

		\item J.M. Serrano, Repository with the source code for the PhD thesis
		manuscript: ``Towards optimal resource management in solar thermal
		applications: CSP and desalination''.\\
		DOI:~\doiHref
	\end{itemize}
	
\end{kaobox}